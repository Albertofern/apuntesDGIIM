\documentclass[11pt,a4paper, titlepage]{article}


% Packages
\usepackage[utf8]{inputenc}
\usepackage[spanish, es-tabla, es-lcroman]{babel}
\usepackage{amssymb, amsmath, amsthm}
\usepackage{upgreek}
\usepackage[margin=1in]{geometry}
\usepackage{enumitem}


% Info
\title{\textbf{Análisis Matemático I}\\ Ejercicios resueltos\\ \vspace{2em}\large{Doble Grado de Informática y Matemáticas}}
\author{}
\date{\vspace{-5em}Curso 2016/17}


% Custom
\providecommand{\abs}[1]{\lvert#1\rvert}    % \abs{} para valor absoluto
\setlength\parindent{0pt}       % no indentar
\newcommand\ddfrac[2]{\frac{\displaystyle #1}{\displaystyle #2}} % fracción grande
\setlist{leftmargin=.5in}    % indentación para las listas
\let\epsilon\upvarepsilon    % letra épsilon redefinida
\newcommand{\bm}[1]{\boldsymbol{#1}}    % comando para negrita en math mode


% Environments

%% dar nombre a teoremas y definiciones
\makeatletter
\def\th@plain{
  \thm@notefont{}
  \itshape
}
\def\th@definition{
  \thm@notefont{}
  \normalfont
}
\makeatother

%% teoremas, definiciones, proposiciones y notas
\theoremstyle{plain}
\newtheorem*{nth}{Teorema}
\newtheorem*{nprop}{Proposición} 
\newtheorem*{props}{Propiedades}
\theoremstyle{remark}
\newtheorem*{nota}{Nota}
\theoremstyle{definition}
\newtheorem*{ndef}{Definición}

%% listas ordenadas con (i), (ii), etc.
\newenvironment{nlist}
  {\begin{enumerate}\renewcommand\labelenumi{(\emph{\roman{enumi})}}}
  {\end{enumerate}}


% Document
\begin{document}
\maketitle


\section{Topología de un espacio métrico.}

\textbf{Ejercicio.} Dado el conjunto $A = \{ (x,y)\in \mathbb{R}^2: 0 < x \le 1 \}$, ¿es abierto?

\begin{proof}
Tenemos que comprobar si $A$ es abierto, es decir, si es cierto que\\
$\forall a \in A \ \ \exists s>0\ tal\ que\ B(a,s)\subset A$. Para ello, fijo $y \in \mathbb{R}$ y escojo $x_o = (1,y) \in A$. Además, tomo $s>0$ cualquiera. Veamos que hay puntos $z \in B(x_o, s)$ que no pertenecen a $A$.\\

Sea $\displaystyle z = (1 + \frac{s}{2}, y)$. Entonces, se tiene que $\displaystyle d(z,x_o)= \sqrt{\left(\frac{s}{2}\right)^2 + 0} = \frac{s}{2} < s$, y por tanto $z \in B(x_o,s)$.\\

Claramente $z\notin A$, pues $\displaystyle 1 + \frac{s}{2} > 1$. Así, concluimos que $z$ es un punto de $B(x_o,s)$ que no pertenece a $A$, por lo que $\displaystyle B(x_o,s) \not \subset A$, y $A$ no es abierto.\\

\textbf{@ Antonio Coín.}

\end{proof}

\end{document}