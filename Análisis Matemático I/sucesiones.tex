\documentclass[11pt]{article}

\usepackage[utf8]{inputenc}
\usepackage[spanish]{babel}
\usepackage[left=1.7cm,top=2cm,right=1.7cm,bottom=2.5cm]{geometry}
\usepackage{ upgreek }
\usepackage{amsmath}
\usepackage{ amssymb }
\usepackage{lmodern}
\usepackage[T1]{fontenc}
\usepackage{amsthm}
\usepackage{tikz-cd}
\newtheorem{theorem}{Teorema}
\newtheorem{fact}{Proposición}
\newtheorem{definition}{Definición}
\newtheorem{proofs}{Demostración}


\title{\textbf{Sucesiones en $\mathbb{R}^N$}}
\date{}
\begin{document}

\maketitle

\section*{Definición de sucesión en $\mathbb{R}^N$.}

Una sucesión en $\mathbb{R}^N$ es una aplicación 

\begin{align}
&x : \mathbb{N} \longrightarrow { \mathbb{R}^N }\notag \\
&\quad\;\; n \longmapsto { x(n) }\notag
\end{align}

A la que denotamos $x_n = x(n)$.

\section*{Convergencia de sucesiones en $\mathbb{R}^N$.}

De forma análoga a la convergencia de sucesiones en $\mathbb{R}$:

\[
	\{x_n\} \rightarrow x \iff \forall \varepsilon > 0\quad \exists m\in\mathbb{N} : n \ge m \implies d(x,x_n) < \varepsilon
\]

\begin{fact}
	Sean $x = (x^1, \dots, x^N)$ y $\{x_n\}$ con $x_n = (x^1_n, \dots, x^N_n)$.
	
	\[
		\{x_n\} \rightarrow x \iff \{x^i_n\} \rightarrow x^i
	\]
\end{fact}

%\begin{proofs}
%\fbox{$\impliedby$}

%Para esta demostración vamos a usar la distancia del máximo en lugar de la usual (todas las distancias son equivalentes en $\mathbb{R}^N$).\\

%Si $\forall i=1, ..., N$ tenemos que $\forall \varepsilon_i > 0\quad \exists m_i : n \ge m_i \implies |x^i_n - x^i| < \varepsilon_i$, se sigue que \[\displaystyle\max_{i=1,...,N} |x^i_n - x^i| < \displaystyle\max_{i=1,...,N} \varepsilon_i\]
%luego $d_\infty(x_n, x) < \displaystyle\max_{i=1,...,N} \varepsilon_i$.
%\end{proofs}

\begin{theorem} (de Bolzano-Weierstrass)
Sea $\{x_n\}$ una sucesión en $\mathbb{R}^N$ acotada. Entonces $\exists \sigma : \mathbb{N} \longrightarrow \mathbb{N}$ creciente tal que $\{x_{\sigma(n)}\}$ es convergente.
\end{theorem}

\begin{theorem} (de Weierstrass)
	Sea $f : A \subset \mathbb{R}^N \longrightarrow \mathbb{R}$ continua con $A$ compacto (?). Entonces existen $x_1,x_2$ tales que
	\[
		x_1 \le f(x) \le x_2\quad \forall x \in A 
	\]
\end{theorem}


\end{document}