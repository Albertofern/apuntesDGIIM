%%%%%%%%%%%%%%%%%%%%%%%%%%%%%%%%%%%%%%%%%%%%%%%%%%%%%%%%%%%%%%%%
%
% Apuntes de la asignatura Análisis Matemático I.
% Doble Grado de Informática y Matemáticas.
% Universidad de Granada.
% Curso 2016/17.
% 
% 
% Colaboradores:
% Javier Sáez (@fjsaezm)
% Daniel Pozo (@danipozodg)
% Pedro Bonilla (@pedrobn23)
% Guillermo Galindo
% Antonio Coín (@antcc)
%
% Agradecimientos:
% Andrés Herrera (@andreshp) y Mario Román (@M42) por
% las plantillas base.
%
% Sitio original:
% https://github.com/libreim/apuntesDGIIM/
%
% Licencia:
% CC BY 4.0 (https://creativecommons.org/licenses/by/4.0/)
%
%%%%%%%%%%%%%%%%%%%%%%%%%%%%%%%%%%%%%%%%%%%%%%%%%%%%%%%%%%%%%%%


%------------------------------------------------------------------------------
%   ACKNOWLEDGMENTS
%------------------------------------------------------------------------------

%%%%%%%%%%%%%%%%%%%%%%%%%%%%%%%%%%%%%%%%%%%%%%%%%%%%%%%%%%%%%%%%%%%%%%%%
% Plantilla básica de Latex en Español.
%
% Autor: Andrés Herrera Poyatos (https://github.com/andreshp) 
%
% Es una plantilla básica para redactar documentos. Utiliza el paquete  fancyhdr para darle un
% estilo moderno pero serio.
%
% La plantilla se encuentra adaptada al español.
%
%%%%%%%%%%%%%%%%%%%%%%%%%%%%%%%%%%%%%%%%%%%%%%%%%%%%%%%%%%%%%%%%%%%%%%%%%

%%%
% Plantilla de Trabajo
% Modificación de una plantilla de Latex de Frits Wenneker para adaptarla 
% al castellano y a las necesidades de escribir informática y matemáticas.
%
% Editada por: Mario Román
%
% License:
% CC BY-NC-SA 3.0 (http://creativecommons.org/licenses/by-nc-sa/3.0/)
%%%

%%%%%%%%%%%%%%%%%%%%%%%%%%%%%%%%%%%%%%%%
% Short Sectioned Assignment
% LaTeX Template
% Version 1.0 (5/5/12)
%
% This template has been downloaded from:
% http://www.LaTeXTemplates.com
%
% Original author:
% Frits Wenneker (http://www.howtotex.com)
%
% License:
% CC BY-NC-SA 3.0 (http://creativecommons.org/licenses/by-nc-sa/3.0/)
%
%%%%%%%%%%%%%%%%%%%%%%%%%%%%%%%%%%%%%%%%%


% Tipo de documento y opciones.
\documentclass[11pt, a4paper, titlepage]{article}


%---------------------------------------------------------------------------
%   PAQUETES
%---------------------------------------------------------------------------

% Idioma y codificación para Español.
\usepackage[utf8]{inputenc}
\usepackage[spanish, es-tabla, es-lcroman, es-noquoting]{babel}
\selectlanguage{spanish} 
%\usepackage[T1]{fontenc}

% Fuente utilizada.
\usepackage{courier}    % Fuente Courier.
\usepackage{microtype}  % Mejora la letra final de cara al lector.

% Diseño de página.
\usepackage{fancyhdr}   % Utilizado para hacer títulos propios.
\usepackage{lastpage}   % Referencia a la última página.
\usepackage{extramarks} % Marcas extras. Utilizado en pie de página y cabecera.
\usepackage[parfill]{parskip}    % Crea una nueva línea entre párrafos.
\usepackage{geometry}            % Geometría de las páginas.

% Símbolos y matemáticas.
\usepackage{amssymb, amsmath, amsthm, amsfonts, amscd}
\usepackage{upgreek}

% Otros.
\usepackage{enumitem}   % Listas mejoradas.
\usepackage[hidelinks]{hyperref}


%---------------------------------------------------------------------------
%   OPCIONES PERSONALIZADAS
%---------------------------------------------------------------------------

% Redefinir letra griega épsilon.
\let\epsilon\upvarepsilon

% Formato de texto.
\linespread{1.1}            % Espaciado entre líneas.
\setlength\parindent{0pt}   % No indentar el texto por defecto.
\setlist{leftmargin=.5in}   % Indentación para las listas.

% Estilo de página.
\pagestyle{fancy}
\fancyhf{}
\geometry{left=3cm,right=3cm,top=3cm,bottom=3cm,headheight=1cm,headsep=0.5cm}   % Márgenes y cabecera.

% Redefinir entorno de demostración (reducir espacio superior)
\makeatletter
\renewenvironment{proof}[1][\proofname] {\vspace{-15pt}\par\pushQED{\qed}\normalfont\topsep6\p@\@plus6\p@\relax\trivlist\item[\hskip\labelsep\it#1\@addpunct{.}]\ignorespaces}{\popQED\endtrivlist\@endpefalse}
\makeatother


%---------------------------------------------------------------------------
%   COMANDOS PERSONALIZADOS
%---------------------------------------------------------------------------

% Valor absoluto: \abs{}
\providecommand{\abs}[1]{\lvert#1\rvert}    

% Fracción grande: \ddfrac{}{}
\newcommand\ddfrac[2]{\frac{\displaystyle #1}{\displaystyle #2}}

% Texto en negrita en modo matemática: \bm{}
\newcommand{\bm}[1]{\boldsymbol{#1}}

% Línea horizontal.
\newcommand{\horrule}[1]{\rule{\linewidth}{#1}}


%---------------------------------------------------------------------------
%   CABECERA Y PIE DE PÁGINA
%---------------------------------------------------------------------------

% Cabecera del documento.
\renewcommand\headrule{
	\begin{minipage}{1\textwidth}
		\hrule width \hsize 
	\end{minipage}
}

% Texto de la cabecera.
\lhead{\subject}  % Izquierda.
\chead{}            % Centro.
\rhead{\docauthor}    % Derecha.

% Pie de página del documento.
\renewcommand\footrule{                                 
	\begin{minipage}{1\textwidth}
		\hrule width \hsize   
	\end{minipage}\par
}

% Texto del pie de página.
\lfoot{}                                                 % Izquierda
\cfoot{}                                                 % Centro.
\rfoot{Página\ \thepage\ de\ \protect\pageref{LastPage}} % Derecha.


%---------------------------------------------------------------------------
%   ENTORNOS PARA MATEMÁTICAS
%---------------------------------------------------------------------------

% Nuevo estilo para definiciones.
\newtheoremstyle{definition-style} % Nombre del estilo.
{10pt}               % Espacio por encima.
{10pt}               % Espacio por debajo.
{}                   % Fuente del cuerpo.
{}                   % Identación.
{\bf}                % Fuente para la cabecera.
{.}                  % Puntuación tras la cabecera.
{.5em}               % Espacio tras la cabecera.
{\thmname{#1}\thmnumber{ #2}\thmnote{ (#3)}}     % Especificación de la cabecera (actual: nombre en negrita).

% Nuevo estilo para notas.
\newtheoremstyle{remark-style} 
{10pt}                
{10pt}                
{}                   
{}                   
{\itshape}          
{.}                  
{.5em}               
{}                  

% Nuevo estilo para teoremas y proposiciones.
\newtheoremstyle{theorem-style}
{10pt}                
{10pt}                
{\itshape}           
{}                  
{\bf}             
{.}                
{.5em}               
{\thmname{#1}\thmnumber{ #2}\thmnote{ (#3)}}                   

% Nuevo estilo para ejemplos.
\newtheoremstyle{example-style}
{10pt}                
{10pt}                
{}                  
{}                   
{\scshape}              
{:}                 
{.5em}               
{}                   

% Teoremas, proposiciones y corolarios.
\theoremstyle{theorem-style}
\newtheorem*{nth}{Teorema}
\newtheorem*{nprop}{Proposición}
\newtheorem{ncor}{Corolario}

% Definiciones.
\theoremstyle{definition-style}
\newtheorem*{ndef}{Definición}

% Notas.
\theoremstyle{remark-style}
\newtheorem*{nota}{Nota}

% Ejemplos.
\theoremstyle{example-style}
\newtheorem*{ejemplo}{Ejemplo}

% Listas ordenadas con números romanos (i), (ii), etc.
\newenvironment{nlist}
{\begin{enumerate}
\renewcommand\labelenumi{(\emph{\roman{enumi})}}}
{\end{enumerate}}

% División por casos con llave a la derecha.
\newenvironment{rcases}
  {\left.\begin{aligned}}
  {\end{aligned}\right\rbrace}

%---------------------------------------------------------------------------
%   PÁGINA DE TÍTULO
%---------------------------------------------------------------------------

% Título del documento.
\newcommand{\subject}{Análisis Matemático I}

% Autor del documento.
\newcommand{\docauthor}{Doble Grado de Informática y Matemáticas}

% Título
\title{
  \normalfont \normalsize 
  \textsc{Universidad de Granada} \\ [25pt]    % Texto por encima.
  \horrule{0.5pt} \\[0.4cm] % Línea horizontal fina.
  \huge \subject\\ % Título.
  \horrule{2pt} \\[0.5cm] % Línea horizontal gruesa.
}

% Autor.
\author{\Large{\docauthor}}

% Fecha.
\date{\vspace{-1.5em} \normalsize Curso 2016/17}


%---------------------------------------------------------------------------
%   COMIENZO DEL DOCUMENTO
%---------------------------------------------------------------------------
\begin{document}

\maketitle  % Título.
\tableofcontents    % Índice
\newpage



%----------------------------
%   Introducción.
%----------------------------
\section*{Introducción.}

El objetivo de este curso es el estudio de las funciones de varias variables, es decir, de funciones $f: \mathbb{R}^N \longrightarrow \mathbb{R}^M$. Para ello, empezaremos caracterizando el espacio $\mathbb{R}^N$, y proseguiremos intentando traspasar los resultados principales sobre funciones reales de variable real a nuestro campo de estudio, así como enunciando otros nuevos.

Es por esto que es fundamental haber cursado con aprovechamiento las asignaturas de \emph{Cálculo I y II}, que tratan exclusivamente sobre funciones reales de variable real.

Aunque nos centraremos en funciones en el espacio $\mathbb{R}^N$, muchos de los resultados que obtendremos son igual de válidos en un espacio métrico en general, e incluso en espacios topológicos.

\newpage


%--------------------------------------
%   Topología de un espacio métrico.
%--------------------------------------
\section{Topología de un espacio métrico.}

%%% El espacio métrico R^N.
\subsection{Concepto de espacio métrico. El espacio métrico $\mathbb{R}^N$.}

\begin{ndef}[Espacio métrico]
Consideremos un conjunto $X$ cualquiera, y una aplicación \mbox{$d:X\times X \longrightarrow \mathbb{R}$} que cumple las siguientes propiedades:

\begin{nlist}
\item $d(x,y) \ge 0\ \ \forall x,y \in X$.

\item $d(x,y) = 0 \iff x = y\ \ \forall x,y \in X$.

\item $d(x,y) = d(y,x)\ \ \forall x,y \in X$.

\item $d(x,y) \leq d(x,z) + d(z,y)\ \ \forall x,y,z \in X. \quad \quad(desigualdad\ triangular)$

\end{nlist}

Entonces, se dice que el par $(X,d)$ es un \emph{espacio métrico}.
\end{ndef}



\begin{nota}
En adelante, entenderemos $\mathbb{R}^N$ como el espacio métrico $(\mathbb{R}^N,d)$, siendo $d$ la distancia usual \textbf{(distancia euclídea)} dada por: $$d(x,y) = \sqrt{\sum_{i=1}^N (y_i - x_i)^2}\ \ \forall x,y\in \mathbb{R}^N.$$

Existen otras distancias en $\mathbb{R}^N$. Las más destacadas son las siguientes:

\begin{nlist}
\item $\displaystyle d_1(x,y) = \sum_{i=1}^N \abs{x_i - y_i}\ \ \forall x,y \in \mathbb{R}^N$.

\item $\displaystyle d_{\infty}(x,y) = m\acute{a}x \{\abs{x_i - y_i}: \ i=1,\dots,N \}\ \ \forall x,y\in \mathbb{R}^N$.

\item $\displaystyle d_p(x,y) = \left( \sum_{i=1}^N \abs{x_i - y_i}^p \right)^{1/p}\ \ \forall x,y\in \mathbb{R}^N$.\\
\end{nlist}

\end{nota}



\begin{ndef}
Sean $(X,d)$ y $(X,d')$ dos espacios métricos sobre un mismo conjunto $X$. Se dice que las distancias $d$ y $d'$ son \textit{equivalentes} si, y solo si, $$\exists k_1,k_2 > 0 :\ k_1d(x,y)\le d'(x,y) \le k_2d(x,y)\ \ \forall x,y\in X.$$
\end{ndef}



\begin{nprop}
En $\mathbb{R}^N$, todas las distancias mencionadas anteriormente son equivalentes entre sí. En particular, la distancia euclídea es equivalente a todas ellas.
\end{nprop}


%%% Conceptos topológicos.
\subsection{Conceptos topológicos.} 

\begin{ndef}[Bola abierta]
Sea $(X,d)$ un espacio métrico, y fijemos un $x\in X$ y un $\epsilon > 0$. Se llama \emph{bola abierta de centro $x$ y radio $\epsilon$} al conjunto $B(x,\epsilon) = \{ y\in X \ | \ d(x,y)<\epsilon\}$.
\end{ndef}



\begin{ndef}[Bola cerrada]
De forma análoga, se define la \emph{bola cerrada de centro $x$ y radio $\epsilon$} como el conjunto $\bar{B}(x,\epsilon) = \{y\in X \ | \ d(x,y)\leq \epsilon \}$.
\end{ndef}



\begin{ndef}[Conjunto abierto]
Sea $(X,d)$ un espacio métrico, y sea $A\subseteq X$. Decimos que \mbox{$A\ es\ abierto \iff \forall a \in A\ \exists \epsilon > 0: B(x,\epsilon) \subseteq A$}.	
\end{ndef}



\begin{nprop}
Sea $(X,d)$ un espacio métrico. Entonces, $\forall x \in X \ \forall \epsilon > 0$ se tiene que $B(x,\epsilon)$ es un conjunto abierto.
\end{nprop}

\begin{proof}

Sea $x \in B(x_0,\epsilon_0) $ arbitrario. Para demostrar que $B(x_0,\epsilon_0)$ es un abierto, tenemos que encontrar un $\epsilon > 0$ tal que $B(x,\epsilon) \subseteq B(x_0,\epsilon_0)  $, y por lo tanto comprobar que se verifica que $\forall y \in B(x,\epsilon) \Rightarrow y \in B(x_0,\epsilon_0).$

Sea $y \in B(x, \epsilon )$ cualquiera. Consideremos $r = d(x, x_0)$, y tomemos $\epsilon = \epsilon_0 - r$. Queremos demostrar que $y \in B(x_0, \epsilon_0)$. Para ello, veamos que $d(x_0, y) < \epsilon_0$. En efecto, por la desigualdad triangular se cumple que: $$d(x_0, y) \leq d(x,x_0) + d(x,y) < r + \epsilon = r + \epsilon_0 - r = \epsilon_0 $$
Luego queda demostrado que $y \in B(x_0, \epsilon_0)$, y por tanto podemos afirmar que para todo punto $x \in B(x_0, \epsilon_0)$ se puede encontrar una bola abierta centrada en él, tal que todos sus puntos están en el conjunto de origen.
\end{proof}



\begin{nprop}
Sea $(X,d)$ un espacio métrico. Entonces, se verifican las siguientes propiedades:

\begin{nlist}
\item $Si\ \{A_\lambda \ | \ \lambda \in \Lambda \}$ es una familia de subconjuntos abiertos de $X$, entonces $\displaystyle \bigcup_{\lambda \in \Lambda} A_\lambda$ es abierto.

\item Si $\{A_1,\dots, A_n\}$ es una familia finita de abiertos de $X$, entonces $\displaystyle \bigcap_{i=1}^n A_i$ es abierto.

\item $X,\emptyset$ son abiertos.
\end{nlist}

\end{nprop}



\begin{ndef}[Punto interior]
Sea $(X,d)$ un espacio métrico, y consideremos $A\subseteq X$, $a\in A$. Se dice que $a$ \emph{es un punto interior de} $A$ si, y solo si, $\exists \epsilon_0 > 0: B(a,\epsilon_0)\subseteq A$.  Definimos $int(A) = \mathring{A} = \{ a\in A \ | \ a\ es\ punto\ interior\ de\ A\}$.
\end{ndef}



\begin{nprop}
Sea $(X,d)$ un espacio métrico, y $A\subseteq X$. Entonces, se verifican las siguientes propiedades:

\begin{nlist}
\item $\mathring{A} \subseteq A$.

\item $\mathring{A}$ es abierto.

\item Si $B\subseteq A$ es un subconjunto abierto de $A$, entonces $B \subseteq \mathring{A}$. Es decir, $\mathring{A}$ es el abierto más grande contenido en $A$.

\item $\displaystyle \mathring{A}  = \bigcup \{ B\subseteq A \ | \ B\ es\ abierto \}$.

\item $A$ es abierto $\iff \mathring{A} =A$.

\item $int(int(A)) = int(A).$

\item Si $A\subseteq B$, entonces $\mathring{A} \subseteq \mathring{B}$.
\end{nlist}

\end{nprop}



\begin{ndef}[Conjunto cerrado]
Sea $(X,d)$ un espacio métrico, y $F\subseteq X$. Se dice que el conjunto $F\ es\ cerrado \iff X-F\ es\ abierto$.
\end{ndef}



\begin{nprop}
Sea $(X,d)$ un espacio métrico. Entonces, $\forall x\in X \ \forall \epsilon > 0$ se tiene que $\bar{B}(x,\epsilon)$ es un conjunto cerrado.
\end{nprop}



\begin{nprop}
Sea $(X,d)$ un espacio métrico. Entonces, se verifican las siguientes propiedades:

\begin{nlist}
\item $Si\ \{F_\lambda \ | \ \lambda \in \Lambda \}$ es una familia de cerrados de $X$, entonces $\displaystyle \bigcap_{\lambda \in \Lambda} F_\lambda$ es cerrado.

\item Si $\{F_1,\dots, F_n\}$ es una familia finita de cerrados de $X$, entonces $\displaystyle \bigcup_{i=1}^n F_i$ es cerrado.

\item $X,\emptyset$ son cerrados.
\end{nlist}

\end{nprop}



\begin{ndef}[Clausura]
Sea $(X,d)$ un espacio métrico. Se llama \textit{clausura o cierre de A} al conjunto $\bar{A} = X - int(X-A)$.
\end{ndef}



\begin{nprop}
Sea $(X,d)$ un espacio métrico, y $A\subseteq X$. Entonces, se verifican las siguientes propiedades:
\begin{nlist}
\item $A \subseteq \bar{A}$.

\item $\bar{A}$ es cerrado.

\item Si $B\subseteq X$ es un subconjunto cerrado de $X$ tal que $A\subseteq B$, entonces $\bar{A} \subseteq B$. Es decir, $\bar{A}$ es el cerrado más pequeño que contiene a $A$.

\item $\displaystyle \bar{A}  = \bigcap \{ F\subseteq X \ | \ F\ es\ cerrado\ y\ A\subseteq F \}$.

\item $A$ es cerrado $\iff \bar{A} = A$.

\item $\bar{\bar{A}} = \bar{A}.$

\item Si $A\subseteq B$, entonces $\bar{A} \subseteq \bar{B}$.
\end{nlist}

\end{nprop}



\begin{ndef}[Frontera]
Sea $(X,d)$ un espacio métrico, y $A\subseteq X$. Llamamos \textit{frontera de A} al conjunto $\partial A = \bar{A}-\mathring{A}$.	
\end{ndef}



\begin{nprop}
Sea $(X,d)$ un espacio métrico, y $A\subseteq X$. Entonces, se verifica lo siguiente:
$x\in \partial A \iff \forall \epsilon > 0\ B(x,\epsilon)\cap A \neq \emptyset \ y\ B(x,\epsilon)\cap (X-A) \neq \emptyset$.
\end{nprop}



\begin{ndef}[Punto de acumulación]
Sea $(X,d)$ un espacio métrico, y $A\subseteq X$. Dado $x\in X$, decimos que \textit{x es punto de acumulación de} $A \iff \forall \epsilon > 0\ B(x,\epsilon)\cap (A-\{x\})\neq \emptyset$. Definimos $A' = \{ x\in X \ | \ x\ es\ punto\ de\ acumulaci\acute{o}n\ de\ A \}$.
\end{ndef}



\begin{nprop}
Sea $(X,d)$ un espacio métrico. Entonces, se verifican las siguientes afirmaciones:

\begin{nlist}
\item $\mathring{A} = X - \overline{X-A}$

\item $\bar{A} = A \cup \partial A$.

\item $\bar{A} = A \cup A'$

\item $\partial A \subseteq A'$

\item $X = int(A) \cup \partial A \cup int(X-A)$. Además, la unión es disjunta dos a dos.
\end{nlist}

\end{nprop}

\newpage


%--------------------------------------
%   Sucesiones en R^N.
%--------------------------------------
\section{Sucesiones en $\mathbb{R}^N$.}



\begin{ndef}[Sucesión en $\mathbb{R}^N$]
Una sucesión en $\mathbb{R}^N$ es una aplicación $x: \mathbb{N} \longrightarrow \mathbb{R}^N$ que a cada $n\in \mathbb{N}$ le hace corresponder un $x(n) \in \mathbb{R}^N$. Por simplicidad, al elemento imagen de $n$ se le denomina $x_n$, y la aplicación $x$ se denota $\{x_n\}$.
\end{ndef}



\begin{ndef}[Convergencia de sucesiones]
Sea $(X,d)$ un espacio métrico, $A\subseteq X$ y $x\in X$. Decimos que una sucesión $\{x_n\}$ de puntos de $A$ converge a $x$ si, y solo si: $$ \forall \epsilon > 0\ \ \exists n_o \in \mathbb{N}: \ n\ge n_o \Rightarrow d(x_n,x) < \epsilon.$$
\end{ndef}



\begin{nota}
Este concepto no depende de la distancia equivalente elegida.
\end{nota}



\begin{nprop}
Sea $A\subseteq \mathbb{R}^N$, $x\in \mathbb{R}^N$, y $\{x_n\}$ una sucesión de puntos de $A$. Adoptemos la notación $x_n = (x_n^1, x_n^2,\dots,x_n^N)$, y $x=(x^1, x^2,\dots, x^N)$. Entonces, se verifica que:
$$\{x_n\} \rightarrow x \iff \{x_n^j\} \rightarrow x^j.$$
\end{nprop}



%\begin{proof}
%\fbox{$\impliedby$}

%Para esta demostración vamos a usar la distancia del máximo en lugar de la usual (todas las distancias son equivalentes en $\mathbb{R}^N$).\\

%Si $\forall i=1, ..., N$ tenemos que $\forall \varepsilon_i > 0\quad \exists m_i : n \ge m_i \implies |x^i_n - x^i| < \varepsilon_i$, se sigue que \[\displaystyle\max_{i=1,...,N} |x^i_n - x^i| < \displaystyle\max_{i=1,...,N} \varepsilon_i\]
%luego $d_\inf



\begin{ndef}
Sea $(X,d)$ un espacio métrico, y $x\in X$. Consideremos, para cada $n\in \mathbb{N}$, un punto $a_n \in X$. Entonces, decimos que $d(a_n,x) \rightarrow 0$ $\iff \{a_n\} \rightarrow x$.
\end{ndef}



\begin{ndef}[Conjunto acotado]
Sea $A \subseteq \mathbb{R}^N$. Decimos que $A$ \textit{está acotado} si, y solo si, $\exists R>0: A\subseteq B(0,R)$.
\end{ndef}



\begin{ndef}[Sucesión acotada]
Sea $\{x_n\}$ una sucesión de puntos de $R^N$. Entonces, decimos que $\{x_n\}$ \textit{está acotada} sí, y solo sí, $\{x_n\ | \ n\in \mathbb{N}\}$ está acotado.
\end{ndef}



\begin{nprop}
Si una sucesión $\{x_n\} \subseteq \mathbb{R}^N$ es acotada, entonces $\forall i=1,\dots,n$ la sucesión $\{x_n^i\}$ es acotada (en $\mathbb{R}$).
\end{nprop}



\begin{nota}
Si un conjunto $A\subseteq \mathbb{R}^N$ es acotado, entonces cualquier sucesión de puntos de $A$ es acotada.
\end{nota}



\begin{nth}[Bolzano-Weierstrass]
Sea $\{x_n\}\subseteq \mathbb{R}^N$ acotada. Entonces, existe una sucesión parcial suya $\{x_{\sigma_{(n)}}\}$ convergente.
\end{nth}



\begin{ndef}[Sucesión de Cauchy]
Sea $\{x_n\}\subseteq \mathbb{R}^N$. Decimos que $\{x_n\}$ es una \textit{sucesión de Cauchy} $\iff \forall \epsilon > 0\ \ \exists n_o \in \mathbb{N}: \ n,m\ge n_o \Rightarrow d(x_n,x_m) < \epsilon$.
\end{ndef}



\begin{nth} [$\bm{\mathbb{R}^N}$ es completo]
Sea $\{x_n\}\subseteq \mathbb{R}^N$. Entonces: $$\{x_n\}\ es\ de\ Cacuchy\ \iff \{x_n\}\ es\ convergente.$$
\end{nth}



\begin{nprop}
Sea $\{x_n\} \subseteq \mathbb{R}^N\ con\ \{x_n\} \rightarrow x \in \mathbb{R}^N$. Entonces, toda sucesión parcial de $\{x_n\}$ es convergente a $x$.
\end{nprop}

\newpage


%-----------------------------------
%   Funciones continuas en R^N.
%-----------------------------------
\section{Funciones continuas en $\mathbb{R}^N$.}



\begin{ndef}[Función continua]
Sea $\emptyset \ne A\subseteq \mathbb{R}^N$, $f: A \longrightarrow \mathbb{R}^M$ y $a \in A$. Decimos que \textit{f es continua en $a$} si, y solo si: $$\forall \epsilon > 0\ \ \exists \delta > 0: \ x\in A, \ d(x,a)<\delta \Rightarrow d(f(x),f(a))<\epsilon.$$
Además, se dice que $f$ es continua si lo es en todos sus puntos.	
\end{ndef}



\begin{nprop}[Caracterización de continuidad]
Sea $\emptyset \ne A \subseteq \mathbb{R}^N$, y $f:A\longrightarrow \mathbb{R}^M$. Entonces: $$f\ es\ continua\ en\ a \iff \forall \{x_n\}\subseteq A\ con\ \{x_n\} \rightarrow a \Rightarrow \{f(x_n)\} \rightarrow f(a).$$
\end{nprop}



\begin{ndef}[Continuidad uniforme]
Sea $\emptyset \ne A \subseteq \mathbb{R}^N$, $f:A \longrightarrow \mathbb{R}^M$. Se dice que $f$ es uniformemente continua si, y solo si: $$\forall \epsilon > 0 \ \ \exists \delta > 0 : \ x,y \in A,\ d(x,y) < \delta \Rightarrow d(f(x),f(a)) < \epsilon.$$
\end{ndef}


\begin{ndef}[Conjunto compacto]
Sea $(X,d)$ un espacio métrico, y sea $\emptyset \ne A \subseteq X$. $$A\ es\ compacto \iff \forall \{x_n\} \subseteq A\ \ \exists \{x_{\sigma(n)}\} \rightarrow x\in A.$$
\end{ndef}

\begin{ndef}[Recubrimiento abierto]
	Sea $A \subseteq \mathbb{R}^N$. Se dice que una familia $\{O_i, i\in I\}$ de abiertos es un \emph{recubrimiento abierto} de $A$ si
	\[
		A \subseteq \bigcup_{i\in I} O_i
	\]
	
	También, si $R_1$ y $R_2$ son recubrimientos abiertos de $A$ y $R_1 \subseteq R_2$, se dice que $R_1$ es un \emph{subrecubrimiento abierto} de $R_2$.
\end{ndef}

\begin{nprop}
	Sea $A \subseteq \mathbb{R}^N$ compacto. Entonces existe un subrecubrimiento finito de $A$.
\end{nprop}

\begin{nprop}[Caracterización de cerrados]
Sea $(X,d)$ un espacio métrico, y $A\subseteq X$. Entonces, son equivalentes:

\begin{nlist}
\item $A$ es cerrado.
\item $\forall \{x_n\} \subseteq A$ convergente a un $x \in X$, se verifica que $x\in A$.
\end{nlist}

\end{nprop}

\begin{proof} 
Veamos las dos implicaciones:

$\displaystyle \boxed{\Rightarrow}\ $ Supongamos $A \subseteq X$ un conjunto cerrado. Entonces, $X - A$ es abierto. Sea $\{x_n\}$ una sucesión de puntos de $A$ que converge a un $x\in X$. Para comprobar que, de hecho, $x\in A$, argumentamos por reducción al absurdo:

\underline{Supongamos $x\notin A$}. Entonces, $x\in X - A$, y por ser este último conjunto abierto, encontramos un $\epsilon > 0\ tal\ que\ B(x,\epsilon)\subseteq (X-A)$. Pero por ser $x$ el límite de la sucesión $\{x_n\}$, se tiene que $\exists n_o \in \mathbb{N}: n\geq n_o \Rightarrow d(x_n,x)<\epsilon$. Es decir, a partir de cierto índice en adelante, $x_n \in B(x,\epsilon)\ con\ x_n \in A\ \forall n \in \mathbb{N}$. Esto se contradice con el hecho de que $B(x,\epsilon)\subseteq (X-A)$, pues encontramos en dicha bola puntos $x_n$ que no pertenecen a $X-A$.

Por tanto, concluimos que $x\in A$.

$\displaystyle \boxed{\Leftarrow}\ $ Sea $A\subseteq X$, y supongamos que se verifica que $\forall \{x_n\} \subseteq A\ tal\ que\ \{x_n\} \rightarrow x \in X,\ se\ tiene\ que\ x\in A$. Para ver que $A$ es cerrado, utilizaremos la siguiente caracterización de conjuntos cerrados:
\vspace{0.5em}
$$A\ es\ cerrado\ \iff \bar{A} = A $$

Si recordamos, se define la frontera de $A$ como $\partial A = \bar{A} - \mathring{A}$. Por tanto, la equivalencia anterior quedaría así: $A\ es\ cerrado \iff \partial A \cup \mathring{A} = A$. Para comprobar esta última igualdad, veamos las dos inclusiones:

\begin{description}
\item $\displaystyle \boxed{\subseteq}\ $ Sabemos por la definición del conjunto de puntos interiores de $A$, que $\mathring{A} \subseteq A$. \\ Comprobemos entonces que $\partial A \subseteq A$:

Sea $x\in \partial A$ cualquiera. Por una caracterización de la frontera de $A$, sabemos que $\forall \epsilon > 0\ B(x,\epsilon)\cap A \neq \emptyset$. Si tomamos $\epsilon = \frac{1}{n} > 0\ con \ n\in \mathbb{N}$, tenemos que $B(x,\frac{1}{n})\cap A \neq \emptyset$, es decir, $\exists a_n \in B(x,\frac{1}{n})\cap A\ tal\ que\ d(x,a_n)<\epsilon = \frac{1}{n}$. Podemos construir entonces, para cada $n \in \mathbb{N}$, la sucesión $\{a_n\}$.

Así, se tiene que $0 < d(x,a_n) < \frac{1}{n}\ \forall n\in \mathbb{N}$, de donde concluimos que $d(x,a_n) \rightarrow 0$. Por definición, esto significa que $\{a_n\} \rightarrow x$, lo que por hipótesis implica, al ser $\{a_n\}$ una sucesión convergente de puntos de $A$, que  $x\in A$. Por tanto, se verifica que $\partial A \subseteq A$.

\item $\displaystyle \boxed{\supseteq}\ $ Esta inclusión es trivial, pues sabemos que $A \subseteq \bar{A}$, y por tanto $A \subseteq \partial A \cup \mathring{A} = \bar{A}$.
\end{description}

De esta forma, queda probada la equivalencia.
\end{proof}



\begin{nprop}[Caracterización de compactos]
Sea $(X,d)$ un espacio métrico, y sea $A \subseteq X$. Entonces: $$A\ es\ compacto \iff A\ es\ cerrado\ y\ acotado.$$
\end{nprop}



\begin{nprop}
Sea $\{x_n\} \subseteq \mathbb{R}^N$ convergente a un $x_o \in \mathbb{R}^N$. Entonces, el conjunto \hfill \\$A = \{x_n:\ n=0,1,2,\dots \}$ es compacto.
\end{nprop}



%%% Clasificación de conjuntos en R^N.
\subsection{Clasificación de conjuntos en $\mathbb{R}^N$}

\begin{ndef}[Conjunto convexo]
Un conjunto $A\subseteq \mathbb{R}^N$ se dice \textit{convexo} si $\forall x,y \in A$ se tiene que el segmento de extremos $x$ e $y$ está incluido en $A$. En otras palabras: $$A\ convexo\ \iff [x,y] = \{tx + (1-t)y: \ t\in [0,1]\} \subseteq A.$$
\end{ndef}



\begin{ndef}[Poligonalmente convexo]
Un conjunto $A\subseteq \mathbb{R}^N$ se dice \textit{poligonalmente convexo} si  $\forall x,y \in A$ existe una poligonal que los une y no se sale de $A$. En otras palabras:  $A\ poligonalmente\ convexo \iff \exists \{x= a_0, a_1,\dots,a_k=y \}\subseteq A$ tal que: $$\bigcup_{i=1}^k [a_{i-1},a_i] \subseteq A.$$
\end{ndef}



\begin{ndef}[Conjunto arco-conexo]
Un conjunto $A \subseteq \mathbb{R}^N$ se dice \emph{arco-conexo(conexo por arcos)} si $\forall x,y \in A$ existe un camino incluido en $A$ que los une. En otras palabras,  $A\ es\ conexo\ por\ arcos \iff \exists \varphi:[a,b] \longrightarrow \mathbb{R}^N$ verificando: $$\varphi(a) = x;\quad \varphi(b) = y;\quad \varphi([a,b]) \subseteq A.$$
\end{ndef}



\begin{ndef}[Conjunto no conexo]
Decimos que un conjunto $A\in \mathbb{R}^N$ es \textit{NO conexo} si existen $U,\ V$ abiertos en $\mathbb{R}^N$ tales que: $$U \cap A \ne \emptyset;\quad V \cap A \ne \emptyset;\quad A \subseteq U \cup V;\quad A \cap U \cap V = \emptyset.$$
\end{ndef}



\begin{nota}
La misma definición se aplica para un espacio topológico $(X,\tau).$
\end{nota}



\begin{ndef}[Conjunto conexo]
Un conjunto $A\subseteq \mathbb{R}^N$ se dice conexo si no es no conexo. Equivalentemente, $\forall \ U,V$ abiertos en $\mathbb{R}^N$ tales que $U \cap A \ne \emptyset, \ V \cap A \ne \emptyset,\ A \subseteq U \cup V,$ se tiene que forzosamente $A \cap U \cap V \ne \emptyset$.
\end{ndef}



\begin{nprop}
Sea $A\subseteq \mathbb{R}$ un conjunto arco-conexo. Entonces, $A$ es convexo.
\end{nprop}

\begin{proof}
Sean $x,y \in A$, y supongamos sin pérdida de generalidad que $x \le y$. Sabemos que por ser $A$ arco-conexo, $\exists \varphi : [a,b] \longrightarrow \mathbb{R}$ función continua verificando: $$\varphi(a) = x;\quad \varphi(b) = y;\quad \varphi([a,b]) \subseteq A.$$
Como $\varphi$ es una función continua definida en un intervalo cerrado y acotado, aplicamos el \textbf{teorema del valor intermedio} en $\mathbb{R}$, y obtenemos que $\varphi([a,b])$ es un intervalo. Por ser un intervalo, verificará que $\forall \alpha, \beta \in \varphi([a,b])\ con\ \alpha \le \beta,\ \ [\alpha,\beta] \subseteq \varphi([a,b])$.

Por tanto, como $\varphi(a), \varphi(b) \in \varphi([a,b])$, concluimos que: $$[\varphi(a),\ \varphi(b)] = [x,y] \subseteq \varphi([a,b]) \subseteq A.$$
Así, hemos demostrado que $\forall x,y \in A\ \ [x,y] \subseteq A$, y por tanto, $A$ es convexo.
\end{proof}



\begin{nprop}
Sea $A \subseteq \mathbb{R}^N$ convexo. Entonces, $A$ es arco-conexo.
\end{nprop}

\begin{proof} \hfill \\
Fijemos $x,y \in A$ arbitrarios, y construyamos la aplicación $\varphi: [0,1] \longrightarrow \mathbb{R}^N$ dada por: $$\varphi(t) = (1-t)x + ty\ \ \forall t\in [0,1]$$
Una primera observación es que $\varphi([0,1]) = \{(1-t)x + ty: t \in [0,1]\} = [x,y] \subseteq A$ por ser $A$ convexo. También se desprende de la definición de $\varphi$ que $\varphi(0) = x$ y $\varphi(1) = y$.

Para comprobar que $\varphi$ es continua, utilicemos la caracterización de la continuidad por sucesiones:

Sea $\{x_n\} \subseteq [0,1]\ con\ \{x_n\} \rightarrow a \in [0,1]$. Entonces, $\{\varphi(x_n) \} = \{(1- x_n)x + x_n y\}$. \\Apliquemos ahora propiedades de las sucesiones convergentes, y obtenemos que: $$\{\varphi(x_n) \} \rightarrow (1-a)x + ay = \varphi(a).$$
Entonces, $\forall \{x_n\} \subseteq [0,1]\ con\ \{x_n\} \rightarrow a \Rightarrow \{\varphi(x_n) \} \rightarrow \varphi(a)$, por lo que $\varphi$ es continua.

Así, queda probado que $A$ es conexo por arcos.
\end{proof}



\begin{nprop}
Sea $A\in \mathbb{R}^N$ arco-conexo. Entonces, $A$ es conexo.
\end{nprop}



\begin{nprop}
Sea $A\subseteq \mathbb{R}^N$ abierto y conexo por arcos. Entonces, $A$ es poligonalmente convexo.
\end{nprop}


%%% Continuidad en espacios topológicos. Topología inducida.
\subsection{Continuidad en espacios topológicos. Topología inducida.}

\begin{ndef}[Continuidad en espacios topológicos]
Sean $(X,\tau_x),\ (Y,\tau_y)$ dos espacios topológicos, y sea $f:X\longrightarrow Y$. Entonces: $$f\ es\ continua \iff f^{-1}(B) \in \tau_x \ \ \forall B \in \tau_y.$$
\end{ndef}



\begin{ndef}[Topología inducida]
Sea $(X,\tau)$ un espacio topológico, y $A\subseteq X$. Entonces, $\tau_A = \{B\cap A: \ B \in \tau\}$ es la \textit{topología inducida en $A$}.
\end{ndef}



\begin{nprop}[Caracterización de abiertos en topología inducida]
Sea $(X,\tau)$ un espacio topológico, y $A\subseteq X$. Si $(A,\tau_A)$ es el espacio topológico inducido en $A$, entonces: $$B' \in \tau_A \iff \exists B\in \tau: \ B' = B\cap A.$$
\end{nprop}



\begin{nprop}
Sea $(X,\tau)$ un espacio topológico, y $A\subseteq X$. Entonces, $A$ es no conexo si, y solo si, existen $U,V$ \textbf{abiertos en $\bm{(A,\tau_A)}$} tales que: $$U \ne \emptyset \ne V;\quad A \subseteq U \cup V;\quad U \cap V = \emptyset.$$
\end{nprop}



\begin{ndef}[Continuidad en topología inducida]
Sean $(X,\tau_x),\ (Y,\tau_y)$ dos espacios topológicos, $A\subseteq X$, y $f:A\longrightarrow Y$. Entonces: $$f\ es\ continua \iff f\ es\ continua\ en\ (A,\tau_A).$$
\end{ndef}



%%% Teoremas sobre funciones continuas en R^N.
\subsection{Teoremas sobre funciones continuas en $\mathbb{R}^N$}



\begin{nth}[Weierstrass]
Sea $(X,d)$ un espacio métrico, $\emptyset \ne A \subseteq X$ compacto, y $f:A \longrightarrow \mathbb{R}$ continua en $A$. Entonces, $\exists x_1,x_2 \in A: \ f(x_1)\le f(x) \le f(x_2)\ \ \forall x\in A$. En otras palabras, la función $f$ alcanza su mínimo y su máximo.
\end{nth}



\begin{nth}[Weierstrass generalizado]
Sean $(X,d)$, $(Y,d)$ espacios métricos, $\emptyset \ne A \subseteq X$ compacto, y $f: A \longrightarrow Y$ continua. Entonces, $f(A)$ es compacto.
\end{nth}



\begin{nth}[Valor Intermedio]
Sea $\emptyset \ne A \subseteq \mathbb{R}^N$ arco conexo, y $f: A \longrightarrow \mathbb{R}^M$ continua. Entonces, $f(A)$ es arco-conexo en $\mathbb{R}^M$.
\end{nth}

\begin{proof}
Sean $X,Y\in f(A)$. Entonces, $\exists x,y \in A : X=f(x), \ Y=f(y)$. Como $A$ es arco-conexo, $\exists\varphi : [a,b]\longrightarrow \mathbb{R}^N \text{ continua tal que}\ \varphi(a) = x,\; \varphi(b)=y,\; \varphi([a,b]) \subseteq A$.


Ahora, definimos $\psi := f\circ \varphi : [a,b] \longrightarrow \mathbb{R}^M$, que es continua por ser composición de funciones continuas. Entonces, se verifica que: $$\psi(a) = f(\varphi(a)) = f(x) = X;\quad \psi(b)= f(\varphi(b)) = f(y) = Y;\quad \psi([a,b]) = f(\varphi([a,b])) \subseteq f(A).$$
Por tanto, queda probado que $f(A)$ es arco-conexo en $\mathbb{R}^M$.
\end{proof}



\begin{nth}[Valor Intermedio revisitado]
Sea $\emptyset \ne A\subseteq \mathbb{R}^N$ conexo, y $f:A \longrightarrow \mathbb{R}^M$ continua. Entonces, $f(A)$ es conexo en $\mathbb{R}^M$.
\end{nth}

\begin{nth}[Heine-Cantor]
	Sea $\emptyset\ne A \subseteq \mathbb{R}^N$ compacto, y $f : A \longrightarrow \mathbb{R}^N$ continua. Entonces f es uniformemente continua en $A$.
\end{nth}


\begin{proof}
	$f$ es continua en A $\implies$ $f$ es continua en $a\;\; \forall a \in A$. Ahora, sea $\epsilon>0$ fijo.
	\[
	    \forall a \in A \quad \exists \delta = \delta_a > 0\;\; \forall x\in A\;\; d(x,a) < \delta_a \implies d(f(x),f(a))<\epsilon
	\]

	Tomamos un recubrimiento abierto de $A$, y como $A$ es compacto, encontramos un subrecubrimiento finito.
	
	\[
		A \subseteq \bigcup_{a\in A} B(a, \frac{\delta_a}{2}) \implies \exists a_1,\dots,a_n \in A: A \subseteq \bigcup_{i=1}^n B\left(a_i, \frac{\delta_{a_i}}{2}\right)
	\]
	
	Por esta última inclusión:
	\[
		\forall x\in A\quad \exists i \in \left \{ 1,\dots,n \right \} : x\in B\left(a_i,\frac{\delta_{a_i}}{2}\right)\cap A \implies f(x)\in B(f(a_i),\epsilon)
	\]
	
	Sean $\delta = \min\left\{\ddfrac{\delta_{a_i}}{2} : i \in \left \{ 1,\dots,n \right \}\right\} > 0$ y $y\in A : d(x,y) < \delta < \delta_{a_i}$ para un $x\in A$ fijo. Tomamos el $a_i$ proporcionado por la proposición anterior para $x$.
	\[
		d(y,a_i) \le d(y,x)+d(x,a_i) < \delta_{a_i} \implies y\in B(a_i,\delta_{a_i}) \implies f(y) \in B(f(a_i), \epsilon)
	\]
	
	Finalmente,
	\[
		d(f(x), f(y)) \le d(f(x),f(a_i)) + d(f(a_i), f(y)) < \epsilon
	\]
	
	Para cualquier $\epsilon$ para el que se desee que se verifique la condición de la continuidad uniforme, basta tomar $\ddfrac{\epsilon}{2}$ en la continuidad.
\end{proof}

\begin{proof}[Demostración alternativa]
	La condición para la continuidad uniforme es la siguiente:
	\[
		\forall \varepsilon > 0 \  \exists \delta > 0 \ \forall x, y \in A : d(x,y) < \delta \implies d (f(x) , f(y) ) < \varepsilon
	\]
	
	Vamos a proceder por reducción al absurdo, para lo cual negamos esta condición:
	\[
		\exists \varepsilon_0 > 0 \ \forall \delta > 0 \ \exists x, y \in A : d (x,y) < \delta \wedge d (f(x) , f(y) ) \ge \varepsilon_0
	\]
	
	Tomamos este $\epsilon_0$, lo que nos da, para cada $\delta>0$, un par de puntos $x$ e $y$ que cumplen la propiedad expresada arriba. Tomamos $\delta = \frac{1}{n} \ \forall n\in \mathbb{N}$. Esto nos da dos sucesiones $\{x_n\}$ e $\{y_n\}$ tales que
	\[
		d(x_n,y_n) < \frac{1}{n} \wedge d(f(x_n),f(y_n)) \ge \epsilon_0
	\]
	
	Por ser A compacto, el teorema de Bolzano-Weierstrass nos da dos sucesiones parciales $\{x_{n_k}\}$ a $x_0$ e $\{y_{n_k}\}$ a $y_0$. Por tanto:
	\[
		d(x_{n_k},y_{n_k}) < \frac{1}{n_k} \wedge d(f(x_{n_k}),f(y_{n_k})) \ge \epsilon_0
	\]
	
	Sin embargo, $\{x_{n_k}\}$ e $\{y_{n_k}\}$ convergen al mismo punto (por converger su distancia a cero), y como $f$ es continua, esta proposición no puede ser verdadera. Hemos llegado por tanto a una contradicción, luego $f$ debe ser uniformemente continua.
\end{proof}


\section{Límite funcional en $\mathbb{R}^N$.}

\begin{ndef}[Límite funcional]
	Sean $\emptyset \ne A \subseteq \mathbb{R}^N$, $a\in A'$ y $f: A \longrightarrow \mathbb{R}^M$. Entonces $f$ tiene límite $l$ en $x=a$, y se denota $\displaystyle\lim_{x\to a} f(x)$ si:
	
\[
	\forall \epsilon>0\quad \exists \delta>0 : \begin{rcases}
	0<d(x,a)<\delta\\
	x\in A
\end{rcases} \implies d(f(x), l) < \epsilon
\]
\end{ndef}



\begin{nprop}[Caracterización punto de acumulación]
Sea $(X,d)$ un espacio métrico, y $A\subseteq X$. Consideremos un punto $x\in X$. Son equivalentes:

\begin{nlist}
\item x es punto de acumulación de A.
\item $\exists \{a_n\}\subseteq A-\{x\}$ tal que $\{a_n\} \rightarrow x$.
\item $\forall \epsilon > 0\ B(x,\epsilon)\cap (A-\{x\})$ es un conjunto infinito. 
\end{nlist}
\end{nprop}



\section{Funciones derivables en $\mathbb{R}^N$.}
\subsection{Concepto de función derivable.}

Sea $A$ un abierto de $\mathbb{R}^N$. Partimos de la siguiente observación:
\[
	\forall x_0 \in A\quad \exists \delta >0 : B(x_0, \delta) \subset A \implies \forall v \in \mathbb{R}^N\quad \exists \epsilon > 0 : [\; t \in (-\epsilon, \epsilon) \implies x_0+tv\in B(x_0,\delta)\;]
\]
 En particular, si $|v| = 1 \implies \epsilon = \delta$.
 
\begin{ndef}[Función derivable] Sean $f : A \longrightarrow \mathbb{R}^M$, y $x_0\in A$. Se dice que $f$ es derivable en $x_0$, según Fréchet, si
\[
	\exists L\in Lin(\mathbb{R}^N, \mathbb{R}^M) : \lim_{x\to x_0} \frac{|f(x)-f(x_0)- L(x-x_0)|}{|x-x_0|} =0
\]

Notamos $Df(x_0) = L$.
\end{ndef}

\begin{nota}[1]\hfill
	\begin{nlist}
		\item El límite tiene sentido porque $x_0\in A'$.
		\item El límite anterior es equivalente a $\lim_{y\to 0} \frac{|f(x_0+y)-f(x_0)-L(y)|}{|y|}$.
	\end{nlist}
\end{nota}

\begin{nota}[2]
	$A$ es abierto $\implies$ $L$ (si existe) es única. De aquí se exige que $A$ sea un abierto. 
\end{nota}

\begin{proof}[Demostración (Nota 2)]
	Suponemos que $\exists L_1,L_2\in Lin(\mathbb{R}^N, \mathbb{R}^M)$ tales que 
\[
\lim_{x\to x_0} \frac{|f(x)-f(x_0)- L_1(x-x_0)|}{|x-x_0|} =0= \frac{|f(x)-f(x_0)- L_2(x-x_0)|}{|x-x_0|}
\]

Entonces, dado un $x\in A$:

\begin{align}
	\frac{|L_1(x-x_0) - L_2(x-x_0)|}{|x-x_0|} \le \frac{|f(x)-f(x_0)- L_1(x-x_0)|}{|x-x_0|} + \frac{|f(x)-f(x_0)- L_2(x-x_0)|}{|x-x_0|}\notag
\end{align}

\[
\implies \lim_{x\to x_0} \frac{|(L_1-L_2)(x-x_0)|}{|x-x_0|} = 0
\]
(Terminar)
\end{proof}

\begin{nprop} En los mismos términos: si $f$ es derivable en $x_0$ $\implies$ $f$ es continua en $x_0$.
	
\end{nprop}

\begin{proof} Para probar esta proposición, hay que probar que $\lim_{x\to x_0} (f(x)-f(x_0)) = 0$
\[
	\lim_{x\to x_0} (f(x)-f(x_0)) = \underbrace{\lim_{x\to x_0} (f(x)-f(x_0)-L(x-x_0))}_{=\;0 \text{ ($f$ derivable)}} + \underbrace{\lim_{x-x_0} L(x-x_0)}_{=\; 0 \text{ ($L$ lineal $\implies$ continua)}} = 0
\]
\end{proof}

\begin{ndef}[Derivada direccional]
	Sea $v\in \mathbb{R}^N$, con $|v| = 1$. $f$ es derivable en $x_0$ en la dirección $v$ si:
	\[
		\exists \lim_{t\to 0} \frac{f(x_0+tv)-f(x_0)}{t} = D_v f(x_0) \iff
	\]
	
	\begin{center}
	$\iff f_1,f_2,\cdots f_m$ derivable direccionalmente en $x_0$ en la dirección v. $\iff$
	\[
	\iff D_vf(x_0) = (D_vf_1(x_0),\cdots,D_vf_m(x_0))
	\]
\end{center}
\end{ndef}


\begin{nprop}
	Sea $f$ derivable en $x_0\implies f$ derivable a lo largo de la dirección v y $D_vf(x_0) = Df(x_0)(v)$
	\begin{proof}
	$f$ derivable en $x_0$. Tomo $y=tv  $. Podemos ver entonces:
	
	\[
	\lim_{t \to 0} \frac{f(x_0 +tv) - f(x_0) - Df(x_0)(tv)}{t} = 0 \implies
	\]
	\[
	\implies \lim_{t \to 0} \abs{\frac{f(x_0 +tv) - f(x_0)}{t} - Df(x_0)(\frac{tv}{t})} \implies \lim_{t \to 0}\frac{f(x_0 +tv) -f(x_0)}{t} - Df(x_0)(v)
	\]
	\[
	\implies \lim_{t \to 0}\frac{f(x_0 +tv) - f(x_0)}{t} = Df(x_0)(v)
	\]
Hemos probado que $\exists D_vf(x_0)$
\end{proof}
\end{nprop}




\end{document}