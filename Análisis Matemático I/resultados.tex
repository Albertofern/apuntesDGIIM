\documentclass[11pt,a4paper, titlepage]{article}


% Packages
\usepackage[utf8]{inputenc}
\usepackage[spanish, es-tabla, es-lcroman]{babel}
\usepackage{amssymb, amsmath, amsthm}
\usepackage{upgreek}
\usepackage[margin=1in]{geometry}
\usepackage{enumitem}


% Info
\title{\textbf{Análisis Matemático I}\\ Teoremas, proposiciones, definiciones y propiedades\\ \vspace{2em}\large{Doble Grado de Informática y Matemáticas}}
\author{}
\date{\vspace{-5em}Curso 2016/17}


% Custom
\providecommand{\abs}[1]{\lvert#1\rvert}    % \abs{} para valor absoluto
\setlength\parindent{0pt}       % no indentar
\newcommand\ddfrac[2]{\frac{\displaystyle #1}{\displaystyle #2}} % fracción grande
\setlist{leftmargin=.5in}    % indentación para las listas
\let\epsilon\upvarepsilon    % letra épsilon redefinida
\newcommand{\bm}[1]{\boldsymbol{#1}}    % comando para negrita en math mode


% Environments

%% dar nombre a teoremas y definiciones
\makeatletter
\def\th@plain{
  \thm@notefont{}
  \itshape
}
\def\th@definition{
  \thm@notefont{}
  \normalfont
}
\makeatother

%% teoremas, definiciones, proposiciones y notas
\theoremstyle{plain}
\newtheorem*{nth}{Teorema}
\newtheorem*{nprop}{Proposición} 
\newtheorem*{props}{Propiedades}
\theoremstyle{remark}
\newtheorem*{nota}{Nota}
\theoremstyle{definition}
\newtheorem*{ndef}{Definición}

%% listas ordenadas con (i), (ii), etc.
\newenvironment{nlist}
  {\begin{enumerate}\renewcommand\labelenumi{(\emph{\roman{enumi})}}}
  {\end{enumerate}}


% Document
\begin{document}
\maketitle



\section{Topología de un espacio métrico.}



\subsection{Concepto de espacio métrico. El espacio métrico $\mathbb{R}^N$.}



\begin{ndef}[Espacio métrico]
Consideremos un conjunto $X$ y una aplicación \mbox{$d:X\times X \longrightarrow \mathbb{R}$} que cumple las siguientes propiedades:

\begin{nlist}
\item $d(x,y) \ge 0\ \ \forall x,y \in X$.
\item $d(x,y) = 0 \iff x = y\ \ \forall x,y \in X$.
\item $d(x,y) = d(y,x)\ \ \forall x,y \in X$.
\item $d(x,y) \leq d(x,z) + d(z,y)\ \ \forall x,y,z \in X. \quad \quad(desigualdad\ triangular)$
\end{nlist}

Entonces, se dice que el par $(X,d)$ es un \emph{espacio métrico}.
\end{ndef}



\begin{nota}
En adelante, entenderemos $\mathbb{R}^N$ como el espacio métrico $(\mathbb{R}^N,d)$, siendo $d$ la distancia usual \textbf{(distancia euclídea)} dada por: $$d(x,y) = \sqrt{\sum_{i=1}^N (y_i - x_i)^2}\ \ \forall x,y\in \mathbb{R}^N.$$

Existen otras distancias en $\mathbb{R}^N$. Las más destacadas son las siguientes:

\begin{nlist}
\item $\displaystyle d_1(x,y) = \sum_{i=1}^N \abs{x_i - y_i}\ \ \forall x,y \in \mathbb{R}^N$.
\item $\displaystyle d_{\infty}(x,y) = m\acute{a}x \{\abs{x_i - y_i}: \ i=1,\dots,N \}\ \ \forall x,y\in \mathbb{R}^N$.
\item $\displaystyle d_p(x,y) = \left( \sum_{i=1}^N \abs{x_i - y_i}^p \right)^{1/p}\ \ \forall x,y\in \mathbb{R}^N$.
\end{nlist}

\end{nota}



\begin{ndef}
Sean $(X,d)$ y $(X,d')$ dos espacios métricos sobre un mismo conjunto $X$. Se dice que las distancias $d$ y $d'$ son \textit{equivalentes} \mbox{$\iff \exists k_1,k_2 > 0 :\ k_1d(x,y)\le d'(x,y) \le k_2d(x,y)\ \ \forall x,y\in X$.}
\end{ndef}



\begin{nprop}
En $\mathbb{R}^N$, todas las distancias mencionadas anteriormente son equivalentes entre sí. En particular, la distancia euclídea es equivalente a todas ellas.\\
\end{nprop}



\subsection{Conceptos topológicos.} 



\begin{ndef}[Bola abierta]
Sea $(X,d)$ un espacio métrico, y fijemos un $x\in X$ y un $\epsilon > 0$. Se llama \emph{bola abierta de centro $x$ y radio $\epsilon$} al conjunto $B(x,\epsilon) = \{ y\in X \ | \ d(x,y)<\epsilon\}$.
\end{ndef}



\begin{ndef}[Bola cerrada]
De forma análoga, se define la \emph{bola cerrada de centro $x$ y radio $\epsilon$} como el conjunto $\bar{B}(x,\epsilon) = \{y\in X \ | \ d(x,y)\leq \epsilon \}$.
	
\end{ndef}



\begin{ndef}[Conjunto abierto]
Sea $(X,d)$ un espacio métrico, y sea $A\subseteq X$. Decimos que \mbox{$A\ es\ abierto \iff \forall a \in A\ \exists \epsilon > 0: B(x,\epsilon) \subseteq A$}.	
\end{ndef}



\begin{nota}
Sea $(X,d)$ un espacio métrico. Entonces, $\forall x \in X \ \forall \epsilon > 0$ se tiene que $B(x,\epsilon)$ es un conjunto abierto.
\end{nota}

\newpage

\begin{nprop}
Sea $(X,d)$ un espacio métrico. Entonces, se verifican las siguientes propiedades:

\begin{nlist}
\item $Si\ \{A_\lambda \ | \ \lambda \in \Lambda \}$ es una familia de subconjuntos abiertos de $X$, entonces $\displaystyle \bigcup_{\lambda \in \Lambda} A_\lambda$ es abierto.

\item Si $\{A_1,\dots, A_n\}$ es una familia finita de abiertos de $X$, entonces $\displaystyle \bigcap_{i=1}^n A_i$ es abierto.

\item $X,\emptyset$ son abiertos.
\end{nlist}

\end{nprop}



\begin{ndef}[Punto interior]
Sea $(X,d)$ un espacio métrico, y consideremos $A\subseteq X$, $a\in A$. Se dice que $a$ \emph{es un punto interior de} $A$ si, y solo si, $\exists \epsilon_0 > 0: B(a,\epsilon_0)\subseteq A$.  Definimos $int(A) = \mathring{A} = \{ a\in A \ | \ a\ es\ punto\ interior\ de\ A\}$.
\end{ndef}



\begin{nprop}
Sea $(X,d)$ un espacio métrico, y $A\subseteq X$. Entonces, se verifican las siguientes propiedades:

\begin{nlist}
\item $\mathring{A} \subseteq A$.

\item $\mathring{A}$ es abierto.

\item Si $B\subseteq A$ es un subconjunto abierto de $A$, entonces $B \subseteq \mathring{A}$. Es decir, $\mathring{A}$ es el abierto más grande contenido en $A$.

\item $\displaystyle \mathring{A}  = \bigcup \{ B\subseteq A \ | \ B\ es\ abierto \}$.

\item $A$ es abierto $\iff \mathring{A} =A$.

\item $int(int(A)) = int(A).$

\item Si $A\subseteq B$, entonces $\mathring{A} \subseteq \mathring{B}$.
\end{nlist}

\end{nprop}



\begin{ndef}[Conjunto cerrado]
Sea $(X,d)$ un espacio métrico, y $F\subseteq X$. Se dice que el conjunto $F\ es\ cerrado \iff X-F\ es\ abierto$.
\end{ndef}



\begin{nota}
Sea $(X,d)$ un espacio métrico. Entonces, $\forall x\in X \ \forall \epsilon > 0$ se tiene que $\bar{B}(x,\epsilon)$ es un conjunto cerrado.
\end{nota}



\begin{nprop}
Sea $(X,d)$ un espacio métrico. Entonces, se verifican las siguientes propiedades:

\begin{nlist}
\item $Si\ \{F_\lambda \ | \ \lambda \in \Lambda \}$ es una familia de cerrados de $X$, entonces $\displaystyle \bigcap_{\lambda \in \Lambda} F_\lambda$ es cerrado.

\item Si $\{F_1,\dots, F_n\}$ es una familia finita de cerrados de $X$, entonces $\displaystyle \bigcup_{i=1}^n F_i$ es cerrado.

\item $X,\emptyset$ son cerrados.
\end{nlist}

\end{nprop}



\begin{ndef}[Clausura]
Sea $(X,d)$ un espacio métrico. Se llama \textit{clausura o cierre de A} al conjunto $\bar{A} = X - int(X-A)$.
\end{ndef}



\begin{nprop}
Sea $(X,d)$ un espacio métrico, y $A\subseteq X$. Entonces, se verifican las siguientes propiedades:
\begin{nlist}
\item $A \subseteq \bar{A}$.

\item $\bar{A}$ es cerrado.

\item Si $B\subseteq X$ es un subconjunto cerrado de $X$ tal que $A\subseteq B$, entonces $\bar{A} \subseteq B$. Es decir, $\bar{A}$ es el cerrado más pequeño que contiene a $A$.

\item $\displaystyle \bar{A}  = \bigcap \{ F\subseteq X \ | \ F\ es\ cerrado\ y\ A\subseteq F \}$.

\item $A$ es cerrado $\iff \bar{A} = A$.

\item $\bar{\bar{A}} = \bar{A}.$

\item Si $A\subseteq B$, entonces $\bar{A} \subseteq \bar{B}$.
\end{nlist}

\end{nprop}



\begin{ndef}[Frontera]
Sea $(X,d)$ un espacio métrico, y $A\subseteq X$. Llamamos \textit{frontera de A} al conjunto $\partial A = \bar{A}-\mathring{A}$.	
\end{ndef}



\begin{nprop}
Sea $(X,d)$ un espacio métrico, y $A\subseteq X$. Entonces, se verifica lo siguiente:
$x\in \partial A \iff \forall \epsilon > 0\ B(x,\epsilon)\cap A \neq \emptyset \ y\ B(x,\epsilon)\cap (X-A) \neq \emptyset$.
\end{nprop}



\begin{ndef}[Punto de acumulación]
Sea $(X,d)$ un espacio métrico, y $A\subseteq X$. Dado $x\in X$, decimos que \textit{x es punto de acumulación de} $A \iff \forall \epsilon > 0\ B(x,\epsilon)\cap (A-\{x\})\neq \emptyset$. Definimos $A' = \{ x\in X \ | \ x\ es\ punto\ de\ acumulaci\acute{o}n\ de\ A \}$.
\end{ndef}



\begin{nprop}
Sea $(X,d)$ un espacio métrico. Entonces, se verifican las siguientes afirmaciones:

\begin{nlist}
\item $\mathring{A} = X - \overline{X-A}$
\item $\bar{A} = A \cup \partial A$.
\item $\bar{A} = A \cup A'$
\item $\partial A \subseteq A'$
\item $X = int(A) \cup \partial A \cup int(X-A)$. Además, la unión es disjunta dos a dos.
\end{nlist}

\end{nprop}

\newpage






\section{Sucesiones en $\mathbb{R}^N$.}



\begin{ndef}[Sucesión en $\mathbb{R}^N$]
Una sucesión en $\mathbb{R}^N$ es una aplicación $x: \mathbb{N} \longrightarrow \mathbb{R}^N$ que a cada $n\in \mathbb{N}$ le hace corresponder un $x(n) \in \mathbb{R}^N$. Por simplicidad, al elemento imagen de $n$ se le denomina $x_n$, y la aplicación $x$ se denota $\{x_n\}$.
\end{ndef}



\begin{ndef}[Convergencia de sucesiones]
Sea $(X,d)$ un espacio métrico, $A\subseteq X$ y $x\in X$. Decimos que una sucesión $\{x_n\}$ de puntos de $A$ converge a $x$ si, y solo si: $$ \forall \epsilon > 0\ \ \exists n_o \in \mathbb{N}: \ n\ge n_o \Rightarrow d(x_n,x) < \epsilon.$$
\end{ndef}



\begin{nota}
Este concepto no depende de la distancia equivalente elegida.
\end{nota}



\begin{nprop}
Sea $A\subseteq \mathbb{R}^N$, $x\in \mathbb{R}^N$, y $\{x_n\}$ una sucesión de puntos de $A$. Adoptemos la notación $x_n = (x_n^1, x_n^2,\dots,x_n^N)$, y $x=(x^1, x^2,\dots, x^N)$. Entonces, se verifica que:
$$\{x_n\} \rightarrow x \iff \{x_n^j\} \rightarrow x^j.$$
\end{nprop}



\begin{ndef}
Sea $(X,d)$ un espacio métrico, y $x\in X$. Consideremos, para cada $n\in \mathbb{N}$, un punto $a_n \in X$. Entonces, decimos que $d(a_n,x) \rightarrow 0$ $\iff \{a_n\} \rightarrow x$.
\end{ndef}



\begin{ndef}[Conjunto acotado]
Sea $A \subseteq \mathbb{R}^N$. Decimos que $A$ \textit{está acotado} si, y solo si, $\exists R>0: A\subseteq B(0,R)$.
\end{ndef}



\begin{ndef}[Sucesión acotada]
Sea $\{x_n\}$ una sucesión de puntos de $R^N$. Entonces, decimos que $\{x_n\}$ \textit{está acotada} sí, y solo sí, $\{x_n\ | \ n\in \mathbb{N}\}$ está acotado.
\end{ndef}



\begin{nota}
Si una sucesión $\{x_n\} \subseteq \mathbb{R}^N$ es acotada, entonces $\forall i=1,\dots,n$ la sucesión $\{x_n^i\}$ es acotada (en $\mathbb{R}$).
\end{nota}



\begin{nota}
Si un conjunto $A\subseteq \mathbb{R}^N$ es acotado, entonces cualquier sucesión de puntos de $A$ es acotada.
\end{nota}



\begin{nth}[Bolzano-Weierstrass]
Sea $\{x_n\}\subseteq \mathbb{R}^N$ acotada. Entonces, existe una subsucesión $\{x_{\sigma_n}\}$ convergente.
\end{nth}



\begin{ndef}[Sucesión de Cauchy]
Sea $\{x_n\}\subseteq \mathbb{R}^N$. Decimos que $\{x_n\}$ es una \textit{sucesión de Cauchy} $\iff \forall \epsilon > 0\ \ \exists n_o \in \mathbb{N}: \ n,m\ge n_o \Rightarrow d(x_n,x_m) < \epsilon$.
\end{ndef}



\begin{nth} [$\bm{\mathbb{R}^N}$ es completo]
Sea $\{x_n\}\subseteq \mathbb{R}^N$. Entonces: $$\{x_n\}\ es\ de\ Cacuchy\ \iff \{x_n\}\ es\ convergente.$$
\end{nth}



\begin{nprop}
Sea $\{x_n\} \subseteq \mathbb{R}^N\ con\ \{x_n\} \rightarrow x \in \mathbb{R}^N$. Entonces, toda sucesión parcial de $\{x_n\}$ es convergente a $x$.
\end{nprop}

\newpage






\section{Funciones continuas en $\mathbb{R}^N$.}



\begin{ndef}[Función continua]
Sea $\emptyset \ne A\subseteq \mathbb{R}^N$, $f: A \longrightarrow \mathbb{R}^M$ y $a \in A$. Decimos que \textit{f es continua en $a$} si, y solo si: $$\forall \epsilon > 0\ \ \exists \delta > 0: \ x\in A, \ d(x,a)<\delta \Rightarrow d(f(x),f(a))<\epsilon.$$

Además, se dice que $f$ es continua si lo es en todos sus puntos.	
\end{ndef}



\begin{nprop}[Caracterización de continuidad]
Sea $\emptyset \ne A \subseteq \mathbb{R}^N$, y $f:A\longrightarrow \mathbb{R}^M$. Entonces: $$f\ es\ continua\ en\ a \iff \forall \{x_n\}\subseteq A\ con\ \{x_n\} \rightarrow a \Rightarrow \{f(x_n)\} \rightarrow f(a).$$
\end{nprop}



\begin{ndef}[Continuidad uniforme]
Sea $\emptyset \ne A \subseteq \mathbb{R}^N$, $f:A \longrightarrow \mathbb{R}^M$. Se dice que $f$ es uniformemente continua si, y solo si: $$\forall \epsilon > 0 \ \ \exists \delta > 0 : \ x,y \in A,\ d(x,y) < \delta \Rightarrow d(f(x),f(a)) < \epsilon.$$
\end{ndef}


\begin{ndef}[Conjunto compacto]
Sea $(X,d)$ un espacio métrico, y sea $\emptyset \ne A \subseteq X$. $$A\ es\ compacto \iff \forall \{x_n\} \subseteq A\ \ \exists \{x_{\sigma(n)}\} \rightarrow x\in A.$$
\end{ndef}



\begin{nprop}[Caracterización de cerrados]
Sea $(X,d)$ un espacio métrico, y $A\subseteq X$. Entonces, son equivalentes:

\begin{nlist}
\item $A$ es cerrado.
\item $\forall \{x_n\} \subseteq A$ convergente a un $x \in X$, se verifica que $x\in A$.
\end{nlist}

\end{nprop}



\begin{nprop}[Caracterización de compactos]
Sea $(X,d)$ un espacio métrico, y $A \subseteq X$. Entonces: $$A\ es\ compacto \iff A\ es\ cerrado\ y\ acotado.$$
\end{nprop}

\vspace{0.10em}

\begin{nprop}
Sea $\{x_n\} \subseteq \mathbb{R}^N$ convergente a un $x_o \in \mathbb{R}^N$. Entonces, el conjunto \hfill \\$A = \{x_n:\ n=0,1,2,\dots \}$ es compacto.\\
\end{nprop}



\subsection{Clasificación de conjuntos en $\mathbb{R}^N$}



\begin{ndef}[Conjunto convexo]
Un conjunto $A\subseteq \mathbb{R}^N$ se dice \textit{convexo} si $\forall x,y \in A$ se tiene que el segmento de extremos $x$ e $y$ está incluido en $A$. En otras palabras: $$A\ convexo\ \iff [x,y] = \{tx + (1-t)y: \ t\in [0,1]\} \subseteq A.$$
\end{ndef}



\begin{ndef}[Poligonalmente convexo]
Un conjunto $A\subseteq \mathbb{R}^N$ se dice \textit{poligonalmente convexo} si  $\forall x,y \in A$ existe una poligonal que los une y no se sale de $A$. En otras palabras:  $A\ poligonalmente\ convexo \iff \exists \{x= a_0, a_1,\dots,a_k=y \}\subseteq A$ tal que: $$\bigcup_{i=1}^k [a_{i-1},a_i] \subseteq A.$$
\end{ndef}



\begin{ndef}[Conjunto arco-conexo]
Un conjunto $A \subseteq \mathbb{R}^N$ se dice \emph{arco-conexo(conexo por arcos)} si $\forall x,y \in A$ existe un camino incluido en $A$ que los une. En otras palabras,  $A\ es\ conexo\ por\ arcos \iff \exists \varphi:[a,b] \longrightarrow \mathbb{R}^N$ verificando: $$\varphi(a) = x;\quad \varphi(b) = y;\quad \varphi([a,b]) \subseteq A.$$
\end{ndef}



\begin{ndef}[Conjunto no conexo]
Decimos que un conjunto $A\in \mathbb{R}^N$ es \textit{NO conexo} si existen $U,\ V$ abiertos en $\mathbb{R}^N$ tales que: $$U \cap A \ne \emptyset;\quad V \cap A \ne \emptyset;\quad A \subseteq U \cup V;\quad A \cap U \cap V = \emptyset.$$
\end{ndef}



\begin{nota}
La misma definición se aplica para un espacio topológico $(X,\tau).$
\end{nota}



\begin{ndef}[Conjunto conexo]
Un conjunto $A\subseteq \mathbb{R}^N$ se dice conexo si no es no conexo. Equivalentemente, $\forall \ U,V$ abiertos en $\mathbb{R}^N$ tales que $U \cap A \ne \emptyset, \ V \cap A \ne \emptyset,\ A \subseteq U \cup V,$ se tiene que forzosamente $A \cap U \cap V \ne \emptyset$.
\end{ndef}



\begin{nprop}
Sea $A\subseteq \mathbb{R}^N$. Entonces, se verifica lo siguiente:

\begin{nlist}
\item $A$ es abierto y conexo por arcos $\Rightarrow$ $A$ es poligonalmente convexo.
\item $A$ es convexo $\Rightarrow$ $A$ es arco-conexo.
\item $A$ es arco-conexo $\Rightarrow$ $A$ es conexo.
\end{nlist}
\end{nprop}



\begin{nprop}
Sea $A\subseteq \mathbb{R}$ un conjunto arco-conexo. Entonces, $A$ es convexo.\\
\end{nprop}



\subsection{Continuidad en espacios topológicos. Topología inducida.}



\begin{ndef}[Continuidad en espacios topológicos]
Sean $(X,\tau_x),\ (Y,\tau_y)$ dos espacios topológicos, y sea $f:X\longrightarrow Y$. Entonces: $$f\ es\ continua \iff f^{-1}(B) \in \tau_x \ \ \forall B \in \tau_y.$$
\end{ndef}



\begin{ndef}[Topología inducida]
Sea $(X,\tau)$ un espacio topológico, y $A\subseteq X$. Entonces, $\tau_A = \{B\cap A: \ B \in \tau\}$ es la \textit{topología inducida en $A$}.
\end{ndef}



\begin{nprop}[Caracterización de abiertos en topología inducida]
Sea $(X,\tau)$ un espacio topológico, y $A\subseteq X$. Si $(A,\tau_A)$ es el espacio topológico inducido en $A$, entonces: $$B' \in \tau_A \iff \exists B\in \tau: \ B' = B\cap A.$$
\end{nprop}



\begin{nprop}
Sea $(X,\tau)$ un espacio topológico, y $A\subseteq X$. Entonces, $A$ es no conexo si, y solo si, existen $U,V$ \textbf{abiertos en $\bm{(A,\tau_A)}$} tales que: $$U \ne \emptyset \ne V;\quad A \subseteq U \cup V;\quad U \cap V = \emptyset.$$
\end{nprop}



\begin{ndef}[Continuidad en topología inducida]
Sean $(X,\tau_x),\ (Y,\tau_y)$ dos espacios topológicos, $A\subseteq X$, y $f:A\longrightarrow Y$. Entonces: $$f\ es\ continua \iff f\ es\ continua\ en\ (A,\tau_A).$$
\end{ndef}


\vspace{0.7em}
\subsection{Teoremas sobre funciones continuas en $\mathbb{R}^N$}



\begin{nth}[Weierstrass]
Sea $(X,d)$ un espacio métrico, $\emptyset \ne A \subseteq X$ compacto, y $f:A \longrightarrow \mathbb{R}$ continua en $A$. Entonces, $\exists x_1,x_2 \in A: \ f(x_1)\le f(x) \le f(x_2)\ \ \forall x\in A$. En otras palabras, la función $f$ alcanza su mínimo y su máximo.
\end{nth}



\begin{nth}[Weierstrass generalizado]
Sean $(X,d)$, $(Y,d)$ espacios métricos, $\emptyset \ne A \subseteq X$ compacto, y $f: A \longrightarrow Y$ continua. Entonces, $f(A)$ es compacto.
\end{nth}



\begin{nth}[Valor Intermedio]
Sea $\emptyset \ne A \subseteq \mathbb{R}^N$ arco conexo, y $f: A \longrightarrow \mathbb{R}^M$ continua. Entonces, $f(A)$ es arco-conexo en $\mathbb{R}^M$.
\end{nth}



\begin{nth}[Valor Intermedio revisitado]
Sea $\emptyset \ne A\subseteq \mathbb{R}^N$ conexo, y $f:A \longrightarrow \mathbb{R}^M$ continua. Entonces, $f(A)$ es conexo en $\mathbb{R}^M$.
\end{nth}















%%% Para la parte de límites:

%\begin{nprop}[Caracterización punto de acumulación]
%Sea $(X,d)$ un espacio métrico, y $A\subseteq X$. Consideremos un punto %$x\in X$. Son equivalentes:

%\begin{nlist}
%\item x es punto de acumulación de A.
%\item $\exists \{a_n\}\subseteq A-\{x\}$ tal que $\{a_n\} \rightarrow x$.
%\item $\forall \epsilon > 0\ B(x,\epsilon)\cap (A-\{x\})$ es un conjunto infinito. 
%\end{nlist}
%\end{nprop}



\end{document}