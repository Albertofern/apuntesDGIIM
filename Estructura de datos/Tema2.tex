% Created 2016-10-17 lun 11:01
\documentclass[11pt]{article}
\usepackage[utf8]{inputenc}
\usepackage[T1]{fontenc}
\usepackage{fixltx2e}
\usepackage{graphicx}
\usepackage{longtable}
\usepackage{float}
\usepackage{wrapfig}
\usepackage{rotating}
\usepackage[normalem]{ulem}
\usepackage{amsmath}
\usepackage{textcomp}
\usepackage{marvosym}
\usepackage{wasysym}
\usepackage{amssymb}
\usepackage{hyperref}
\tolerance=1000
\usepackage{minted}
\usepackage[spanish]{babel}
\usepackage[T1]{fontenc}
\usepackage{amsmath}
\usepackage[left=2.5cm,top=2cm,right=2.5cm,bottom=2.5cm]{geometry}
\usemintedstyle{manni}
\setminted{linenos=true}
\date{\today}
\title{Apuntes Estructura de datos.}
\hypersetup{
  pdfkeywords={},
  pdfsubject={},
  pdfcreator={Emacs 24.5.1 (Org mode 8.2.10)}}
\begin{document}

\maketitle

\section{Estructuras de datos}
\label{sec-1}
\subsection{Tema 2 - Abstracción.}
\label{sec-1-1}
\subsubsection{Uso de Template.}
\label{sec-1-1-1}

Nos permite seleccionar el tipo de dato que vamos a utilizar en tiempo de ejecución.
Declarando:

\begin{minted}[]{c++}
template <class T, int n>

class array_n {
private:
    T items[n];
  };
\end{minted}

Creando un objeto de esta clase de la forma:

\begin{minted}[]{c++}
array_n<int,1000> w
\end{minted}
Creando un metodo de la forma:

\begin{minted}[]{c++}
template <class T>
T VectorDinamico::componente(int i) const
{
  return datos[i];
}
\end{minted}

La compilación a la hora de usar templates es distinta a la que estamos acostumbrados. En lugar de hacer un \emph{\#include} ''\emph{clase.h}'' en el archivo \uline{.cpp}, se incluirá el archivo \emph{\#include} ''\emph{clase.cpp}'' al final del archivo \uline{.h}.

\subsubsection{Clase vector dinámico}
\label{sec-1-1-2}

Vamos a estudiar una clase que tenga un vector de datos y el número de elementos.Un primer ejemplo con tipo de dato float sería:

\begin{minted}[]{c++}
class VectorDinamico{

  float* datos;
  int ns;
}
\end{minted}


\subsubsection{Iteradores}
\label{sec-1-1-3}

Pretendemos hacer recorridos mucho más rápido. No volvermos a recorrer vectores haciendo v[i]. Usaremos los punteros para iterar, de la forma:


\begin{minted}[]{c++}
double *v = &a.
double *p;
double * fin;
fin = v+n;
for(p = v; p!= fin; ++p)
  cout << *p << endl;
\end{minted}

Nota: en un compilador moderno, simplemente activando la optimización de código se consigue el mismo aumento en el rendimiento.

Definiremos incluso un nuevo tipo de dato llamado iterator, haciéndolo de la forma:

\begin{minted}[]{c++}
typedef double* iterator;
iterator begin(double* v, int n){
  return v;
}
iterator end(double* v, int n){
  return v+n;
}

/**--------------------------*/
iterator p;
for(p=begin(v,n); p!=end(v,n);++p)
  cout << *p << endl;
\end{minted}

\subsubsection{Pilas}
\label{sec-1-1-4}
Son estructuras de datos lineales, secuencias de elementos dispuestos en una dimensión. Tienen estructura \emph{LIFO} (last in, first out)

No usaremos el concepto de posición, sólo trabajaremos con el tope de la estructura. En realidad, sí podemos recorrerla pero debemos salvaguardar los elementos
, pues para acceder al siguiente elemento tenemos que borrar el anterior.

\begin{itemize}
\item Para insertar un elemento, se inserta siempre al principio de la pila, añadiendo físicamente
\item Para borrar un elemento, se elimina y el puntero que había al primero se lleva al segundo
\item Como la pila no tiene recorridos, no tienen iteradores.
\item Las funciones básicas son Tope, Poner, Quitar, Vaciar.
\end{itemize}

Hay que recordar que los elementos de la pila se guardan en orden contrario al que fueron insertados. Por tanto si los imprimimos, por ser de tipo \emph{LIFO} salen del 
primero que hemos insertado al último.
En la práctica, usaremos las pilas con Celdas Enlazadas.


\begin{minted}[]{c++}
#ifndef __PILA_H__
#define __PILA_H__

typedef char Tbase;

struct CeldaPila{
    Tbase elemento;
    CeldaPila *sig;
};

class Pila{

    CeldaPila *primera;

  public:

    Pila();
    Pila(const Pila& p);
    ~Pila();
    Pila& operator= (const Pila& pila);

    void poner(Tbase c);
    void quitar();
    Tbase tope() const;
    bool vacia() const;
};
\end{minted}

\subsubsection{Colas}
\label{sec-1-1-5}

Una cola es una estructura de datos lineal en la que lso elementos se suprimen e insertan por extremos opuestos. Son estructuras \emph{FIFO}. 
En una cola, si tenemos dos elementos

Usaremos el operador \emph{\%}, que transforma una estructura lineal en una linear circular.
Al igual que en las colas, no tengo acceso a los elementos si no destruimos la cola.
% Emacs 24.5.1 (Org mode 8.2.10)
\end{document}
