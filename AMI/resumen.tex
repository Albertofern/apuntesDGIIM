
% Tipo de documento y opciones.
\documentclass[11pt, a4paper]{article}


%---------------------------------------------------------------------------
%   PAQUETES
%---------------------------------------------------------------------------

% Idioma y codificación para Español.
\usepackage[utf8]{inputenc}
\usepackage[spanish, es-tabla, es-lcroman, es-noquoting]{babel}
\selectlanguage{spanish} 
%\usepackage[T1]{fontenc}

% Fuente utilizada.
\usepackage{courier}    % Fuente Courier.
\usepackage{microtype}  % Mejora la letra final de cara al lector.

% Diseño de página.
\usepackage{fancyhdr}   % Utilizado para hacer títulos propios.
\usepackage{lastpage}   % Referencia a la última página.
\usepackage[parfill]{parskip}    % Crea una nueva línea entre párrafos.
\usepackage{geometry}            % Geometría de las páginas.

% Símbolos y matemáticas.
\usepackage{amssymb, amsmath, amsthm, amsfonts, amscd}
\usepackage{upgreek}

% Otros.
\usepackage{enumitem}   % Listas mejoradas.
\usepackage[hidelinks]{hyperref}


%---------------------------------------------------------------------------
%   OPCIONES PERSONALIZADAS
%---------------------------------------------------------------------------

% Redefinir letra griega épsilon.
\let\epsilon\upvarepsilon

% Formato de texto.
\linespread{1.1}            % Espaciado entre líneas.
\setlength\parindent{0pt}   % No indentar el texto por defecto.
\setlist{leftmargin=.5in}   % Indentación para las listas.

% Estilo de página.

\geometry{left=3cm,right=3cm,top=3cm,bottom=3cm}   % Márgenes y cabecera.

% Redefinir entorno de demostración (reducir espacio superior)
\makeatletter
\renewenvironment{proof}[1][\proofname] {\vspace{-15pt}\par\pushQED{\qed}\normalfont\topsep6\p@\@plus6\p@\relax\trivlist\item[\hskip\labelsep\it#1\@addpunct{.}]\ignorespaces}{\popQED\endtrivlist\@endpefalse}
\makeatother

% Aumentar el tamaño del interlineado
\linespread{1.3}
%---------------------------------------------------------------------------
%   COMANDOS PERSONALIZADOS
%---------------------------------------------------------------------------

% Valor absoluto: \abs{}
\providecommand{\abs}[1]{\lvert#1\rvert}    

% Fracción grande: \ddfrac{}{}
\newcommand\ddfrac[2]{\frac{\displaystyle #1}{\displaystyle #2}}

% Texto en negrita en modo matemática: \bm{}
\newcommand{\bm}[1]{\boldsymbol{#1}}

\newcommand{\overbar}[1]{\mkern 1.5mu\overline{\mkern-1.5mu#1\mkern-1.5mu}\mkern 1.5mu}

% Línea horizontal.
\newcommand{\horrule}[1]{\rule{\linewidth}{#1}}

\newcommand{\R}{\mathbb{R}}

%---------------------------------------------------------------------------
%   ENTORNOS PARA MATEMÁTICAS
%---------------------------------------------------------------------------

% Nuevo estilo para definiciones.
\newtheoremstyle{definition-style} % Nombre del estilo.
{10pt}               % Espacio por encima.
{10pt}               % Espacio por debajo.
{}                   % Fuente del cuerpo.
{}                   % Identación.
{\bf}                % Fuente para la cabecera.
{.}                  % Puntuación tras la cabecera.
{.5em}               % Espacio tras la cabecera.
{\thmname{#1}\thmnumber{ #2}\thmnote{ (#3)}}     % Especificación de la cabecera (actual: nombre en negrita).

% Nuevo estilo para notas.
\newtheoremstyle{remark-style} 
{10pt}                
{10pt}                
{}                   
{}                   
{\itshape}          
{.}                  
{.5em}               
{}                  

% Nuevo estilo para teoremas y proposiciones.
\newtheoremstyle{theorem-style}
{10pt}                
{10pt}                
{\itshape}           
{}                  
{\bf}             
{.}                
{.5em}               
{\thmname{#1}\thmnumber{ #2}\thmnote{ (#3)}}                   

% Nuevo estilo para ejemplos.
\newtheoremstyle{example-style}
{10pt}                
{10pt}                
{}                  
{}                   
{\scshape}              
{:}                 
{.5em}               
{}                   

% Teoremas, proposiciones y corolarios.
\theoremstyle{theorem-style}
\newtheorem*{nth}{Teorema}
\newtheorem*{nprop}{Proposición}
\newtheorem{ncor}{Corolario}

% Definiciones.
\theoremstyle{definition-style}
\newtheorem*{ndef}{Definición}

% Notas.
\theoremstyle{remark-style}
\newtheorem*{nota}{Nota}

% Ejemplos.
\theoremstyle{example-style}
\newtheorem*{ejemplo}{Ejemplo}

% Listas ordenadas con números romanos (i), (ii), etc.
\newenvironment{nlist}
{\begin{enumerate}
\renewcommand\labelenumi{(\emph{\roman{enumi})}}}
{\end{enumerate}}

% División por casos con llave a la derecha.
\newenvironment{rcases}
  {\left.\begin{aligned}}
  {\end{aligned}\right\rbrace}

\title{\textbf{Resúmenes Análisis}}
\date{}
%---------------------------------------------------------------------------
%   COMIENZO DEL DOCUMENTO
%---------------------------------------------------------------------------
\begin{document}
\maketitle

\section{Conjuntos.}
\begin{ndef}[Punto de acumulación]

Un punto $x$ en un espacio métrico $M$ es un punto de acumulación de un conjunto $A\subset M$ si todo conjunto abierto $U$ que contiene a $x$ también contiene a algún punto de A distinto de $x$.
	
\end{ndef}

\begin{nth}
	Un conjunto $A\subset M$ es cerrado $\iff$ todos los puntos de acumulación de A pertenecen a A.
\end{nth}
\section{Sucesiones.}

\begin{nprop}
	Una sucesión $\{x_n\}$ en M converge a $x \in M \iff \ \ \forall \epsilon > 0 \ \ \exists N : \ \ k \geq N \implies d(x,x_k) < \epsilon$
	
	
\end{nprop}

\begin{nprop}
	$\{v_n\} \to v \in \mathbb{R}^n \iff$ cada sucesión de coordenadas converge a la coordenada correspondiente de $v$ como una sucesión en $\mathbb{R}$
\end{nprop}

\begin{nprop}\hfill\\
	\begin{itemize}
	\item Un conjunto $A\subset M$ es cerrado $\iff \forall \{x_n\} \subset A $ con $\{x_n\}$ convergente, el límite es un elemento de A.
	
	\item Para un conjunto $B \subset M, \ \ x \in \overbar{B} \iff \ $ existe una sucesión $\{x_n\}\in B$ tal que $\{x_n\} \to x$

\end{itemize}
\end{nprop}

\begin{proof}
	Demostraremos el primero.\\
	Sea $A$ un conjunto cerrado y $\{x_n\} \to x$. Entonces $x$ es un punto de acumulación de $A$. $x$ es un punto de acumulación de A, pues cualquier entorno de $x$ contiene algún punto de $\{x_n\}\subset A$. Ahora, como sabemos que $A$ es cerrado si contiene a todos sus puntos de acumulación, A es cerrado y $x$ es un punto de acumulación entonces $x\in A$.
	
De la misma forma, sea $x\in A$ un punto de acumulación de A y elegimos $\{x_n\}\in B(x, \frac{1}{n}) \cap A$. De esta forma, $\{x_n\} \to x \ $ (pues $ \forall \epsilon > 0 \ \exists N \geq 1 /\epsilon
$ con lo que $k\geq N \implies x_n \in B(x,\epsilon)$). Así, tenememos una sucesión de elementos de $A$ que converge a un elemento de $A$, y su límite es un punto de acumulación, por tanto por el teorema anterior, $A$ es cerrado.
\end{proof}

\begin{ndef}[Sucesión de Cauchy]
	Una sucesión de Cauchy es una sucesión $\{x_n\} \in M$ tal que $\forall \epsilon > 0 \ \ \exists N$ tal que si $p,q \geq N$ entonces $d(x_p,x_q) < \epsilon $. M es completo $\iff$ toda sucesión de Cauchy en M converge a un punto de M.
\end{ndef}

\begin{nprop}
Una sucesión convergente en un espacio normado o métrico está acotada.
	
\end{nprop}
\begin{proof}
	Sea $\{x_n\} \to x$, sabemos que $\exists N : d(x_n,x) < 1$ si $n \geq N$, así que $x_n \in B(x,1)$ si $n \geq N$. Basta tomar $R = max\{1,d(x_1,x),...,d(x_{N-1},x)\}$ y así $d(x,x_n) \leq R \ \forall n$ por lo que $x_n \in B(x,R) \ \forall n$ y así está acotada.
\end{proof}

\begin{nth}[Teorema de Bolzano Weierstrass]

Sea $\{x_n\}$ una sucesión de $\mathbb{R}^N$ acotada. Entonces existe una sucesión parcial suya $\{x_{\sigma(n)}\}$ convergente.
	
\end{nth}

\begin{proof}
	Notaremos $x_n = (x_n^1, \dots, x_n^N)$. Como $\{x_n^1\}$ es acotada en $\mathbb{R}$, existe $\sigma_1 : \mathbb{N} \rightarrow \mathbb{N}$ estrictamente creciente tal que $\{x_{\sigma_1(n)}^1\}$ es convergente.

Ahora, como $\{x_n^2\}$ es acotada, $\{x_{\sigma_1(n)}^2\}$ también es acotada, y existe $\sigma_2 : \mathbb{N} \rightarrow \mathbb{N}$ estrictamente creciente tal que $\{x_{(\sigma_2\circ\sigma_1)(n)}^2\}$ es convergente.

Procediendo de esta forma con cada componente de $x_n$, obtenemos $\sigma_1, \dots, \sigma_N$, y\\ $\{x_{\sigma_1(n)}^1\}, \{x_{(\sigma_2\circ\sigma_1)(n)}^2\}, \dots, \{x_{(\sigma_N\circ\dots\circ\sigma_2\circ\sigma_1)(n)}^N\}$ sucesiones convergentes en $\mathbb{R}$. Al ser $\sigma_i$ estrictamente creciente $\forall i=1,\dots,N$, $\{x_{(\sigma_N(n)\circ\dots\circ\sigma_{i+1}\sigma_i\circ\dots\circ\sigma_1)(n)}^i\}$ también es convergente (toda sucesión parcial de una sucesión convergente es convergente).

Así, tomando $\sigma = \sigma_1\circ\dots\circ\sigma_N$, $\{x_{\sigma(n)}\}$ es convergente.
\end{proof}

\begin{nprop}\hfill\\
	\begin{nlist}
	\item Toda sucesión convergente en un espacio métrico es una sucesión de Cauchy.
	\item Una sucesión de Cauchy en un espacio métrico debe estar acotada.
	\item Si una subsucesión de una sucesión de Cauchy converge a $x$, entonces la sucesión converge a $x$.
\end{nlist}
\end{nprop}

\begin{nth}[$\mathbb{R}^n$ es completo]
	Una sucesión $x_n\in \mathbb{R}^n$ converge a un punto de $\mathbb{R}^N \iff $ es una sucesión de Cauchy
\end{nth}
\begin{proof}\hfill\\
\boxed{\Rightarrow}\\
Dado $\varepsilon > 0$, existe $m\in \mathbb{N}$ tal que si $n \ge m$ entonces $d(x_n, x) < \displaystyle\frac{\varepsilon}{2}$, y si $p,q \ge m$ entonces $d(x_p, x_q) \le d(x_p, x) + d(x, x_q) < \varepsilon $


	\boxed{\Leftarrow}\\
Como $\{x_n\}$ es de Cauchy, $\{x_n^i\}$ es de Cauchy $\forall i = 1,\dots,N$ (porque $|x_n^i-x_m^i| \le |x_n-x_m|$). $\implies$ $\{x_n^i\}\rightarrow x^i$ es convergente, por ser $\mathbb{R}$ completo. Luego $\{x_n\}$ es convergente.


\end{proof}



\section{Conjuntos compactos y conexos.}

\begin{ndef}[Compacto]
	Un subconjunto $A$ de un espacio métrico $M$ es compacto si todo recubrimiento abierto de $A$ contiene un subrecubrimiento finito.
\end{ndef}

\begin{ndef}[Otra definición de compacto.]
Sea $A \subset X$ con $X$ espacio métrico.
\begin{center}
	A es compacto $\iff \forall \{x_n\} \subset A \ \ \exists \{x_{\sigma(n)}\}$ parcial de $\{x_n\}$ con $\{x_{\sigma(n)}\}\to x \in A$
\end{center}	
\end{ndef}

\begin{nth}[Teorema de Heine-Borel]
	
	Un conjunto $A \subset \mathbb{R}^n$ es compacto $\iff$ es cerrado y acotado.
\end{nth}
\begin{proof}\hfill\\
	\boxed{\Rightarrow}\\
	Suponemos que $A \subset \mathbb{R}^n$ es compacto. Entonces, por su definición, $\forall \{x_n\} \subset A \ \exists \{x_{\sigma(n)}\}$ parcial de $\{x_n\}$ con $\{x_{\sigma(n)}\} \to x \in A$. Supongamos que $A$ no está acotado. Entonces, $\forall n \in \mathbb{N}, \ \exists a_n \in A : \ |a_n| \geq n$, por lo que $\{a_n\}$ no converge y por tanto $\sigma(n) \geq n \implies \{a_{\sigma(n)}\}$ no converge, por lo que $A$ está acotado.
	
	Supongamos ahora que $\{x_n\} \to x \implies \exists \{x_{\sigma(n)}\} \to x \in A$, y como sabemos que si una sucesión es convergente todas sus parciales convergen al mismo límite, entonces eso implica que $\{x_n\}\to x \in A$ por lo que toda sucesión converge a un punto de $A$, y así $A$ es cerrado.
	\\
	\boxed{\Leftarrow}\\
	Supongamos ahora que $A$ es cerrado y acotado. Sea $\{x_n\}$ una sucesión cualquiera de puntos de $A$.
	
	Como $A$ es acotado, entonces $\exists R > 0 : \ \ A \subset B(0,R)$. Además, como $\{x_n\} \subset A \ \forall n \implies |x_n| < R \ \forall n \in \mathbb{N}$, así $\{x_n\}$ es acotada.
	
	Como $\{x_n\}$ es acotada, por el teorema de Bolzano Weierstrass, $\exists \sigma: \mathbb{N} \to \mathbb{N}$ estrictamente creciente con $\{x_{\sigma(n)}\}\to x \in \mathbb{R}^n$, y como $\{x_{\sigma(n)}\}$ es una subsucesión de puntos de $A$ que converge a $x$ y el conjunto $A$ es cerrado, entonces el límite de esta sucesión está en $A$ , es decir:
	\[
	\{x_{\sigma(n)}\}\to x \in A
	\]
Por lo que tenemos la definición de conjunto compacto.

\end{proof}
\begin{ndef}[Función continua]
	Una aplicación $f:A \to M$ es continua si $\{x_n\} \to x \implies \{f(x_n)\} \to f(x)$ para toda sucesión $x_n$ convergente a un punto de A con $x_n\in A$
\end{ndef}

\begin{nprop}[Caracterización de continuidad]
Sea $\emptyset \ne A \subseteq \mathbb{R}^N$, y $f:A\longrightarrow \mathbb{R}^M$. Entonces: $$f\ es\ continua\ en\ a \iff \forall \{x_n\}\subseteq A\ con\ \{x_n\} \rightarrow a \Rightarrow \{f(x_n)\} \rightarrow f(a).$$
\end{nprop}

\begin{ndef}[Conjunto convexo]
Un conjunto $A\subseteq \mathbb{R}^N$ se dice \textit{convexo} si $\forall x,y \in A$ se tiene que el segmento de extremos $x$ e $y$ está incluido en $A$. En otras palabras: $$A\ convexo\ \iff [x,y] = \{tx + (1-t)y: \ t\in [0,1]\} \subseteq A.$$
\end{ndef}



\begin{ndef}[Poligonalmente conexo]
Un conjunto $A\subseteq \mathbb{R}^N$ se dice \textit{poligonalmente conexo} si  $\forall x,y \in A$ existe una poligonal que los une y no se sale de $A$. En otras palabras:  $A\ poligonalmente\ conexo \iff \exists \{x= a_0, a_1,\dots,a_k=y \}\subseteq A$ tal que: $$\bigcup_{i=1}^k [a_{i-1},a_i] \subseteq A.$$
\end{ndef}



\begin{ndef}[Conjunto arco-conexo]
Un conjunto $A \subseteq \mathbb{R}^N$ se dice \emph{arco-conexo(conexo por arcos)} si $\forall x,y \in A$ existe un camino incluido en $A$ que los une. En otras palabras,  $A\ es\ conexo\ por\ arcos \iff \exists \varphi:[a,b] \longrightarrow \mathbb{R}^N$ continua verificando: $$\varphi(a) = x;\quad \varphi(b) = y;\quad \varphi([a,b]) \subseteq A.$$
\end{ndef}



\begin{ndef}[Conjunto no conexo]
Decimos que un conjunto $A\in \mathbb{R}^N$ es \textit{NO conexo} si existen $U,\ V$ abiertos en $\mathbb{R}^N$ tales que: $$U \cap A \ne \emptyset;\quad V \cap A \ne \emptyset;\quad A \subseteq U \cup V;\quad A \cap U \cap V = \emptyset.$$
\end{ndef}



\begin{nota}
La misma definición se aplica para un espacio topológico $(X,\tau).$
\end{nota}



\begin{ndef}[Conjunto conexo]
Un conjunto $A\subseteq \mathbb{R}^N$ se dice conexo si no es no conexo. Equivalentemente, $\forall \ U,V$ abiertos en $\mathbb{R}^N$ tales que $U \cap A \ne \emptyset, \ V \cap A \ne \emptyset,\ A \subseteq U \cup V,$ se tiene que forzosamente $A \cap U \cap V \ne \emptyset$.
\end{ndef}

\begin{nth}
	Los conjuntos conexos por arcos en $\mathbb{R}$ son convexos.
\end{nth}
\begin{proof}
Sean $x,y$ dos puntos de $A$. Suponemos $x \leq y$ (si fuera al revés, cambiamos los nombres).

Como A es arco conexo$\implies \exists \varphi:[a,b]\to \R$ continua con $\varphi(a) = x$	, $\varphi(b) = y$ y $\varphi([a,b])$ es un intervalo por el teorema del valor intermedio en R.

Ahora, $\forall \ \alpha, \beta \in \varphi([a,b])$ con $\alpha \leq \beta \implies [\alpha,\beta ]  \subseteq \varphi([a,b]) $ y así tenemos $\varphi(a), \varphi(b) \in \varphi([a.b]) \implies [\varphi(a),\varphi(b)] = [x,y] \subseteq \varphi([a,b])\subseteq A$
\end{proof}



\section{Funciones continuas}

\begin{ndef}[Límite]
	Supongamos que $x_0$ es un punto de acumulación de A. Decimos que $b\in N$ es el límite de $f$ en $x_0$, denotado por:
	\[
	\lim_{x\to x_0}f(x) = b
	\]
	Si $\forall\epsilon > 0  \ \exists \delta > 0 $ tal que $\forall x \in A$ que sea distinto de $x_0$ y $d(x_0,x)$, entonces $d'(f(x),b) < \epsilon$ 
\end{ndef}

\begin{ndef}[Función continua]
	Una función $f:A \to B$ es continua en un punto $x_0$ de su dominio $\iff \forall \epsilon > 0 \ \ \exists \delta > 0 : \ \forall x \in A$ que cumpla que $d(x,x_0)< \delta \implies d'(f(x),f(x_0)) < \epsilon$
\end{ndef}

\begin{nth}
	Sea $f:A \to B$ continua y $K \subset A$ conexo. Entonces, $f(K)$ es conexo. Análogamente, si K es arco-conexo, entonces, f(K) es arco-conexo.
\end{nth}

\begin{nth}[Teorema de Weierstrass]
Sea $f:A \to B$ continua y $K\subset A$ un compacto. Entonces, $f(K)$ es compacto.	
\end{nth}
\begin{proof}
	Sea $\{y_n\}\subseteq f(K)$ una sucesión cualquiera en $f(K)$. Para demostrar que $f(K)$ es compacto debemos demostrar que $\{y_n\}$ tiene una subsucesión convergente a algún punto de $f(K)$. Sea $y_n$ = $f(x_n)$ con $x_n \in K$, por ser A compacto $\{x_n\}$ tiene una subsucesión $\{x_{\sigma(n)}\}$ convergente a un $a \in K$. Por lo tanto, $\{f(x_{\sigma(n)})\}$ convergente a un $f(a) \in f(K)$, siendo $\{f(x_{\sigma(n)})\}$ una subsucesión de $\{f(x_{n})\}$, es decir, de $\{y_n\}$.

\end{proof}

\begin{nth}
	Sean M,N,P espacios métricos y supongamos que $f:A\subset M \to N$ y $g: B \subset N \to P$ son transformaciones continuas tales que $f(A) \subset B$. Entonces, $g \circ f : A \subset M \to P $ es continua.
\end{nth}

\begin{nth}[Teorema del máximo-mínimo]
	Sea (M,d) un espacio métrico, $A\subset M$ y $f:A \to \R$ una función continua. Sea $K \subset A$ un conjunto compacto. Entonces f está acotada en K, es decir: $B=\{f(x) : x \in K\}\subset \R$ es un conjunto acotado. Además, existen puntos $x_0,x_1 \in K $ tales que $f(x_0) = inf(B)$ y $f(x_1) = sup(B)$. Decimos que sup(B) es el máximo de f en K e inf(B) el mínimo de f en K.
	
\end{nth}
\begin{nth}
	[Teorema de los valores intermedios]
Sean M un espacio métrico, $A\subset M$ y $f:A \to \R$ continua. Supongamos que $K\in A$ es conexo y que x,y $\in K$. Para cada número $c \in \R$ tal que $f(x) < c f(y)$ existe un punto $z \in K : f(z) = c$
\end{nth}


\subsection{Continuidad uniforme}

\begin{ndef}[Uniformemente continua]
	Sean $(M,d)$ y $(N,p)$ espacios métricos, $A \subset M$ , $f:A \to N$ y $B \subset A$. Decimos que $f$ es uniformemente continua en el conjunto $B$ si $\forall \epsilon > 0 \ \exists \delta > 0 :  \ x,y \in B \ \ y \ \ d(x,y) < \delta \implies p(f(x),f(y)) < \epsilon$
\end{ndef}

\begin{nth}[Teorema de la continuidad uniforme. Teorema de Heine-Cantor.]
	Sean $f:A \to N$ continua y $K\subset A $ un compacto. Entonces, f es uniformemente continua en K.
\end{nth}
\begin{proof}
		La condición para la continuidad uniforme es la siguiente:
	\[
		\forall \varepsilon > 0 \  \exists \delta > 0 \ \forall x, y \in A : d(x,y) < \delta \implies d (f(x) , f(y) ) < \varepsilon
	\]
	
	Vamos a proceder por reducción al absurdo, para lo cual negamos esta condición:
	\[
		\exists \varepsilon_0 > 0 \ \forall \delta > 0 \ \exists x, y \in A : d (x,y) < \delta \wedge d (f(x) , f(y) ) \ge \varepsilon_0
	\]
	
	Tomamos este $\epsilon_0$, lo que nos da, para cada $\delta>0$, un par de puntos $x$ e $y$ que cumplen la propiedad expresada arriba. Tomamos $\delta = \frac{1}{n} \ \forall n\in \mathbb{N}$. Esto nos da dos sucesiones $\{x_n\}$ e $\{y_n\}$ tales que
	\[
		d(x_n,y_n) < \frac{1}{n} \wedge d(f(x_n),f(y_n)) \ge \epsilon_0
	\]
	
	Por ser A compacto, el teorema de Bolzano-Weierstrass nos da dos sucesiones parciales $\{x_{n_k}\}$ a $x_0$ e $\{y_{n_k}\}$ a $y_0$. Por tanto:
	\[
		d(x_{n_k},y_{n_k}) < \frac{1}{n_k} \wedge d(f(x_{n_k}),f(y_{n_k})) \ge \epsilon_0
	\]
	
	Sin embargo, $\{x_{n_k}\}$ e $\{y_{n_k}\}$ convergen al mismo punto (por converger su distancia a cero), y como $f$ es continua, esta proposición no puede ser verdadera. Hemos llegado por tanto a una contradicción, luego $f$ debe ser uniformemente continua.
\end{proof}



\section{Transformaciones diferenciables}

\begin{ndef}
	[Diferenciable]
	Una transformación $f:A\subset \R^n \to \R^m$ es diferenciable en $x_0 \in A$ si existe una transformación lineal, denotada $Df(x_0): \R^n \to \R^m$ llamada diferencial de $f$ en $x_0$ tal que:
	 \[
	 \lim_{x\to x_0}\frac{||f(x)-f(x_0)-Df(x_0)(x-x_0)||}{||x-x_0||} = 0
	 \]
	 Donde $Df(x_0)(x-x_0)$ es el valor de la aplicación lineal aplicada al vector $(x-x_0)$.\\
	 
	Equivalentemente, podemos decir que $\forall \epsilon > 0 \ \exists \delta > 0$ tal que si $x\in A $ y $||x-x_0||< \delta$ entonces:
	\[
	||f(x)-f(x_0) - Df(x_0)(x-x_0) ||  \leq \epsilon ||x-x_0||
	\]
\end{ndef}

\begin{nth}[Matriz jacobiana de $f$]
Sea $A\subset \R^n$ un conjunto abierto y $f:A \to \R^m$ es diferenciable en $A$. Entonces, las derivadas parciales $\frac{\partial f_j}{\partial x_i}$ existen y la matriz de la transformación lineal $Df(x)$ con respecto de las bases canónicas en $\R^n \ y \ \R^m$ es:
\[
\begin{pmatrix}
 \frac{\partial f_1}{\partial x_1} & \cdots & \frac{\partial f_1}{\partial x_n} \\
 \vdots & \ddots&\vdots \\
 \frac{\partial f_m}{\partial x_i} & \cdots & \frac{\partial f_m}{\partial x_n}
\end{pmatrix} 
\]
donde cada derivada parcial se evalua en $x=(x_1,...,x_n)$. Esta matriz es la matriz jacobiana de f.
	
\end{nth}
\begin{ndef}[Gradiente (caso $f:A \to \mathbb{R}$)]
	En el caso de que $f:A \to \R$, entonces $Df(x)$ es una matriz $1\times n$. El vector cuyas componentes son iguales a las de $Df(x)$ se denomina gradiente de $f$ y se denota $\triangledown f$.
	\[
	\triangledown f = (\frac{\partial f}{\partial x_1},..., \frac{\partial f}{\partial x_n})
	\]
\end{ndef}

\begin{nth}
	Sean $A \subset \R^n$ un conjunto abierto y $f:A \subset \R^n \to \R^m$. Si $f=(f_1,...,f_m)$ y cada $\frac{\partial f_j}{\partial x_i}$ existe y es continua en $A$, entonces $f$ es diferenciable en $A$
\end{nth}

\begin{ndef}[Derivada direccional]
	Las derivadas parciales de una función miden su variación en las direcciones paralelas a los ejes. Las derivadas direccionales hacen lo mismo en otras direcciones.
	
	Sea $f$ una función escalar definida en un entorno de $x_0\in \R^n$ y sea $e\in \R^n$ un vector unitario, entonces:
	\[
	\frac{d}{dt}f(x_0+te)|_{t= 0} =  \lim_{t\to 0} \frac{f(x_0+te) -f(x_0)}{t}
	\]
	es la derivada direccional de $f$ en $x_0$ en la dirección e.
	
	Se puede afirmar que la derivada direccional en la dirección de $e$ es igual a $Df(x_0)(e)$. Se suele notar $D_ef(x)$.
\end{ndef}
\begin{nprop}[Regla de la cadena]
	Sea $A\subset \R^n$ abierto y $f:A \to \R^m$ diferenciable en $x_0 \in A$. Sean $B\subset \R^n$ abierto y $f(A) \subset B$ y $g:B \to \R^p$ diferenciable en $f(x_0)$. Entonces, la composición $g\circ f$ es diferenciable en $x_0$ y 
	\[
	D(g \circ f)(x_0) = Dg(f(x_0)) \circ Df(x_0)
	\]
\end{nprop} 


\begin{nth}[Teorema del valor medio]\hfill\\
	\begin{nlist}
	\item Sea $f:A \subset \R^n\to \R$ diferenciable en un abierto A. $\forall a,b \in A$ tales que el segmento de recta que une a con b esté en A, existe un punto $c$ en ese segmento tal que:
	\[
	f(b)-f(a) = Df(c)(b-a)
	\]
	\item Sea $f:A \subset \R^n \to \R^m$ diferenciable en el conjunto abierto A. Supongamos que el segmento de recta que une x e y está contenido en A y $f=(f_1,...,f_m)$. Entonces, existen puntos $c_1,...,c_m$ en ese segmento tales que:
	\[
	f_i(y) - f_i(x) = Df_i(c_i)(y-x) \ \ \ i = 1,...,m
	\]
\end{nlist}
\end{nth}
\begin{proof}
	Lo vamos a hacer para una función $f:A \subset \R^n \to \R$ y lo aplicamos luego a cada coordenada de $f$.
	
	Consideremos la función $h:[0,1] \to \R$ definida por $h(t) = f((1-t)a +tb) \ \ \forall t \in [0,1]$ es continua en [0,1] y por la regla de la cadena es derivable en ]0,1[ con:
	\[
	h'(t) = Df((1-t)a +tb)(b-a)
	\].
	El TVM para funciones de $\R$ en $\R$ nos da un $t_0 \in ]0,1[$ tal que:
	\[
	f(b)-f(a) = h(1)-h(0) = h'(t_0) = Df((1-t_0)a+t_0b)(b-a)
	\]
	Si tomamos $c=(1-t_0)a+t_0b \in [a,b]$ obtenemos el $c$ que buscábamos.
	
\end{proof}

\begin{nth}[Matriz Hessiana]
	Sea $f:A\subset \R^n \to \R$ dos veces diferenciable en el conjunto abierto A. Entonces, la matriz $D^2f(x): \R^n\times\R^n \to \R$ en la base canónica está dada por:
	\[
	\begin{pmatrix}
 \frac{\partial^2 f}{\partial x_1 \partial x_1} & \cdots & \frac{\partial^2 f}{\partial x_1\partial x_n} \\
 \vdots & \ddots&\vdots \\
 \frac{\partial^2 f}{\partial x_n \partial x_1} & \cdots & \frac{\partial^2 f}{\partial x_n \partial x_n}
\end{pmatrix} 
	\]
	donde cada derivada parcial está evaluada en el punto $x=(x_1,...,x_n)$
\end{nth}

\begin{nprop}
	Si $f:A \to \R^m$ es dos veces diferenciable en el conjunto abierto A con $D^2 f(x)$ continua $\forall x\in A$, entonces $D^2f(x)$ es simétrica $\forall x \in A$, es decir:
	\[
	\frac{\partial^2 f_k}{\partial x_i \partial x_j} =  \frac{\partial^2 f_k}{\partial x_j \partial x_i}
	\]
\end{nprop}

\begin{nth}[Teorema de Taylor. Caso n=1]
	Sea $A \subseteq \R^n$ abierto, $a \in A$ y $f:A \to \R$ con $f\in \mathcal{C}^2(A)$. Entonces, $\forall x \in A \ \exists c$ comprendido entre x y a tal que:
	\[
	f(x) = f(a) + \frac{Df(a)(x-a)}{1!} + \frac{(x-a)^t Hf(c)(x-a)}{2!}
	\]
	Donde (x-a) es un vector columna.
\end{nth}


\section{Máximos y mínimos}

\begin{ndef}
	Sea $f:A\subset \R^n \to \R$ donde A es abierto. Si existe un entorno de $x_0 \in A$ en el que $f(x_0)$ es máximo, es decir si $f(x_0) \geq f(x) \ \forall x$ en el entorno, entonces $x_0$ es un punto de máximo local y $f(x_0)$ es un máximo local de $f$. Análogamente se define un mínimo local de $f$. Un punto es extremo si es un máximo o un mínimo local de $f$. Un punto $x_0$ es un punto crítico si $f$ es diferenciable en ese punto y $Df(x_0) = 0$.
\end{ndef}

\begin{nth}\hfill
	\begin{nlist}
	\item Si $f:A \subset \R^n\to \R$ es una función $\mathcal{C}^2$ definida en un abierto A y $x_0$ es un punto crítico de f tal que $H_{x_0}(f)$ es definida negativa, entonces f tiene un máximo local en $x_0$.
	\item Si f tiene un máximo local en $x_0$, entonces $H_{x_0}(f) $ es semidefinida negativa.
\end{nlist}
Para el caso del mínimo, reemplazamos negativa por positiva. La matriz hessiana, es la vista anteriormente.
\end{nth}

\section{Clasificación de matrices según su signo}
 Si $A_k$ es el determinante del menor de orden K de una matriz A, entonces:
 \begin{itemize}
	\item A es definida positiva $\iff$ $A_k > 0 \ \forall k$
	\item A es definida negativa $\iff$ $A_k > 0$ si K es par y $A_k < 0 $ si K es impar.
\end{itemize}

\section{Teorema de la función Inversa}
\begin{nth}[Teorema de la función inversa. Caso general.]
	Sea $A \subset \R^n$ abierto, $a\in A$ y $f:A \to \R^n$ de clase 1. Si  
$Df(a): \R^n \to \R^n$ es invertible \\
$\implies \exists U\subset A$ abierto y $\exists V$ abierto en $\R^n$ tales que $a\in U $, $f(a) \in V$ y $f|_U : U \to V$ es biyectiva, y por tanto existe la inversa de la función $g = (f|_U)^{-1}: V \to U$ es de clase 1 y además:
	\[
	Dg(f(a)) = (Df(a))^{-1}
	\]
Además, decir que $Df(a)$ es invertible es lo mismo que decir que $det(Jf(a)) \ne 0$
Y decir que $Dg(f(a)) = (Df(a))^{-1}$ es lo mismo que decir $Jg(f(a)) = (Jf(a))^{-1}$
\end{nth}

\section{Teorema de la función implícita}
\begin{nth}[Teorema de la función implícita]
	Sea $A \subset \R^n \times\R^m$ abierto y no vacío. Sea $F: A \to \R^n$ una función de clase 1. Fijando un $(x_0,y_0) \in A$ tal que $F(x_0,y_0) = 0$. Si:
	\[
	det\begin{pmatrix}
 \frac{\partial F_1}{\partial x_1}(x_0,y_0) & \cdots & \frac{\partial F_1}{\partial x_n}(x_0,y_0) \\
 \cdots& & \cdots\\
 
 \frac{\partial F_n}{\partial x_1}(x_0,y_0) & \cdots & \frac{\partial F_n}{\partial x_n} (x_0,y_0)
\end{pmatrix} \ne 0
	\]
	
	Entonces, existe $U$ un entorno abierto de $x_0$ en $\R^n$ y también existe $V$ un entorno abierto de $y_0$ en $\R^m$ tal que $U\times V \subset A$ y $\forall y \in V \ \ \exists! x \in U : F(x,y) = 0$
\end{nth}


\section{Teorema de Lagrange.}

\begin{nth} Sean $\emptyset \ne A \subset \mathbb{R}^N$ un abierto, $f\in \mathcal{C}^1(A, \R)$, $g_1, \dots, g_k \in \mathcal{C}^1(A,\R)$ ($k$ restricciones). Llamo $S = \{ x\in A :\ g_1(x) = \dots = g_k(x) = 0 \}$ y supongo $1\le k < N$.
	
Sea $a \in S$ tal que $f$ presenta un mínimo (respectivamente máximo) relativo condicionado a $S$. Entonces $\exists \lambda_0, \dots, \lambda_k\in \R$ que verifican:

\begin{nlist}
	\item $(\lambda_0, \dots, \lambda_k) \ne (0, \dots, 0)$, $\lambda_0 \ge 0$
	\item $\lambda_0 \displaystyle\frac{\partial f}{\partial x_j}(a) + \sum_{i=1}^k \lambda_i \displaystyle\frac{\partial g_i}{\partial x_j} = 0 \quad \forall j=1,\dots, N$
\end{nlist}

Si, además $r(Jg(a)) = k, \text{ donde } g = (g_1, \dots, g_k) : A \to \R^k$ entonces puedo escoger $\lambda_0 = 1$.
\end{nth}
\begin{nota}Demostraremos el teorema para el caso en el que $f$ presente un mínimo relativo. Para la prueba con máximo en lugar de mínimo, basta aplicar el resultado a $-f$.
	\end{nota}

\begin{proof}[Demostración (método de penalización)]\hfill\\
	
	$f$ presenta un mínimo relativo en $a$ condicionado a $S$, luego
	\[
	\exists \varepsilon_0 >0 \text{ tal que } B(a, \varepsilon_0) \subseteq A \text{ y } f(a) \le f(x) \;\;\forall x\in B(a, \varepsilon_0)\cap S
	\]
	\boxed{Paso\hspace{1mm} 1}\hfill\\
	Afirmamos que $\forall \varepsilon\in (0,\varepsilon_0)\ \exists M>0$ tal que
	\[
		f(x) + \abs{x-a}^2 + M\sum_{i=1}^k g_i(x)^2 > f(a)\ \forall x \text{ tal que } \abs{x-a} = \epsilon
	\]
	
	Para probarlo, supongamos lo contrario. Entonces
	\[
		\exists\epsilon>0\ \ \forall M>0\ \ \exists x\in A : \abs{x-a} = \epsilon \text{ y } f(x) + \abs{x-a}^2 + M \sum_{i=1}^k g_i(x)^2 \le f(a)
	\]
	
	Tomando $M = n\in \mathbb{N}$,
	\[
	\exists x_n\in A \text{ t.q. } \begin{cases}
	\abs{x_n-a} = \epsilon\\
	f(x_n) + \abs{x_n-a}^2+n\sum_{i=1}^k g_i(x_n)^2 \le f(a)
\end{cases}
	\]
	
	$\{x_n\}$ está acotada, pues $\abs{x_n-a} = \epsilon \ \ \forall n\in \mathbb{N}$. Luego existe $\{x_{\sigma(n)}\}$ sucesión parcial de $\{x_n\}$ tal que $\{x_{\sigma(n)}\}\to x^*$, y se verifica que $\abs{x^*-a} = \epsilon$ ($\abs{x^*-a} = \lim \abs{x_{\sigma(n)}-a} = \epsilon$).
	
	Reescribiendo tenemos que 
	\[f(x_n) + \abs{x_n-a}^2-f(a) \le -n\sum_{i=1}^k g_i(x_n)^2\quad(1)
	\]
	
	Dividiendo por $-n$ y tomando para cada $n$ $\sigma(n)$:
	\[
		\frac{f(x_\sigma(n))}{-\sigma(n)} + \frac{\epsilon^2}{-\sigma(n)}-\frac{f(a)}{-\sigma(n)} \ge \sum_{i=1}^k g_i(x_{\sigma(n)})^2
	\]
	
	Y tomando límites:
	\[0 \ge \sum_{i=1}^k g_i(x^*) \implies g_i(x^*) = 0 \ \ \forall i=1,\dots,k \implies x^*\in S\]
	
	Es decir, $x^*\in B(a,\epsilon_0)\cap S \implies f(x^*) \ge f(a)\quad (*)$. Sabemos que
	\[
		f(x_{\sigma(n)}) \le f(a) - \epsilon^2 -\underbrace{n\sum_{i=1}^k g_i(x_{\sigma(n)})^2}_{\ge 0 \text{ por (1)}}
	\]
	
	Luego,
	\[
	f(x_{\sigma(n)}) \le f(a) - \epsilon^2 \implies f(x^*) \le f(a) - \epsilon^2 < f(a)
	\]
	
	lo cual es una contradicción con (*), por lo que \textbf{queda probado el paso 1}.\\
	
	\boxed{Paso\hspace{1mm} 2}\hfill\\
	Veamos que $\forall \epsilon\in (0,\epsilon_0)\ \exists x^{\epsilon}\in A$ que verifica que $\abs{x-a} < \epsilon$ y $\exists(\lambda^{\epsilon}_0, \dots, \lambda_k^{\epsilon})\in \mathbb{R}^{k+1}$ tal que
	\[
		\begin{cases}
	\abs{(\lambda_0^{\epsilon}, \dots, \lambda_k^{\epsilon})} = 1\\
	\lambda_0^{\epsilon}\left[\displaystyle\frac{\partial f}{\partial x_j}(x^{\epsilon})+2(x_j^{\epsilon}-a_j)\right] + \displaystyle\sum_{i=1}^k \lambda_i^\epsilon \frac{\partial g_i}{\partial x_j}(x^{\epsilon}) = 0 \ \ \forall j=1,\dots,N
\end{cases}
	\]
	
	Recordemos: $f(a) = f(a)+\abs{a-a}^2+M\displaystyle\sum_{i=1}^k g_i(a)^2$ (porque $a\in S$).
	
	Definimos $F(x) := f(x) + \abs{x-a}^2 + M\displaystyle\sum_{i=1}^k g_i(x)^2 \in \mathcal{C}^1$. Por el teorema de Weierstrass, $\exists \min_{\bar{B}(a,\epsilon)} F \implies \exists x^{\epsilon}\in \bar{B}(a,\epsilon)$ tal que
	\[
	F(x^{\epsilon}) = \min_{\bar{B}(a, \epsilon)} F \left( \le \underbrace{F(a)}_{=f(a)} \underbrace{<F\restriction_{\partial B(a,\epsilon)} (x)}_{\text{Paso 1}}\implies x^{\epsilon}\in B(a,\epsilon) \text{ (es interior)}  \right)
	\]
	
	Por tanto $x^\epsilon$ es un mínimo relativo de $F$ en $B(a,\epsilon)$, que es un abierto, luego \[\displaystyle \frac{\partial F}{\partial x_j}(x^{\epsilon}) = 0<\quad \forall j=1,\dots, N\] 
	
	Y vemos que por el paso 1, además:
	
	\[
	\left[\frac{\partial f}{\partial x_j}(x^{\epsilon}) + 2(x_j^{\epsilon}-a_j)\right]+\sum_{i=1}^k M2g_i(x^{\epsilon})\frac{\partial g_i}{\partial x_j}(x^{\epsilon}) = 0\quad \forall j=1,\dots,N
	\]
	
	Ahora ajustamos las constantes, dividiendo por el módulo del vector de los candidatos a $\lambda_i$, que es $\sqrt{\underbrace{1}_{\lambda_0}+\underbrace{\sum_{i=1}^k 4M^2g_i(x^{\epsilon})^2}_{\lambda_i}}$  ($\lambda_i$ son los candidatos).
	
\[
	\frac{1}{\sqrt{1+\sum_{i=1}^k 4M^2g_i(x^{\epsilon})^2}}\left( \frac{\partial f}{\partial x_j}(x^{\epsilon})+2(x_j^{\epsilon}-a_j) \right)+\sum_{i=1}^k \frac{M2g_i(x^{\epsilon})}{\sqrt{1+\sum_{i=1}^k 4M^2g_i(x^{\epsilon})^2}}\frac{\partial g_i}{\partial x_j}(x^{\epsilon}) = 0\ \ \forall j=1,\dots, N
\]
	
	Ahora, escogemos $\lambda_0^{\epsilon} = \frac{1}{\sqrt{1+\sum_{i=1}^k 4M^2g_i(x^{\epsilon})^2}}$, $\lambda_i^{\epsilon} = \frac{M2g_i(x^{\epsilon})}{\sqrt{1+\sum_{i=1}^k 4M^2g_i(x^{\epsilon})^2}}\quad \forall i = 1,\dots,k$ y se verifica que $\abs{(\lambda_0^{\epsilon},\dots,\lambda_k^{\epsilon})} = 1$, y además $\lambda_0^\epsilon > 0$.\\
	
	\boxed{\text{Conclusión.}}\hfill\\
	Escojo $\epsilon=\frac{\epsilon_0}{n},\ n\in \mathbb{N}-\{1\} \stackrel{\text{Paso 2}}{\implies} \exists(\lambda_0^n,\dots,\lambda_k^n)$ tales que
	
	\begin{nlist}
	\item $\abs{x^n-a}<\frac{\epsilon_0}{n}$
	\item $\abs{(\lambda_0^n,\dots,\lambda_k^n)} = 1,\ \lambda_0^n>0$
	\item $\lambda_0^n\left[\frac{\partial f}{\partial x_j}(x^n)+2(x_j^n-a_j)\right]+\sum_{i=1}^k \lambda_i^n \frac{\partial g_i}{\partial x_j} (x^n) = 0 \quad \forall j=1,\dots,N$
\end{nlist}

Ahora, por (i), $\{x^n\}\to a$.
En (ii) tenemos una sucesión acotada de vectores\\ $\stackrel{\text{Bolz.-Weierstrass}}{\implies}\ \exists \{(\lambda_0^{\sigma(n)},\dots,\lambda_k^{\sigma(n)})\}\to (\lambda_0,\dots,\lambda_k)$ con módulo 1 y $\lambda_0 \ge 0$.

En (iii) reescribo sustituyendo $n$ por $\sigma(n)$:
\[
	\lambda_0^{\sigma(n)}\left[\frac{\partial f}{\partial x_j}(x^{\sigma(n)})+2(x_j^{\sigma(n)}-a_j)\right]+\sum_{i=1}^k \lambda_i^{\sigma(n)} \frac{\partial g_i}{\partial x_j} (x^{\sigma(n)}) = 0 \quad \forall j=1,\dots,N
\]

Y tomando límites:

\[
	\lambda_0\left[\frac{\partial f}{\partial x_j}(a)+0\right]+\sum_{i=1}^k \lambda_i \frac{\partial g_i}{\partial x_j} (a) = 0 \quad \forall j=1,\dots,N
\]

Donde $(\lambda_0,\dots,\lambda_k)\ne (0,\dots,0)$, porque $\abs{(\lambda_0,\dots,\lambda_k)} = 1$. \textbf{Queda probada la primera afirmación del teorema}.

%Ahora, veamos que si $r(Jg(a)) = k\implies \lambda_0 \ne 0$
\end{proof}





\end{document}
