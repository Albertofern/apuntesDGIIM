\begin{ejer}
	Sean $\alpha\in\mathbb{C}$ y $r\in\mathbb{R}^+$. Probar que $\int_{[\alpha,\alpha+r]}f(z)cdz = \int_{0}^{r} f(\alpha +s)ds$ para cualquier función $f\in C([\alpha,\alpha+r]^{\ast})$. ¿Cuál es la igualdad análoga para un segmento vertical?
\end{ejer}

\begin{sol}

Definimos $\gamma :[0,r] \rightarrow \mathbb{C}, \ \gamma (s) = \alpha + s, s\in[0,r]$
$$\int_{[\alpha,\alpha + r]} f(z)dz = \int_{0}^{r} f(\alpha +s)\gamma ' (s)ds \hspace{1cm}\text{ donde }\gamma ' (s) = 1 \forall s\in [0,r]$$

Para la igualdad respecto a un segmento vertical definimos $\xi : [0,r] \rightarrow \mathbb{C}$, $\xi(s) = \alpha +is \forall s\in[0,r]$

$$\int_{[\alpha,\alpha + ir]} f(z)dz = \int_0^r f(\alpha + is) \gamma ' (s)ds = i\int_0^r f(\alpha +is)ds$$

\end{sol}



\begin{ejer}
	Para $r\in]1,+\infty[$ se define $I(r) = \int_{\gamma_r} \frac{zdz}{z^3+1}$ y $J(r) = \int_{\sigma_r} \frac{z^2e^z}{z+1}dz$ donde $\gamma_r=C(0,r)$ y $\sigma_r = [-r,-r+i]$. Probar que $\lim_{r\rightarrow+\infty} I(r) = \lim_{r\rightarrow+\infty} J(r) = 0$.
\end{ejer}

\begin{sol}
Para la primera integral buscamos una función $M(r)$ de forma que:
$$\left| \int_{\gamma_r} \frac{z}{z^3+1}dz \right| \leq \int_{\gamma_r} \frac{|z|}{|z^3+1|} dz \leq l(\gamma_r)M(r)$$
donde $M(r)>0$ satisface que $\frac{|z|}{|z^3+1|} \leq M(r) \ \ \forall z\in\gamma_r^{\ast}$. Dado $z\in C(0,r)^{\ast}$, $|z| = r $
$$\frac{|z|}{|z^3+1|} \leq \frac{r}{|z^3|-1} = \frac{r}{r^3-1} \implies l(\gamma_r)M(r) = \lim_{r\rightarrow\infty} 2\pi r\frac{r}{r^3-1} = 0$$



Para la segunda integral
$$\left| \int_{\gamma_r} \frac{z^2 e^z}{z+1}dz \right| < \int_{\gamma_r} \frac{|z|^2 |e^z|}{z+1}dz$$

Si $z\in\sigma_r^{\ast}$
$$ |z|^2 = |-r+is|^2 \leq (r+1)^2 \hspace{1cm} s\in[0,1] $$
$$ |z+1| \geq r-1 $$

Como $z\in\sigma_r^{\ast}$, $|e^z| = e^{Rez} = e^{-r}$ y podemos acotar el valor absoluto de la integral por $0$.

\end{sol}


\begin{ejer}
	Probar que $\int_{C(0,r)} \frac{\log(1+z)}{z}dz = 0$ y deducir que $\int_0^{\pi}\log(1+r^2+2r\cos(t))dt = 0$, para todo $r\in]0,1[.$
\end{ejer}

\begin{sol}
Sabemos que $C(0,r)^{\ast} \subset D(0,1)$, definimos $f(z) = \log(1+z)$, $f\in\mathcal{H}(D(0,1))$

Usando la fórmula de Cauchy para la circunferencia (Tema 7):
$$\int_{C(0,r)} \frac{\log(1+z)}{z} dz = f(0)2\pi i = 0$$

Sin usar la fórmula de Cauchy para la circunferencia vemos:
$$\log (1+z) = \sum_{n=1}^{\infty} (-1)^{n+1} \frac{z^n}{n}$$
La serie $\sum_{n\geq 1} (-1)^{n+1} \frac{z^n}{n}$ converge uniformemente sobre compactos de $D(0,1)$ y la sucesión $\{ \frac{1}{z} \}$ está acotada en el compacto $C(0,1)^{\ast}$. 
Por lo tanto 
$\frac{1}{z} \sum_{n\geq 1} (-1)^{n+1} \frac{z^n}{n}$ converge uniformemente en $C(0,r)^{\ast}$ a $\frac{\log (1+z)}{z}$
$$0 = \int_{C(0,r)} \frac{\log(1+z)}{z} dz = \int_{-\pi}^{\pi}  \frac{\log(1+re^{it})}{re^{it}} dt
= i\int_{-pi}^{\pi} \log(1+re^{it}) dt$$
$$i\int_{-\pi}^{\pi} \ln|1+re^{it}|dt -\int_{-\pi}^{\pi} arg(1+re^{it})dt$$
Por tanto
$$ \int_{-\pi}^{\pi} \ln |1+re^{it}|dt = 0$$

Sabemos que 
$$\log(1+re^{it}) = \ln|1+re^{it}| + i*arg(1+re^{it})$$
y además
$$\gamma(t) = re^{it}, \gamma:[-\pi,\pi] \rightarrow \mathbb{C}, \ \gamma '(t) = ire^{it}$$


también sabemos que
$$|1+re^{it}| = ((1+r\cos(t))^2 + r^2\sin^2(t))^{1/2} = (1+2r\cos(t) + r^2\cos^2(t) + r^2\sin^2(t))^{1/2} = (1+2r\cos(t)+r^2)^{1/2}$$

por tanto
$$0 = \int_{-\pi}^{\pi} \ln|1+re^{it}| dt = 1/2 \int_{-\pi}^{\pi} \ln(1++2r\cos(t) +r^2)dt$$

Como sabemos que el coseno es par
$ 1/2 \int_{-\pi}^{\pi} \ln(1++2r\cos(t) +r^2)dt = \int_{0}^{\pi} \ln (1+2r\cos(t)+r^2) dt$
\end{sol}

\begin{ejer}
	Sea $f\in\mathcal{H}(D(0,1))$ verificando que $|f(z)-1|<1$ para todo $z\in D(0,1)$. Admitiendo que $f'$ es continua, probar que $\int_{C(0,r)} \frac{f'(z)}{f(z)}dz = 0$ para todo $r\in]0,1[$.
\end{ejer}

\begin{sol}
% Sabemos que $f\in D(0,1)$, teniendo que $|f(z)-1|<1, \forall z\in D(0,1)$

Sabemos que $f$ no se anula en $D(0,1)$ (de lo contrario en la integral se dividiría por $0$) y del hecho de que $|f(z)-1|<1$ deducimos que
$f(D(0,1)) \subset D(1,1)$. 
Por tanto $\log(f) \in \mathcal{H}(D(0,1))$

Como $\log(f(z))' = \frac{f'(z)}{f(z)} \implies \frac{f'}{f}$ admite primitiva holomorfa en $D(0,1)$
y $C(0,r) \subset D(0,1)$ es un camino cerrado, por tanto
$$\int_{C(0,r)} \frac{f'(z)}{f(z)}dz = 0$$

\end{sol}

\begin{ejer}
	Sean $\Omega = \mathbb{C}\backslash \{i,-i\}$ y $f\in\mathcal{H}(\Omega)$ dada por $f(z)=1/(1+z^2)$ para todo $z\in\Omega$. Probar que $f$ no admite una primitiva en $\Omega$.
\end{ejer}

\begin{sol}
Sabemos que
$$\frac{1}{1+z^2} = \frac{1}{(z-i)(z+i)} = \frac{A}{z-i} + \frac{B}{z+i} = \frac{1}{2i} \left[ \frac{1}{z-i} - \frac{1}{z+i} \right]$$
Por tanto
$$\int_{C(i,1)} \frac{1}{1+z^2} dz = \frac{1}{2i} \left[ \int_{C(i,1)}\frac{1}{z-i}dz - \int_{C(i,1)} \frac{1}{z+i}dz \right] \not = 0$$

La primera es $2\pi i$ y la segunda integral es $0$ por el teorema de Cauchy para dominios estrellados (Tema 7), teniendo que $C(i,1) \subset D(i,3/2)$.
\end{sol}


\begin{ejer}
	Probar que $\int_{\sigma} \frac{dz}{1+z^2} = 0$, donde $\sigma(t) = \cos(t)+(i/2)\sin(t)$ para todo $t\in[0,2\pi]$.
\end{ejer}
\textbf{Idea}

$\Omega = \mathbb{C}\backslash \{ iy : |y|>1 \}$
$\arctan \in \mathcal{H}(\Omega)$ y $\arctan$ es una primitiva de $\frac{1}{1+z^2}$

$\sigma ^{\ast} \subset \Omega$ y $\sigma$ es un camino cerrado.

Con esa información sabremos que 
$\int_{\sigma} \frac{1}{1+z^2} dz = 0$

