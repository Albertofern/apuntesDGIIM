\subsection{Ejercicio 1.}
Estudiar la continuidad de la función argumento principal, $arg : \mathbb{C}^{\ast} \rightarrow \mathbb{R}$ .

\textbf{Solución}

Usamos la fórmula de la relación anterior para probar que es continua en $\mathbb{C}^{\ast} \backslash \mathbb{R}$
$$  
arg z = 2\arctan (\frac{Imz}{Rez + |z|}) \hspace{1cm} \forall z\in\mathbb{C}^{\ast}\backslash\mathbb{R}
$$
Luego nos paroximamos por sucesiones

$$  
\{ z+\frac{i}{n} \} \rightarrow z \hspace{1cm}
\lim_n(arg(z+\frac{i}{n})) = \lim_n( \arctan(\frac{1}{nz}+\pi) ) = \pi
$$

$$
\{ z-\frac{i}{n} \} \rightarrow z
\lim_n(arg(z-\frac{i}{n})) = \lim_n( \arctan(\frac{-1}{nz}-\pi) ) = -\pi
$$


\subsection{Ejercicio 2.}
Dado $\theta\in\mathbb{R}$, se considera el conjunto $S_{\theta} = \{ z\in\mathbb{C}^{\ast} : \theta\not\in Argz \}$. 
Probar que existe una función $\phi\in\mathcal{C}(S_{\theta})$ que verifica


\textbf{Solución}

Definimos
$$
f(z) = z( \cos(\pi-\theta) + i\sin(\pi-\theta) ) \hspace{2cm} f:\mathbb{C}\rightarrow \mathbb{C} \text{ es continua}
$$
$$
\phi(z) = arg(f(z)) - (\pi-\theta) \hspace{2cm} \phi: S_{\theta} \rightarrow \mathbb{C} \text{ $\phi$ es continua}
$$

$$ 
\phi(z) \in Arg (f(z)) + \theta-\pi \subset Arg (f(z)) + Arg( \cos(\theta-\pi)+i\sin(\theta-\pi) ) 
$$
$$= Arg ( f(z)-\cos(\theta-\pi)+i(\theta-\pi) ) = Argz
$$


\subsection{Ejercicio 3.}
Probar que no existe ninguna función $\phi\in\mathcal{C}^{\ast}$ tal que $\phi(z)\in Argz$ para todo $z\in\mathcal{C}^{\ast}$



\textbf{Solución}

$\mathbb{T}$ compacto y conexo $\implies \phi(\mathbb{T})$ intervalo cerrado y acotado

Sea $\alpha\in\mathbb{T} : \phi(\alpha) \not =\min(\phi(\mathbb{T})), \max(\phi(\mathbb{T}))$ 
$\phi ( \mathbb{T}-\{ \alpha \} )$ conexo $\implies [ \min(\phi(\mathbb{T})), \phi(\alpha)[ \cup ]\phi(\alpha), \max( \phi(\mathbb{T}) ) ] $ que es una contradicción.

Si $\exists h:\mathbb{C}^{\ast} \rightarrow \mathbb{R}$ con $h(z)\in Arg(z)$ $\forall z\in\mathbb{C}^{\ast}$ y $h$ continua, entonces
$h_{| \mathbb{T}} : \mathbb{T} \rightarrow \mathbb{T} \rightarrow \mathbb{R}$ es continua y $h_{|\mathbb{T}} (z) \in Arg(z) \ \forall z\in\mathbb{T}$



\subsection{Ejercicio 4.}


\textbf{Solución}
$$
\{ z_n \} \rightarrow z \implies \forall \epsilon>0\ \exists m\in\mathbb{N} : n\geq m \ |z_n-z|z\epsilon
$$
Sea $\epsilon = \frac{|z|}{2}$, entonces 
$\exists m\in\mathbb{N} :  n\geq m\ z_n\in D(z, |z|/2)$

Usamos el ejercicio 2




\subsection{Ejercicio 5.}


\textbf{Idea}


$$
|1+\frac{z}{n}|^n \rightarrow e^{Re z}
$$
$$
\lim_{n\rightarrow\infty} |1+\frac{z}{n}|^n 
= e^{ \lim_{n\rightarrow\infty} n(|1+\frac{z}{n}|-1) }
$$

$$
arg((1+\frac{z}{n})^n) \rightarrow Im z
$$

\textbf{Solución}

$z_n\in\mathbb{C}, \phi_n\in Arg(z_n)$

$\{ |z_n| \} \rightarrow |z|$ y $\{y_n\} \rightarrow y\in Arg(z)$

Vamos a usar que is tiene sun sucesión de numeros complejos y la sucesion de los modulos converge y hay una sucesion de los argumentos de forma q convergen, la sucesion converge al modulo por el cos +isen

$\phi_n \in Arg(z_n)$, $z_n = (1+\frac{z}{n})^n = (1+\frac{a+ib}{n})^n
= [(1+\frac{a}{n}) + (\frac{ib}{n})] ^n$
$|z_n| = [| (1+\frac{a}{n}) + \frac{ib}{n} |] ^n
= [ (1+\frac{a}{n})^2 + (\frac{b}{n})^n ]^{1/2}
= lim n [ (1+\frac{a^2}{n^2} + (\frac{a}{n}+\frac{b^2}{n^2}) )]^{n/2}
= e^{  \lim n/2 (\frac{a^2}{n^2} + 2a/n + b^2/n^2 )  } = e^a = e^{Re z}$ 

Vamos a utilizar la fórmula de Moivre.

$ z_m = (1+z/n)^n$
$Arg(a+ \frac{a+ib}{n})^n = n Arg(a+\frac{a+ib}{n}) = n*\arctan (\frac{b/n}{1+a/n}) = n*\arctan(\frac{b}{n+a})$
Llamamos $y=b/(n+a)$ y usamos $\lim_{y\rightarrow0} \frac{\arctan(y)}{y} = 1$,
$n*\arctan(\frac{b}{n+a}) = \frac{\arctan(\frac{b}{n+a})}{ \frac{n+a}{n}\frac{b}{n+a}\frac{1}{b} } = b$