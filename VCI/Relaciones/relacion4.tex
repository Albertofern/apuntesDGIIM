\begin{ejer}
	Calcular el radio de convergencia de las siguientes series de potencias:
	\begin{enumerate}[label=(\alph*)]
		\item $\sum_{n\geq1} \frac{n!}{n^n}z^n$
		\item $\sum_{n\geq 0} z^{2n}$
		\item $\sum_{n\geq 0} 2^nz^{n!}$
		\item $\sum_{n\geq 0} (3+(-1)^n)^nz^n$
		\item $\sum_{n\geq 0} (n+a^n)z^n \ (a\in\mathbb{R}^+)$
		\item $\sum_{n\geq 0} a^{n^2}z^n \ (a\in\mathbb{C})$
	\end{enumerate}
\end{ejer}

\begin{comment}


\textbf{b)} %(*)

$\sum_{n\geq 0} z^{2n} = \sum_{n\geq 0} \alpha_nz^n$
con $\alpha_nz{2n+1} =0, \alpha_{2n} = 1$

Vemos el límite superior de la sucesión
$\limsup |\alpha_n| = 1 \implies R=\frac{1}{1} = 1$
\end{comment}

\begin{sol}


\textbf{c)}
$$\sum_{n\geq 0} 2^nz^{n!}$$
Si $|z|\geq 1$ entonces la serie diverge, en el caso $|z|<1$:
$$\sqrt[n]{|2^nz^{n!}|} = \sqrt[n]{2^n|z|^{n!}} = 2|z|^{(n-1)!}$$
Dicha sucesión tiende a $0$ cuando $n\rightarrow\infty$, por tanto %(*)
$\sum_{n\geq 0} |2^nz^{n!}| < \infty \implies \sum_{n\geq 0} 2^nz^{n!} < \infty$, entonces el radio de convergencia de la serie es $R=1$.



\textbf{e)}
$$\sum_{n\geq 0} (n+a^n)z^n \hspace{1cm} a\in\mathbb{R}^+$$
Podríamos separar en dos series, calcular sus radios de convergencia y cogemos el menor:
$$\sum_{n\geq 0} nz^n \hspace{1.5cm}\sum_{n\geq 0} a^nz^n\hspace{1cm} a\in\mathbb{R}^+$$
\begin{comment}
Observamos que $a^n \leq n+a^n \leq (n+1)a^n$,
en los extremos el radio de convergencia es $1/a$, por tanto el radio de convergencia de $n+a^n$ es $1/a$
$$\sum_{n\geq 0} a^n |z|^n \leq \sum_{n\geq 0} (n+a^n)|z|^n \leq \sum_{n\geq 0} (n+1)a^n|z|^n$$
Por tanto el radio de convergencia es $1/a$.
\end{comment}

También podemos usar otros criterios:
$$\lim_{n\rightarrow\infty} \sqrt[n]{n+a^n} = \lim_{n\rightarrow\infty} \frac{(n+1)+a^{n+1}}{n+a^n} $$
por tanto el radio de convergencia buscado es:
$$ R = \left\{ \begin{array}{lcc}
1/a &   si  & a>1 \\
1 &  si & a\leq 1
\end{array}
\right. $$


\begin{comment}
\textbf{f)}
$\sum_{n\geq 0} a^{n^2}z^n,a\in\mathbb{C}$
raíz n-esima 
\end{comment}

\end{sol}

\begin{ejer}
	Conocido el radio de convergencia $R$ de la serie $\sum_{n\geq 0} \alpha_nz^n$, calcular el de las siguientes:
	\begin{enumerate}[label=(\alph*)]
		\item $\sum_{n\geq 0} n^k\alpha_nz^n \ (k\in\mathbb{N} \text{ fijo})$
		\item $\sum_{n\geq0} \frac{\alpha_n}{n!}z^n$	
	\end{enumerate}
\end{ejer}
\begin{sol}

\textbf{a)}

El radio de convergencia no se reduce, podemos comprobarlo utilizando el criterio de la raíz.

\textbf{b)}

Vemos el cociente entre los términos $n+1$ y $n$:
$$ \frac{\alpha_{n+1}/(n+1)!}{\alpha_n/n} = \frac{\alpha_{n+1}}{\alpha_n (n+1)}$$
Tomando límites:
$$ \lim_{n\rightarrow\infty} \frac{\alpha_{n+1}}{\alpha_n (n+1)} = \lim_{n\rightarrow\infty}\frac{\alpha_{n+1}}{\alpha_n}\lim_{n\rightarrow\infty}\frac{1}{n+1} = \frac{1}{R} \lim_{n\rightarrow\infty}\frac{1}{n+1} = 0$$
Por tanto el radio de convergencia de la serie es infinito.
\end{sol}

\begin{ejer}
	Caracterizar las series de potencias que convergen uniformemente en todo el plano.
\end{ejer}
\begin{sol}

Convergen en todo el plano las que a partir de cierto término son 0, o sea, las que son una suma finita

Sea $\sum_{n\geq 0} \alpha_n (z-a)^n$ que converge uniformemente
$$S_n=\sum_{k=0}^{n-1} \alpha_k (z-a)^k \text{ \ converge uniformemente } \Longleftrightarrow S_n \text{ es uniformemente de Cauchy }$$
Lo cual, por definición, es:
$$\forall\epsilon>0,\exists m\in\mathbb{N}: p\geq q\geq n \implies |S_p(z)-S_q(z)| < \epsilon \hspace{1cm}\forall z\in\mathbb{C}$$
Sea un polinomio $p\in\mathbb{P}[\mathbb{C}]$, tenemos que $\lim_{|z|\rightarrow\infty} p(z) = \infty \Longleftrightarrow gr(p)>0$ (grado del polinomio).

Vemos que $|S_p(z)-S_q(z)|$ es un polinomio que está acotado, tiene que ser constante para no diverger en infinito, en particular es la constante $0$, ya que $S_p(a)=S_q(a)$, por tanto $$\sum_{q=m}^{p-10} \alpha_n (z-\alpha)^n = 0 \implies  \alpha_n=0 \hspace{1cm}\forall n\geq m$$ %(*)

\end{sol}

\begin{ejer}
	Estudiar la convergencia puntual, absoluta y uniforme, de la serie $\sum_{n\geq 0} f_n$ donde
	$$ f_n(z) = \left( \frac{z-1}{z+1} \right)^n\hspace{1cm} \forall z\in\mathbb{C}\backslash\{-1\} $$
\end{ejer}

\begin{comment}
Idea:

llamar w a (w-1)/(z-1), y mirar cuando la función tiene imagen que cae en el disco de centro 0 y radio 1, afinar para saber cuando un subconjunto de C po esta función cae dentro de un subconjunto compacto donde el sumatorio converja
\end{comment}