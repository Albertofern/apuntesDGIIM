4 
Existe la posibilidad de que $f$ sea constante.

En otro caso, si $f$ no es constante usando Liouille sabemos que $f(\mathbb{C})$ es denso en $\mathbb{C}$
Entonces
$\forall z\in f(\mathbb{C})$, $\exists  : f(c)=z \implies f(z) = f(f(c)) = f(c) = z$


Por tanto

$f(\mathbb{C}) \cap \mathbb{C} \implies f(z)=z \forall z\in \mathbb{C}$

también se puede razonar de la siguiente forma:
$\forall z\in\mathbb{C}, \exists \{z_n\}_{n\in\mathbb{N}}$ tal que 
$z_n\in f(\mathbb{C}) \forall n\in\mathbb{C} \forall n\in\mathbb{N}$ y $\{z_n\}\rightarrow z \implies \{f(z_n)\}\rightarrow f(z)=z \implies f(z)=z \forall z\in\mathbb{C}$


Ejercicio 8

\

$f'(g(z)) g'(z) = 0 $
entonces puede darse
$f'(g)=0$ o $g'=0$

Ejercicio 9

\

Definimos
$h(z) = \frac{1}{f(1/z)}$, como $\lim_{w\rightarrow\infty}f(w) = \infty$ por tanto $\exists R>0$ tal que si $|w|>R$, entonces $|f(w)|>1$

Si $z\in D(0,1/R)\backslash\{0\}$, $|z|<1/R \Longleftrightarrow 1/|z|>R$, entonces

podemos definir
$h : D(0,1/R)\backslash\{0\} \rightarrow \mathbb{C}$,
$h\in\mathbb{H}(D(0,1/R)\backslash\{0\})$ y $\lim_{z\rightarrow 0} h(z) = \lim_{z\rightarrow 0} \frac{1}{f(1/z)} = 0$
y deducimos por el Teorema de Extensión de Riemman
$h\in\mathbb{H}(D(0,1/R))$ y $h(0)=0$

$\exists g\in\mathbb{H}(D(0,1/R)))$ con $g(0) \not = 0$, $\exists k\in\mathbb{N}$ tal que $h(z) = \frac{1}{f(1/z)} = z^k g(z)$

$ f(1/z) = \frac{1}{z^k}\frac{1}{g(z)}$
donde el valor absoluto del segundo término está acotado, por tanto
$f(w) = w^k \frac{1}{g(1/w)}$,
como $g(0)\not = 0 \exists\delta>0$, $\exists C>0$ tal que $D(0,\delta)\subset D(0,1/R)$ tal que $|g(z)| \geq C > 0 \forall z\in D(0,\delta) \implies$
$\frac{1}{|g(z)|} \leq \frac{1}{C} \forall z\in D(0,\delta) \implies \frac{1}{|g(1/w)|} \leq \frac{1}{C} \forall x\in \mathbb{C}\backslash \overline{D}(0,1/\delta)$

$|f(w)| = |w|^k \frac{1}{g(1/w)} \leq \frac{1}{C}|w|^k$ $\forall w\in \mathbb{C}\backslash\overline{D}(0,1/\delta)$

por el ejercicio 2 de esta relación

$f$ es un polinomio, en particular de grado menor o igual que $k$


10 pista
LEMA
Si f verifica que $|z|=1 \implies |f(z)|=1$, entonces $f(0)=0$ o $f$ es constante.
PISTA LEMA:
$1=|f(z)|^2 = f(z)\overline{f(z)} \forall z\in\mathbb{T}$ 
$1= f(\overline{z})\overline{f(\overline{z})} = f(1/z)\overline{f(\overline{z})}\forall $ 
--
Si hemos probado eso y f es constante es trivial, en caso contrario
$f(0)=0 \implies \exists g\in\mathbb{H}(\mathbb{C})$, $\exists k\in\mathbb{N}$ con $g(0) \not =0$ tal que
$f(z) = z^kg(z) \forall z\in\mathbb{C}$

$1=|f(z)| = |z|^k|g(z)| z\in\mathbb{T}$
$g$ está en las mismas condiciones que $f$ y por el lema anterior $g$ es constante