\begin{ejer}
	Sea $f\in\mathcal{H}(D(0,1))$ tal que
	$$ |f(z)| \leq \frac{1}{1-|z|} \hspace{1cm} \forall a\in D(0,1) $$
	Probar que $|f^{(n)}(0)| \leq e(n+1)!$ para todo $n\in\mathbb{N}$.
\end{ejer}

\begin{ejer}
	Sea $f$ una función entera verificando que existen constantes $\alpha,\beta,\rho\in\mathbb{R}^+$ tales que
	$$ z\in\mathbb{C}, \ \ |z|>\rho \implies |f(z)| \leq \alpha |\alpha|^{\beta} $$
	Probar que $f$ es una función polinómica de grado menor o igual que $\beta$.
\end{ejer}

\begin{ejer}
	Sea $f$ una función entera verificando que
	$$ f(z) = f(z+1)=f(z+i) \hspace{1cm} \forall z\in\mathbb{C} $$
	Probar que $f$ es constante.
\end{ejer}

\begin{ejer}
	Sea $f$ una función entera verificando que $f(f(z))=f(z)$ para todo $z\in\mathbb{C}$. ¿Qué se puede afirmar sobre $f$?
\end{ejer}
Existe la posibilidad de que $f$ sea constante.

En otro caso, si $f$ no es constante usando el \textit{teorema de Liouville} sabemos que $f(\mathbb{C})$ es denso en $\mathbb{C}$
Entonces
$\forall z\in f(\mathbb{C})$, $\exists  : f(c)=z \implies f(z) = f(f(c)) = f(c) = z$


Por tanto

$f(\mathbb{C}) \cap \mathbb{C} \implies f(z)=z \forall z\in \mathbb{C}$

también se puede razonar de la siguiente forma:
$\forall z\in\mathbb{C}, \exists \{z_n\}_{n\in\mathbb{N}}$ tal que 
$z_n\in f(\mathbb{C}) \forall n\in\mathbb{C} \forall n\in\mathbb{N}$ y $\{z_n\}\rightarrow z \implies \{f(z_n)\}\rightarrow f(z)=z \implies f(z)=z \forall z\in\mathbb{C}$

\begin{ejer}
	En cada uno de los siguientes casos, decidir si existe una función $f$, holomorfa en un entorno del origen, y verificando que $f(1/n)=a_n$ para todo $n\in\mathbb{N}$ suficientemente grande:
	\begin{enumerate}[label=(\alph*)]
		\item $a_{2n} = 0, \ \ a_{2n-1}=1 \hspace{1cm} \forall n\in\mathbb{N}$
		\item $a_{2n} = a_{2n-1} = \frac{1}{2n} \hspace{1cm} \forall n\in\mathbb{N}$
		\item $a_n = \frac{n}{n+1} \hspace{1cm} \forall n\in\mathbb{N}$
	\end{enumerate}
\end{ejer}

\begin{ejer}
	Enunciar y demostrar un resultado referente al orden de los ceros de una suma, producto o cociente de funciones holomorfas.
\end{ejer}

\begin{ejer}
	Dado un abierto $\Omega$ del plano, probar que el anillo $\mathcal{H}(\Omega)$ es un dominio de integridad si, y sólo si, $\Omega$ es conexo.
\end{ejer}


\begin{ejer}
	¿Qué se puede afirmar sobre dos funciones enteras cuya composición es constante?
\end{ejer}
$f'(g(z)) g'(z) = 0 $
entonces puede darse
$f'(g)=0$ o $g'=0$


\begin{ejer}
	Sea $f$ una función entera verificando que $f(z)\rightarrow\infty (z\rightarrow\infty)$. Probar que $f$ es una función polinómica.
\end{ejer}
Definimos
$h(z) = \frac{1}{f(1/z)}$, como $\lim_{w\rightarrow\infty}f(w) = \infty$ por tanto $\exists R>0$ tal que si $|w|>R$, entonces $|f(w)|>1$

Si $z\in D(0,1/R)\backslash\{0\}$, $|z|<1/R \Longleftrightarrow 1/|z|>R$, entonces

podemos definir
$h : D(0,1/R)\backslash\{0\} \rightarrow \mathbb{C}$,
$h\in\mathbb{H}(D(0,1/R)\backslash\{0\})$ y $\lim_{z\rightarrow 0} h(z) = \lim_{z\rightarrow 0} \frac{1}{f(1/z)} = 0$
y deducimos por el Teorema de Extensión de Riemman
$h\in\mathbb{H}(D(0,1/R))$ y $h(0)=0$

$\exists g\in\mathbb{H}(D(0,1/R)))$ con $g(0) \not = 0$, $\exists k\in\mathbb{N}$ tal que $h(z) = \frac{1}{f(1/z)} = z^k g(z)$

$ f(1/z) = \frac{1}{z^k}\frac{1}{g(z)}$
donde el valor absoluto del segundo término está acotado, por tanto
$f(w) = w^k \frac{1}{g(1/w)}$,
como $g(0)\not = 0 \exists\delta>0$, $\exists C>0$ tal que $D(0,\delta)\subset D(0,1/R)$ tal que $|g(z)| \geq C > 0 \forall z\in D(0,\delta) \implies$
$\frac{1}{|g(z)|} \leq \frac{1}{C} \forall z\in D(0,\delta) \implies \frac{1}{|g(1/w)|} \leq \frac{1}{C} \forall x\in \mathbb{C}\backslash \overline{D}(0,1/\delta)$

$|f(w)| = |w|^k \frac{1}{g(1/w)} \leq \frac{1}{C}|w|^k$ $\forall w\in \mathbb{C}\backslash\overline{D}(0,1/\delta)$

por el ejercicio $2$ de esta relación

$f$ es un polinomio, en particular de grado menor o igual que $k$


\begin{ejer}
	Sea $f$ una función entera verificando que
	$$ z\in\mathbb{C}, \ \ |z|=1 \implies |f(z)|=1 $$
	Probar que existen $\alpha\in\mathbb{C}$ on $|\alpha|=1$ y $n\in\mathbb{N}\cup \{0\}$ tales que $f(z) = \alpha z^n$ para todo $z\in\mathbb{N}$.
\end{ejer}

\textbf{Pista}

LEMA
Si f verifica que $|z|=1 \implies |f(z)|=1$, entonces $f(0)=0$ o $f$ es constante.
PISTA LEMA:
$1=|f(z)|^2 = f(z)\overline{f(z)} \forall z\in\mathbb{T}$ 
$1= f(\overline{z})\overline{f(\overline{z})} = f(1/z)\overline{f(\overline{z})}\forall $ 
--
Si hemos probado eso y f es constante es trivial, en caso contrario
$f(0)=0 \implies \exists g\in\mathbb{H}(\mathbb{C})$, $\exists k\in\mathbb{N}$ con $g(0) \not =0$ tal que
$f(z) = z^kg(z) \forall z\in\mathbb{C}$

$1=|f(z)| = |z|^k|g(z)| z\in\mathbb{T}$
$g$ está en las mismas condiciones que $f$ y por el lema anterior $g$ es constante