\subsection{Ejercicio 1}
\subsubsection{a)}
Cauchy Riemman

\textbf{Solución}


\textbf{c)}

$f(x+iy) = \frac{x^3+iy^3}{x^2+y^2}$, $\forall (x,y)\in\mathbf{R}^2 \backslash \{ (0,0) \}$, $f(0)=0$

$u(x,y) = Re f = \frac{x^3}{x^2+y^2}, v(x,y) = Imf = \frac{y^3}{x^2+y^2}$

En $\mathbf{C}^{\ast}$

$ \frac{\partial u}{\partial x} = \frac{3x^2}{x^2+y^2} + \frac{x^4(-1)2}{(x^2+y^2)^2}$
$ \frac{\partial v}{\partial y} = \frac{3y^2}{x^2+y^2} = \frac{y^4(-1)2}{(x^2+y^2)^2}$

entonces
$\frac{\partial u}{\partial y} = \frac{x^3 (-1)2y}{(x^2+y^2)^2}$
$\frac{\partial v}{\partial x} = \frac{y^3(-1)2x}{(x^2+y^2)^2}$

De las dos ecuaciones primeras deducimos

$3x^2(x²+y^2) - 2x^4 = 3y^2(x^2+y^2)-2y^4$  (1)

de últimas dos deducimos

$2yx^3=-2xy^3 \implies xy(x^2+y^2) = 0 \implies xy=0 ó x^2+y^2=0$

por tanto

$x^4-y^4 = 0 \implies |x|=|y|$



\subsection{Ejercicio 2}
2 cauchy riemman
$Imf = 4x^3*y - 4xy^3$

\subsection{Ejercicio 3}
3 cauchy rieman
$a=-c$


\subsection{Ejercicio 4}
\textbf{Enunciado}


\textbf{Solución}
\textbf{4 cauchy rieman, derivar respecto x y respecto y}
$Ref (x,y) = u(x,y)$ $Imf(x,y) = v(x,y)$

Los casos extremos son $a=0$ o $b=0$, donde o la imagen o la parte real que quedan serían constantes.
$au(x,y)+bv(x,y)=c$
$-bv(x,y) = c-au(x,y)$
$v(x,y)= c'-a'u(x,y)$
Derivamos

$\frac{\partial v(x,y)}{\partial x} = \frac{-a' \partial u(x,y)}{\partial x} = \frac{-a' \partial v(x,y)}{\partial y}$
$= \frac{\partial v(x,y)}{\partial y} = \frac{-a' \partial u(x,y)}{\partial y} = -a' \frac{\partial v(x,y)}{\partial x}$
implica
$\frac{\partial u (x,y)}{\partial x} = -a'\frac{\partial v(x,y)}{\partial y} = -a'a'\frac{\partial v(x,y)}{\partial x}$
por tanto
$\frac{\partial v(x,y)}{\partial x} = 0 = \frac{\partial v(x,y)}{\partial y}$ en $\Omega$


\subsection{Ejercicio 5}
\textbf{Enunciado}

Sea $\Omega$ un dominio y $f\in\mathcal{H}(\Omega)$. Probar que si $\overline{f}\in\mathcal{H}(\Omega)$, entonces $f$ es constante.

\textbf{Solución}

$\Omega$ dominio, $f\in\mathcal{H}(C)$
$f+\overline{f} = 2Re f$
$g$ es holomorfa por ser suma de holomorfas, $g\in\mathcal{H}(C)$
$g\in\mathcal{H}(\Omega)$ y su parte imaginaria es constante, por tanto $g$ es constante, entonces

$Re f$ es constante, $f\in\mathbb{H}(\Omega), \Omega$ dominio implican que $f$ es constante.


\subsection{Ejercicio 6}
\textbf{Enunciado}





\textbf{Solución}

$P(z) = a_n z^n + a_{n-1}z^{n-1}+...+a_1z + a_0$
$P(\overline{z}) = a_n \overline{z}^n + a_{n-1}\overline{z}^{n-1}+...+a_1\overline{z} + a_0 $

$\overline{ P(\overline{z}) } = \overline{a_n}z^n + \overline{a_{n-1}}z^{n-1}+...+\overline{a_1}z + \overline{a_0}$

$\overline{z}^n  = \overline{z} ... \overline{z} = \overline{(z^n)}$


resolvemos

$a\in\Omega^{\ast} \longleftrightarrow \overline{a}\in\Omega$
$\lim_{z\rightarrow a} \frac{f^{\ast}(z)-f^{\ast}(a)}{z-a} = \lim_{z\rightarrow a} \frac{\overline{f(\overline{z})} - \overline{f(\overline{a})}}{z-a} =
 \lim_{z\rightarrow a} \frac{\overline{\overline{f(\overline{z})} - \overline{f(\overline{a})}}}{z-a} =
 \lim_{z\rightarrow a} \overline{\left( \frac{f(\overline{z})-f(\overline{a})}{z-a} \right)} =
 \overline{f'(\overline{a})}$


\subsection{Ejercicio 7}
\textbf{Enunciado}



\textbf{Solución}

\textbf{Idea} 7 jugar con los grados del polinomio

Suponemos que  existe una función racional que es la exponencial:

$R(z) = \frac{\sum_{i=0}^k \lambda_i z^i}{\sum_{i=0}^m \mu_i z^i} = \frac{A(z)}{B(z)}$

$R(z) = R'(z) = \frac{A'(z)B(z)-A(z)B'(z)}{B(z)^2} = \frac{A(z)}{B(z)}$ 
(Tachamos los $B(z)$)

$A(z) = A'(z) - \frac{B'(z)A(z)}{B(z)}$,
$B(z)A(z) = A'(z)B(z) - B'(z)A(z)$, esto se da $\forall z\in D(a,r)$ con $a\in\mathbb{C}, r>0$

Tienes dos polinomios que son iguales en un entorno no vació. Para deducir lo último tienes que pensar que son polinomios iguales en un montón de puntos.


%$gr(A(z)) \leq gr(A'(z))$
$gr(B(z)) + gr(A(z)) \leq \max \{ gr(A'(z)) + gr(B(z)), gr(A(z))+gr(B'(z)) \}$

Por tanto llegamos a una contradicción, nuestra hipótesis no era cierto.