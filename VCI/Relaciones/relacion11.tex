\begin{ejer}
	Sea $R\in\mathbb{R}^+$ y $f\in\mathcal{H}(D(0,R))$, no constante.
	Probar que la función $M : ]0,R[\rightarrow\mathbb{R}$ definida por
	$$ M(r) = \max\{ |f(z)| : z\in C(0,r)^{\ast} \} \hspace{1cm} \forall r\in ]0,R[ $$
	es estrictamente creciente.
\end{ejer}

\begin{sol}
Vamos a suponer que $M(r)$ no es estrictamente creciente, existirían $0<r1<r2<R$ tal que
$$M(r_1) \geq M(r_2) = \max\{ |f(z)| : z\in C(0,r_2)^{\ast} \} = \max\{ |f(z)| : z\in\overline{D}(0,r_2) \}$$
por tanto
$\exists z_0 \in D(0,r_1)$ en el que $|f|$ alcanza un máximo relativo para el disco $D(0,r_2)$.
Ahora usamos el principio de módulo máximo para deducir que 
$f$ es constante en $D(0,r_2)$ y por el principio de identidad
$f$ es constante en $D(0,R)$. 
Hemos llegado a una contradicción, por lo tanto nuestra hipótesis no es cierta y $M(r)$ es estrictamente creciente.

\end{sol}

\begin{ejer}
	Sea $\Omega = \{ z\in\mathbb{C} : |z|>1 \}$ y $f:\overline{\Omega}\rightarrow\mathbb{C}$ una función continua en $\overline{\Omega}$ y holomorfa en $\Omega$, que tenga límite en $\infty$. 
	Probar que $|f|$ tiene un máximo absoluto en un punto de $\mathbb{T}$.
	Suponiendo que $f$ no es constante, probar que la función $M:[1,+\infty[\rightarrow\mathbb{R}$ definida por
	$$ M(r) = \max\{ |f(z)| : z\in C(0,r)^{\ast} \}  \hspace{1cm} \forall r\in ]1,+\infty[$$
	es estrictamente decreciente.
\end{ejer}
\begin{sol}
Definimos
$$g:\overline{D}(0,1)\backslash\{0\} \rightarrow \mathbb{C}, \ \ g(z) = f(1/z) \hspace{1cm} g\in\mathcal{H}(D(0,1)\backslash\{0\}), \
g\in C(\overline{D}(0,1)\backslash\{0\})$$
Vemos el límite en el punto problemático
$$\lim_{z\rightarrow0} g(z) = \lim_{w\rightarrow\infty} f(w) \in\mathbb{C}$$
Por tanto, por el teorema de extensión de Riemann
$g\in\mathcal{H}(D(0,1))$

Ahora se aplica el resultado del ejercicio $1$ a $g$ y sacamos la conclusión sobre $f$.
\end{sol}


\begin{ejer}
	Sea $P$ un polinomio de grado $n\in\mathbb{N}$ y $M = \max\{ |P(z)| : z\in\mathbb{T} \}$. Probar que
	$$ z\in\mathbb{C},\ \ |z|\geq 1 \implies |P(z)| \leq M|z|^n $$
\end{ejer}
\begin{sol}
$$\lim_{z\rightarrow\infty} \frac{P(z)}{z^n} = a_n$$
$$f(z) = \frac{P(z)}{z^n}, \ \ f\in\mathcal{H}(\{ z\in\mathbb{C} : |z|>1 \}) \text{ y $f$ es continua en }\{ z\in\mathbb{C} : |z|\geq 1 \}$$
y usando ejercicio $2$ sabemos que
$|f|$ alcanza su máximo en el conjunto $\{ z\in\mathbb{C} : |z|\geq 1 \}$ en un punto de $\mathbb{T}$.
$$\left| \frac{P(z)}{z^n} \right| = |f(z)| \leq M \hspace{1cm} \forall z\in\mathbb{C} : |z|\geq 1$$
\end{sol}


\begin{ejer}
	Sea $f:\overline{D}(0,1)\rightarrow\mathbb{C}$ una función continua en $\overline{D}(0,1)$ y holomorfa en $D(0,1)$, que verifica la siguiente condición:
	$$ z\in\mathbb{T} \implies |f(z)| \leq \left\{ \begin{array}{lcc}
	1 &  si  & Re(z) \leq 0 \\
	2 &  si  & Re(z) > 0
	\end{array}
	\right. $$
	Probar que $|f(0)| \leq \sqrt{2}$.
\end{ejer}
\begin{sol}
$$|f(a)| = \left| \frac{1}{2\pi i}\int_{C(a,r)} \frac{f(w)}{w-a} dw \right|$$
$$|f(a)| \leq \frac{1}{2\pi} \int_{C(a,r)} \frac{|f(w)|}{|w-a|} dw \leq \frac{l(C(a,r))}{2\pi r} \max\{ f(w) : w\in C(a,r) \} = \max\{ f(w) : w\in C(a,r) \} $$

Definimos $g(z) = f(z)f(-z)$, 
$|f(0)|^2 = |g(0)| = \max\{ |g(w)| : w\in\mathbb{T} \} \leq 2$ $\forall w\in\mathbb{T}$
\end{sol}


\begin{ejer}
	Sea $f\in\mathcal{H}(D(0,1))$ y supongamos que existe $n\in\mathbb{N}$ tal que, para todo $r\in ]0,1[$ se tiene
	$$ \max \{ |f(z)| : |z|=r \} = r^n $$
	Probar que existe $\alpha\in\mathbb{T}$ tal que $f(z)=\alpha z^n$ para todo $z\in D(0,1)$.
\end{ejer}
\begin{sol}
Sabemos que $f\in\mathcal{H}(D(0,1))$ y usamos la desigualdad de la media en $C(0,r)$ 
$$|f(0)| \leq \max\{ |f(z)| : z\in C(0,r)^{\ast} \} = r^n \hspace{1cm} 0<r<1$$
tomamos límite con $r\rightarrow 0$ obtenemos que $f(0)=0$

Si $f$ es constante no hay nada que demostrar.
Si $f$ no es constante, entonces, por el principio de los ceros de una función holomorfa $\exists g\in\mathcal{H}(D(0,1)), \exists k\in\mathbb{N}$ con $g(0) \not=0$ tal que $f(z) = z^kg(z)$ $\forall z\in D(0,1)$

Si tenemos que $k<n$
$$r^n = \max\{ |f(z)| : |z|=r \} = \max\{ |z^kg(z)| : |z|=r \} = r^k \max\{ |g(z)| : |z|=r \}$$
por tanto
$\max\{ |g(z)| : |z|=r \} = r^{n-k}$ tomando límite en $r\rightarrow0$ tenemos que $g$ diverge en $z=0$, lo que es una contradicción.

Si tenemos que $n>k$,
como $\max\{ |g(z)| : |z|=r \} = r^{n-k}$ tomando límite en $r\rightarrow 0$ deducimos que $g(0)=0$, lo que también es una contradicción.

Luego $k=n$, tenemos $f(z)=z^ng(z) \ \ \forall z\in D(0,1)$. Como $\max\{ |g(z)| : |z|=r \} = r^{n-k}$ deducimos que $\max\{ |g(z)| : |z|=r \} = 1$ $\forall r\in]0,1[$
por el ejercicio $1$ tenemos que $g$ es constante.


%  \subsection{Ejercicio 6} Consecuencia del ejercicio 1

\begin{comment}
	ENUNCIADO
	Si $f\in\mathbb{H}(\mathbb{C})$ y es inyectiva, ¿Qué se puede decir de $f$?
	
	SOLUCIÓN
	Como $f$ es entera tenemos que $f(z) = \sum_{n=0}^{\infty} \frac{f^{(n)}(0)}{n!}z^n \forall z\in\mathbb{C}$
	
	Si $f$ es un polinomio de grado $k$
	si $k\geq 2$ el teorema fundamental del álgebra nos dice que $f$ tiene al menos $2$ ceros (contando multiplicidad), los ceros no pueden ser distintos porque $f$ es inyectiva, por tanto $f$ tiene un cero de orden al menos $2$.
	
	$f$ no es inyectiva en un entorno de un punto donde se anule la derivada, por tanto tenemos una contradicción ya que $f$ es inyectiva.
	
	Si $f$ es entera no polinómica, como consecuencia del teorema de Casorati $f(\mathbb{C}\backslash\overline{D}(0,R))$ es denso en $\mathbb{C}$
	
	Por el teorema de la aplicación abierta sabemos que $f(D(0,r))$
	Entonces hay dos puntos donde la función vale lo mismo, por tanto no puede ser inyectiva.
	
\end{comment}
\end{sol}



\begin{ejer}
	Sea $f\in\mathcal{H}(D(0,1))$ verificando que
	$$ |f(z)| \leq |f(z^2)| \hspace{1cm} \forall z\in D(0,1) $$
	Probar que $f$ es constante.
\end{ejer}



\begin{ejer}
	Mostrar que el teorema fundamental del Álgebra se deduce directamente del principio del módulo mínimo.
\end{ejer}
\begin{sol}
$\lim_{z\rightarrow\infty} P(z) = \infty$
En $\overline{D}(0,R), |P|$ alcanza su mínimo absoluto en $\overline{D}(0,R)$ que vale $$k=\min\{ |P(z)| : z\in\overline{D}(0,R) \}$$
Entonces
$\exists R^+>R$ tal que si $|z|\geq R^+$ entonces $|P(z)| > k$

En $\overline{D}(0,R) \subset \overline{D}(0,R^+)$ $|P|$ alcanza su mínimo absoluto en $z_0\in\overline{D}(0,R^+)$ y sabemos que es menor o igual que $k$.
Como en $C(0,R^+)$ $|P|>k$ entonces $z_0\in D(0,R^+)$

$P\in\mathcal{H}(D(0,R^+))$ y $|P|$ alcanza un mínimo relativo en $z_0$, por el principio del módulo mínimo $P(z_0)=0$.
\end{sol}


\begin{ejer}
	Sea $\Omega$ un dominio y $f\in\mathcal{H}(\Omega)$.
	Probar que, si la función $Re(f)$ tiene un extremo relativo en un punto de $\Omega$, entonces $f$ es constante.
\end{ejer}
\begin{sol}
Tenemos un punto $a\in\Omega$ tal que $Re(f)$ tiene un extremo relativo en $a$.
Sabemos que $\exists R>0$ tal que $D(a,R)\subset\Omega$ y 
$Re(f(z)) \geq Re(f(a))$ (respectivamente $Re(f(z))\leq Re(f(a))$) $\forall z\in D(a,R)$.

Suponemos que $f$ no es constante y por el teorema de la aplicación abierta, $f(a)\in(D(a,R))$, que es abierto.
Sabemos que $\exists r>0$ tal que $D(f(a),r) \subset f(D(a,R))$, lo que es una contradicción con que $Re(f)$ alcance un extremo relativo en $a$.
\end{sol}


\begin{ejer}
	Sea $f:\overline{D}(0,1)\rightarrow\mathbb{C}$ una función continua en $\overline{D}(0,1)$ y holomorfa en $D(0,1)$, tal que $Im(f(z)) = 0$ para todo $z\in\mathbb{T}$.
	Probar que $f$ es constante.
\end{ejer}
\begin{sol}
Como $C(0,1)$ es cerrado y acotado su imagen por $f$ (continua) es un compacto.
Si $Im(f(z))=0 	\forall z\in D(0,1)$, entonces $f$ es constante en $\overline{D}(0,1)$.
En caso contrario $\exists z_0 \in D(0,1)$ tal que $Im(f(z_0)) \not=0$, se continuará suponiendo que $Im(f(z_0))>0$, en caso de ser menor que $0$ se razonaría igual pero usando el mínimo.
$$Im(f):\overline{D}(0,1)\rightarrow \mathbb{R} \hspace{1cm} \text{ es continua en un compacto }$$
Por tanto $\exists w_0\in\overline{D}(0,1)$ tal que $Im(f(w_0)) = \max\{ Im(f(z)) : z\in\overline{D}(0,1) \} > 0$
Como $Im(f(w_0))>0$ tenemos que $w_0\in D(0,1)$ y $Im(f)$ tiene un máximo relativo en $w_0$.
Por el ejercicio $8$ tenemos que $f$ es constante.
\end{sol}


\begin{ejer}
	Sea $\Omega$ un abierto de $\mathbb{C}$ y $f\in\mathcal{H}(\Omega)$ una función inyectiva. 
	Dados $a\in\Omega$ y $r\in\mathbb{R}^+$ tales que $\overline{D}(a,r)\subset\Omega$, calcular, para cada $z\in D(a,r)$, la integral
	$$ \int_{C(a,r)} \frac{wf'(w)}{f(w)-f(z)} dw $$
\end{ejer}
\begin{sol}
Tenemos un abierto $\Omega\in\mathcal{C}$, $f\in\mathcal{H}(\mathbb{C})$ inyectiva.
$\overline{D}(a,r) \subseteq \Omega \forall z_0\in D(a,r)$

Aplicando la fórmula del cambio de variable con 
$$\gamma = C(a,R)\hspace{1cm} \varphi(w) = f(w) \hspace{1cm} g(z) = \frac{f^{-1}(z)}{den}$$
nos queda
$$g(\varphi(w)) = \frac{w}{f(w)-f(z_0)}$$
Por lo tanto
$$\int_{C(a,r)} \frac{wf'(w)}{f(w)-f(z_0)} dw = \int_{f(C(a,r))} \frac{f^{-1}(z)}{z-f(z_0)} dz$$
Por el teorema general de Cauchy ($\frac{1}{2\pi i} \int_{\Gamma}\frac{f(z)}{z-z_0} dz = Ind_{\Gamma} (z_0)f(z_0)$, tema $12$) nos queda
$$\int_{f(C(a,r))} \frac{f^{-1}(z)}{z-f(z_0)} dz = 2\pi i Ind_{f(C(a,r))} (f(z_0)) * f^{-1}(f(z_0)) = 2\pi i Ind_{f(C(a,r))} (f(z_0)) * z_0$$
\end{sol}