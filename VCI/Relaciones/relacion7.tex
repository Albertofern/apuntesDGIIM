
\begin{ejer}
	Sean $a\in\mathbb{C}$ y $r\in\mathbb{R}^+$. Probar que para $z\in\mathbb{C}$ con $|z-a|>r$ se tiene
	$$ \int_{C(a,r)} \frac{dw}{w-z} = 0 $$
\end{ejer}
\begin{sol}
Fijo $z\in\mathbb{C}$ con $|z-a|>r, \exists r'>0 : |z-a|>r'>r$.
Consideramos $\Omega = D(a,r'), f(w) = \frac{1}{w-z}$ , $f\in\mathcal{H}(D(a,r'))$. Sabemos que $D(a,r')$ es convexo (en particular estrellado), por tanto, usando el teorema de Cauchy para dominios estrellados tenemos % y $TLC$
$$\int_{C(a,r)} \frac{dw}{w-z} = 0$$
\end{sol}


\begin{ejer}
	Probar la siguiente versión, más general, de la fórmula de Cauchy:
	
	Sean $a\in\mathbb{C},R\in\mathbb{R}^+$ y $f:\overline{D}(a,R) \rightarrow \mathbb{C}$ una función continua en $\overline{D}(a,R)$ y holomorfa en $D(a,R)$. Se tiene entonces:
	$$ f(z) = \frac{1}{2\pi i} \int_{C(a,R)} \frac{f(w)}{w-z}dw \hspace{1cm} \forall z\in D(a,R) $$
\end{ejer}

\begin{sol}

Fijo $z\in D(a,R)$
Fijo $0<r<R : |z-a|<r<R$, $\overline{D}(a,r) \subset D(a,R)$

Por la fórmula de Cauchy para la circunferencia $f(z) = \frac{1}{2\pi i} \int_{C(a,r)} \frac{f(w)}{w-z} dw$
$$ \left|\frac{1}{2\pi i} \int_{C(0,R)} \frac{f(w)}{w-z} dz - \frac{1}{2\pi i} \int_{C(0,r)} \frac{f(w)}{w-z}dw \right| = \frac{1}{2\pi} \left|\int_{-\pi}^{\pi} \frac{f(a+Re^{it}) iRe^{it} }{ a+Re^{it}-z } dt - \int_{-\pi}^{\pi} \frac{f(a+re^{it}) ire^{it}}{a+re^{it}-z} dt\right|$$
$$\leq \frac{1}{2\pi} \int_{-\pi}^{\pi} \left| \frac{f(a+Re^{it})R}{a+Re^{it}-z} -\frac{f(a+re^{it})r}{a+re^{it}-z} \right| dt$$
Definimos
$$g:[r,R]\times]-\pi, \pi[ \rightarrow \mathbb{C} \hspace{1cm}
g(p,t) = \frac{f(a+pe^{it})p}{a+pe^{it}-z}$$
por lo que el término de la integral queda $|g(R,t)-g(r,t)|$. Vemos que $g$ es uniformemente continua por ser una función continua en un compacto.

Dado $\epsilon >0 \ \exists \delta>0 : si \| (p,t)-(p',t') \| _{\infty}<\delta \implies |g(p,t)-g(p',t')| < \epsilon$,
y lo utilizamos para $(p,t)=(R,t)$ y $(p',t')=(r,t)$. Si $|R-r|<\delta$ entonces $|g(R,t)-g(r,t)| < \epsilon$:
$$\frac{1}{2\pi} \int_{-\pi}^{\pi} \left| \frac{f(a+Re^{it})R}{a+Re^{it}-z} -\frac{f(a+re^{it})r}{a+re^{it}-z} \right| dt \leq \frac{1}{2\pi}\epsilon 2\pi = \epsilon$$
Ahora solamente queda hacer que $r\rightarrow R$.

\end{sol}


\begin{ejer}
	Dados $a\in\mathbb{C}, r\in\mathbb{R}^+$ y $b,c\in\mathbb{C}\backslash C(a,r)^{\ast}$, calcular todos los posibles valores de la integral
	$$ \int_{C(a,r)} \frac{dz}{(z-b)(z-c)} $$
	dependiendo de la posición relativa de $b,c$ respecto a la circunferencia $C(a,r)^{\ast}$.
\end{ejer}

\begin{sol}
	Primero dividimos la integral en dos partes
$$\frac{1}{(z-b)(z-c)} = \frac{1}{b-c} \left[\frac{1}{z-b}-\frac{1}{z-c}\right]$$

Vemos primero el caso $b\not = c$, tendríamos tres subcasos:
\begin{itemize}
	\item b interior, c exterior, la segunda parte es $0$ por el ejercicio $1$
	$$\int_{C(a,r)} \frac{dz}{(z-b)(z-c)} = \frac{1}{b-c} \left[ \int_{C(a,r)} \frac{dz}{z-b} - \int_{C(a,r)} \frac{dz}{z-c} \right] = \frac{2\pi i}{b-c} (1 - 0)$$
	
	\item b y c interiores 
	$$\int_{C(a,r)} \frac{dz}{(z-b)(z-c)} =  \frac{1}{b-c}[2\pi i - 2\pi i] = 0$$
	
	\item b y c exteriores, por el ejercicio 1 ambas partes son $0$
	$$\int_{C(a,r)} \frac{dz}{(z-b)(z-c)} = 0-0 = 0$$
\end{itemize}

Ahora vemos el caso $b=c$, se tiene primitiva holomorfa en $\mathbb{C}^{\ast}, \frac{-1}{z-b}$
$$ \int_{C(a,r)} \frac{dz}{(z-b)^2} = 0 $$

\end{sol}


\begin{ejer}
	Calcular las siguientes integrales:
	\begin{enumerate}[label=(\alph*)]
		\item $\int_{C(0,r)} \frac{z+1}{z(z^2+4)}dz \hspace{1cm} (r\in\mathbb{R}^+, r\not =2)$
		\item $\int_{C(0,1)} \frac{\cos(z)dz}{(a^2+1)z-a(z^2+1)} \hspace{1cm} (a\in\mathbb{C}, |a|\not =1)$
	\end{enumerate}
\end{ejer}

\begin{sol}

\textbf{a)}
Tenemos $\int_{C(0,r)} \frac{z+1}{z(z^2+4)} dz$, donde $r\not = 2$, $z\not = \pm 2i$
$$\frac{z+1}{z(z^2+4)} = \frac{A}{z} + \frac{B}{z-2i} + \frac{C}{z+2i}$$
Nos queda $A = 1/4, B=1/8-i/4, C = -1/8+i/4$, por tanto
$$\int_{C(0,r)} \frac{z+1}{z(z^2+4)} dz = \frac{1}{4} \int_{C(0,r)} \frac{dz}{z} 
+ \left(\frac{1}{8}-\frac{i}{4}\right)\int_{C(0,r)} \frac{dz}{z-2i} 
+
\left(-\frac{1}{8}+\frac{i}{4}\right) \int_{C(0,r)} \frac{dz}{z+2i}$$

Tenemos para las dos últimas integrales y $r\not = 2$ dos casos, si $r<2$ ambas integrales son $0$, si $r>2$ ambas integrales valen lo mismo ($2\pi i *1$) y se anulan entre sí. Por tanto
$$\int_{C(0,r)} \frac{z+1}{z(z^2+4)} dz
=
\frac{1}{4}\int_{C(0,r)} \frac{dz}{z} = \frac{2\pi i}{4} = \frac{\pi}{2}$$

\end{sol}


\begin{ejer}
	Dados $a,b\in\mathbb{C}$ con $a\not =b$, sea $R\in\mathbb{R}^+$ tal que $R>\max \{|a|,|b|\}$. Probar que, si $f$ es una función entera, se tiene:
	$$ \int_{C(0,R)} \frac{f(z)dz}{(z-a)(z-b)} = 2\pi i\frac{f(b)-f(a)}{b-a} $$
	Deducir que toda función entera y acotada es constante.
\end{ejer}
\begin{sol}

Como $R>\max\{ |a|, |b| \}$
$$\int_{C(0,R)} \frac{f(z)}{(z-a)(z-b)}dz = \frac{1}{b-a} \left[ \int_{C(0,R)} \frac{f(z)}{z-b}dz - \int_{C(0,R)} \frac{f(z)}{z-a}dz \right]$$

Por el ejercicio $3$ tenemos
$$\frac{f(z)}{(z-a)(z-b)} = \frac{f(z)}{b-a} \left[ \frac{1}{z-b} - \frac{1}{z-a} \right]$$

Por la fórmula de Cauchy para la circunferencia
$$\frac{1}{b-a} \left[ \int_{C(0,R)} \frac{f(z)}{z-b}dz - \int_{C(0,R)} \frac{f(z)}{z-a}dz \right] = 2\pi i\frac{f(b) - f(a)}{b-a}$$

\textbf{Consecuencia}

Si $\exists M>0 : |f(z)|\leq M\ \ \forall z \in \mathbb{C}$
$$\frac{2\pi |f(b)-f(a)|}{|b-a|} = \frac{|2\pi i(f(b)-f(a))|}{|b-a|} 
\leq \int_{C(0,R)} \frac{|f(z)|}{|z-a||z-b|} dz \leq \frac{M}{(R-|a|)(R-|b|)} 2\pi R$$
que tiende a $0$ cuando $R\rightarrow \infty$ ya que $|z-a|\geq R-|a|$ y $|z-b|\geq R - |b|$
\end{sol}