\subsection{Ejercicio 1}


\textbf{Solución}

Fijo $z\in\mathbf{C}$ con $|z-a|>r, \exists r'>0 : |z-a|>r'>r$
Consideramos $\Omega = D(a,r'), f(w) = \frac{1}{w-z}$ , $f\in\mathcal{H}(D(a,r'))$
$D(a,r')$ es convexo, en particular estrellado, por tanto, usando $TLC$ y $TCDE$ $\int_{C(a,r)} \frac{dw}{w-z} = 0$

\subsection{Ejercicio 2}


\textbf{Solución}

Fijo $z\in D(a,R)$
Fijo $0<r<R : |z-a|<r<R$, $\overline{D}(a,r) \subset D(a,R)$

Por la fórmula de Cauchy para la circunferencia $f(z) = \frac{1}{2\pi i} \int_{C(a,r)} \frac{f(w)}{w-z} dw$

$ \left|\frac{1}{2\pi i} \int_{C(0,R)} \frac{f(w)}{w-z} dz - \frac{1}{2\pi i} \int_{C(0,r)} \frac{f(w)}{w-z}dw \right| = \frac{1}{2\pi} |\int_{-\pi}^{\pi} \frac{f(a+Re^{it}) iRe^{it} }{ a+Re^{it}-z } dt - \int_{-\pi}^{\pi} \frac{f(a+re^{it}) ire^{it}}{a+re^{it}-z} dt|$

$\leq \frac{1}{2\pi} \int_{-\pi}^{\pi} \left| \frac{f(a+Re^{it})R}{a+Re^{it}-z} -\frac{f(a+re^{it})r}{a+re^{it}-z} \right| dt$

$g:[r,R]x]-\pi, \pi[ \rightarrow \mathbb{C}$,
$g(p,t) = \frac{f(a+pe^{it})p}{a+pe^{it}-z}$
por lo que el término de la integral queda $|g(R,t)-g(r,t)|$

$g$ es una función continua en un compacto, por lo tanto es uniformemente continua.

Dado $\epsilon >0 \exists \delta : si \| (p,t)-(p',t') \| _{\infty}<\delta \implies |g(p,t)-g(p',t')| < \epsilon$,
y lo utilizamos para $(p,t)=(R,t)$ y $(p',t')=r,t$, si $|R-r|<\delta$ entonces $|g(R,t)-g(r,t)| < \epsilon$

Por lo tanto

$\frac{1}{2\pi} \int_{-\pi}^{\pi} \left| \frac{f(a+Re^{it})R}{a+Re^{it}-z} -\frac{f(a+re^{it})r}{a+re^{it}-z} \right| dt \leq \frac{1}{2\pi}\epsilon 2\pi$ en el caso de que $|R-r|<\delta$


\subsection{Ejercicio 3}

\textbf{Solución}

$\frac{1}{(z-b)(z-c)} = \frac{1}{b-c} [\frac{1}{z-b}-\frac{1}{z-c}]$

Vemos primero el caso $b\not = c$

Tenemos tres subcasos
\begin{itemize}
	\item b interior, c exterior $\int_{C(a,r)} f(z)dz = \frac{1}{b-c} [ \int_{C(a,r)} \frac{dz}{z-b} - \int_{C(a,r)} \frac{dz}{z-c} ] = \frac{1}{b-c} (2\pi i (b interior) - 0 (ej1))$
	
	\item b y c interiores $\int_{C(a,r)} f(z)dz =  \frac{1}{b-c}[2\pi i - 2\pi i] = 0$
	
	\item b y c exteriores, por el ejercicio 1 ambas partes son $0$: $\int_{C(a,r)} f(z)dz = 0-0 = 0$
\end{itemize}

Ahora vemos el caso $b=c$
$f(z) = \frac{1}{(z-b)^2}$ tiene primitiva holomorfa en $\mathbb{C}^{\ast}, \frac{-1}{z-b} \implies \int_{C(a,r)} f(z)dz = 0 $


\subsection{Ejercicio 4}

\textbf{Solución}
\subsubsection{a}

$\int_{C(0,r)} \frac{z+1}{z(z^2+4)} dz$, donde $r\not = 2$, $z\not = \pm 2i$

$\frac{z+1}{z(z^2+4)} = \frac{A}{z} + \frac{B}{z-2i} + \frac{C}{z+2i}$, donde $A = 1/4, B=1/8-i/4, C = -1/8-i/4$

Por tanto

$\int_{C(0,r)} \frac{z+1}{z(z^2+4)} dz = \frac{1}{4} \int_{C(0,r)} \frac{dz}{z} + (\frac{1}{8}-\frac{i}{4})\int_{C(0,r)} \frac{1}{z-2i}dz + (-\frac{1}{8}+\frac{i}{4}) \int_{C(0,r)} \frac{1}{z+2i}$

para $r\not = 2$ las dos últimas integrales se anulan, y por tanto $\frac{1}{4}\int_{C(0,r)} \frac{dz}{z} = \frac{2\pi i}{4} = \frac{\pi}{2}$



\subsection{Ejercicio 5}

\textbf{Solución}

$R>\max\{ |a|, |b| \}$

$\int_{C(0,R)} \frac{f(z)}{(z-a)(z-b)}dz = \frac{1}{b-a} \left[ \int_{C(0,R)} \frac{f(z)}{z-b}dz - \int_{C(0,R)} \frac{f(z)}{z-a}dz \right]$

Por el ejercicio 3 tenemos

$\frac{f(z)}{(z-a)(z-b)} = \frac{f(z)}{b-a} \left[ \frac{1}{z-b} - \frac{1}{z-a} \right]$

Por la fórmula de Cauchy para la circunferencia

$= \frac{1}{b-a} \left[ \int_{C(0,R)} \frac{f(z)}{z-b}dz - \int_{C(0,R)} \frac{f(z)}{z-a}dz \right] = 2\pi i f(b) - 2\pi if(a)$

\textbf{Consecuencia}

Si $\exists M>0 : |f(z)|\leq M \forall z \in \mathbb{C}$

$\frac{2\pi |f(b)-f(a)|}{|b-a|} = \frac{|2\pi i(f(b)-f(a))|}{|b-a|} = \left| \int_{C(0,R)} \frac{f(z)}{(z-a)(z-b)} dz \right|$
$\leq \int_{C(0,R)} \frac{|f(z)|}{|z-a||z-b|} dz \leq \frac{M}{(R-|a|)(R-|b|)} 2\pi R$ que tiende a $0$ cuando $R\rightarrow \infty$ ya que $|z-a|\geq R-|a|$ y $|z-b|\geq R - |b|$