\begin{ejer}
	Probar que el conjunto de matrices
	$$ M = \left\{ 
	\left( \begin{array}{cc}
	 a & b \\
	-b & a \end{array} \right)
	: \ a, b\in \mathbb{R}\right\} $$
	con las operaciones de suma y producto de matrices, es un cuerpo isomorfo a $\mathbb{C}$.
\end{ejer}


\begin{ejer}
	Calcular la parte real, la parte imaginaria y el módulo de los números complejos
	$$ \frac{i-\sqrt{3}}{1+i}\hspace{1cm}\text{y}\hspace{1cm} \frac{1}{i\sqrt{3}-1} $$
\end{ejer}



\begin{ejer}
	Sea $U=\{ z\in\mathbb{C} : |z|<1 \}$.
	Fijado $a \in U$, se considera la función $f: U\rightarrow\mathbb{C}$ dada por
	$$ f(z) = \frac{z-a}{1-\overline{a}z}\hspace{1cm} \forall z\in U $$
	Probar que $f$ es una biyección de U sobre sí mismo y calcular su inversa.
\end{ejer}

\begin{sol}

$$|f(z)| < 1 \hspace{1cm} \forall z\in\mathbb{C} \text{ tal que } |z| <1$$
$$f^{-1}(z)= \frac{z+a}{1+\overline{a}z}$$
Si $|z|=1$, entonces $$|f(z)| = 			
	\left|\frac{z-a}{1-\overline{a}z}\right|$$
multiplicando en esta expresión por $\overline{z}$
$$ \frac{z-a}{1-\overline{a}z} \overline{z}  =  \frac{1-a\overline{z}}{1-\overline{a}z}$$
Tenemos entonces que $f$ es holomorfa en el disco $D(0,1)$, lleva la frontera en la frontera.
\end{sol}


\begin{ejer}
	Dados $z_1, z_2, \ldots , z_n \in C^{\ast}$, encontrar una condición necesaria y suficiente para que se verifique la siguiente igualdad:
	$$ \left|\sum_{k=1}^n z_k\right| = \sum_{k=1}^n |z_k|$$
\end{ejer}


\begin{sol}



Por inducción, todos los números complejos deben tener el mismo argumento, son vectores linealmente dependientes sin que se invierta el signo de ninguno de ellos.

$\exists\lambda_1,...,\lambda_n>0 : \lambda_1 z_1 = \lambda_2 z_2 = ... = \lambda_n z_n$

para probar que es necesaria no hace falta hacer inducción
$ |\sum_{k=1}^n z_n| = |\sum_{k=1}^n \frac{\lambda_1}{\lambda_k} z_1| 
=
|z_1|\sum_{k=1}^n \frac{\lambda_1}{\lambda_k} 
=
|z_1|\sum_{k=1}^n \frac{|z_k|}{|z_1|} 
=
\sum_{k=1}^n |z_k|$

$n=2 \implies |z_1+z_2| = |z_1| + |z_2| \longleftrightarrow z_2 = \lambda z_1$ con $\lambda \in \mathbb{R}$

Sea cierto para $n\in\mathbb{N}$
$|\sum_{k=1}^{n+1} z_k| 
=
|z_{n+1} + \sum_{k=1}^n z_k|
=
|z_{n+1}| + |\sum_{k=1}^n z_k|$

Hemos usado en este último paso que
$|z_{n+1}|+|\sum_{k=1}^n| \geq |\sum_{k=1}^{n+1} z_k|=\sum_{k=1}^{n+1} |z_k| = |z_{n+1}| + \sum_{k=1}^n |z_k|$
donde la otra desigualdad entre los extremos se sabe de la desigualdad triangular.
%Ahora usamos la hipótesis de inducción
$|z_{n+1}| + |\sum_{k=1}^n z_k| 
=
|z_{n+1}| + \sum_{k=1}^n |z_k|$
Por la hipótesis de inducción para $k=2$, tenemos que 
$\exists \lambda >0 : z_{n+1} = \lambda \sum_{k=1}^n z_k$

Nos queda por hacer lo análogo con el resto de números complejos $z_k,k=1...n$
Por la hipótesis de inducción para $k=n, \exists \mu_1,...\mu_n : \mu_1 z_1 = \mu_2 z_2=...= \mu_n z_n$,
como $z_{n+1} = \lambda \sum_{k=1}^n z_k 
=
\lambda \sum_{k=1}^n \frac{\mu_1}{\mu_k} z_1$

$$ |z_1+z_2| = |z_1|+|z_2| 
\Longleftrightarrow
|z_1+z_2|^2 = (|z_1|+|z_2|)^2 
=
|z_1|^2 + 2|z_1||z_2|+|z_2|^2$$
 entonces tenemos
$$z_1 \overline{z_2}=\lambda>0 \text{ y que } z_1 = \frac{\lambda}{|z_2|^2 }z_2 
$$

\end{sol}

\begin{ejer}
	Describir geométricamente los subconjuntos del plano dados por
	$$ A=\{ z\in\mathbb{C} : |z+i|=2|z-i| \} \text{ y } B=\{ z\in\mathbb{C} : |z-i| + |z+i| = 4 \} $$
\end{ejer}

\begin{sol}

\

Veamos el conjunto A, notamos $z=(a,b)$,
$$ \sqrt{a^2+(b+1)^2} = 2\sqrt{a^2+(b-1)^2}$$
Entonces $$a^2+b^2+1+2 b = a^2+(b+1)^2 = 4(a^2+(b-1)^2) = 4a^2+4b^2+4-8b$$
Luego
$$3a^2+3b^2-10b+3=0 \implies a^2+b^2-\frac{10}{3}b+1=0$$
Sumamos y restamos $\frac{25}{9}$
$$a^2+(b-\frac{5}{3})^2-\frac{25}{9}+1 \implies a^2+(b-\frac{5}{3})^2=\frac{16}{9}$$
Por lo tanto $A$ es la circunferencia con $c=(0, \frac{5}{3})$ y radio $r=\frac{4}{3}$

\

Veamos el B, elevando dos veces al cuadrado tenemos como resultado una elipse
$$ \frac{a^2}{\sqrt{3}^2} + \frac{b^2}{2^2} = 1 $$
\end{sol}



\begin{ejer}
	Probar que $arg(z) = 2\arctan\left( \frac{Im(z)}{Re(z) + |z|} \right)$ para todo  $z\in\mathbb{C}^{\ast}\backslash\mathbb{R}^{-}$.
\end{ejer}

\begin{sol}

Usamos la fórmula del ángulo doble
$$ \cos(2\alpha) = \cos^2(\alpha)-\sin^2(\alpha) $$
$$ \sin(2\alpha) = 2\sin(\alpha)\cos(\alpha) $$

$\phi = \arctan(\frac{Im z}{Re z + |z|}) \implies \phi = 2\alpha = \arctan(\frac{Im z}{Re z + |z|}) $

\textbf{Pista para otra forma de hacerlo}
$$ \phi = 2\arctan ( \frac{Im z}{Re z + |z|} ) 
|z|(\cos(\phi)+i\sin(\phi)) =? z $$
Para ello usamos, si $t=\theta/2$
$$ \cos(\theta) = \frac{1-\tan^2(\theta/2)}{1+\tan^2(\theta/2)}
\sin(\theta) = \frac{2\tan(\theta/2)}{1+\tan^2(\theta/2)} $$
\end{sol}


\begin{ejer}
	Probar que, si $z=x+iy\in\mathbb{C}^{\ast}$, con $x,y\in\mathbb{R}$, se tiene
	
	$$
	arg(z) = \left\{ 
		\begin{array}{lcc}
			\arctan(y/x) &   si  & x>0 \\
			\arctan(y/x) + \pi & si & x<0, y>0 \\
			\arctan(y/x)-\pi & si & x<0,y<0 \\
			\pi/2 & si & x=0,y>0 \\
			-\pi/2 & si & x=0,y<0 
		\end{array}
	\right.
	$$
\end{ejer}


\begin{ejer}
	Probar las \textit{fórmulas de De Moivre}:
	$$ \cos(n\theta) + i\sin(n\theta) = (\cos(\theta)+i\sin(\theta))^n \hspace{1cm}\forall\theta\in\mathbb{R}, \forall n\in\mathbb{N} $$
\end{ejer}


\begin{sol}
Haremos la prueba por inducción, para el caso inicial $n=1$ es trivial, suponemos cierto para un $n\in\mathbb{N}$ genérico.
Probamos que en ese caso es cierto para $n+1$.

$$ 
\cos((n+1)\theta)+i\sin((n+1)\theta) 
=
\cos(n\theta+\theta)+i\sin(n\theta+\theta)
$$
$$ =
\cos(n\theta) \cos(\theta)- \sin(n\theta)\sin(\theta) +i\cos(n\theta)\sin(\theta) + i\sin(n\theta)\cos(\theta) $$
$$ =
\cos(n\theta)(cos(\theta)+i\sin(\theta)) + i\sin(n\theta)(cos(\theta)+i\sin(\theta))
$$
$$=
(\cos(theta)+i\sin(\theta)) (\cos(n\theta)+i\sin(n\theta))
=
(cos(\theta)+i\sin(theta))^{n+1}
$$
La última igualdad la tenemos por la hipótesis de inducción.
\end{sol}



\begin{ejer}
	Calcular las partes real e imaginaria del número complejo 
	$\left( \frac{1+i\sqrt{3}}{2} \right)^8$
\end{ejer}


\begin{ejer}
	Probar que, para todo $x\in\mathbb{R}$, se tiene
	\begin{enumerate}[label=(\alph*)]
		\item $\sin\left( \frac{x}{2} \right) \sum_{k=0}^{n} \cos(kx) = \cos\left( \frac{nx}{2} \right) \sin\left( \frac{(n+1)x}{2} \right)$
		\item $\sin\left( \frac{x}{2} \right) \sum_{k=1}^n \sin(kx) = \sin\left( \frac{nx}{2} \right)\sin\left( \frac{(n+1)x}{2} \right)$
	\end{enumerate}
\end{ejer}
\begin{sol}


\textbf{Pista}
Probar las dos simultáneamente y usar la fórmula de Moivre
$ \sum_{k=0}^n ( cos(x)+sin(x) )^k $



$$\sin(x/2) \sum_{k=0}^{n} \cos(kx) + i\sin(x/2) \sum_{k=0}^{n} \sin(kx)$$
Sacamos factor común 
$$ \sin(x/2)( \sum_{k=0}^{n} \cos(kx)+i\sin(kx) ) $$
Usamos la fórmula de Moivre
$$ \sin(x/2) \sum_{k=0}^{n} (\cos(x)+i\sin(x)^k $$
Usamos que $x\not\in 2\pi\mathbb{Z}$
$$ \sin(x/2) \frac{1-(\cos(x)+i\sin(x)^{n+1}}{1-(\cos(x)+i\sin(x)} $$
$$= \sin(x/2) \frac{ 1- (\cos((n+1)x)+i\sin((n+1)x)) }{1 - \cos(x) - i\sin(x)} $$
Usamos que $2\sin^2(x/2) = 1-\cos(x)$ y que $\sin(x) = \sin(x/2)\cos(x/2)$,
$$ 
\sin(x/2) \frac{ 1- (\cos((n+1)x)+i\sin((n+1)x)) }{ 2\sin^2(x/2)-2i\sin(x/2)\cos(x/2) }
=
\frac{ 1- (\cos((n+1)x)+i\sin((n+1)x)) }{ 2\sin(x/2)-2i\cos(x/2) }
 $$
$$
\frac{ 2\sin^2 (\frac{n+1}{x}x)-i2\sin(\frac{n+1}{2}x)\cos(\frac{n+1}{2}x) }{ 2\sin(x/2) -2i\cos(x/2) }
$$
Multiplicamos numerador y denominador por el conjugado del denominador
$$
\sin(\frac{n+1}{2}x)( \sin(\frac{n+1}{2}x)-i\cos(\frac{n+1}{2}x) )( \sin(x/2)+i\cos(x/2) )
$$
$$ 
=  \sin(\frac{n+1}{2}x) ( \cos(\frac{n+1}{2}x)\cos(x/2) + \sin(\frac{n+1}{2}x)\sin(x/2)  + i( \sin(\frac{n+1}{2}x)\cos(x/2) - \cos(\frac{n+1}{2}x)\sin(x/2) ) )
$$
$$ = \sin(\frac{n+1}{2}x) (\cos(\frac{nx}{2}) + i\sin(\frac{nx}{2}))
$$
\end{sol}
