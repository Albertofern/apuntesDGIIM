\begin{ejer}
	Sea $\{f_n\}$ una sucesión de funciones continuas en un abierto $\Omega$ del plano, y sea $f:\Omega\rightarrow\mathbb{C}$ otra función continua. Probar que las siguientes afirmaciones son equivalentes:
	\begin{itemize}
		\item $\{f_n\}$ converge uniformemente a $f$ en cada subconjunto compacto de $\Omega$.
		\item Para toda sucesión $\{z_n\}$ de puntos de $\Omega$ que converja a un punto $z\in\Omega$ se tiene que $\{f_n(z_n)\}\rightarrow f(z)$.
	\end{itemize}
\end{ejer}

\begin{ejer}
	Sea $\{f_n\}$ la sucesión de funciones enteras definida por $f_n(z) = z\exp(-n^2z^2/2)$ para cualesquiera $z\in\mathbb{C}$ y $n\in\mathbb{N}$. Probar que $\{f_n\}$ converge uniformemente en $\mathbb{R}$ pero no converge uniformemente en ningún entorno del origen.
\end{ejer}

\begin{ejer}
	Probar que la serie de funciones $\sum_{n\geq 1} f(z^n)$ converge en $D(0,1)$ y que su suma es una función holomorfa en $D(0,1)$.
\end{ejer}

\begin{ejer}
	Probar que la sucesión de funciones enteras definida por
	$$ f_n(z) = \frac{1}{n}\sin(nz) \hspace{1cm} \forall z\in\mathbb{C}, \ \ \forall n\in\mathbb{N} $$
	converge uniformemente en $\mathbb{R}$ pero no converge uniformemente en ningún subconjunto de $\mathbb{C}$ que tenga interior no vacío.
\end{ejer}
Sea $U$ con interior no vacío, entonces $\exists z_o\in U$ con $|Im(z_o)| \geq r > 0$

$\frac{1}{n}|\sin(nz_O)| = \frac{1}{n}\left| \frac{e^{inz_0}-e^{-inz_0}}{2i} \right|= \frac{1}{2n} \left| e^{inz_0} - e^{inz_0} \right| $
entonces si $Im(z_0)\geq r$
$\frac{1}{2n} \left| e^{inz_0} - e^{inz_0} \right|  \geq \frac{1}{2n} \left( e^{nIm(z_0)} -e^{-nIm(z_0)} \right)  $
que tiende a infinito cuando $n\rightarrow\infty$


\begin{ejer}
	Probar que la serie $\sum_{n\geq 0} \frac{\sin(nz)}{3^n}$ converge en la banda $\Omega = \{ z\in\mathbb{C} : |Im(z)|<\log(3) \}$ y que su suma es una función $f\in\mathcal{H}(\Omega)$. Calcular $f'(0)$.
\end{ejer}
$\frac{\sin(nz)}{3^n} = \frac{1}{2i} \left( \frac{e^{inz}}{3^n} - \frac{e^{-inz}}{3^n} \right)$

$\left| \frac{e^{inz}}{3^n} \right| = \frac{|e^{inRe(z)-nIm(z)}|}{3^n} = \frac{e^{nIm(z)}}{3^n}$
tenemos que hacer que sea menor que $\delta^n$ con $0<\delta<1$, lo que pasa solamente cuando $-Im(z) < \log(3)+\log(\delta)$


\begin{ejer}
	Sea $\Phi [0,1]\rightarrow\mathbb{C}$ una función continua. Probar que definiendo
	$$ f(z) = \int_{0}^{1} \Phi(t)e^{itz}dt \hspace{1cm} \forall z\in\mathbb{C} $$
	se obtiene una función entera y calcular el desarrollo en serie de Taylor de $f$ centrado en el origen.
\end{ejer}
Es entera por el teorema de holomorfía de integrales dependientes de un parámetro.




\begin{ejer}
	Para cada $n\in\mathbb{N}$, se considera la función $f_n:\mathbb{C}\rightarrow\mathbb{C}$ definida por
	$$ f_n(z) = \int_{0}^{n} \sqrt{t}e^{-tz} \hspace{1cm} \forall z\in\mathbb{C} $$
	\begin{enumerate}[label=\alph*)]
		\item Probar que $f_n$ es una función entera y calcular su desarrollo en serie de Taylor centrado en el origen.
		\item Estudiar la convergencia de la sucesión $\{f_n\}$ en $\Omega = \{ z\in\mathbb{C} : Re(z)>0 \}$.
		\item Deducir que $f\in\mathcal{H}(\Omega)$, donde $f(z) = \int_{0}^{+\infty} \sqrt{t}e^{-tz}dt$ para todo $z\in\Omega$.
	\end{enumerate}
\end{ejer}

\textbf{a)}

$f_n(z) = \int_0^n \sqrt{t}e^{-tz}dt$

Usaremos el teorema de holomorfía de integrales dependientes de un parámetro.

Definimos
$\gamma_n :[0,n] \rightarrow \mathbb{C}$, $\gamma_n(t)=t$, $\gamma'(t) = 1$
Tenemos que buscar una función $\Phi_n : \gamma_n^{\ast}\times\mathbb{C} \rightarrow \mathbb{C}$, $\Phi_n(w,z) = \sqrt{w}e^{-wz}$
Tenemos que
\begin{itemize}
	\item $\Phi_n$ es continua
	\item $\Phi_n(w,\dot)$ es entera $\forall w\in\gamma_n^{\ast}$
\end{itemize}

por el teorema de holomorfía
$f_n \in\mathbb{H}(\mathbb{C})$ y $f_n'(z) = \int_{\gamma_n} \frac{d}{dz} (\Phi_n(w,z)) dw$

entonces

$\int_{\gamma_n}(\Phi_n(w,z)) dw = \int_0^n \sqrt{t}e^{-tz} dt$


\textbf{b)}

Sea $n\in\mathbb{N}$,

$|f_n(z) - f(z)| = |\int_{0}^{n} \sqrt{t} e^{-tz}dt - \int_{0}^{\infty} \sqrt{t}e^{-tz}dt| = |\int_{n}^{\int_n^{\infty}} \sqrt{t} e^{-tz}dt|$

Usamos que 
$|e^{-tz}| = |e^{-tRe(z)-iIm(z)}| = e^{-tRe(z)}$, de esta forma


$|\int_{n}^{\int_n^{\infty}} \sqrt{t} e^{-tz}dt| \leq \int_{n}^{\infty} \sqrt{t}e^{-tRe(z)} dt \leq \int_{n}^{\infty} \sqrt{t}e^{-tk}dt $


Y $\forall k\in \{ z\in\mathbb{C} : Re(z)\geq k > 0 \}$ tenemos
$\int_{n}^{\infty} \sqrt{t}e^{-tk}dt \leq \int_{n}^{\infty} \sqrt{t}e^{-tk} dt \leq \int_{n}^{\infty} te^{-tk}$

