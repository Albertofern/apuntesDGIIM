%%%%%%%%%%%%%%%%%%%%%%%%%%%%%%%%%%%%%%%%%%%%%%%%%%%%%%%%%%%%%%%%
%
% Ejercicios de la asignatura Álgebra I.
% Doble Grado de Informática y Matemáticas.
% Universidad de Granada.
% Curso 2016/17.
% 
% 
% Colaboradores:
% Javier Sáez (@fjsaezm)
% Daniel Pozo (@danipozodg)
% Pedro Bonilla (@pedrobn23)
% Guillermo Galindo
% Antonio Coín (@antcc)
%
% Agradecimientos:
% Andrés Herrera (@andreshp) y Mario Román (@M42) por
% las plantillas base.
%
% Sitio original:
% https://github.com/libreim/apuntesDGIIM/
%
% Licencia:
% CC BY 4.0 (https://creativecommons.org/licenses/by/4.0/)
%
%%%%%%%%%%%%%%%%%%%%%%%%%%%%%%%%%%%%%%%%%%%%%%%%%%%%%%%%%%%%%%%


%------------------------------------------------------------------------------
%   ACKNOWLEDGMENTS
%------------------------------------------------------------------------------

%%%%%%%%%%%%%%%%%%%%%%%%%%%%%%%%%%%%%%%%%%%%%%%%%%%%%%%%%%%%%%%%%%%%%%%%
% Plantilla básica de Latex en Español.
%
% Autor: Andrés Herrera Poyatos (https://github.com/andreshp) 
%
% Es una plantilla básica para redactar documentos. Utiliza el paquete  fancyhdr para darle un
% estilo moderno pero serio.
%
% La plantilla se encuentra adaptada al español.
%
%%%%%%%%%%%%%%%%%%%%%%%%%%%%%%%%%%%%%%%%%%%%%%%%%%%%%%%%%%%%%%%%%%%%%%%%%

%%%
% Plantilla de Trabajo
% Modificación de una plantilla de Latex de Frits Wenneker para adaptarla 
% al castellano y a las necesidades de escribir informática y matemáticas.
%
% Editada por: Mario Román
%
% License:
% CC BY-NC-SA 3.0 (http://creativecommons.org/licenses/by-nc-sa/3.0/)
%%%

%%%%%%%%%%%%%%%%%%%%%%%%%%%%%%%%%%%%%%%%
% Short Sectioned Assignment
% LaTeX Template
% Version 1.0 (5/5/12)
%
% This template has been downloaded from:
% http://www.LaTeXTemplates.com
%
% Original author:
% Frits Wenneker (http://www.howtotex.com)
%
% License:
% CC BY-NC-SA 3.0 (http://creativecommons.org/licenses/by-nc-sa/3.0/)
%
%%%%%%%%%%%%%%%%%%%%%%%%%%%%%%%%%%%%%%%%%


% Tipo de documento y opciones.
\documentclass[11pt, a4paper, titlepage]{article}


%---------------------------------------------------------------------------
%   PAQUETES
%---------------------------------------------------------------------------

% Idioma y codificación para Español.
\usepackage[utf8]{inputenc}
\usepackage[spanish, es-tabla, es-lcroman, es-noquoting]{babel}
\selectlanguage{spanish} 
%\usepackage[T1]{fontenc}

% Fuente utilizada.
\usepackage{courier}    % Fuente Courier.
\usepackage{microtype}  % Mejora la letra final de cara al lector.

% Diseño de página.
\usepackage{fancyhdr}   % Utilizado para hacer títulos propios.
\usepackage{lastpage}   % Referencia a la última página.
\usepackage{extramarks} % Marcas extras. Utilizado en pie de página y cabecera.
\usepackage[parfill]{parskip}    % Crea una nueva línea entre párrafos.
\usepackage{geometry}            % Geometría de las páginas.

% Símbolos y matemáticas.
\usepackage{amssymb, amsmath, amsthm, amsfonts, amscd}
\usepackage{upgreek}

% Otros.
\usepackage{enumitem}   % Listas mejoradas.
\usepackage[hidelinks]{hyperref}


%---------------------------------------------------------------------------
%   OPCIONES PERSONALIZADAS
%---------------------------------------------------------------------------

% Redefinir letra griega épsilon.
\let\epsilon\upvarepsilon

% Formato de texto.
\linespread{1.1}            % Espaciado entre líneas.
\setlength\parindent{0pt}   % No indentar el texto por defecto.
\setlist{leftmargin=.5in}   % Indentación para las listas.

% Estilo de página.
\pagestyle{fancy}
\fancyhf{}
\geometry{left=3cm,right=3cm,top=3cm,bottom=3cm,headheight=1cm,headsep=0.5cm}   % Márgenes y cabecera.

% Redefinir entorno de demostración (reducir espacio superior)
\makeatletter
\renewenvironment{proof}[1][\proofname] {\vspace{-15pt}\par\pushQED{\qed}\normalfont\topsep6\p@\@plus6\p@\relax\trivlist\item[\hskip\labelsep\it#1\@addpunct{.}]\ignorespaces}{\popQED\endtrivlist\@endpefalse}
\makeatother


%---------------------------------------------------------------------------
%   COMANDOS PERSONALIZADOS
%---------------------------------------------------------------------------

% Números enteros: \ent
\providecommand{\ent}{\mathbb{Z}}

% Números racionales: \rac
\providecommand{\rac}{\mathbb{Q}}

% Números naturales: \nat
\providecommand{\nat}{\mathbb{N}}


% Valor absoluto: \abs{}
\providecommand{\abs}[1]{\lvert#1\rvert}    

% Fracción grande: \ddfrac{}{}
\newcommand\ddfrac[2]{\frac{\displaystyle #1}{\displaystyle #2}}

% Texto en negrita en modo matemática: \bm{}
\newcommand{\bm}[1]{\boldsymbol{#1}}

% Línea horizontal.
\newcommand{\horrule}[1]{\rule{\linewidth}{#1}}


%---------------------------------------------------------------------------
%   CABECERA Y PIE DE PÁGINA
%---------------------------------------------------------------------------

% Cabecera del documento.
\renewcommand\headrule{
	\begin{minipage}{1\textwidth}
		\hrule width \hsize 
	\end{minipage}
}

% Texto de la cabecera.
\lhead{\subject}  % Izquierda.
\chead{}            % Centro.
\rhead{\docauthor}    % Derecha.

% Pie de página del documento.
\renewcommand\footrule{                                 
	\begin{minipage}{1\textwidth}
		\hrule width \hsize   
	\end{minipage}\par
}

% Texto del pie de página.
\lfoot{}                                                 % Izquierda
\cfoot{}                                                 % Centro.
\rfoot{Página\ \thepage\ de\ \protect\pageref{LastPage}} % Derecha.


%---------------------------------------------------------------------------
%   ENTORNOS PARA MATEMÁTICAS
%---------------------------------------------------------------------------

% Nuevo estilo para definiciones.
\newtheoremstyle{definition-style} % Nombre del estilo.
{10pt}               % Espacio por encima.
{10pt}               % Espacio por debajo.
{}                   % Fuente del cuerpo.
{}                   % Identación.
{\bf}                % Fuente para la cabecera.
{.}                  % Puntuación tras la cabecera.
{.5em}               % Espacio tras la cabecera.
{\thmname{#1}\thmnumber{ #2}\thmnote{ (#3)}}     % Especificación de la cabecera (actual: nombre en negrita).

% Nuevo estilo para notas.
\newtheoremstyle{remark-style} 
{10pt}                
{10pt}                
{}                   
{}                   
{\itshape}          
{.}                  
{.5em}               
{}                  

% Nuevo estilo para teoremas y proposiciones.
\newtheoremstyle{theorem-style}
{10pt}                
{10pt}                
{\itshape}           
{}                  
{\bf}             
{.}                
{.5em}               
{\thmname{#1}\thmnumber{ #2}\thmnote{ (#3)}}                   

% Nuevo estilo para ejemplos.
\newtheoremstyle{example-style}
{10pt}                
{10pt}                
{}                  
{}                   
{\scshape}              
{:}                 
{.5em}               
{}                   

% Teoremas, proposiciones y corolarios.
\theoremstyle{theorem-style}
\newtheorem*{nth}{Teorema}
\newtheorem*{nprop}{Proposición}
\newtheorem{ncor}{Corolario}

% Definiciones.
\theoremstyle{definition-style}
\newtheorem*{ndef}{Definición}

% Notas.
\theoremstyle{remark-style}
\newtheorem*{nota}{Nota}

% Ejemplos.
\theoremstyle{example-style}
\newtheorem*{ejemplo}{Ejemplo}

% Listas ordenadas con números romanos (i), (ii), etc.
\newenvironment{nlist}
{\begin{enumerate}
\renewcommand\labelenumi{(\emph{\roman{enumi})}}}
{\end{enumerate}}

% División por casos con llave a la derecha.
\newenvironment{rcases}
  {\left.\begin{aligned}}
  {\end{aligned}\right\rbrace}



%---------------------------------------------------------------------------
%   PÁGINA DE TÍTULO
%---------------------------------------------------------------------------

% Título del documento.
\newcommand{\subject}{Ejercicios resueltos Álgebra I}

% Autor del documento.
\newcommand{\docauthor}{Doble Grado de Informática y Matemáticas}

% Título
\title{
  \normalfont \normalsize 
  \textsc{Universidad de Granada} \\ [25pt]    % Texto por encima.
  \horrule{0.5pt} \\[0.4cm] % Línea horizontal fina.
  \huge \subject\\ % Título.
  \horrule{2pt} \\[0.5cm] % Línea horizontal gruesa.
}

% Autor.
\author{\Large{\docauthor}}

% Fecha.
\date{\vspace{-1.5em} \normalsize Curso 2016/17}


%---------------------------------------------------------------------------
%   COMIENZO DEL DOCUMENTO
%---------------------------------------------------------------------------
\begin{document}
\maketitle



\section{Ejercicio 1 de la relación 3 - Comprobar si dos números son congruentes.}


\textbf{Enunciado:}\textbf{ Discutir, usando congruencias, la validez de las siguientes afirmaciones:}

\underline{\textbf{1) $320^{207}$ y $2^{42}$ dan el mismo resto al dividirlos por 13.}}

Tenemos que reducir ambos números, hacemos las bases.

$300^{207}\equiv 8^{207}mod(13) \equiv (2^3)^{207} mod(13) \equiv 2^{621}$

Ahora, calculamos los restos de las potencias de 2, hasta ver cuándo se repite uno de los restos:

\begin{itemize}
	\item $2^1 \equiv_{13} 2$
	\item $2^2 \equiv_{13} 4$
	\item ...
	\item $2^{13} \equiv_{13} 2$
\end{itemize}

Ahora, como nos ha salido que la potencia es 13, aplicamos la función $\varphi$ de Euler a 13 para ver con qué número tienen que ser las potencias congruentes.

\[
\varphi(13) = 12
\]

Por último, como tenemos los dos números en la misma base, tenemos que ver si los exmponentes son congruentes módulo 12.

\[
621 \equiv_{12} 42 \implies 9 \equiv_{12} 8
\]

Por lo que no son congruentes módulo 12, luego los dos números iniciales no dan el mismo resto al dividirlos por 13.

\textbf{\underline{2) $5^{2n+1} + 2^{2n+1}$ es divisible por 7 cualquiera que sea el entero $n \geq 1$}}

Tenemos que ver si $5^{2n+1} + 2^{2n+1} \equiv_7 0$ para $n\geq 1$. Pero $5\equiv_7 2$, luego:

\[
5^{2n+1} + 2^{2n+1} \equiv_7 -(2)^{2n+1} + 2^{2n+1} \equiv_7 0
\]

Luego siempre es divisible por 7 para cualquier entero.\\


\underline{\textbf{4)Las dos últimas cifras del número $7^{355}$ son 4 y 3.}}

Para esto, bastaría ver si $7^{355} \equiv 43 mod(100)$. Para ello, vamos a facilitar el cálculo usando la función $\varphi$ de Euler. $\varphi(100) = 100*1/2 * 4/5 = 40$.

Esto implica que $ 7^{40} \equiv 1 mod(100)$

Vamos a reducir el $7^{355}$ con módulo 40. Si dividimos 355 entre 40 nos queda un resto de 35, luego $7^{355} \equiv 7^{35} mod(100)$.

Ahora, vamos a calcular las potencias de 7 para ver cuándo se repite el resto al ir añadiendo exponentes.
\begin{itemize}
	\item $7 \equiv_{100} 7$
	\item $7^2 \equiv 29 mod(100)$
	\item $7^3 \equiv 43 mod(100)$
	\item ...
	\item $7^5 \equiv_{100} 7$
\end{itemize}

Luego $7^{35} \equiv_{100} 7^3 \equiv_{100} 43$, pues ya habíamos obtenido ese 43 como resultado de hacer las congruencias de las potencias sucesivas de 7.


\underline{\textbf{4) $3*5^{2n+1} + 2^{3n+1}$ es divisible por 17 cualquiera que sea el entero $n\geq 1$}}

Tenemos que volver a ver en este caso, si el número es congruente con 0 módulo 17.
Ahora, quitando el n+1 en el 5:
\[
3*5^{2n+1} + 2^{3n+1} \equiv_{17} 0  \implies 15*5^{2n} + 2^{3n+1}
\]

Y seguimos desarrollando.

\[
 15*5^{2n} + 2^{3n+1} \equiv  \implies -2^{2n} + 2^{3n+1} \equiv_{17} -2*8^n + 2^{3n+1} = -2*2^{3n} = -(2)^{3n+1} + 2^{3n+1} = 0
\]

\textbf{6) Un número es divisible por 4 si y solo si el número formado por sus dos
últimas cifras es múltiplo de 4.}

Vamos ver si : $a_n a_{n-1}...a_1a_0 \iff a_1 a_0 \equiv_4 0$.

El número vendrá dado: $a_n*10^{n}+ a_{n-1}*10^{n-1} + ... + a_1 * 10 + a_0$.

Pero, tomando todas la demás cifras menos las dos últimas, su suma es congruente con 0 módulo 4, luego  basta ver si los dos últimos es congruente con 0 módulo 4, pero eso es el enunciado, luego queda probado.


\newpage
\section{Ejercicio 6 de la relación 3.}

\textbf{Enunciado: Antonio, Pepe y Juan son tres campesinos que principalmente se
dedican al cultivo de la aceituna. Este año la producción de los olivos de
Antonio fue tres veces la de los de Juan y la de Pepe cinco veces la de los
de Juan. Los molinos a los que estos campesinos llevan la aceituna, usan
recipientes de 25 litros el de Juan, 7 litros el de Antonio y 16 litros el de
Pepe. Al envasar el aceite producido por los olivos de Juan sobraron 21
litros, al envasar el producido por Antonio sobraron 3 litros y al envasar el
producido por Pepe sobraron 11 litros. Sabiendo que la producción de Juan
está entre 1000 y 2000 litros ¿cual fue la producción de cada uno de ellos?.}

Vamos a plantear primero el problema: Vamos a llamar:
\begin{enumerate}
	\item A = 3J; P = 5J
	\item La capacidad es: J = 25, A = 7 y P = 16.
	\item Sobran: J = 21, A = 3, y P = 11.
\end{enumerate}

Así, el sistema a plantear es:

\begin{itemize}
	\item $J \equiv 21 mod(25)$
	\item $A \equiv 3 mod(7) \equiv 3J$
	\item $P \equiv 11 mod(16) \equiv 5J$
\end{itemize}

Por lo que el sistema resultante es

\[
\begin{rcases}
	x \equiv 21 mod(25)\\
	3x \equiv 3 mod(7)\\
	5x \equiv 11 mod(16)
\end{rcases}
\]

Tenemos que hacer transformaciones hasta llegar a dejar la $x$ sola en cada una de las ecuaciones en congruencia. Si las hacemos, la transformación es:



\[
\begin{rcases}
	x \equiv 21 mod(25)\\
	3x \equiv 3 mod(7)\\
	5x \equiv 11 mod(16)
\end{rcases} \implies \begin{cases}
	 x \equiv 21 mod(25)\\
	 x \equiv 1 mod(7)\\
	 x \equiv 15 mod(16)
\end{cases}
\]
Y este es el sistema final a resolver. Para ello, resolvemos dos primero y luego resolvemos esos dos con el siguiente. Resolvemos el de las dos primeras:

Vemos que la primera ecuación es : $x = 21+y*25$ lo quenos lleva a la congruencia: $21 +y*25 \equiv 1 mod(7) \implies 4y \equiv 1 mod(7) \implies y \equiv 2 mod(7)$ por tanto la solución óptima es $y_0 = 2$ y ahora si $y=2$, eso implica que (volviendo a la x que habíamos despejado) $x_0 = 21+50 = 71$ y si volvemos a expresarlo como congruencias nos queda $x \equiv 71 mod(175)$. 

Hemos reducido una ecuación, ahora tendríamos que volver a resolver el sistema
\[
\begin{cases}
	x \equiv 71 mod(175)\\
	x \equiv 15 mod(16)
\end{cases}
\]



\section{ Ejercicio 1 de la relación 4.}

\textbf{Enunciado:Resuelve las ecuaciones siguientes en los anillos que se indican:}

\underline{\textbf{1) 12x = 8 en el anillo $\ent_{20}$ .}}

Lo primero es comprobar si tiene solución. Para ello, tenemos que ver si $(20,12) = 4(5,3) = 4$ divide a 8, que sabemos que sí.

Ahora, planteamos la ecuación en congruencias:
	\[
	12x \equiv 8 mod(20) \implies 3x \equiv 2 mod(5) \implies x \equiv 4 mod(5)
	\]
Luego una solución particular de nuestro problema es 4. Además, es la óptima pues $R_{20}(4) = 4$.

Ahora, las soluciones vendrán dadas por $4+k*5$ en $\ent_{20}$, luego son $\{4,9,14,19\}$.


\underline{\textbf{4) $5^{30}$ x = 2 en $\ent_7$ }}

Podemos despejar un poco la ecuación viendo que:

\[
5^{30} \equiv_7 -2^{30} = 2^{30}
\]

Ahora, como 2 y 7 son primos entre sí, usamos la función $\varphi $ de Euler y vemos: $2^{\varphi(7)} = 2^6 \equiv_7 1$

Luego resulta que $2^{30}\equiv_7 2^6 \equiv_7 1$

Por lo que $x_0 = 2$ y la solución es $x=2$

\section{Ejercicio 2 de la relación 4}

Enunciado: Determina cuántas unidades y cuántos divisores de cero tienen los anillos:

1) $\ent_{125}$

Para ello, basta calcular $|U(\ent_{125})| = \varphi(125) = 125 *(1-1/5) = 100$, luego como tiene 100 unidades, tiene 25 divisores de cero.

2) $\ent_{1000}$

Volvemos a hacer lo mismo, $\varphi(1000) = 1000*(1-1/2)*(1-1/5) = 400$ y por tanto hay 600 divisores de cero.


\section{Ejercicio 2 (parte 2 ) de la relación 4}

\textbf{Enunciado: Sea $\mathcal{F}_9 = \ent_3[x]_{ x^2 +1}$ el anillo de restos del anillo $\ent_3[x]$ módulo $x^2 +1$.}

%Si $f(x)$ no es irreducible $\exists x-a : x-a/f \implies f(a) = 0$

Vamos primero a describir los polinomios que hay:
\[
\mathcal{F}_9 =\ent_3[x]_{x^2+1}= \{0,1,2,x,x+1,x+2,2x,2x+1,2x+2\}
\]
Por tanto, este anillo tiene 9 polinomios:

\underline{\textbf{1) Argumentar que $\mathcal{F}_9$ es un cuerpo}}

Para ello, tenemos que ver si $x^2+1$ es un irreducible en $\ent_3[x]$. Vemos si tiene raíces, dándole los valores 0, 1 y 2 y vemos que en ningún caso el resultado es cero, por tanto es irreducible por la afirmación: Si $f(x)$ no es irreducible $\exists x-a : x-a/f \implies f(a) = 0$

Entonces, $\mathcal{F}_9$ es un cuerpo.



\end{document}
