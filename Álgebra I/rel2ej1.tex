
\documentclass[13pt]{article}
\usepackage[utf8]{inputenc}
\usepackage[margin=3cm]{geometry}
\usepackage{enumerate}
\usepackage{amsmath}
\usepackage{float}





\begin{document}
\textbf{Ejercicio 1:} Calcular las soluciones enteras positivas de la ecuación diofántica 138x+30y= 12150.\\

Sea d = (138,30). La ecuación tendrá solución si d/12150. Calculamos d mediante el algoritmo de Euclides.

\begin{figure}[H]
\begin{center}
\caption{mcd}
\label{my-label}
\begin{tabular}[(b)]{|c|cc}
\cline{1-1}
\textbf{138} & 1                    & 0                    \\ \cline{1-1}
\textbf{30}  & 0                    & 1                    \\ \cline{1-1}
\textbf{18}  & 1                    & -4                   \\ \cline{1-1}
\textbf{12}  & -1                   & 5                    \\ \cline{1-1}
\textbf{6}  & 2                    & -9                    \\ \cline{1-1}
\textbf{0}           & \multicolumn{1}{l}{} & \multicolumn{1}{l}{} \\ \cline{1-1}
\end{tabular}
\end{center}
\end{figure}

(138,30) = 6. Comprobemos si la ecuación diofántica tiene solución. Efectivamente, sí la tiene porque 6/12150. Hallemos una solución particular mediante la ecuación reducida.
 \\
 
Dividmos la ecuación (138,30) = 6
\\


$\frac{138}{6}$x + $\frac{30}{6}$y = $\frac{12150}{6}$
 \\

23x+5y=2025
 \\

Sabemos que 6 = 138(2) + 30(-9). Dividiendo entre 6
\\

 
1 = 23(2) + 5(-9)
 \\
 
Multiplicamos por 2025:
 \\
 
2025 = 23(4050) + 5(-18225)
 \\

Obteniendo así la solución particular 
 \\
 

$x_{0}$ = 4050, $y_{0}$ = -18225

Por lo tanto todas las posibles soluciones a la ecuación son las de la forma

\[
	\begin{cases}
	\text{x = 4050 + 5K}\\
	\text{y = -18225 -23k}\\
	\end{cases}
\]

con k cualquier número entero.

\end{document}