%%%%%%%%%%%%%%%%%%%%%%%%%%%%%%%%%%%%%%%%%%%%%%%%%%%%%%%%%%%%%%%%
%
% Apuntes de la asignatura Análisis Matemático I.
% Doble Grado de Informática y Matemáticas.
% Universidad de Granada.
% Curso 2016/17.
% 
% 
% Colaboradores:
% Javier Sáez (@fjsaezm)
% Daniel Pozo (@danipozodg)
% Pedro Bonilla (@pedrobn23)
% Guillermo Galindo
% Antonio Coín (@antcc)
%
% Agradecimientos:
% Andrés Herrera (@andreshp) y Mario Román (@M42) por
% las plantillas base.
%
% Sitio original:
% https://github.com/libreim/apuntesDGIIM/
%
% Licencia:
% CC BY 4.0 (https://creativecommons.org/licenses/by/4.0/)
%
%%%%%%%%%%%%%%%%%%%%%%%%%%%%%%%%%%%%%%%%%%%%%%%%%%%%%%%%%%%%%%%


%------------------------------------------------------------------------------
%   ACKNOWLEDGMENTS
%------------------------------------------------------------------------------

%%%%%%%%%%%%%%%%%%%%%%%%%%%%%%%%%%%%%%%%%%%%%%%%%%%%%%%%%%%%%%%%%%%%%%%%
% Plantilla básica de Latex en Español.
%
% Autor: Andrés Herrera Poyatos (https://github.com/andreshp) 
%
% Es una plantilla básica para redactar documentos. Utiliza el paquete  fancyhdr para darle un
% estilo moderno pero serio.
%
% La plantilla se encuentra adaptada al español.
%
%%%%%%%%%%%%%%%%%%%%%%%%%%%%%%%%%%%%%%%%%%%%%%%%%%%%%%%%%%%%%%%%%%%%%%%%%

%%%
% Plantilla de Trabajo
% Modificación de una plantilla de Latex de Frits Wenneker para adaptarla 
% al castellano y a las necesidades de escribir informática y matemáticas.
%
% Editada por: Mario Román
%
% License:
% CC BY-NC-SA 3.0 (http://creativecommons.org/licenses/by-nc-sa/3.0/)
%%%

%%%%%%%%%%%%%%%%%%%%%%%%%%%%%%%%%%%%%%%%
% Short Sectioned Assignment
% LaTeX Template
% Version 1.0 (5/5/12)
%
% This template has been downloaded from:
% http://www.LaTeXTemplates.com
%
% Original author:
% Frits Wenneker (http://www.howtotex.com)
%
% License:
% CC BY-NC-SA 3.0 (http://creativecommons.org/licenses/by-nc-sa/3.0/)
%
%%%%%%%%%%%%%%%%%%%%%%%%%%%%%%%%%%%%%%%%%


% Tipo de documento y opciones.
\documentclass[11pt, a4paper, titlepage]{article}


%---------------------------------------------------------------------------
%   PAQUETES
%---------------------------------------------------------------------------

% Idioma y codificación para Español.
\usepackage[utf8]{inputenc}
\usepackage[spanish, es-tabla, es-lcroman, es-noquoting]{babel}
\selectlanguage{spanish} 
%\usepackage[T1]{fontenc}

% Fuente utilizada.
\usepackage{courier}    % Fuente Courier.
\usepackage{microtype}  % Mejora la letra final de cara al lector.

% Diseño de página.
\usepackage{fancyhdr}   % Utilizado para hacer títulos propios.
\usepackage{lastpage}   % Referencia a la última página.
\usepackage{extramarks} % Marcas extras. Utilizado en pie de página y cabecera.
\usepackage[parfill]{parskip}    % Crea una nueva línea entre párrafos.
\usepackage{geometry}            % Geometría de las páginas.

% Símbolos y matemáticas.
\usepackage{amssymb, amsmath, amsthm, amsfonts, amscd}
\usepackage{upgreek}

% Otros.
\usepackage{enumitem}   % Listas mejoradas.


%---------------------------------------------------------------------------
%   OPCIONES PERSONALIZADAS
%---------------------------------------------------------------------------

% Redefinir letra griega épsilon.
\let\epsilon\upvarepsilon

% Formato de texto.
\linespread{1.1}            % Espaciado entre líneas.
\setlength\parindent{0pt}   % No indentar el texto por defecto.
\setlist{leftmargin=.5in}   % Indentación para las listas.

% Estilo de página.
\pagestyle{fancy}
\fancyhf{}
\geometry{left=3cm,right=3cm,top=3cm,bottom=3cm,headheight=1cm,headsep=0.5cm}   % Márgenes y cabecera.

% Redefinir entorno de demostración (reducir espacio superior)
\makeatletter
\renewenvironment{proof}[1][\proofname] {\vspace{-15pt}\par\pushQED{\qed}\normalfont\topsep6\p@\@plus6\p@\relax\trivlist\item[\hskip\labelsep\it#1\@addpunct{.}]\ignorespaces}{\popQED\endtrivlist\@endpefalse}
\makeatother


%---------------------------------------------------------------------------
%   COMANDOS PERSONALIZADOS
%---------------------------------------------------------------------------

% Números enteros: \ent
\providecommand{\ent}{\mathcal{Z}}

% Números racionales: \rac
\providecommand{\rac}{\mathcal{Q}}


% Valor absoluto: \abs{}
\providecommand{\abs}[1]{\lvert#1\rvert}    

% Fracción grande: \ddfrac{}{}
\newcommand\ddfrac[2]{\frac{\displaystyle #1}{\displaystyle #2}}

% Texto en negrita en modo matemática: \bm{}
\newcommand{\bm}[1]{\boldsymbol{#1}}

% Línea horizontal.
\newcommand{\horrule}[1]{\rule{\linewidth}{#1}}


%---------------------------------------------------------------------------
%   CABECERA Y PIE DE PÁGINA
%---------------------------------------------------------------------------

% Cabecera del documento.
\renewcommand\headrule{
	\begin{minipage}{1\textwidth}
		\hrule width \hsize 
	\end{minipage}
}

% Texto de la cabecera.
\lhead{\subject}  % Izquierda.
\chead{}            % Centro.
\rhead{\docauthor}    % Derecha.

% Pie de página del documento.
\renewcommand\footrule{                                 
	\begin{minipage}{1\textwidth}
		\hrule width \hsize   
	\end{minipage}\par
}

% Texto del pie de página.
\lfoot{}                                                 % Izquierda
\cfoot{}                                                 % Centro.
\rfoot{Página\ \thepage\ de\ \protect\pageref{LastPage}} % Derecha.


%---------------------------------------------------------------------------
%   ENTORNOS PARA MATEMÁTICAS
%---------------------------------------------------------------------------

% Nuevo estilo para definiciones.
\newtheoremstyle{definition-style} % Nombre del estilo.
{10pt}               % Espacio por encima.
{10pt}               % Espacio por debajo.
{}                   % Fuente del cuerpo.
{}                   % Identación.
{\bf}                % Fuente para la cabecera.
{.}                  % Puntuación tras la cabecera.
{.5em}               % Espacio tras la cabecera.
{\thmname{#1}\thmnumber{ #2}\thmnote{ (#3)}}     % Especificación de la cabecera (actual: nombre en negrita).

% Nuevo estilo para notas.
\newtheoremstyle{remark-style} 
{10pt}                
{10pt}                
{}                   
{}                   
{\itshape}          
{.}                  
{.5em}               
{}                  

% Nuevo estilo para teoremas y proposiciones.
\newtheoremstyle{theorem-style}
{10pt}                
{10pt}                
{\itshape}           
{}                  
{\bf}             
{.}                
{.5em}               
{\thmname{#1}\thmnumber{ #2}\thmnote{ (#3)}}                   

% Nuevo estilo para ejemplos.
\newtheoremstyle{example-style}
{10pt}                
{10pt}                
{}                  
{}                   
{\scshape}              
{:}                 
{.5em}               
{}                   

% Teoremas, proposiciones y corolarios.
\theoremstyle{theorem-style}
\newtheorem*{nth}{Teorema}
\newtheorem*{nprop}{Proposición}
\newtheorem{ncor}{Corolario}

% Definiciones.
\theoremstyle{definition-style}
\newtheorem*{ndef}{Definición}

% Notas.
\theoremstyle{remark-style}
\newtheorem*{nota}{Nota}

% Ejemplos.
\theoremstyle{example-style}
\newtheorem*{ejemplo}{Ejemplo}

% Listas ordenadas con números romanos (i), (ii), etc.
\newenvironment{nlist}
{\begin{enumerate}
\renewcommand\labelenumi{(\emph{\roman{enumi})}}}
{\end{enumerate}}


%---------------------------------------------------------------------------
%   PÁGINA DE TÍTULO
%---------------------------------------------------------------------------

% Título del documento.
\newcommand{\subject}{Álgebra I}

% Autor del documento.
\newcommand{\docauthor}{Doble Grado de Informática y Matemáticas}

% Título
\title{
  \normalfont \normalsize 
  \textsc{Universidad de Granada} \\ [25pt]    % Texto por encima.
  \horrule{0.5pt} \\[0.4cm] % Línea horizontal fina.
  \huge \subject\\ % Título.
  \horrule{2pt} \\[0.5cm] % Línea horizontal gruesa.
}

% Autor.
\author{\Large{\docauthor}}

% Fecha.
\date{\vspace{-1.5em} \normalsize Curso 2016/17}


%---------------------------------------------------------------------------
%   COMIENZO DEL DOCUMENTO
%---------------------------------------------------------------------------
\begin{document}
\maketitle

\section{Anillo conmutativo}

\begin{ndef}[Anillo conmutativo]
Un conjunto A es un anillo conmutativo si en él hay definidas dos operaciones; una aplicacion de adición y una aplicación de multiplicación, tales que cumplen las siguientes propiedades:
\begin{nlist}
\item Asociativa: $a+(b+c) = (a+b)+c$\hspace{1cm} $a(bc) = (ab)c$
\item Conmutativa: $a+b = b+a$ \hspace{1cm} $ab = ba$
\item Existencia elemento neutro: $a+0 = a$ \hspace{1cm} $a*1 = a$
\item Existencia del elemento opuesto: $a+(-a) = 0$
\item Distributiva del producto en la suma: $a(b+c) = ab+ ac$

\end{nlist}

\end{ndef}

\begin{ndef}[Grupo conmutativo]
	Denominamos un grupo conmutativo o abeliano a aquellos conjuntos que cumplen las propiedades asociativa, conmutativa y existencia de elemento neutro para la suma, y existencia de elemento opuesto.
\end{ndef}

\begin{ndef}[monoide]
	Denominamos monoide a un conjunto con una operación binaria interna que cumple la propiedad asociativa y tiene un elemento neutro a izquierda y derecha. En el caso del producto, se denomina monoide multiplicativo.
\end{ndef}

\begin{nota}
	Llamaremos anillo aquellos conjuntos que cumplan todas las propiedades excepto la propiedad conmutativa para la multiplicación.
\end{nota}


\section{Caracterización de $\mathbb{Z}_{n}$.}

Llamaremos $R_n:\mathbb{N} \rightarrow \mathbb{Z}_n$ a la aplicación definida como:

\[
R_n(a) = a - nq = a- nE(\frac{a}{n})
\] 
Para esta aplicación, definimos las siguientes propiedades:

\begin{itemize}
\item Si $0 \leq a < n-1  \rightarrow R_n(a) = a$
\item $\forall a,b \in \mathbb{N}$

\begin{itemize}
	\item  $R_n(a+b) = R_n(R_n(a) + R_n(b))$
	\item $R_n(ab) = R_n(R_n(a)*R_n(b))$
\end{itemize}

\end{itemize} 
\pagebreak

Una vez que tenemos definida una suma y producto con la aplicación $R_n$, definimos las suma y el producto de $\mathbb{Z}_n$.
\begin{ndef}[Suma y producto en $\ent_n$] Se define la suma y el producto en $\ent_n$ de la forma:
	\begin{itemize}
	\item $a\oplus b = R_n(a+b)$
	\item $a\otimes b = R_n(ab)$
\end{itemize}

\end{ndef}



Es fácil verificar que $\mathbb{Z}_n$ es un anillo conmutativo con estas operaciones.\\

\begin{ndef}[Unidad]
	Si A es un anillo conmutativo (a.c) $a \in A$ es una "\textbf{unidad}" o "\textbf{invertible}" si $\exists a^{-1}$ t.q. $ aa^{-1} = 1$.\\ $U(A) = \{ a \in A$ t.q. a es una unidad\} = conjunto de las unidades de A.\\
\end{ndef}



\begin{ndef}[Cuerpo] Se dice que A es un \textbf{cuerpo} si siendo un anillo conmutativo, $U(A) = A - \{0\}$, es decir, $\exists a^{-1}$ $\forall a \in A$ con $a \neq 0$.
\end{ndef}


\begin{nprop}[Asociatividad generalizada]
	Sea A un anillo conmutativo, y $a_1 ... a_n$ una lista de elementos de A.
La propiedad de la \textbf{asociatividad generalizada} nos dice que:
$\forall m$ tal que $1 \leq m < n$ entonces: \[\sum_{i=1}^{n}a_i = (\sum_{i=1}^{m}a_i) + (\sum_{i=m+1}^{n}a_i)\]\[\prod_{i=1}^{n}a_i = (\prod_{i=1}^{m}a_i)(\prod_{i=m+1}^{n}a_i)\]\\
\end{nprop}


\begin{ndef}[Distributividad generalizada]
Definimos también la distributividad generalizada en un anillo como:
	\[(\sum_{i=0}^{n} a_i)(\sum_{j=1}^{m}b_j) = \sum_{i=1}^{n}\sum_{j=1}^{m}a_i b_j \hspace{1cm} \forall a,b\in A\]
\end{ndef}

\begin{ndef}[Subanillo]

Si A es un anillo conmutativo y B es un subconjunto de A. Se dice que B es un \textbf{subanillo} de A ($B \leq A$) si se verifican:
\begin{itemize}
\item $1,-1 \in B$
\item B es cerrado para la suma y el producto.
\end{itemize}
	
\end{ndef}


\subsection{Ejemplos- Anillos de números cuadráticos}

\begin{itemize}
\item $\mathbb{Z}[\sqrt{n}]$.
Definimos este conjunto de la siguiente forma:
\begin{center}$\mathbb{Z}[\sqrt{n}] = \{a+b\sqrt{n} \in \mathbb{C}$ : $a,b \in \mathbb{Z} \}$ $\leq \mathbb{C}$\end{center}

Podemos definir también $\mathbb{Q}[\sqrt{n}]$ de la misma forma:

\begin{center}$\mathbb{Q}[\sqrt{n}] = \{a+b\sqrt{n} \in \mathbb{C}$ : $a,b \in \mathbb{Q} \}$ $\leq \mathbb{C}$\end{center}

Se puede comprobar que $\mathbb{Z}[\sqrt{n}]$ $\leq$ $\mathbb{Q}[\sqrt{n}]$ y que $\mathbb{Q}[\sqrt{n}]$ es un cuerpo.

\begin{ndef}[Conjugado]
	Si $\alpha = a+b\sqrt{n}\in \mathbb{Q}[\sqrt{n}]$ se define su conjugado como $\bar{\alpha} = a - b\sqrt{n}$. Este verifica que:
\begin{enumerate}

\item $ \overline{(\alpha+ \beta)} $ = $\bar{\alpha} + \bar{\beta}$
\item $\overline{\alpha \beta}$ = $\bar{\alpha}\bar{\beta}$
\item $\alpha = \bar{\alpha} \Leftrightarrow b = 0$
	
\end{enumerate} 
\end{ndef}

\begin{ndef}[Norma]
	Se define entonces la Norma $N(\alpha) = \alpha \bar{\alpha} = a^2 - nb^2 \in \rac $. Así:
\begin{enumerate}
	\item $N(\alpha \beta) = N(\alpha) * N(\beta)$
	\item $N(\alpha) = 0 \iff \alpha = 0$
\end{enumerate}
\end{ndef}


\begin{nprop}
	$\alpha \in a + b \sqrt{n} \in \ent[\sqrt{n]}$ es invertible $\iff$ $N(\alpha) \in \{-1,1\}$
\end{nprop}



\item Anillos de series.
\begin{ndef}
	Si A es un anillo conmutativo y X es un símbolo que no denota ningún elemento de A. El anillo de series con coeficientes en A, denotado con A[[x]] esta definido como:\begin{center}
$A[[x]] = \{a = \sum_{i=1}^{n}a_i x^i = a_0 + a_1 x^1 + ... + a_n x^n\}$ $a_i \in A$\end{center}
Y definimos la suma y el producto de la siguiente forma:\\
\[(a+b) = \sum_{i=0}^{n}(a_i+b_i)x^i\]\[ (ab) = \sum_{k=0}^n\sum_{i=0}^{k}a_ib_{k-i}\]
\end{ndef}
\end{itemize}

Se puede probar que con estas operaciones de suma y producto, $A[[x]]$ es un anillo y $A[x]$ es un subanillo de $A[[x]]$  

\section{Homomorfismos}

\begin{ndef}
Si $A,B$ son anillos conmutativos, una aplicacion $\varphi: A \to B$ es un homomorfismo si:

\begin{enumerate}
	\item $\varphi(1) = 1$
	\item $\varphi(a+b) = \varphi(a) + \varphi(b)$
	\item $\varphi(ab) = \varphi(a)  \varphi(b)$
	
\end{enumerate}
	
\end{ndef}

Además, decimos que:

\begin{enumerate}
	\item Es monomorfismo si es inyectivo.
	\item Es epimorfismo si es sobreyectivo.
	\item Es isomorfismo si es biyectivo.
\end{enumerate}

\textbf{Propiedades de los homomorfismos}

\begin{itemize}

\item $\varphi(0) = 0$

\item $\varphi(-a) = -\varphi(a) $

\item $\varphi(\sum_{i = 1}^n a_i) = \sum_{i = 1}^n\varphi(a_i)$. 

$\varphi(\prod_{i = 1}^n a_i) = \prod_{i = 1}^n\varphi(a_i)$

\item $\varphi(na) = n\varphi(a)$

	
\end{itemize}


Ya sabemos que $Im(\varphi) = \{ \varphi(x): x \in A\} \leq B$ es un subanillo.

\begin{nprop}
	Si $\varphi$ es monomorfismo, entonces la aplicación restringida:

\[
A \to Im(\varphi)\]
\[
a \mapsto \varphi(a)
\]

es un epimorfismo y por ello es un isomorfismo, podemos decir que $A \cong Im(\varphi)$.\\
\end{nprop}


\begin{nota}
	Se puede probar que $R_n: \ent \to \ent_n$ es un homomorfismo, llamado \emph{Homomorfismo de reducción módulo n}\\
\end{nota}


\begin{nprop}[1]
 Dado $A$ cualquier anillo conmutativo, conocido $A[x]$.
 
 Si $\varphi:A \to B$ es homomorfismo de anillos conmutativos, entonces:
 \[
 \exists \varphi: A[x] \to B[x] : \varphi\left(\sum_i a_i x^i\right) = \sum_i\varphi(a_i) x^i
 \]

	
\end{nprop}

\begin{nprop}[Sustición en un polinomio(2)]
	
Si $A$ es cualquier conjunto y $a \in A$ entonces: existe un homomorfismo $E_a: A[x] \to A$ tal que $E_a(\sum_i a_i x^i) = \sum_i a_i a^i$.
	
\end{nprop}

\begin{nprop}[3]
Si $A \leq B$ es un subanillo y $b\in B$, la aplicación $E_b:A[x] \to B$ definida como $E_b(\sum_i a_i x_i) = \sum_i a_i b^i$ es un homomorfismo
	
\end{nprop}

\begin{nprop}[Engloba a las anteriores]
Si $\varphi:A \to B$ es un homomorfismo y $b\in B$, la aplicación $\Phi:A[x] \to B$ definida como $\Phi ( \sum_i a_i x_i) = \sum_i \varphi(a_i)b^i \in B$ es un homomorfismo 

\end{nprop}

\begin{proof}
Veamos primero cómo (4) engloba a las demás:
\begin{nlist}

\item $4 \Rightarrow 3$. Se ve tomando como $\varphi$ la inclusión en $B$
\item $4 \Rightarrow 2$. Tomamos esta vez como $\varphi$ la identidad
\item $4 \Rightarrow 1$. Suponemos 4 válido. Probaremos que $\exists\varphi: A \to B[x]$ que lleva $a \to \varphi(a)$.
Ahora, podemos ver que esa aplicación es como usar primero $\varphi$ para ir de A a B y luego usar la inclusión de $B$ en $B[x]$:
\[
A\to B \to B[x]
\]
\[
a \to a \to \varphi(a)
\]

De esta forma, tomamos $x\in B[x]$. Entonces:
\begin{center}
	$A[x]\to B[x]$\\
	$\sum_i a_i x_i \to \sum_i \varphi(a_i)x_i$
\end{center}

Que es justamente el enunciado de la primera proposición.

\end{nlist}

Pasamos ahora a la demostración de la Proposición 4.


Sean $f = \sum a_ix_i$ y $g = \sum b_i x_i \in A[x]$. Entonces: $f+g = \sum c_i x_i$ con $c_i = a_i + b_i$

Si ahora aplicamos $\Phi(f+g) = \sum \varphi(c_i)b^i = \sum \varphi(a_i + b_i)b^i$. 

Como $\varphi$ es homomorfismo , eso es igual a:
 $\sum (\varphi(a_i) + \varphi(b_i))b^i$.
 
 Usando que $B$ es un anillo y por ello hay distributividad, eso es igual a: $\sum (\varphi(a_i)b^i + \varphi(b_i)b^i$.
 
 Por la asociatividad generalizada eso es igual a: $\sum \varphi(a_i)b^i + \sum \varphi(b_i)b^i = \Phi(f) + \Phi(g)$ Por lo que queda probado para la suma.
 
 Ahora probaremos el producto:
 
 $fg = \sum c_i x^i$ con $c_i  = \sum_{i+j = n} a_ib_j$\\
 
 Así: \[\Phi(f+g) = \sum_n\varphi(c_n)b^n = \sum \varphi(\sum_{i+j = n} a_ib_j)b^n = \sum_n ( \sum_{i+j = n} \varphi(a_i)\varphi(b_j))b^n\]
 
 Desarrollamos por otro lado
 \[
 \Phi(f) + \Phi(g) = (\sum_i \varphi(a_i)b^i)(\sum_j \varphi(b_j)b^i) =^{(1)} \sum_{i,j} \varphi(a_i)b^i\varphi(b_j)b^j =^{(2)} \sum_{i,j} \varphi(a_ib_j)b^{i+j}=
 \]
 
 \[
  = \sum_n(\sum_{i,j: i+j = n} \varphi(a_ib_j)b^n)
 \]
 
 Donde en $(1)$ hemos usado la distributividad general y en $(2)$ hemos usado que estamos en un anillo conmutativo y que $\varphi$ es un homomorfismo. 
 
 Así, hemos llegado a dos expresiones que son iguales, probando así el resultado.
 
\end{proof}


Sabemos que cada polinomio $f(x)$ constituye una función de evaluación $f(x) \in A[x]$
\[
f(x):B \to B
\]
\[
\hspace{1cm}b \to f(b)
\]

Sin embargo, un polinomio es mucho más que la función de evaluación que él mismo define. Estudiaremos el caso $A[x_1,...,x_r]$

\begin{ndef}[Polinomios de r variables con coeficientes en A]
Sea A un anillo conmutativo. Consideramos $A[x_1,...x_r]$ inductivamente en $r$:\\
Si $r>1$ entonces $A[x_1,...,x_r] = A[x_1,...,x_{r-1}][x_r]$\\
	
\begin{proof}\hfill 

\begin{itemize}

	\item $r=1$:
\[
f(x_1) \in A[x_1] \hspace{1cm} \sum_{i \geq 0}a_i x_i \hspace{0.5cm} a_i \in A \hspace{0.5cm} \exists K: a_{i1} = 0 \hspace{0.3cm} \forall i > K
\]
	\item $r> 1$

\[
f(x_1,...,x_r) = \sum_{i1,...ir} a_{i1},...,a_{ir} x_1^{i1},...,x_r^{ir}: \hspace{0.3cm} \exists K: a_i,...,a_r  = 0 \iff i_s > K
\]
Ahora, si vemos que:
\[
f_{ir}(x_1,...,x_{r-1}) = \sum_{i1,...ir > 0} a_{i1},...,a_{ir} x_1^{i1},...,x_r^{ir-1} \in A[x_1,...,x_{r-1}]
\]
Entonces:
\[
\sum_{ir \geq 0} f_{ir}(x_1,...,x_{r-1})x_r^{ir}= \sum_{ir \geq 0}(\sum_{i1,...,ir > 0}a_1,...,a_r x_1^{i1},...,x_{r-1}^{ir-1})x_r^{ir} = 
\]
\[
=\sum_{i1,...ir} a_{i1},...,a_{ir} x_1^{i1},...,x_r^{ir}
\]

Ahora, definimos $g(x_1,...,x_r) = \sum_{i1,...ir} b_{i1},...,b_{ir} x_1^{i1},...,x_r^{ir}$. Ahora, sumamos:

\[
f(x_1,...,x_r)+g(x_1,...,x_r) = \sum_{i1,...ir} a_{i1},...,a_{ir} x_1^{i1},...,x_r^{ir} + \sum_{i1,...ir} b_{i1},...,b_{ir} x_1^{i1},...,x_r^{ir} = 
\]
\[
 = \sum_{i1,...,ir} (a_i+bi)x^{i1+ij}
\]

Ahora, podemos desarrollar de la misma forma el producto y ver que:

\[
(ax_1^{i1},...,x_r^{ir})(bx_1^{j1},...,x_r^{jr}) = abx_i^{i+j}x_2^{i_2+j_2}...x_r^{i_r+j_r}
\]

\end{itemize}
Por lo que queda probado nuestro resultado.
\end{proof}

\end{ndef}





\end{document}
