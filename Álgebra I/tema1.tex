\documentclass[11pt]{article}
\usepackage[utf8]{inputenc}
\usepackage{amssymb}

\providecommand{\rac}{\mathbb{Q}}
\providecommand{\ent}{\mathbb{Z}}
\providecommand{\sii}{\Leftrightarrow}
%Gummi|065|=)
\title{\textbf{Álgebra}}
\date{}
\begin{document}

\maketitle

\section{Anillo conmutativo}

Un conjunto A, en el que hay definidas dos operaciones; una aplicacion de adición y una aplicación de multiplicación, tales que cumplen las siguientes propiedades:
\begin{itemize}
\item Asociativa: $a+(b+c) = (a+b)+c$\hspace{1cm} $a(bc) = (ab)c$
\item Conmutativa: $a+b = b+a$ \hspace{1cm} $ab = ba$
\item Existencia elemento neutro: $a+0 = a$ \hspace{1cm} $a*1 = a$
\item Existencia del elemento opuesto: $a+(-a) = 0$
\item Distributiva del producto en la suma: $a(b+c) = ab+ ac$

\end{itemize}

Con estas propiedades tenemos un anillo conmutativo.\\

Denominamos un grupo conmutativo o abeliano a aquellos conjuntos que cumplen las propiedades asociativa, conmutativa y existencia de elemento neutro para la suma, y existencia de elemento opuesto.\\

Denominamos monoide a un conjunto con una operación binaria interna que cumple la propiedad asociativa y tiene un elemento neutro a izquierda y derecha. En el caso del producto, se denomina monoide multiplicativo.\\

Llamaremos anillo aquellos conjuntos que cumplan todas las propiedades excepto la propiedad conmutativa para la multiplicación.\\

\section{Definición de $\mathbb{Z}_{n}$.}

LLamaremos $R_n:\mathbb{N} \rightarrow \mathbb{Z}_n$ a la aplicación definida como $R_n(a) = a - nq = a- nE(\frac{a}{n})$. Así, definimos las siguientes propiedades:

\begin{itemize}
\item Si $0 \leq a < n-1  \rightarrow R_n(a) = a$
\item $\forall a,b \in \mathbb{N}, R_n(a+b) = R_n(R_n(a) + R_n(b))$\\$R_n(ab) = R_n(R_n(a)*R_n(b))$

\end{itemize} 

Definimos las suma y el producto de $\mathbb{Z}_n$:

- $a\oplus b = R_n(a+b)$\\
\hspace*{0.6cm}- $a\otimes b = R_n(ab)$\\\\

<<<<<<< HEAD
Es facil verificar que $\mathbb{Z}_n$ es un anillo conmutativo con estas operaciones.\\
=======
Es fácil verificar que $\mathbb{Z}_n$ es un anillo conmutativo con estas operaciones.\\
>>>>>>> f330c22ce02ff106582537df6df2d0d8502e15f4

\textbf{Definición}:
Si A es un anillo conmutativo (a.c) $a \in A$ es una "\textbf{unidad}" o "\textbf{invertible}" si $\exists a^{-1}$ t.q. $ aa^{-1} = 1$.\\ $U(A) = \{ a \in A$ t.q. a es una unidad\} = conjunto de las unidades de A.\\

\textbf{Definición}: Se dice que A es un \textbf{cuerpo} si siendo un anillo conmutativo, $U(A) = A - \{0\}$, es decir, $\exists a^{-1}$ $\forall a \in A$ con $a \neq 0$.\\

Sea A un anillo conmutativo, y $a_1 ... a_n$ una lista de elementos de A.
La propiedad de la \textbf{asociatividad generalizada} nos dice que:
$\forall m$ tal que $1 \leq m < n$ entonces: \[\sum_{i=1}^{n}a_i = (\sum_{i=1}^{m}a_i) + (\sum_{i=m+1}^{n}a_i)\]\[\prod_{i=1}^{n}a_i = (\prod_{i=1}^{m}a_i)(\prod_{i=m+1}^{n}a_i)\]\\

También definimos la propiedad \textbf{distributiva generalizada} como:\\

\[(\sum_{i=0}^{n} a_i)(\sum_{j=1}^{m}b_j) = \sum_{i=1}^{n}\sum_{j=1}^{m}a_i b_j\]\\\\

\textbf{Definición}:
Si A es un anillo conmutativo y B es un subconjunto de A. Se dice que B es un \textbf{subanillo} de A ($B \leq A$) si se verifican:
\begin{itemize}
\item $1,-1 \in B$
\item B es cerrado para la suma y el producto.
\end{itemize}

\subsection{Ejemplos- Anillos de números cuadráticos}

\begin{itemize}
\item $\mathbb{Z}[\sqrt{n}]$ y $\mathbb{Q}[\sqrt{n}]$\\\\
Definimos este conjunto de la siguiente forma:\\
\begin{center}$\mathbb{Z}[\sqrt{n}] = \{a+b\sqrt{n} \in \mathbb{C}$ : $a,b \in \mathbb{Z} \}$ $\leq \mathbb{C}$\end{center}

Podemos definir también $\mathbb{Q}[\sqrt{n}]$ de la misma forma pero estando los coeficientes $a$ y $b$ en $\mathbb{Q}$. Además se puede comprobar que $\mathbb{Z}[\sqrt{n}]$ $\leq$ $\mathbb{Q}[\sqrt{n}]$ y que $\mathbb{Q}[\sqrt{n}]$ es un cuerpo.

Definición: Si $\alpha = a+b\sqrt{n}\in \mathbb{Q}[\sqrt{n}]$ se define su conjugado como $\bar{\alpha} = a - b\sqrt{n}$. Este verifica que:
\begin{enumerate}

\item $ \overline{(\alpha+ \beta)} $ = $\bar{\alpha} + \bar{\beta}$
\item $\overline{\alpha \beta}$ = $\bar{\alpha}\bar{\beta}$
\item $\alpha = \bar{\alpha} \Leftrightarrow b = 0$
	
\end{enumerate} 

Se define entonces la Norma $N(\alpha) = \alpha \bar{\alpha} = a^2 - nb^2 \in \rac $. Así:
\begin{enumerate}
	\item $N(\alpha \beta) = N(\alpha) * N(\beta)$
	\item $N(\alpha) = 0 \sii \alpha = 0$
\end{enumerate}

\textbf{Proposición}.
$\alpha \in a + b \sqrt{n} \in \ent[\sqrt{n]}$ es invertible $\sii$ $N(\alpha) \in \{-1,1\}$

\item Anillos de series.\\
Si A es un anillo conmutativo y X es un símbolo que no denota ningún elemento de A. El anillo de series con coeficientes en A, denotado con A[[x]] esta definido como:\begin{center}
$A[[x]] = \{a = \sum_{i=1}^{n}a_i x^i = a_0 + a_1 x^1 + ... + a_n x^n\}$ $a_i \in A$\end{center}
Y definimos la suma y el producto de la siguiente forma:\\
\[(a+b) = \sum_{i=0}^{n}(a_i+b_i)x^i\]\[ (ab) = \sum_{k=0}^n\sum_{i=0}^{k}a_ib_{k-i}\]
\end{itemize}

Se puede probar que con estas operaciones de suma y producto, $A[[x]]$ es un anillo y $A[x]$ es un subanillo de $A[[x]]$  

\section{Homomorfismo}

Si $A,B$ son anillos conmutativos, una aplicacion $\varphi: A \to B$ es un homomorfismo si:

\begin{enumerate}
	\item $\varphi(1) = 1$
	\item $\varphi(a+b) = \varphi(a) + \varphi(b)$
	\item $\varphi(ab) = \varphi(a)  \varphi(b)$
	
\end{enumerate}

Además, decimos que:

\begin{enumerate}
	\item Es monomorfismo si es inyectivo.
	\item Es epimorfismo si es sobreyectivo.
	\item Es isomorfismo si es biyectivo.
\end{enumerate}

\textbf{Propiedades de los homomorfismos}

\begin{itemize}

\item $\varphi(0) = 0$

\item $\varphi(-a) = -\varphi(a) $

\item $\varphi(\sum_{i = 1}^n a_i) = \sum_{i = 1}^n\varphi(a_i)$. 

$\varphi(\prod_{i = 1}^n a_i) = \prod_{i = 1}^n\varphi(a_i)$

\item $\varphi(na) = n\varphi(a)$

	
\end{itemize}


Ya sabemos que $Im(\varphi) = \{ \varphi(x): x \in A\} \leq B$ es un subanillo.

Ahora, si $\varphi$ es monomorfismo, entonces la aplicación restringida:

\[
A \to Im(\varphi)\]
\[
a \mapsto \varphi(a)
\]

es un epimorfismo y por ello es un isomorfismo, podemos decir que $A \cong Im(\varphi)$.\\


\textbf{Ejemplo:} Se puede probar que $R_n: \ent \to \ent_n$ es un homomorfismo, llamado \emph{Homomorfismo de reducción módulo n}\\


\textbf{Proposición}
 Dado $A$ cualquier anillo conmutativo, conocido $A[x]$.
 
 Si $\varphi:A \to B$ es homomorfismo de anillos conmutativos, entonces:
 \[
 \exists \varphi: A[x] \to B[x] : \varphi\left(\sum_i a_i x^i\right) = \sum_i\varphi(a_i) x^i
 \]


\textbf{Proposición (sustitución en un polinomio)}

Si $A$ es cualquier conjunto y $a \in A$ entonces: existe un homomorfismo $E_a: A[x] \to A$ tal que $E_a(\sum_i a_i x^i) = \sum_i a_i a^i$.





















\end{document}
