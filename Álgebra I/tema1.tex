%%%%%%%%%%%%%%%%%%%%%%%%%%%%%%%%%%%%%%%%%%%%%%%%%%%%%%%%%%%%%%%%
%
% Apuntes de la asignatura Álgebra I.
% Doble Grado de Informática y Matemáticas.
% Universidad de Granada.
% Curso 2016/17.
% 
% 
% Colaboradores:
% Javier Sáez (@fjsaezm)
% Daniel Pozo (@danipozodg)
% Pedro Bonilla (@pedrobn23)
% Guillermo Galindo
% Antonio Coín (@antcc)
% Sofía Almeida (@SofiaAlmeida)
%
% Agradecimientos:
% Andrés Herrera (@andreshp) y Mario Román (@M42) por
% las plantillas base.
%
% Sitio original:
% https://github.com/libreim/apuntesDGIIM/
%
% Licencia:
% CC BY 4.0 (https://creativecommons.org/licenses/by/4.0/)
%
%%%%%%%%%%%%%%%%%%%%%%%%%%%%%%%%%%%%%%%%%%%%%%%%%%%%%%%%%%%%%%%


%------------------------------------------------------------------------------
%   ACKNOWLEDGMENTS
%------------------------------------------------------------------------------

%%%%%%%%%%%%%%%%%%%%%%%%%%%%%%%%%%%%%%%%%%%%%%%%%%%%%%%%%%%%%%%%%%%%%%%%
% Plantilla básica de Latex en Español.
%
% Autor: Andrés Herrera Poyatos (https://github.com/andreshp) 
%
% Es una plantilla básica para redactar documentos. Utiliza el paquete  fancyhdr para darle un
% estilo moderno pero serio.
%
% La plantilla se encuentra adaptada al español.
%
%%%%%%%%%%%%%%%%%%%%%%%%%%%%%%%%%%%%%%%%%%%%%%%%%%%%%%%%%%%%%%%%%%%%%%%%%

%%%
% Plantilla de Trabajo
% Modificación de una plantilla de Latex de Frits Wenneker para adaptarla 
% al castellano y a las necesidades de escribir informática y matemáticas.
%
% Editada por: Mario Román
%
% License:
% CC BY-NC-SA 3.0 (http://creativecommons.org/licenses/by-nc-sa/3.0/)
%%%

%%%%%%%%%%%%%%%%%%%%%%%%%%%%%%%%%%%%%%%%
% Short Sectioned Assignment
% LaTeX Template
% Version 1.0 (5/5/12)
%
% This template has been downloaded from:
% http://www.LaTeXTemplates.com
%
% Original author:
% Frits Wenneker (http://www.howtotex.com)
%
% License:
% CC BY-NC-SA 3.0 (http://creativecommons.org/licenses/by-nc-sa/3.0/)
%
%%%%%%%%%%%%%%%%%%%%%%%%%%%%%%%%%%%%%%%%%


% Tipo de documento y opciones.
\documentclass[11pt, a4paper, titlepage]{article}


%---------------------------------------------------------------------------
%   PAQUETES
%---------------------------------------------------------------------------

% Idioma y codificación para Español.
\usepackage[utf8]{inputenc}
\usepackage[spanish, es-tabla, es-lcroman, es-noquoting]{babel}
\selectlanguage{spanish} 
%\usepackage[T1]{fontenc}

% Fuente utilizada.
\usepackage{courier}    % Fuente Courier.
\usepackage{microtype}  % Mejora la letra final de cara al lector.

% Diseño de página.
\usepackage{fancyhdr}   % Utilizado para hacer títulos propios.
\usepackage{lastpage}   % Referencia a la última página.
\usepackage{extramarks} % Marcas extras. Utilizado en pie de página y cabecera.
\usepackage[parfill]{parskip}    % Crea una nueva línea entre párrafos.
\usepackage{geometry}            % Geometría de las páginas.

% Símbolos y matemáticas.
\usepackage{amssymb, amsmath, amsthm, amsfonts, amscd, stmaryrd}
%\usepackage{upgreek}

% Otros.
\usepackage{enumitem}   % Listas mejoradas.
\usepackage[hidelinks]{hyperref}


%---------------------------------------------------------------------------
%   OPCIONES PERSONALIZADAS
%---------------------------------------------------------------------------

% Redefinir letra griega épsilon.
\let\epsilon\upvarepsilon

% Formato de texto.
\linespread{1.1}            % Espaciado entre líneas.
\setlength\parindent{0pt}   % No indentar el texto por defecto.
\setlist{leftmargin=.5in}   % Indentación para las listas.

% Estilo de página.
\pagestyle{fancy}
\fancyhf{}
\geometry{left=3cm,right=3cm,top=3cm,bottom=3cm,headheight=1cm,headsep=0.5cm}   % Márgenes y cabecera.

% Redefinir entorno de demostración (reducir espacio superior)
\makeatletter
\renewenvironment{proof}[1][\proofname] {\vspace{-15pt}\par\pushQED{\qed}\normalfont\topsep6\p@\@plus6\p@\relax\trivlist\item[\hskip\labelsep\it#1\@addpunct{.}]\ignorespaces}{\popQED\endtrivlist\@endpefalse}
\makeatother

% Aumentar el tamaño del interlineado
\linespread{1.3}
%---------------------------------------------------------------------------
%   COMANDOS PERSONALIZADOS
%---------------------------------------------------------------------------

% Números enteros: \ent
\providecommand{\ent}{\mathbb{Z}}

% Números racionales: \rac
\providecommand{\rac}{\mathbb{Q}}

% Números naturales: \nat
\providecommand{\nat}{\mathbb{N}}


% Valor absoluto: \abs{}
\providecommand{\abs}[1]{\lvert#1\rvert}    

% Fracción grande: \ddfrac{}{}
\newcommand\ddfrac[2]{\frac{\displaystyle #1}{\displaystyle #2}}

% Texto en negrita en modo matemática: \bm{}
\newcommand{\bm}[1]{\boldsymbol{#1}}

% Línea horizontal.
\newcommand{\horrule}[1]{\rule{\linewidth}{#1}}


%---------------------------------------------------------------------------
%   CABECERA Y PIE DE PÁGINA
%---------------------------------------------------------------------------

% Cabecera del documento.
\renewcommand\headrule{
	\begin{minipage}{1\textwidth}
		\hrule width \hsize 
	\end{minipage}
}

% Texto de la cabecera.
\lhead{\subject}  % Izquierda.
\chead{}            % Centro.
\rhead{\docauthor}    % Derecha.

% Pie de página del documento.
\renewcommand\footrule{                                 
	\begin{minipage}{1\textwidth}
		\hrule width \hsize   
	\end{minipage}\par
}

% Texto del pie de página.
\lfoot{}                                                 % Izquierda
\cfoot{}                                                 % Centro.
\rfoot{Página\ \thepage\ de\ \protect\pageref{LastPage}} % Derecha.


%---------------------------------------------------------------------------
%   ENTORNOS PARA MATEMÁTICAS
%---------------------------------------------------------------------------

% Nuevo estilo para definiciones.
\newtheoremstyle{definition-style} % Nombre del estilo.
{10pt}               % Espacio por encima.
{10pt}               % Espacio por debajo.
{}                   % Fuente del cuerpo.
{}                   % Identación.
{\bf}                % Fuente para la cabecera.
{.}                  % Puntuación tras la cabecera.
{.5em}               % Espacio tras la cabecera.
{\thmname{#1}\thmnumber{ #2}\thmnote{ (#3)}}     % Especificación de la cabecera (actual: nombre en negrita).

% Nuevo estilo para notas.
\newtheoremstyle{remark-style} 
{10pt}                
{10pt}                
{}                   
{}                   
{\itshape}          
{.}                  
{.5em}               
{}                  

% Nuevo estilo para teoremas y proposiciones.
\newtheoremstyle{theorem-style}
{10pt}                
{10pt}                
{\itshape}           
{}                  
{\bf}             
{.}                
{.5em}               
{\thmname{#1}\thmnumber{ #2}\thmnote{ (#3)}}                   

% Nuevo estilo para ejemplos.
\newtheoremstyle{example-style}
{10pt}                
{10pt}                
{}                  
{}                   
{\scshape}              
{:}                 
{.5em}               
{}                   

% Teoremas, proposiciones y corolarios.
\theoremstyle{theorem-style}
\newtheorem*{nth}{Teorema}
\newtheorem*{nprop}{Proposición}
\newtheorem{ncor}{Corolario}

% Definiciones.
\theoremstyle{definition-style}
\newtheorem*{ndef}{Definición}

% Notas.
\theoremstyle{remark-style}
\newtheorem*{nota}{Nota}

% Ejemplos.
\theoremstyle{example-style}
\newtheorem*{ejemplo}{Ejemplo}

% Listas ordenadas con números romanos (i), (ii), etc.
\newenvironment{nlist}
{\begin{enumerate}
\renewcommand\labelenumi{(\emph{\roman{enumi})}}}
{\end{enumerate}}

% División por casos con llave a la derecha.
\newenvironment{rcases}
  {\left.\begin{aligned}}
  {\end{aligned}\right\rbrace}



%---------------------------------------------------------------------------
%   PÁGINA DE TÍTULO
%---------------------------------------------------------------------------

% Título del documento.
\newcommand{\subject}{Álgebra I}

% Autor del documento.
\newcommand{\docauthor}{Doble Grado de Informática y Matemáticas}

% Título
\title{
  \normalfont \normalsize 
  \textsc{Universidad de Granada} \\ [25pt]    % Texto por encima.
  \horrule{0.5pt} \\[0.4cm] % Línea horizontal fina.
  \huge \subject\\ % Título.
  \horrule{2pt} \\[0.5cm] % Línea horizontal gruesa.
}

% Autor.
\author{\Large{\docauthor}}

% Fecha.
\date{\vspace{-1.5em} \normalsize Curso 2016/17}


%---------------------------------------------------------------------------
%   COMIENZO DEL DOCUMENTO
%---------------------------------------------------------------------------
\begin{document}
\maketitle
\tableofcontents
\newpage

\section{Anillo conmutativo}

\begin{ndef}[Anillo conmutativo]
Un conjunto A es un anillo conmutativo si en él hay definidas dos operaciones; una aplicacion de adición y una aplicación de multiplicación, tales que cumplen las siguientes propiedades:
\begin{nlist}
\item Asociativa: $a+(b+c) = (a+b)+c$\hspace{1cm} $a(bc) = (ab)c$
\item Conmutativa: $a+b = b+a$ \hspace{1cm} $ab = ba$
\item Existencia elemento neutro: $a+0 = a$ \hspace{1cm} $a*1 = a$
\item Existencia del elemento opuesto para la suma: $a+(-a) = 0$
\item Distributiva del producto en la suma: $a(b+c) = ab+ ac$

\end{nlist}

\end{ndef}

\begin{ndef}[Grupo conmutativo]
	Denominamos un grupo conmutativo o abeliano a aquellos conjuntos que cumplen las propiedades asociativa, conmutativa y existencia de elemento neutro para la suma, y existencia de elemento opuesto.
\end{ndef}

\begin{ndef}[monoide]
	Denominamos monoide a un conjunto con una operación binaria interna que cumple la propiedad asociativa y tiene un elemento neutro a izquierda y derecha. En el caso del producto, se denomina monoide multiplicativo.
\end{ndef}

\begin{nota}
	Llamaremos anillo aquellos conjuntos que cumplan todas las propiedades excepto la propiedad conmutativa para la multiplicación.
\end{nota}


\section*{Caracterización de $\mathbb{Z}_{n}$}

Llamaremos $R_n:\mathbb{N} \rightarrow \mathbb{Z}_n$ a la aplicación definida como:

\[
R_n(a) = a - nq = a- nE(\frac{a}{n})
\] 
Para esta aplicación, definimos las siguientes propiedades:

\begin{itemize}
\item Si $0 \leq a < n  \Rightarrow R_n(a) = a$
\item $\forall a,b \in \mathbb{N}$

\begin{itemize}
	\item  $R_n(a+b) = R_n(R_n(a) + R_n(b))$
	\item $R_n(ab) = R_n(R_n(a)*R_n(b))$
\end{itemize}

\end{itemize} 


Una vez que tenemos definida una suma y producto con la aplicación $R_n$, definimos las suma y el producto de $\mathbb{Z}_n$.
\begin{ndef}[Suma y producto en $\ent_n$] Se define la suma y el producto en $\ent_n$ de la forma:
	\begin{itemize}
	\item $a\oplus b = R_n(a+b)$
	\item $a\otimes b = R_n(ab)$
\end{itemize}

\end{ndef}



Es fácil verificar que $\mathbb{Z}_n$ es un anillo conmutativo con estas operaciones.\\

\begin{ndef}[Unidad]
	Si A es un anillo conmutativo (a.c) $a \in A$ es una "\textbf{unidad}” o "\textbf{invertible}” si $\exists a^{-1}$ tal que $ aa^{-1} = 1$.\\ $U(A) = \{ a \in A$ : $a$ es una unidad\} = conjunto de las unidades de A.\\
\end{ndef}



\begin{ndef}[Cuerpo] Se dice que A es un \textbf{cuerpo} si siendo un anillo conmutativo, $U(A) = A - \{0\}$, es decir, $\exists a^{-1}$ $\forall a \in A$ con $a \neq 0$.
\end{ndef}


\begin{nprop}[Asociatividad generalizada]
	Sea A un anillo conmutativo, y $a_1, ..., a_n$ una lista de elementos de A.
La propiedad de la \textbf{asociatividad generalizada} nos dice que:
$\forall m$ tal que $1 \leq m < n$ se verifican: \[\sum_{i=1}^{n}a_i = (\sum_{i=1}^{m}a_i) + (\sum_{i=m+1}^{n}a_i)\]\[\prod_{i=1}^{n}a_i = (\prod_{i=1}^{m}a_i)(\prod_{i=m+1}^{n}a_i)\]\\
\end{nprop}


\begin{ndef}[Distributividad generalizada]
Definimos también la distributividad generalizada en un anillo como:
	\[(\sum_{i=1}^{n} a_i)(\sum_{j=1}^{m}b_j) = \sum_{i=1}^{n}\sum_{j=1}^{m}a_i b_j \hspace{1cm} \forall a,b\in A\]
\end{ndef}

\begin{ndef}[Subanillo]

Si A es un anillo conmutativo y B es un subconjunto de A. Se dice que B es un \textbf{subanillo} de A ($B \leq A$) si se verifican:
\begin{itemize}
\item $1,-1 \in B$
\item B es cerrado para la suma y el producto.
\end{itemize}
	
\end{ndef}

\newpage

\subsection*{Anillos de números cuadráticos}

\begin{itemize}
\item $\mathbb{Z}[\sqrt{n}]$.
Definimos este conjunto de la siguiente forma:
\begin{center}$\mathbb{Z}[\sqrt{n}] = \{a+b\sqrt{n} \in \mathbb{C}$ : $a,b \in \mathbb{Z} \}$ $\leq \mathbb{C}$\end{center}

Podemos definir también $\mathbb{Q}[\sqrt{n}]$ de la misma forma:

\begin{center}$\mathbb{Q}[\sqrt{n}] = \{a+b\sqrt{n} \in \mathbb{C}$ : $a,b \in \mathbb{Q} \}$ $\leq \mathbb{C}$\end{center}

Se puede comprobar que $\mathbb{Z}[\sqrt{n}]$ $\leq$ $\mathbb{Q}[\sqrt{n}]$ y que $\mathbb{Q}[\sqrt{n}]$ es un cuerpo.

\begin{ndef}[Conjugado]
	Si $\alpha = a+b\sqrt{n}\in \mathbb{Q}[\sqrt{n}]$ se define su conjugado como $\bar{\alpha} = a - b\sqrt{n}$. Este verifica que:
\begin{enumerate}

\item $ \overline{(\alpha+ \beta)} $ = $\bar{\alpha} + \bar{\beta}$
\item $\overline{\alpha \beta}$ = $\bar{\alpha}\bar{\beta}$
\item $\alpha = \bar{\alpha} \Leftrightarrow b = 0$
	
\end{enumerate} 
\end{ndef}

\begin{ndef}[Norma]
	Se define entonces la Norma $N(\alpha) = \alpha \bar{\alpha} = a^2 - nb^2 \in \rac $. Así:
\begin{enumerate}
	\item $N(\alpha \beta) = N(\alpha) * N(\beta)$
	\item $N(\alpha) = 0 \iff \alpha = 0$
\end{enumerate}
\end{ndef}


\begin{nprop}
	$\alpha \in a + b \sqrt{n} \in \ent[\sqrt{n]}$ es invertible $\iff$ $N(\alpha) \in \{-1,1\}$
\end{nprop}



\item Anillos de series.
\begin{ndef}
	Si A es un anillo conmutativo y x es un símbolo que no denota ningún elemento de A. El anillo de series con coeficientes en A, denotado con A[[x]] esta definido como:\begin{center}
$A[[x]] = \{a = \sum_{i=0}^{n}a_i x^i = a_0 + a_1 x^1 + ... + a_n x^n\}$ $a_i \in A$\end{center}
Y definimos la suma y el producto de la siguiente forma:\\
\[(a+b) = \sum_{i=0}^{n}(a_i+b_i)x^i\]\[ (ab) = \sum_{k=0}^n\sum_{i=0}^{k}a_ib_{k-i}\]
\end{ndef}
\end{itemize}

Se puede probar que con estas operaciones de suma y producto, $A[[x]]$ es un anillo y $A[x]$ es un subanillo de $A[[x]]$  

\section{Homomorfismos}

\begin{ndef}
Si $A,B$ son anillos conmutativos, una aplicacion $\varphi: A \to B$ es un homomorfismo si:

\begin{enumerate}
	\item $\varphi(1) = 1$
	\item $\varphi(a+b) = \varphi(a) + \varphi(b)$
	\item $\varphi(ab) = \varphi(a)  \varphi(b)$
	
\end{enumerate}
	
\end{ndef}

Además, decimos que:

\begin{enumerate}
	\item Es monomorfismo si es inyectivo.
	\item Es epimorfismo si es sobreyectivo.
	\item Es isomorfismo si es biyectivo.
\end{enumerate}

\textbf{Propiedades de los homomorfismos}

\begin{itemize}

\item $\varphi(0) = 0$

\item $\varphi(-a) = -\varphi(a) $

\item $\varphi(\sum_{i = 1}^n a_i) = \sum_{i = 1}^n\varphi(a_i)$. 

$\varphi(\prod_{i = 1}^n a_i) = \prod_{i = 1}^n\varphi(a_i)$

\item $\varphi(na) = n\varphi(a)$

\item $\varphi(a^n) = \varphi(a)^n$
	
\end{itemize}


Ya sabemos que $Im(\varphi) = \{ \varphi(x): x \in A\} \leq B$ es un subanillo.

\begin{nprop}
	Si $\varphi$ es monomorfismo, entonces la aplicación restringida:
\[
A \to Im(\varphi)\]
\[
a \mapsto \varphi(a)
\]

es un epimorfismo y por ello es un isomorfismo, podemos decir que $A \cong Im(\varphi)$.\\
\end{nprop}


\begin{nota}
	Se puede probar que $R_n: \ent \to \ent_n$ es un homomorfismo, llamado \emph{Homomorfismo de reducción módulo n}\\
\end{nota}


\begin{nprop}[Homomorfismo de cambio de coeficientes)(1]
 Dado $A$ cualquier anillo conmutativo, conocido $A[x]$.
 
 Si $\varphi:A \to B$ es un homomorfismo de anillos conmutativos, entonces:
 \[
 \exists \varphi: A[x] \to B[x] : \varphi\left(\sum_i a_i x^i\right) = \sum_i\varphi(a_i) x^i
 \]

	
\end{nprop}

\begin{nprop}[Sustición en un polinomio)(2]
	
Si $A$ es un anillo y $a \in A$ entonces: existe un homomorfismo $E_a: A[x] \to A$ tal que $E_a(\sum_i a_i x^i) = \sum_i a_i a^i$.
	
\end{nprop}

\begin{nprop}[3]
Si $A \leq B$ es un subanillo y $b\in B$, la aplicación $E_b:A[x] \to B$ definida como $E_b(\sum_i a_i x^i) = \sum_i a_i b^i$ es un homomorfismo
	
\end{nprop}

\begin{nprop}[Engloba a las anteriores]
Si $\varphi:A \to B$ es un homomorfismo y $b\in B$, la aplicación $\Phi:A[x] \to B$ definida como $\Phi ( \sum_i a_i x^i) = \sum_i \varphi(a_i)b^i \in B$ es un homomorfismo 

\end{nprop}

\begin{proof}
Veamos primero cómo (4) engloba a las demás:
\begin{nlist}

\item $4 \Rightarrow 3$. Se ve tomando como $\varphi$ la inclusión en $B$
\item $4 \Rightarrow 2$. Tomamos esta vez como $\varphi$ la identidad
\item $4 \Rightarrow 1$. Suponemos 4 válido. Probaremos que $\exists\varphi: A \to B[x]$ que lleva $a \to \varphi(a)$.
Ahora, podemos ver que esa aplicación es como usar primero $\varphi$ para ir de A a B y luego usar la inclusión de $B$ en $B[x]$:
\[
A\to B \to B[x]
\]
\[
a \to a \to \varphi(a)
\]

De esta forma, tomamos $x\in B[x]$. Entonces:
\begin{center}
	$A[x]\to B[x]$\\
	$\sum_i a_i x^i \to \sum_i \varphi(a_i)x^i$
\end{center}

Que es justamente el enunciado de la primera proposición.

\end{nlist}

Pasamos ahora a la demostración de la Proposición 4.


Sean $f = \sum a_i x^i$ y $g = \sum b_i x^i \in A[x]$. Entonces: $f+g = \sum c_i x^i$ con $c_i = a_i + b_i$

Si ahora aplicamos $\Phi(f+g) = \sum \varphi(c_i)b^i = \sum \varphi(a_i + b_i)b^i$. 

Como $\varphi$ es homomorfismo, eso es igual a:
 $\sum (\varphi(a_i) + \varphi(b_i))b^i$.
 
 Usando que $B$ es un anillo y por ello hay distributividad, eso es igual a: $\sum (\varphi(a_i)b^i + \varphi(b_i)b^i)$.
 
 Por la asociatividad generalizada eso es igual a: $\sum \varphi(a_i)b^i + \sum \varphi(b_i)b^i = \Phi(f) + \Phi(g)$ Por lo que queda probado para la suma.
 
 Ahora probaremos el producto:
 
 $fg = \sum c_i x^i$ con $c_n  = \sum_{i+j = n} a_ib_j$\\
 
 Así: \[\Phi(f*g) = \sum_n\varphi(c_n)b^n = \sum \varphi(\sum_{i+j = n} a_ib_j)b^n = \sum_n ( \sum_{i+j = n} \varphi(a_i)\varphi(b_j))b^n\]
 
 Desarrollamos por otro lado
 \[
 \Phi(f) * \Phi(g) = (\sum_i \varphi(a_i)b^i)(\sum_j \varphi(b_j)b^j) =^{(1)} \sum_{i,j} \varphi(a_i)b^i\varphi(b_j)b^j =^{(2)} \sum_{i,j} \varphi(a_ib_j)b^{i+j}=
 \]
 
 \[
  = \sum_n(\sum_{i,j: i+j = n} \varphi(a_ib_j)b^n)
 \]
 
 Donde en $(1)$ hemos usado la distributividad general y en $(2)$ hemos usado que estamos en un anillo conmutativo y que $\varphi$ es un homomorfismo. 
 
 Hemos llegado a dos expresiones que son iguales, probando así el resultado.
 
\end{proof}


Sabemos que cada polinomio $f(x)$ constituye una función de evaluación $f(x) \in A[x]$
\[
f(x):B \to B
\]
\[
\hspace{1cm}b \to f(b)
\]

Sin embargo, un polinomio es mucho más que la función de evaluación que él mismo define. Estudiaremos el caso $A[x_1,...,x_r]$

\begin{ndef}[Polinomios de r variables con coeficientes en A]
Sea A un anillo conmutativo. Consideramos $A[x_1,...x_r]$ inductivamente en $r$:\\
Si $r>1$ entonces $A[x_1,...,x_r] = A[x_1,...,x_{r-1}][x_r]$\\
	
\begin{proof}\hfill 

\begin{itemize}

	\item $r=1$:
\[
f(x_1) \in A[x_1] \hspace{1cm} \sum_{i \geq 0}a_i x^i \hspace{0.5cm} a_i \in A \hspace{0.5cm} \exists K: a_{i1} = 0 \hspace{0.3cm} \forall i > K
\]
	\item $r> 1$
\[
f(x_1,...,x_r) = \sum_{i1,...ir} a_{i1},...,a_{ir} x_1^{i1},...,x_r^{ir}: \hspace{0.3cm} \exists K: a_i,...,a_r  = 0 \iff i_s > K
\]
Ahora, si vemos que:
\[
f_{ir}(x_1,...,x_{r-1}) = \sum_{i1,...ir > 0} a_{i1},...,a_{ir} x_1^{i1},...,x_r^{ir-1} \in A[x_1,...,x_{r-1}]
\]
Entonces:
\[
\sum_{ir \geq 0} f_{ir}(x_1,...,x_{r-1})x_r^{ir}= \sum_{ir \geq 0}(\sum_{i1,...,ir > 0}a_1,...,a_r x_1^{i1},...,x_{r-1}^{ir-1})x_r^{ir} = 
\]
\[
=\sum_{i1,...ir} a_{i1},...,a_{ir} x_1^{i1},...,x_r^{ir}
\]


Ahora, definimos $g(x_1,...,x_r) = \sum_{i1,...ir} b_{i1},...,b_{ir} x_1^{i1},...,x_r^{ir}$. Ahora, sumamos:

\[
f(x_1,...,x_r)+g(x_1,...,x_r) = \sum_{i1,...ir} a_{i1},...,a_{ir} x_1^{i1},...,x_r^{ir} + \sum_{i1,...ir} b_{i1},...,b_{ir} x_1^{i1},...,x_r^{ir} = 
\]
\[
 = \sum_{i1,...,ir} (a_i+bi)x^{i1+ij}
\]

Ahora, podemos desarrollar de la misma forma el producto y ver que:

\[
(ax_1^{i1},...,x_r^{ir})(bx_1^{j1},...,x_r^{jr}) = abx_i^{i+j}x_2^{i_2+j_2}...x_r^{i_r+j_r}
\]

\end{itemize}
Por lo que queda probado nuestro resultado.
\end{proof}

\end{ndef}

\begin{ndef} \textbf{(A[x][y])}
\hfill \\
Definimos $f=\sum f_i y^i\ |\ f_i \in A[x] : f_i = \sum_j a_{ij}x^j$\\
Luego, $f = \sum_i (\sum_j a_{ij}x^j)y^i = \sum_{i,j} a_{ij}x^iy^j$\\

Ahora, tomamos $g =\sum_{i,j} b_{ij}x^iy^j$ y sumamos:

\[
f+g = \sum_{i,j}(a_{ij}+b_{ij})x^iy^j
\]

Y, si $A[x][y]$ es un anillo, vemos que la multiplicación se realiza:
\[
(a_{ij}x^iy^j)(b_{mn}x^my^n) = a_{ij}b_{mn} x^{i+m}y^{j+n}
\]


Además, como es un anillo conmutativo $\Rightarrow$ $A[x][y] = A[y][x] = A[x,y]$

	
\end{ndef}


\begin{ndef}
	$A[x_1,...,x_n] = A[x_1,...,x_{n-1}][x_n]$

Se puede probar que $A[x_1,...,x_n] = A[x_{\sigma(1)},...,x_{\sigma(n)}]$ siendo $\sigma$ una permutación de $\{1,2,...,n \}$
\end{ndef}

\begin{nprop}
	
Si $\varphi :A \to B$ es un homomorfismo, $\forall(b_1,...,b_n) \in B^n$ la aplicación:

\[
\Phi:A[x_1,...x_n] \to B \iff \Phi(\sum_{i1,...,in}a_{i1}...a_{in}x^{i1}... x^{in}) = \sum_{i1,...,in}a_{i1}...a_{in}b^{i1}... b^{in} \in B
\]

es un homomorfismo de anillos conmutativos. Es conocido como evaluación de un polinomio en n variables.

\end{nprop}



\begin{nprop}
	Si $\varphi:A \to B$ es un homomorfismo, $\forall\  b \in B\  \exists ! $ homomorfismo definido como:
	
\[
\Phi:A[x] \to B : \begin{cases}
	\Phi(a) = \varphi(a)\ \forall a \in A \\
	\Phi(x) = b 
\end{cases}
\]
\[
\Phi(\sum a_i x^i) = \sum \Phi(a_i x^i) =  \sum \Phi(a_i) \Phi(x)^i = \sum \varphi(a_i) b^i
\]

Además, ya se probó que esto es un homomorfismo de anillos conmutativos.

\end{nprop}

\begin{ncor}
	$A \leq B$ subanillo, $\forall b \in B$ $\exists !$ homomorfismo
\[
E_b: A[x] \to B: \begin{cases}
E_a(a) = a \hspace{0.3cm} \forall a \in A\\
E_b(x) = b
	
\end{cases}
\] 
\end{ncor}

\begin{nota}
	Si $f(x)\in A[x]$ denota un polinomio de $A[x]$, notaremos: $E_b(f(x)) = f(b)$.
	De la misma forma, si $f(x) = \sum a_i x^i \Rightarrow E_b(f(x)) = \sum a_i b^i$
\end{nota}


\begin{nprop}[Evaluación en r-variables]
  Si $\varphi : A \longrightarrow B$ es un homomorfismo de anillos conmutativos, y $b_1,\cdots, b_r \in B$ una lista ordenada. Entonces
  \[
    \exists!\ \phi : A[x_1,\cdots,x_r] \longrightarrow B :\begin{cases}
      \phi(a) = \varphi(a) & \forall a\in A\\
      \phi(x_1) = b_1\\
      \vdots\\
      \phi(x_r) = b_r
    \end{cases}
  \]
\end{nprop}

\begin{proof}
  Si $r=1$, ya está probado. Para $r>1$:
  \[
    \exists\ \psi : A[x_1,\cdots,x_{r-1}] \longrightarrow B : \begin{cases}
      \psi(a) = \varphi(a)\\
      \psi(x_i) = b_i & \forall i=1,\cdots,r-1
    \end{cases}
  \]
  \[  \exists\ \phi : A[x_1,\cdots,x_r] \longrightarrow B \begin{cases}
      \phi(a) = \psi(a) = \varphi(a)\\
      \phi(x_i) = \psi(x_i) = b_i &\forall i=1,\cdots,r-1\\
      \phi(x_r) = b_r
    \end{cases}
  \]
  
  ¿Es único?
  \[
    \phi(\sum_{i1,\cdots,ir} a_{1i}\cdots a_{ir}x_1^{i1}\cdots x_r^{ir} )= \sum _{i1,\cdots,ir} \varphi(a_{i1}\cdots a_{ir})b_1^{i1}\cdots b_r^{ir}
  \]
\end{proof}

\begin{nprop}[Evaluación en subanillos r-variables]
	Si $A\leq B, \forall b_1,\cdots,b_r \in B$ lista ordenada:
	
\[
\exists !\ E_{b_1,\cdots,b_r}: A[x_1,\cdots,x_r] \to B :
\begin{cases}
	a \to a\\
	x_i \to b_i
\end{cases}
\]

Se suele notar $f(x_1,\cdots,x_r) \to f(b_1,\cdots,b_r)$
\end{nprop}

\section{Dominio de Integridad}

\begin{ndef}[Dominio de integridad]
A (anillo conmutativo) es un dominio de integridad si verifica la propiedad:
\[
a \neq 0\ \wedge\ b \neq 0 \Rightarrow ab \neq 0 \iff si \hspace{0.1cm} ab=0 \begin{cases}
	a = 0\\
	b = 0
\end{cases}
\]
	
\end{ndef}

\begin{nprop}[Propiedad de simplificación]
	
A es un dominio de integridad $\iff$ se verifica: $ax=ay$ con $a\neq 0 \Rightarrow x = y$
\end{nprop}
\begin{proof}
	\boxed{\Rightarrow} $a(x-y) = 0$, por ser A dominio de integridad, $x-y = 0 \Rightarrow x=y$ \\
	\boxed{\Leftarrow} $ab = 0$ con $a\neq 0 \Rightarrow b= 0$ pues $a0 = 0; ab = a0; b = 0;$
\end{proof}

\begin{ndef}[Divisor de 0]
	$a\in A$ es divisor de 0 si $\exists b\neq 0 : ab = 0$
\end{ndef}

\begin{nprop}
	Si A es un dominio de integridad $\Rightarrow$ el 0 es el único divisor de 0.
	
	Equivalentemente: A es dominio de integridad $\iff$ no tiene divisores de cero no nulos.
\begin{nlist}
	\item $A\leq B$ y $B$ es D.I. $\Rightarrow$ A es D.I.
	\item Todo cuerpo es D.I.
	\item Si $u \in U(A) \Rightarrow$ u no es divisor de 0 (Supongamos $u*b = 0 \Rightarrow u*u^{-1}*b = u^{-1}*0 \Rightarrow b=0$)
\end{nlist}
\end{nprop}

\begin{nprop}
	Si $|A|< \infty$, A es dominio de integridad $\iff$ A es un cuerpo
\end{nprop}
\begin{proof}
	\boxed{\Leftarrow} Trivial\\
	\boxed{\Rightarrow} $0 \neq a \in A$.
Tomo $\{1,a,a^2,\cdots,a^n\} = \{a^n: n\in N\} \subseteq A$
Como tiene cardinalidad finita: $\exists k \in N : a^n = a^{n+k}$.\\
Pero, por ello: $a^n = a^n a^k; a^n * 1 = a^n * a^k$, luego $a^n$ no es 0 porque A es Dominio de integridad y por ser D.I entonces:
\[
1=a^k \begin{cases}
	k = 1 \Rightarrow a = 1\\
	k > 1 \Rightarrow a^{k-1}*a = 1
	
\end{cases}\]
Con lo que $\exists$ inverso de a = $a^{k-1}$ y como a es un elemento cualquiera, todo elemento tiene inverso, luego es un cuerpo.
\end{proof}


\begin{nprop}
	Todo D.I. es un subanillo de un cuerpo.
\end{nprop}


Primero, presentaremos otros conceptos:

\begin{ndef}[Cuerpo de fracciones de un D.I.]
Sea A un dominio de integridad con $|A| \geq 2$. Consideramos $A$x$A-\{0\} = \{(a,b), a,b \in A\hspace{0.3cm} | b\neq 0\}$

\begin{ndef} Decimos que (a,b) es equivalente a (c,d):
	$(a,b)\sim(c,d) \iff ad = bc$
	
Esta relación es reflexiva, simétrica y transitiva.
\end{ndef}
	
Ahora, considero $a,b\in A \hspace{0.2cm}$. Llamo $\frac{a}{b} = \{(c,d)\hspace{0.2cm} c,d \in A : (c,d) \sim (a,b)\} \subseteq A$x$A-\{0\}$

Y llamo a $\frac{a}{b}$ la fracción a entre b.

\begin{ncor}
	\[\frac{a}{b}= \frac{u}{v} \iff av = bu \iff (a,b) \sim (u,v)\]
\end{ncor}
\begin{proof}
	
	\boxed{\Rightarrow} $(a,b) \in \frac{a}{b} = \frac{u}{v} \Rightarrow (a,b) \sim (u,v) \Rightarrow (av = ub)$
	
	\boxed{\Leftarrow} $(a,b) \sim (u,v)$ Por la transitividad: $\frac{a}{b} \subseteq \frac{u}{v}$ y $\frac{u}{v} \subseteq \frac{a}{b} \Rightarrow \frac{a}{b}= \frac{u}{v}$
\end{proof}
\end{ndef}

Ahora, llamamos $Q(A) = \{\frac{a}{b} | a,b \in A: \hspace{0.25cm} b\neq 0\}$ que es un conjunto de conjuntos, pues ya habíamos definido la fracción $\frac{a}{b}$ como un conjunto.

Sobre él, definimos unas operaciones que nos permitirán ver que es un cuerpo:
\begin{nlist}
	\item Suma: \[\frac{a}{b}+ \frac{c}{d} = \frac{ad + cb}{bd}\]

Ahora, como la fracción $\frac{a}{b}$ es un conjunto, hay que probar que el resultado es único, es decir:
\[
\frac{a}{b}= \frac{a'}{b'} \hspace{0.25cm}y\hspace{0.25cm} \frac{c}{d}= \frac{c'}{d'}  \Rightarrow ab' = a'b \hspace{0.25cm} y \hspace{0.25cm} cd' = c'd
\]
Hay que probar que se cumple:
\[
\frac{ad+cb}{bd} = \frac{a'd'+c'b'}{b'd'}
\]
Equivalementente, tenemos que probar que se cumple:
\[
b'd'(ad+cb) = bd(a'd'+c'b')
\]

Desarrollamos en la izquierda:
\[
b'd'(ad+cb) = b'd'ad + b'd'cb =^{(1)} a'bd'd + b'bc'd
\]
Donde en (1) hemos usado la equivalencia que habíamos dado de $ab' = a'b$ y $cd' = c'd$. Ahora, desarollamos el producto de la derecha y veremos que es igual al resultado obtenido

\[
bd(a'd'+c'b') = bda'd' + bdc'b' = a'bdd' + bb'c'd
\]

Probando la unicidad. 
	
    \item Producto: \[\frac{a}{b} \frac{c}{d} = \frac{ac}{bd}\]

La unicidad del producto se hace desarrollando de la misma manera.
\end{nlist}



Para finalizar, se puede probar que es un cuerpo probando las propiedades de anillo conmutativo y que existe inverso para todo $\frac{a}{b}$.

\begin{nprop}[Fracciones de denominador 1]
  Existe un homomorfismo
  \begin{align}
    i :& A \longrightarrow \mathbb{Q}(A)\notag\\
     \quad  &a \longmapsto \frac{a}{1} = i(a)\notag
  \end{align}

  Que cumple que $i(a+b) = i(a)+i(b)$ y que $i(ab)=i(a)i(b)$, y además es un monomorfismo.
  Así, $A \stackrel{i}{\cong} Img(i) = \{\frac{a}{1} : a\in A\}$ es un isomorfismo y $A\le\mathbb{Q}(A) \text{ con } a=\frac{a}{1}$.
  Con esta identificaión $\frac{a}{b} = \frac{a}{1}\frac{1}{b} = ab^{-1}$
\end{nprop}

\begin{nprop}
  Sea $K$ un cuerpo y $A\le K$, $a,b\in A\ (b\not=0)$.
  \begin{align}
    \implies& a\in K \text{ y } b^{-1}\in K \implies ab^{-1}\in K\notag\\
    \implies&\mathbb{Q}(A)\le K\notag
  \end{align}
\end{nprop}

\begin{nota}
 Sea $K$ un cuerpo. Entonces  $\mathbb{Q}(K)$ es el cuerpo más pequeño que contiene a $K$.
\end{nota}

\begin{nota}
  $A \subseteq \mathbb{Q}(A), A=\text{D.I.} \implies \rac(\rac(A)) = \rac(A)$
\end{nota}


\begin{nprop}
  Sea K un cuerpo, $A\le K$. Si $\forall\ \alpha\in K\quad \exists\ a\in A,\;\; a \not= 0 : a\alpha\in A \implies \rac(A)=K$
\end{nprop}

\begin{proof}
  $\alpha\in K,\; \exists a \ne 0, a\in A : a\alpha=b\in A \implies \alpha = ba^{-1} = \frac{b}{a}\in \rac(A)$
\end{proof}

\begin{ejemplo}
  $\ent[i] = \{a+bi : a,b\in \ent\} \le \rac[i] = \{a+bi : a,b\in \rac\} \implies \rac(\ent[i]) = \rac[i]$
  \[
    \alpha\in\rac[i]\implies\alpha=\frac{m}{n}+\frac{m'}{n'}i\; \implies \ent[i] \ni nn'\alpha = n'm+nm'i \in \ent[i]\notag
  \]
\end{ejemplo}

\begin{nprop}
	Si $A$ es un D.I. $\implies$ $A[x]$ es un D.I. 
\end{nprop}

\begin{ndef}[Grado de un polinomio]

Si $f=\sum a_i x^i \neq 0$ $\implies gr(f)=n\in \mathbb{N}$ si $a_n\neq 0$ y $a_m=0 \hspace{0.3cm} \forall m>n$ 

El coeficiente $a_n$ se denomina coeficiente líder.

\begin{itemize}
	\item Si A es D.I, $f,g \in A[x]\implies gr(fg) = gr(f)+gr(g)$\\
	(Si no es D.I, tenemos que $gr(fg) \le gr(f)+gr(g)$)
\end{itemize}
	
\end{ndef}

\begin{ndef}[Divisibilidad en D.I.]
	
Sea A un D.I. Sean $a,b\in A$. Decimos entonces que a divide a b (a es un divisor de b, b es un múltiplo de a):
\begin{align}
	\Rightarrow& \exists c \in A : b = ac 
\\
	\iff& La \hspace{0.2cm} ecuacion \hspace{0.2cm} ax=b \hspace{0.2cm} tiene\hspace{0.2cm}solucion
\\
\iff& \frac{b}{a} \in A
\end{align}

\begin{proof}
	$\boxed{\implies}$ Si a divide a b $\implies$ $\exists c: b=ac \implies \frac{b}{a} = \frac{ac}{a} = \frac{c}{1} = c \in A$\\
	$\boxed{\Longleftarrow}$ si $\frac{b}{a}\in A \implies \frac{b}{a} = \frac{c}{1} \implies b = ac$
\end{proof}
\end{ndef}
\textbf{Notación:} Si a divide a b, escribiremos $a/b$

\begin{nlist}
	\item Los divisores de 1 son las unidades del anillo, los elementos del grupo $U(A)$
	\item Las unidades son divisores de todos los elementos del anillo.
	\item Dado $a\in A$, los elementos $ua$ con $u\in U(A)$ se llaman \emph{asociados de a}.
	\item Si $u\in U(A)$, $\forall a \in A$, $ua/a$
	
\end{nlist}

\begin{ndef}
	Los divisores triviales de un número son las unidades y sus asociados.
\end{ndef}

\begin{nprop}
	Sean $a,b\neq 0$. Son equivalentes:
	\begin{nlist}
	\item a es asociado de b
	\item b es asociado de a
	\item $a/b \wedge b/a$, los asociados son los elementos que se dividen mutuamente
	
\end{nlist}
\end{nprop}

\begin{ndef}[Irreducible]
	
	Sea $a\in A$(D.I), $a\neq 0, a \notin U(A)$ es irreducible si sus únicos divisores son los triviales

\begin{align}
	\iff& si\hspace{0.2cm} b/a \implies b\in U(A) \vee b \sim a
\\
	\iff& si \hspace{0.2cm} a=bc \implies b\in U(A) \vee c \in U(A)
\\
	\iff& si \hspace{0.2cm} a=bc \implies a\sim b \vee c \sim a
\\
	\iff& si \hspace{0.2cm}a=bc \wedge b \notin U(A) \implies c \in U(A)
\end{align}
	
Propiedades elementales:
\begin{nlist}
	\item Reflexión: $a/a$
	\item Transitividad: $a/b \wedge b/c \implies a/c$
	\item Si $a/b \wedge a/c \implies a/bx +cy\hspace{0.2cm} \forall x,y \in A$ 
	\item Si $a/b \implies \hspace{0.2cm} \forall c \hspace{0.2cm} a/bc$
	\item Si $c\neq 0$ entonces $a/b \iff ac/bc$
\end{nlist}
\end{ndef}

\section{Dominios euclídeos}
\begin{ndef}[Dominios euclídeos]

	Un dominio euclídeo es un dominio de integridad, A, tal que haya definida una función $\varphi: A- \{0\} \to \mathbb{N} $ verificando:

\begin{nlist}
	\item $\varphi(ab) \ge \varphi(a)$
	\item $\forall a,b \in A,\ b \neq 0 \quad \exists q,r \in A : a = bq + r $ con $r=0 \vee \varphi(r) < \varphi(b)$
	\item $\forall a,b \in A,\ b \neq 0 \quad \exists q \in A : a - bq = 0 \vee  \varphi(a-bq) < \varphi(b)$
\end{nlist}	
\end{ndef}

\begin{nota}
	Si A es dominio euclídeo, entonces: 
$b/a \iff$ un resto de dividir a entre b es cero $\iff$ cualquier resto de dividir a entre b es 0 
\end{nota}
\begin{proof}
	$\boxed{\implies}$ Por definición de b/a, $\implies \exists c \in A$ tal que $a=bc$ y por ser A un dominio euclídeo, $\implies \exists q,r \in 		A : 	a = bq + r $ con $r=0 \vee \varphi(r) < \varphi(b)$. La solución 	es evidentemente correcta para $r = 0$, veamos que sucede para $r \neq 	0$.
\\
Supongamos $r \neq 0$, entonces $\varphi(r) < \varphi(b)$. 
$$r = a - bq = bc - bq = b(c-q) \quad \quad c-q \neq 0$$
$$\varphi(r) = \varphi(b(c-q)) \ge \varphi(b) \implies \textup{CONTRADICCIÓN}$$  
\end{proof}

\begin{nth}[Teorema de Euclídes]

	$\forall a,b \in \ent, b \neq 0, \exists !q,r \in \ent$ tales que $a = bq + r $ con $0 \le r < |b|$
\end{nth}


\begin{proof}
Probaremos primero la unicidad. Supongamos 
$$a=bq+r \quad \quad 0\leq r <\abs{b}$$   
$$a=bq'+r' \quad \quad 0 \leq r' < \abs{b}$$ 

distintos. Vamos a ver que $r=r'$ y $q=q'$

\begin{itemize}
	\item Si $r \neq r'$, supongamos $r > r' \implies 0 < r - r' < \abs{b}$
Ahora: 
$$ r-r' = a-bq - a +bq' = b(q' -q )$$
$$r-r' > 0 \implies r - r' = \abs{b(q' -q)} = \abs{b}\abs{q' -q}$$

Pero, como $q\neq q' \implies q'-q \neq 0$ y $q,q' \in \ent \implies \abs{q' -q} \geq 1 $, luego:

$$ r-r' = \abs{b}\abs{q' -q} \geq \abs{b}$$

Por lo que tenemos una contradicción con el comienzo de la suposición.

	\item Ahora, si $r=r'\implies b(q' -q) = 0$ y $b\neq 0 \implies q' -q=0 \implies q' =q$
\end{itemize}

Probamos ahora la existencia. Sean $a,b \geq 0$

\begin{itemize}
	\item Si $ a < b \implies a = b*0 + a$, luego $q=0$ y $r=a$, ya los tenemos.
	\item Si $a \geq b$, llamamos $R=\{a-bx: x\in \nat \quad a \geq bx\}\subseteq \nat$ que es no vacío, pues está al menos $x=1$.
	
	Ahora, por el Principio de buena ordenación, $R$ tiene mínimo. Tomo $r=min(R)$. 
	
$r=a+bq$ para cierto $q \in \nat$ y $ r \geq 0$.

Veremos ahora que $r < b$, llegando a una contradicción.

Supongamos $ r \geq b\implies r' = r-b \geq 0 \implies r' = a-bq -b = a -b(q+1)\implies r' \in R$.

Podemos ver que $r' < r$ (pues $r'=r-b$) $\implies  r'$ está en $R$ y es menor que el mínimo, luego es una contradicción y tenemos que $r < b$

\end{itemize}
Por último, vamos a probar que $0 \leq r < \abs{b}$

Supongamos: 
\[
r=0 \implies a=bq \begin{cases}
	-a = b(-q)\\
	-a = (-b)q \\
	 a = (-b)(-q)\\	
\end{cases}
\]

Ahora, supongamos $r > 0$:

\begin{itemize}
	\item $-a = b(-q) -r = b(-q) -b + b -r = b(-q-1) +(b-r)$ y como $0 < r < b \implies b > b-r > 0$
	\item $-a = (-b)q -r = (-b)q + b - b -r = -b(q+1) + (b-r)$ y por el mismo motivo, $b > b-r > 0$
	\item $a = (-b)(-q) +r \implies 0 < r < b = \abs{-b}$
\end{itemize}

De esta forma, hemos cubierto todos los casos y hemos acabado la demostración
\end{proof}

\begin{ncor}
$\ent$ es un dominio de euclídes con $\varphi = \abs{.}: \ent \to \nat$
\[
\varphi(a) \begin{cases}
	a  \quad \textup{si } a \ge 0\\
	-a \quad \textup{si } a < 0
	
\end{cases}\]
\end{ncor}

\begin{nth}
		$\forall f,g \in A[x]$ donde $g \neq 0 $ y su coeficiente líder es una unidad de $A$, existen polinomios:
	
	\[
	q,r \in A[x] : f = gq + r \quad con \quad
	 \begin{cases}
		r = 0\\
		o\\
		gr(r) < gr(g)
	\end{cases}
	\]
	y que son únicos.
\end{nth}

\newpage
\begin{proof}
	
	Sean : $f = \sum_{i=0}^n a_i x^i$ y $g = \sum_{i=0}^m b_i x^i$ con $b_m \in U(A)$
	
	\begin{itemize}
	
	\item Si $n< m \implies f = f * 0 + f \implies \exists q,r\in A[x] : f = gq+r$ con $g=0$ y $r = f$
	
	\item Si $n\geq m$, razonamos por inducción en $n=gr(f)$
	\begin{itemize}
	
	\item Si $n=0 \implies m = 0$ por tanto $f=a_0$ y $g=b_0$ pero con $b_0 \in U(A)$
	
	De esta forma:
	
	\[
	f = a_0 = \frac{a_0}{b_0}b_0 = \frac{a_0}{b_0}b_0 + 0 = g\frac{a_0}{b_0}
	\]
	
	Podemos tomar como hemos visto $q=\frac{a_0}{b_0}$ y $r=0$ y ya tenemos el q y r que buscábamos.
	
	\item Si $n> 0$, haremos la inducción
	
	Vamos a considerar que $\frac{a_n}{b_m} = a_n b_m^{-1} \in A$ Tomamos entonces $x^{n-m}$.
	
	Consideramos $x^{n-m}g(x)$ y establecemos $f_1 = f - \frac{a_n}{b_m}x^{n-m}g$. Recordaremos esto como $(1)$.
	
	Entonces, podemos ver que $gr(f_1) < n$. Por hipótesis de inducción $\implies \exists q,r \in A[x] : f_1 = gq_1 + r$, que consideraremos como (2).
	
	Ahora, utilizando (1) y (2):
	
	 \[
	 \implies f = f_1 +\frac{a_n}{b_m}x^{n-m}g = gq_1 + \frac{a_n}{b_m}x^{n-m}g +r = 
	 \]
	 
	 \[
	 g(q_1 + \frac{a_n}{b_m}x^{n-m}) + r
	 \]
	
	Encontramos así el q y el r que queríamos, probando la existencia.
	
	Vamos a probar ahora la unicidad.
	
	Sea $f=gq+r$ y $f= gq' +r'$ con
	\[
    \begin{cases}
		 r,r' \neq 0\\
		  \quad o \\
 gr(r) < m 
 \\ gr(r') < m
		   
\end{cases}
	\quad
	\]
	
	Ahora, si $r\neq r'\implies r-r' \neq 0 \implies r-r' = g(q-q') \neq 0$.
	Vemos que $gr(r-r') = gr(g)+ gr(q-q')$.
	
 Como $q-q' \neq 0 \implies gr(q-q') \geq 0 $ y de esta forma: $gr(g)+ gr(q-q') \geq gr(g) = m$.
 
 Sin embargo, habíamos dicho que $r,r'$ eran ambas de grado menor que m luego $gr(r-r') < m$, llegando a una contradicción y probando así el resultado.
\end{itemize} 
	
	
\end{itemize}
\end{proof}

\begin{ncor}
	Si $K$ es un cuerpo, entonces $K[x]$ es un D.E con función euclídea:
	
	\[
	gr: K[x]-\{0\} \to \nat
	\]
	
(función que asigna a cada polinomio su grado)
\end{ncor}

\begin{nota}
	Hacemos el ejercicio de ver si $3x^2 +1$ es divisor de $2x^3 + 4x^2 +4x +3$ en $\ent _5[x]$. (Solución: El resto de la división es 0, con resultado de la división $= 2/3 x + 4/3$)
\end{nota}


\begin{nth}
	Los anillos $\ent[\sqrt{n}]$ para $n=2,3,-1,-2$ son D.E. con función euclídea:
	\[
	\varphi: \ent[\sqrt{n}] \to \nat : \varphi(a+b\sqrt{n}) = \abs{N(a+b\sqrt{n})} = \abs{a^2 - nb^2}
	\]
	
	
\end{nth}
\begin{proof}
	Probaremos que $\forall \alpha, \beta \in \ent[\sqrt{n}]$ con $\beta \neq 0 \quad \quad \exists q,r \in \ent[\sqrt{n}] : \alpha = \beta q + r$ con $r=0$ ó $\abs{N(r)} < \abs{N(\beta)}$:
	
	\begin{itemize}
	\item Si $\abs{N(\alpha)} < \abs{N(\beta)}$ Basta tomar $\alpha  =  \beta * 0 + \alpha$
	
	\item Si $\abs{N(\alpha)} \geq \abs{N(\beta)}$ consideramos entonces  $\frac{\alpha}{\beta} \in \rac[\sqrt{n}]$.
	
	Ahora, $\frac{\alpha}{\beta} = a_1 + a_2\sqrt{n}$ con $a_1, a_2 \in \rac $. Esos $a_1,a_2$ se obtienen usando el conjugado de $\beta$.
	
	Sean $q_1,q_2 \in \ent : \abs{a_1-q_1}\leq 1/2$ y $\abs{a_2-q_2}\leq 1/2$. Esto quiere decir que $q_1$ y $q_2$ son los enteros más cercanos a $a_1,a_2$ respectivamente.
	
	Sea $q=q_1+q_2\sqrt{n}$  y $r = \alpha - \beta q$.
	
	Tomo $\abs{N(r)} = \abs{N(\alpha - \beta q)} = \abs{N(\beta(\frac{\alpha}{\beta}+ q))} = \abs{N(\beta)}\abs{N(\frac{\alpha}{\beta}+ q)}$
	
	Queremos probar que: $\abs{N(\beta)}\abs{N(\frac{\alpha}{\beta}+ q)} < \abs{N(\beta)}$.
	
	Equivalentemente, queremos probar que:
	\[
	  \abs{N(\frac{\alpha}{\beta}+ q)} < 1 \implies \abs{N(a_1+a_2\sqrt{n}-q_1 -q_2\sqrt{n})} = \abs{N((a_1-q_1) + (a_2-q_2)\sqrt{n})} = \]
	  \[
	 = \abs{(a_1-q_1)^2 - n(a_2-q_2)^2} = m \in \rac
	 \]
	 
Vamos a probarlo para los casos que habíamos anunciado en el teorema, $n = -1,-2,2,3$

\begin{itemize}
	\item $n=-1 \implies m= (a_1-q_1)^2 +(a_2-q_2)^2 \leq 1/4 + 1/4 = 1/2 \implies \abs{m} < 1$
	
	\item $n=-2 \implies m = (a_1-q_1)^2 +2(a_2-q_2)^2 \leq 1/4 + 1/2 = 3/4 \implies \abs{m} < 1$
	
	\item $n=2 \implies m = |(a_1-q_1)^2 -2(a_2-q_2)^2 |\implies -1/2 \leq m \leq 1/4 \implies \abs{m} < 1$
	
	\item $n=3 \implies m = |(a_1-q_1)^2 -3(a_2-q_2)^2| \implies -3/4 \leq m \leq 1/4 \implies \abs{m} < 1$
\end{itemize}

Por lo que queda probado el resultado para esos casos.
	
\end{itemize} 
\end{proof}

\begin{ejemplo}
Vamos a tratar de dividir $\alpha=6+10i$ entre $\beta=1+2i$ en el anillo $\ent[i]$. Tenemos que saber si se puede hacer dicha división o no y para ello averiguaremos la norma de ambos números.

$$\abs{N(6+10i)}=36+100=136$$
$$\abs{N(1+2i)}=1+4=5$$

Como $1+2i$ tiene una norma menor que la norma $6+10i$ podemos hacer la división, primero dividiremos como si fuesen numeros complejos normales para hallar nuestro número cociente que será de la forma $q=q_1+q_2i$:

\[
	\frac{6+10i}{1+2i}=\frac{(6+10i)(1-2i)}{(1+2i)(1-2i)}=\frac{6-12i+10i+20}{5}=\frac{26-2i}{5}=\frac{26}{5}-\frac{2}{5}i
\]

Tenemos que $5<\frac{26}{5}<6$ y $5$ es más cercano a $\frac{26}{5}$ que $6$ escogemos $q_1=5$ y por el mismo razonamiento $q_2=0$, de forma que $q=5+0i=5$. A continuación, para hallar el resto $r$ hacemos la siguiente operación:

\[
	r=\alpha-\beta \cdot q = 6+10i - (1+2i)(5)=6+10i-5-10i=1
\]

Finalmente, comprobamos que no nos hemos equivocado:
\[
	(6+10i)=5(1+2i)+1\\
	\abs{N(1)}<\abs{1+2i} \implies 1<5
\]

Viéndose así que el ejemplo está correcto.

\end{ejemplo}

\section{Máximo Común divisor. Dominios de Ideales principales. Ecuaciones Diofánticas en D.I.P.}


\begin{ndef}[Máximo común divisor]

Dados $a,b \in A$ decimos que un elemento $d\in A$ es un mcd de a y b ($d=(a,b))$ si el conjunto de los divisores comunes a $a$ y a $b$ coinciden con el conjunto de los divisores de d. Esto es:

\begin{nlist}
	\item $d/a$ y $d/b$
	\item Si $c/a$ y $c/b \implies c/d$ 
\end{nlist} 
	
\end{ndef}

Propiedades:
\begin{nlist}

	\item $(a,b) = (b,a)$
	
	\item Si $a \sim a'$ asociados y $b \sim b'$ también $\implies (a,b) = (a',b')$
	
	\item $(a,b) = a \iff a/b$. En particular, $(a,0) = a, \quad  \quad (a,1) = 1, \quad (a,u) = 1 \iff u \in U(A)$
	
	\item Si $(a,b) = 1$, a y b se dicen \underline{primos relativos}
	
	\item $((a,b),c) = (a,(b,c)) = (a,b,c)$
	\item $(ac,bc) = c(a,b)$\\
	\begin{proof}
	Primero, llamamos $(ac,bc) = e$ y $(a,b) = d$.\\
	Si a,b o c son 0, se verifica trivialmente. Si no lo son:\\
	\[
	\begin{rcases}
	d/a \implies dc/ac\\
	d/b \implies dc/bc\\
\end{rcases} \implies dc/e \implies \exists u \in A : e = dcu
	\]
	
	$\begin{rcases}e / ac \implies \exists x \in A : ac = ex \implies ac = dcux \implies a = dux\\
	e/bc\implies \exists y \in A : bc = ey \implies bc = dcuy \implies b = duy\end{rcases} \implies \begin{rcases}du/a\\du/b\end{rcases} du/d \\\implies \exists v \in A : d = duv \stackrel{d\ne 0}{\implies} 1=uv \implies u\in U(A) \implies e \sim  dc$

	
\end{proof}

	\item Si $c/a$ y $c/b\implies (\frac{a}{c},\frac{b}{c}) =  \frac{(a,b)}{c}$
	
	\item
	$(\frac{a}{(a,b)}, \frac{b}{(a,b)}) = 1$
	
	\item Si $a/bc \implies a/(a,b)c$\\
	\begin{proof}
	Supongamos que $\exists x \in A : bc = ax  \implies (a,b)c = (ac,bc) = (ac,ax) = a(c,x) \implies a/(a,b)c$
	
\end{proof}

\item Si $a/bc$ y $(a,b) = 1 \implies a/c$ 

\item Si $a/c$ y $b/c$ y $(a,b) = 1 \implies ab/c$
	\item Si $(a,b)=1 $ y $a/bc \implies a/c$
	\item Si $a/c$ , $b/c$ y $(a,b)=1 \implies ab/c$
	\item Si $a/c \implies \exists x$ : $c=ax$. Y $b/c \implies b/ax$ con $(a,b)=1 \implies b/x \implies \exists y$ : $x=by$
	Entonces: 
	\[
		\begin{cases}
			c=ax \\
			x=by \\
		\end{cases} \implies c=aby \implies ab/c
	\]
	\item Si $(a,b)=1$ y $(a,c)=1 \iff (a,bc)=1$ \\
	\begin{proof}
	$\boxed{\implies}$ Sabiendo que: $(ac,bc)=c(a,b)=c$
	
	Tenemos que: $1=(a,c)=(a,(ac,bc))=((a,ac),bc)=(a(1,c),bc)=(a,bc)$, por tanto: $1=(a,bc)$
	
	$\boxed{\Longleftarrow}$ $1=(a,bc)=(a(1,c),bc)=((a,ac),bc)=(a,(ac,bc))=(a,c(a,b))=(\frac{a}{(a,b)}(a,b),c(a,b))=(a,b)(\frac{a}{(a,b)},c)=1 \implies (a,b)\in U(A)\implies (a,b)=1\implies (a,c)\in U(A)\implies (a,c)=1$
\end{proof}
	\item $(a,b)=(a-kb,b)$ $\forall k \in A$
	\item Si $d/b$, $d/a \iff d/(a-kb)$ \\
	\begin{proof}
	$\boxed{\implies}$ Por la propiedad de combinación lineal se confirma.
	
	$\boxed{\Longleftarrow}$ Igual que la otra implicación pero tomando $a=(a-kb)+kb$
	\end{proof}
\end{nlist}

\begin{nota}
En $\mathbb{Z}[\sqrt{n}]$ si $\alpha$ es un divisor propio de $\beta \implies N(\alpha)$ es un divisor propio de $N(\beta)$ en $\mathbb{Z}$.
\end{nota}

\begin{ejemplo}
Realizamos un ejemplo en el que se puede probar que, usando la Nota anterior, 3 y $(1+\sqrt{5})$ son irreducibles	
\end{ejemplo}


\begin{ndef}[Ideal/Ideal Principal]
	En un anillo se llama ideal a un subconjunto suyo no vacío que es cerrado para la suma y para múltiplos. Dicho de otra manera: \\
	Si A es un anillo conmutativo, un subconjunto $\emptyset \not= I\subseteq A$, es un ideal si:
	\begin{nlist}
		\item $a,b\in I \implies a+b\in I$
		\item $a\in I \implies ax \in I$
	\end{nlist}
	Si $a\in A$, $aA=(a)=$\{$ax: x\in A$\} es el ideal principal generado por a.
\end{ndef}


\begin{ndef}[DIP: Dominio de ideales principales]
	Un DIP es un anillo en el cual todo ideal es principal.
\end{ndef}
\begin{nth}
	Todo dominio euclídeo es un dominio de ideales principales: $DE \implies DIP$
\end{nth}
\begin{proof}
	Sea A un DE con función euclídea $\varphi: A-\{ 0 \} \longrightarrow \mathbb{N}$ y $I\subseteq A$ un ideal:
	\begin{itemize}
		\item Caso $I=\{ 0 \} = (0)= 0A \implies $ trivial
		\item Consideremos $I\neq \{ 0\}$, $\emptyset \neq \{ \varphi (x): x\in I, x\neq 0\} \subseteq \mathbb{N}$, sea $\varphi (b)$ el mínimo de este conjunto, donde $b\in I, b\neq 0 \implies I=(b)$. Probamos esto con la doble inclusión:
		$\boxed{\subseteq}$ $b\in I \implies (b)\subseteq I$ \\
		$\boxed{\supseteq}$ $a\in I$; $\exists q,r\in A: a=bq+r$. Supongamos que $r\neq 0 \implies r=a-bq\in I$ con $\varphi (r) < \varphi (b)$, esto es imposible puesto que b es el mínimo, luego $r=0 \implies a\in (b) \implies I\subseteq (b)$
	\end{itemize}
\end{proof}


\begin{nth}
	Si $A$ es un DIP, $\forall a,b \in A \quad \exists d = (a,b)$. Además, $\exists u,v \in A : d = au+bv$. A esta igualdad se le llama Identidad de Bezout, y u y v son los coeficientes de Bezout, que no son únicos.
	
\end{nth}

\begin{proof}
	Sea $\emptyset \ne I(a,b) = \{ax + by : x,y \in A\} \subseteq A$
	
	Vemos que:\\
$(ax+by)+(ax'+by') = a(x+x') + b(y+y') \implies$ cerrado para la suma.\\
$(ax+by)z = a(xz)+b(xz)\implies$ cerrado para el producto

Ahora, como es un ideal $\implies \exists d\in A : I(a,b) = (d)$ con $(d)=\{dx: x \in A\}$.
$d\in I(a,b) \implies \exists u,v \in A : d=au+bv$.

Ahora, veamos que $d$ es mcd de a y b.\\
$a\in I(a,b) \implies a \in (d) \implies d/a$\\
$b\in I(a,b) \implies b \in (d) \implies b/d$

Por lo que d es divisor común. Ahora, sea $c: c/a$ y $c/b \implies c/(au+bv=d) \implies c/d$.
Hemos encontrado así un divisor común que es dividido por cualquier divisor común, por tanto es el mcd. 
\end{proof}

\subsection{Ecuaciones diofánticas en D.I.P.}

En cualquier anillo, llamamos ecuaciones diofánticas a aquellas que son de la forma:
\[
ax +by  = c
\]
\begin{nlist}
	\item Sea $d=(a,b)\implies$ entonces la ecuación tiene solución $\iff d/c$
	\item Supongamos que tiene solución. Supongamos también que $d= au+bv \circledast$
\[
\frac{a}{d}= a',\quad \frac{b}{d}=b',\quad \frac{c}{d}=c'\implies da'x + db'y = dc' \implies d(a'x+b'y)=dc'\]
\[ \implies a'x+b'y = c'
\]
Esta ecuación tiene las mismas soluciones que la ecuación diofántica inicial. Llamaremos a esta la ecuación 'reducida'.\\
$\circledast\quad d/a ,\quad d/b \implies 1=a'u +b 'v $. Podemos hallar así los coeficientes de Bezout.\\

Como $c'= a'(c'u)+b'(c'v)$ y ahí tenemos una solución particular. Conociendo esta, podemos hallar TODAS las soluciones. Si llamamos $x_0 =c'u$ e $y_0 = c'v$

\item Solución general \[\begin{cases}
	x = x_0 +kb'\\
	y= y_0 -ka'
\end{cases} k\in A\]
Si $(x_0,y_0)$ es la solución particular, entonces la solución general es el conjunto de los $(x,y)$ que hemos dado arriba.
\\
\begin{proof}[Demostración de iii)]

\[
a'x + b'y = a'(x_0 +kb') + b'(y_o - ka') = a'x_0 + a'kb' + b'y_0 - a'kb' = 
\]
\[
a'x_0 +b'y_0 = c'
\]

Suponer ahora que (x,y) es cualquier solución:$\implies a'x +b'y = c'$. Por hipótesis: $a'x_0 +b'y_0 = c'$. Si restamos esas dos ecuaciones queda: $a'(x-x_0) +b(y-y_0) = 0 \implies a'(x-x_0) = b(y_0-y)$. Denotamos a esta ecuación como 3.

Ahora, $b'/(a'(x-x_0))$ pero $b'$ y $a'$ son primos entre sí, luego $b'/(x-x_0)\implies \quad \exists k \in A : (x-x_0) = kb'$. Llamamos a esta ecuación 1, y además despejando en ella vemos $x = x_0 + kb'$, una solución de x.

Análogamente, podemos ver que $a/(b(y_0-y)) \implies a/(y_0-y) \implies \exists h \in A: y_0 -y = a'h \implies y = y_0 - ha'$, solución de y. Llamamos a esa ecuación la 2.

Falta probar que $k = h$, pero sustituyendo las ecuaciones 1 y 2 en 3, vemos que $a'kb'  = b'ha' \implies k = h$
	
\end{proof}

\end{nlist}

\begin{nprop}[Algoritmo de Euclides para el cálculo del MCD]
Supongamos que tenemos dos elementos $a,b$ y queremos hallar su mcd.

\begin{itemize}
	
	\item Si $b=0 \implies (a,b) = (a,0) = a$. Igual si $a=0$
	
	\item Si $a \ne 0 \ne b$
	
	Construimos una sucesión: $r_1,r_2,\cdots, r_n, \cdots , r_m, r_{m+1} = 0$.\\
	Recordamos que A es un D.E con función euclídea $\varphi : A-\{0\} \to \nat $\\
	Si $\varphi(a) \geq \varphi(b)\implies r_1 = a$ y $r_2 = b$. En el otro caso, lo hacemos al revés, es decir  $r_1 = b $ y $r_2 = a$.\\
	Si $r_{n-1} \ne 0 \implies r_n= $ resto de dividir $r_{n-2}$ entre $r_{n-1} \implies$
	
	\[
	r_{n-2} = r_{n-1}q_{n-2} + r_n \begin{cases}
	 r_n = 0 \\
	  \varphi(r_n) \leq \varphi(r_{n-1})
\end{cases}
	 \]
La idea es ir reduciendo de la forma:
\[ (a,b) = (r_1,r_2) = \cdots = (r_n,r_{n+1}) = \cdots = (r_m, r_{m+1}) = (r_m,0) = r_m\]
Obteniendo los cocientes de la forma:
\[\begin{cases}
	r_{n-2} = au_{n-2} + bv_{n-2}\\

r_{n-1} = au_{n-1} + bv_{n-1}\\

r_{n-2} - r_{n-1}q_{n-2} = r_n = a(u_{n-2} - q_{n-2}u_{n-1}) + b(r_{n-2}-q_{n-2}v_{n-1})\\
\cdots \\

d = r_m = au + bv
\end{cases}
\quad
\]
	
	\end{itemize}
\end{nprop}

\begin{ejemplo}
	Un agricultor lleva al mercado 80 sandías y 30 melones. La venta le ha sido rentable, pues ha vendido cada pieza por más de 3 euros, que es lo que le costó producirlos. Vuelve a casa con 600 euros. Calcular precio de sandías y melones.\\
	(El ejercicio se resuelve resolviendo la ecuación diofántica $80x +30y = 600$ , hallando primero la solución general que viene dada por $x = -60 + 3k$ ; $y = 180 - 8k$ y luego tomando que x e y tienen que ser mayores que 3, viendo que la solución es que $k=22$).
\end{ejemplo}

\section{Mínimo común múltiplo. Ecuaciones en congruencias}
\begin{ndef}[Mínimo común múltiplo]
	Sea $a, b \in A =DI \\ m \in A $ es un mínimo común múltiplo de a y b, notando por $m = mcm(a,b) = [a,b]$  \\
Si se verifica que el conjunto de los múltiplos comunes a ambos es igual al conjunto de múltiplos de $m$. Esto implica: 
\begin{enumerate}
\item $a/m \ y \ b/m$
\item Si $a/c$ y $b/c \Rightarrow m/c$ \\
\end{enumerate}

Del mismo modo se define para $[a_1, a_2, ..., a_r],r \in \mathbb{N}$.

\textbf{Propiedades. }
\begin{nlist}
\item Si $a\sim a' \ y \ b\sim b' \Rightarrow [a,b]=[a',b']$
\item $[a,b] = [b,a]$
\item $[a,0] = 0$
\item $[a,1] = a$
\item $[a,[c,b]]= [[a,c],b] = [a,b,c]$
\item $[ac,bc] =[a,b]c$\\

\begin{proof}[Demostración del último.]
	
	Supongamos que $c\ne 0$, pues si no es trivial.\\
	Como $c/ab\implies c/[ca,cb] \implies \exists q \in A : [ac,bc] = cq\quad (1)$\\
	Por otro lado, sea $m = [a,b]; \implies a/m \quad y \quad b/m \implies ac/mc \quad y \quad bc/mc \implies cq/mc$.\\
	Como $c\ne 0 \implies q/m$.\\
	Por otro lado, $ca/cq \quad y \quad cb/cq \implies$ como $c\ne 0 \implies a/q \quad y \quad b/q \implies m/q.$\\
	Hemos llegado a que $q/m \quad y \quad m/q \implies $ son asociados $\implies q=[a,b]$.\\
	Ahora, basta llevarnos esto a $(1)$ en esta demostración para ver que:\\ $[ac,bc] = c[a,b]$
\end{proof}
\end{nlist}
\end{ndef}


\begin{nprop}
	Si A es un DIP $\implies \forall a,b \in A \quad \exists [a,b]$
\end{nprop}
\begin{proof}
	Consideramos $aA = (a)$, el ideal principal generado por a. De la misma forma, consideramos $bA = (b)$, el ideal principal generado por b.\\
	Ahora, tomamos $aA \cap bA \implies$ los números que están simultáneamente en los múltiplos de ambos.
	Ahora, esto es cerrado para sumas y para productos, por tanto también es un ideal.\\
	Por último, por estar en un DIP $\implies$ el ideal es principal y por tanto:\\ $\implies aA\cap bA = mA \implies m = [a,b]$ 
\end{proof}


\begin{nth}
	Sea A un DI en el cual $\exists(a,b) \quad \forall a,b \in A.$ Entonces, $\exists [a,b] \quad \forall a,b \in A $ y se verifica que: $[a,b](a,b) = ab$
\end{nth}
\begin{proof}
	Sean $0 \ne a,b \in A$. Llamamos $d=(a,b) \implies \begin{cases}
	a = a_1 d\\
	b = b_1 d
\end{cases}$

Podemos observar que:

\[
m = \frac{ab}{d} = a_1b = ab_1
\]
De esta forma, nuestra prueba termina si comprobamos que $m = [a,b]$. Tenemos ya que claramente $a/m \quad y \quad b/m$.\\

Sea $m_1 = a/m_1 $ y $b/m_1$, tenemos que probar que $m/m_1$. Para esto, lo que hay que probar es que $(m,m_1) = m$.\\
Para ello, vamos a llamarlo $k = (m,m_1) \implies k/m$. Llamo $d_1 = \frac{m}{k} \implies m =_{(1)} d_1k$ para un cierto $d_1$. Guardamos la igualdad de $(1)$ para usarla después.\\
Ahora, lo que bastaría probar es que $d_1 \in U(A)$:\\
Tenemos que $a/m$ y $a/m_1 \implies a /k \implies k = au$. Podemos hacer lo mismo con b para ver que $k = bv$. Esto ocurre para ciertos $u$ y $v$.\\
Ahora, usando la igualdad del principio ($m = a_1b = ab_1$) y el $(1)$ podemos ver que $\begin{rcases}
	m = a_1b = kd_1 = bvd_1 \implies a_1 = vd_1\\
m = ab_1 = kd_1 = aub_1  \implies  b_1 = ud_1
\end{rcases} \implies \begin{rcases}
	a = a_1d = vd_1 d\\
b = b_1d = ud_1d
\end{rcases} \implies d_1d/a \quad d_1d/b$


$\implies d_1d /d \implies \exists x \in A : d = dd_1 x \implies 1 = d_1 x \implies d_1 \in U(A)$.\\
$\implies m,k$ son asociados y como k era $mcd(m,m1) \implies$ m también lo es.
\end{proof}

\subsection{ Congruencias}
Sea A un anillo, $I \subset A$ un ideal. $a,b \in A$ son 'congruentes módulo I' si $a-b \in I\\$(Equivalentemente, si $\exists x \in I : a = b+x$). La notaremos:
\[
a \equiv b mod(I) \quad o \quad a \equiv_ I b
\]
Otra notación. En un DIP $I = (m) = mA$
\[
a \equiv bmod(mA) \to^{notacion} a \equiv b mod(m) ( \iff m/a-b \iff a-b = qm)
\]
Para algún $q$ en el último paso, y en ese caso $\iff a = b+qm$ para algún q.\\

\textbf{Propiedades}

\begin{nlist}
	\item $\equiv$ es una relación de equivalencia.
	\begin{itemize}
		\item $a \equiv a$
		\item $a \equiv b \iff b \equiv a$ (dem:\ $a-b = (-1)(b-a) \in I$)
		\item $a \equiv b$ y $b \equiv c \implies a \equiv c$ (dem:\ $a-b\in I, b-c \in I \implies a-c \in I$)
	\end{itemize}
	
	\item $a\equiv b \iff \forall c : a+ c \equiv b + c$
	\item $a\equiv b$ y $c \equiv d \implies a+c \equiv b + d$ (dem: usando $(ii)$ y $(i)$)
	\item $a \equiv 0 \iff a \in I$
	\item $a \equiv b \implies \forall c : ac  \equiv bc$
	\item $a \equiv b$, $c \equiv d \implies ac \equiv bd$ (dem:\ $(v)$ y luego uso\ $(i)$)
	\item $ac \equiv bc mod(mc) $ y $c \ne 0 \implies a\equiv b mod(m)$\\

\begin{proof}
	$ac \equiv bc mod(mc) \iff mc/(a-b)c \iff^{c\ne 0} m/a-b \iff a\equiv b mod(m)$
\end{proof}
	\item Si $(c,m) = 1$, entonces: $ac\equiv bcmod(m) \iff a \equiv bmod(m)$\\
	\begin{proof}
	$ac\equiv bc \implies m/(a-b)c \implies$, como $(c,m) = 1 \implies m/a-b$
\end{proof}
	
	
\end{nlist}

\subsection{Ecuaciones en Congruencias}
\begin{nprop}[Ecuaciones en congruencias]

Estudiaremos la ecuación $ax \equiv b mod(m)$ (1)
\begin{itemize}
	\item Si $m = 0 \implies$ la ecuación es $ax = b$
	\item Si $a = 0\implies $ la ecuación es $0x \equiv 0 mod(m) \implies$ tiene solución: todo el anillo
	\item $a,b \ne 0$
	\begin{enumerate}
	
	\item Si $d = (a,m)$ la ecuación tiene solución $\iff  d/b$\\
	\begin{proof}
	(1), tiene solución $\iff \exists x \in A : ax \equiv bmod(m) \iff \exists x \in A : m/ax-b \iff \exists x,y \in A : (ax-b) = my \iff \exists x,y \in A: ax -my  = b$, que es una ecuación diofántica, que sabemos ya que tiene solución $\iff d =(a,m) \ y  \ d/b$
\end{proof}
	\item  Supongamos que tiene solución. Consideramos $a'= \frac{a}{d}$, $b'= \frac{b}{d}$ y $m' = \frac{m}{d}$.
	
	Ahora, usando $(1) = da'x \equiv db'mod(dm')$, esta es equivalente a $a'x \equiv b' mod(m')$ a la que llamaremos (2). Esta es su reducida. Tiene las mismas soluciones pero $(a',m') = 1$.
	
	Podemos hallar los coeficientes de Bezout: $u,v \in A :  1 = a'u + b'v$. Esto nos lleva a ver que:
	\[
	a'u \equiv 1 mod(m') \implies a'ub' \equiv b' mod(m')
	\]
	Y así tenemos que $x_0 = ub'$ es una solución particular.
	
	\item La solución general es de la forma: $x = x_0 +km' \quad k \in A$. Equivalentemente, es de la forma $x \equiv x_0 mod(m')$\\
	\begin{proof}
	Si $x_0$ es una solución particular $\implies a'x_0 \equiv b'mod(m')$\\
	Si sustituimos $x_0$ por x pues son congruentes obtenemos: $a'x \equiv b' mod(m')$.\\
	Vamos a suponer que:
	\[
	\begin{rcases}
	a'x \equiv b' mod(m')\\
	a' x_0 \equiv b' mod(m')
\end{rcases} \implies a'x \equiv a'x_0 mod(m')
	\]
Por la transitividad. Pero $a'$ y $m'$ son primos entre sí, luego $x \equiv x_0 mod(m')$
\end{proof}

\item Diremos que una solución particular $x_1$ es óptima si $x_1 = 0$ ó $\varphi(x_1) < \varphi(m')$ siendo $\varphi$ la función euclídea de A.\\
Si $x_0$ es cualquier solución particular, entonces:
\[
x_0 = m'q + x_1 \begin{cases}
	x_1 = 0\\
	o\\
	\varphi(x_1) < \varphi(m')
	
\end{cases}
\]
Y $x_1$ es una solución parcial óptima. En este caso, la solución general óptima es: $x\equiv x_1 mod(m')$
\end{enumerate}
\end{itemize}
	
\end{nprop}

\subsection{Sistemas de Ecuaciones en Congruencias}

En este caso, vamos a abordar un problema en el que tenemos un sistema de ecuaciones en congruencias, que sabemos que se puede expresar de la forma:

\[
\begin{rcases}
	a_1x \equiv b_1 mod(m1)\\
	a_2x \equiv b_2 mod(m2)
\end{rcases}(1) \begin{rcases}
	x \equiv a mod(m)\\
	x \equiv b mod(n)
\end{rcases}(2)
\]

\begin{nth}[Teorema Chino]
	El sistema tiene solución $\iff a \equiv b mod((m,n))$
\end{nth}
\begin{proof}
	Sea $d=(m,n)$. \\
	Si tomamos $x = a+km; \quad \exists k: a+km \equiv b mod(n) \iff km \equiv b-a mod(n) \iff d/b-a \iff b\equiv amod(d)$
\end{proof}

Ahora, supuesto que tiene solución, vamos a hallar las soluciones particular y general del problema.\\
Si $y_0$ es una solución particular de $my \equiv b-a mod(n)$, entonces su solución general es: $$y = y_0 + k\frac{n}{(m,n)} \quad k \in A$$
Entonces $x_0 = a + my_0$ es una solución particular del sistema dado en (2) y por tanto la solución general de 2 viene dada por:

\[
x = a +m(y_0 +  k\frac{n}{(m,n)}) \quad k \in A
\]
\[
 = a+my_0 + k \frac{mn}{(m,n)} = x_0 +k[m,n] \quad k \in A
\]
\[
\implies x \equiv x_0 mod [m,n]
\]
Pero si $x_0 = [m,n]q + x_1$ con $x_1 = 0 $ ó $\varphi(x_1) < \varphi([m,n])$ entonces tenemos que $$x_0 \equiv x_1 mod([m,n])$$

Y obtenemos que la solución general óptima de nuestro sistema es:
\[
x \equiv x_1 mod([m,n])
\]


\begin{nth}[Teorema de Ruffini]
	Si $f(x) \in A[x], \ a \in A$ entonces $f(a) =$ resto de dividir $f$ entre $x-a$.
	Equivalentemente, $f = (x-a)q+r$ donde $r\in A$. Así, $f(a) = r$.
	
	En forma de congruencias: $f \equiv f(a)mod(x-a)$.
\end{nth}
\section{Anillos de Congruencias. Conjuntos Cocientes}

Sea $A$ un anillo cualquiera. Sea también $I \subseteq A$ un Ideal de $A$.\\
Sabemos que $a \equiv b mod(I) \iff a-b \in I$. Vamos a denotar:
\[
[a] = \{ b: b\equiv a mod(I)\} = \bar{a} = a + I
\]
Que sabemos que es un subconjunto de A y al que llamaremos la clase de congruencia de $a$.\\
Denotaremos también:
\[
 A/I = \{[a]: a \in A\}
\]


\textbf{Propiedades}:\\
\begin{itemize}
	\item $  [a]=[b] \iff a \equiv b mod (I) $
	\item $[a] + [b] = [a+b]$
	\item $[a][b] = [ab] $
	\item Si $[a] = [a']$ y $[b] = [b'] \implies \begin{cases}
	[a+b] = [a' +b']\\
	[ab] = [a'b']
\end{cases}$\\

\begin{proof}
	$a\equiv_I a'$ y $b \equiv_I b' \implies \begin{cases}
	a+b \equiv_I a'+b'\\
	ab \equiv_I a'b'
\end{cases}\implies \begin{cases}
	[a+b] = [a'+b']\\
	[ab] = [a'b']
\end{cases}$
\end{proof}
	\item $[0] = I$
\end{itemize}

\begin{nprop}
	Si $f_i:A \to B$ es un homomorfismo de anillo, $Img(g) = \{f(a): a \in A\} \leq B$ es un subanillo. Entonces, $Ker(f) = \{a \in A: f(a) = 0\}$ es un ideal.
	\begin{proof}
	Vamos a probar que este ideal es cerrado para sumas y para múltiplos. Para ello, en ambos casos usaremos que $f$ es un homomorfismo.
	\[
	Si f(a) = 0 \ y \ f(b) = 0,\]\[
	f(a+b) = f(a)+f(b) = 0 + 0 = 0\]\[
	f(ab) = f(a)f(b) = 0*0 = 0
	\]
\end{proof}
Además, $f$ es un monomorfismo $\iff Ker(f) = 0$
\begin{proof}
	\boxed{\Rightarrow} Trivial\\
	\boxed{\Leftarrow} Si $f(a) = f(b) \implies f(a-b) = 0 \implies a-b \in Ker(f)$ pero hemos dicho que $Ker(f) = 0 \implies a-b = 0 \implies a = b$
\end{proof}
\end{nprop}

\begin{nth}[Teorema de Isomorfía]
	Si $f:A \to B$ es un homomorfismo, se induce un isomorfismo de anillos:
	\[
	A/Ker(f) \cong Im(f)
	\]
	\[
	F:[a] \longmapsto f(a)
	\]
	
Además, $F$ está bien definida: es biyectiva y, por tanto, es un isomorfismo.

\begin{proof}
	Vamos a probar que está bien definida (inyectividad y sobreyectividad) y que es un homomorfismo.
	
	Veamos primero que si $[a] = [b] \implies f(a) = f(b)$\\
	Si $[a] = [b] \implies a \equiv b mod(Ker(f)) \implies a = x+b$ para algún $x\in Ker(f)$\\
	$\implies f(a) = f(b+x) = f(b) +f(x) = f(b) + 0 = f(b)$\\
	
	Vamos a ver ahora que es un homomorfismo $F:[a] \mapsto f(a)$
	\begin{itemize}
		\item $F([a]+[b]) = F([a+b]) = f(a+b)$ Pero como f es un homomorfismo por hipótesis $\implies f(a)+f(b) = F[a] + F[b]$
		\item $F([a][b]) = F([ab]) = f(ab)$ pero f vuelve a ser un homomorfismo, luego $f(a)f(b) = F[a]F[b]$
		\item $F(1) = f(1) = 1$
\end{itemize}
Probamos la inyectividad:\\
Suponemos $F[a] = F[b] \implies f(a) = f(b) \implies f(a-b) = 0 \implies a-b \in Ker(f) \implies a \equiv bmod(Ker(f)) \implies [a] = [b]$.
%Tomo $[a] \in Ker(F) \implies F[a] = 0 \implies f(a) = a \implies a \in Ker(f) \implies [a] = [0] = 0$

Probamos la sobreyectividad:\\
Sea $b \in Im(f) \implies \exists a \in A : f(a) = b \implies F[a] = f(a) = b$. Como $f$ es sobreyectiva, $\forall b \in Im(f), \ \exists [a]$ que se aplica en $b$.
\end{proof}

\end{nth}

\begin{nprop}
	Sea $A$ un Dominio Euclídeo con función euclídea $\varphi :A-\{0\} \to \nat$ tal que en $A$ hay unicidad de cocientes y restos (Esto es: $\forall a,b \in A: b\ne 0 \implies \exists ! q,r\in A : a=bq+r \begin{cases}
	r = 0 \\
	\varphi(r) < \varphi(b)
\end{cases}$). Si seleccionamos un $b \in A, b \ne 0$ tal que $\varphi(1) < \varphi(b)$ entonces:\\
$\forall a \in A, R_b(a)=$ resto de dividir a entre b; $R_b(a) = r \iff \begin{cases}
	a \equiv r mod(b)\\
	r = 0 \quad o \quad \varphi(r) < \varphi(b)
\end{cases}$
Ahora, llamaremos: $$ A_b = \{ R_b(a): a \in A\} \subseteq A$$ que cumple:

\begin{enumerate}
	\item Si $r\in A_b \implies R_b(r) = r$
	\item $R_b(a+a') = R_b(R_b(a) + R_b(a'))$\\
	\begin{proof}
	$R_b(a+a') \equiv a+a' \equiv_b R_b(a) +R_b(a') \equiv_b R_b(R_b(a) + R_b(a'))$
\end{proof}
	\item $R_b(aa') = R_b(R_b(a)R_b(a'))$
\end{enumerate}

Además, se define la suma y el producto de $r,r' \in A_b$ de la forma:
\begin{itemize}
	\item $r+r' = R_b(r+r')$
	\item $rr' = R_b(rr')$
\end{itemize}

Con estas operaciones, $A_b$ es un anillo.
\end{nprop}


Se comprueba que si $f: A \to B$ es un isomorfismo, entonces:

Si $a \in U(A)$,$\exists a^{-1}: aa^{-1} = 1\implies f(a)f(a^{-1}) = f(1) = 1 \implies f(a) \in U(B)$

Así, surge la aplicación: $f: U(A) \to U(B)$ isomorfismo en la que si $b\in U(B) \implies \exists b^{-1} \in B : bb^{-1} = 1$


Pero también:\\
 $\begin{rcases}
	\exists a \in A: f(a) = b\\
	\exists a' \in A: f(a') = b'
\end{rcases} f(aa') = f(a)f(a') =bb' = 1 \implies aa' = 1 \implies a \in U(A)$

\begin{ndef}[Divisores de Cero]
	Si a es divisor de cero de $A$, $\exists a'\ne 0 : aa' = 0 \implies f(a)f(a') = 0$ con $f(a') \ne 0 \implies f(a)$ es divisor de cero en $B$
	
	Análogamente, surge el isomorfismo entre los divisores de cero de dos Anillos A y B:
	$$f: DivCero(A) \to DivCero(B)$$ Si b es divisor de B y $b=f(a)$ para cierto $a \in A\implies  a \in $ D. Cero de A.
	
	$\implies \exists b' \ne 0 : bb' = 0$
	
	Luego:
	
	Si $b' = f(a')\implies f(a)f(a') = 0 \implies f(aa') = 0 \implies aa' = 0$ y con $a'\ne 0 \implies a$ es divisor de cero de A.
\end{ndef}

\begin{nprop}
		Sea A un D.E. con función euclídea $\varphi$ donde hay unicidad en cocientes y restos y si $m\in A: m\ne 0$ y $ \varphi(1) < \varphi(m)$.
	
		Consideramos $A_m = \{R_m(a): a \in A\}$ donde, como ya sabemos, $\begin{cases}
	r+r' = R_m(r+r')\\
	rr' = R_m(rr')
\end{cases}$ Veamos que es un homomorfismo.\\
\begin{proof}
	$R_m(a+b) = R_m(a) + R_m(b) = R_m(R_m(a)+R_m(b))$ donde la primera suma es en A, la segunda es en $A_m$ y la tercera dentro del paréntesis vuelve a ser en $A$\\
	$R_m(ab) = R_m(a)R_m(b) = R_m(R_m(a)R_m(b))$ luego el producto también está bien definido.\\
	Por último: $R_m(1) = 1$ por $\varphi(1) < \varphi(m)$
\end{proof}
\end{nprop}
Además, $Im(R_m) = A_m$ y $Ker(R_m) = (m) = mA = \{mx : x\in A\}$.

También hay un isomorfismo:

\[
A/(m)  \cong A_m
\]\[
[a]\mapsto R_m(a)
\]\[
[r] \mapsfrom r
\]

De esta forma, podemos llevarnos los problemas a otros anillos para facilitar su resolución.

\begin{nprop} Sea $a \in A$
	\begin{nlist}
	\item $[a] \in U(A/(m)) \iff (a,m) = 1 $
	\item $a \in A_m$, entonces $a \in U(A_m) \iff (a,m) = 1$
	\item Todo elemento de $A/(m)$ es una unidad o divisor de cero
	\item Todo elemento de $A_m$ es una unidad o divisor de cero
\end{nlist}
\end{nprop}
\begin{proof}
	Vamos a probar i) y iii). Luego ii) y iv) son consecuencia del isomorfismo entre $A/(m)$ y $A_m$.\\
	Sea $[a] \in U(A/(m)) \iff \exists x \in A: [a][x] = [1] \iff \exists x \in A : ax \equiv 1mod(m) \iff mcd(a,m) = 1$.
	
	Sea $a \in A/(m) : a \notin U(A/(m)) \implies mcd(a,m) = d \ne 1$. Ahora, sea $a = da'$ y $m=dm'$
	Pero m no divide a $m'$ pues en otro caso: $m'=mx \implies m' = dm'x \implies 1 = dx \implies d \in U(A)$, contradicción.
	Ahora, tomo $[a][m'] = [am'] = [da'm'] = [a'm'] = [0]$ por ser un múltiplo de m' (congruencia módulo m'), así que [a] es divisor de cero.
\end{proof}

\begin{ncor}
	En las mismas condiciones, son equivalentes:
	
	\begin{nlist}
	\item m es irreducible
	\item A/(m) ó $(A_m)$ es DI
	\item A/(m) ó $(A_m)$ es un cuerpo
\end{nlist}
\end{ncor}
\begin{proof}
	$\boxed{iii) \implies ii)}$ Todo cuerpo es un dominio de integridad.\\
	$\boxed{ii) \implies iii)}$ Supongamos $[a] \in A/(m) $ si $[a] \ne [0]$ entonces, por la proposición anterior, $a$ es una unidad.
	
	$\boxed{i) \implies iii)}$ Sea $[a] \in A/(m), [a] \ne [0] = 0 \implies m$ no divide a $a \implies (a,m) = 1$, como m es irreducible, sus únicos divisores son m y 1 salvo asociados $\implies [a] \in U(A/(m))$
	
	$\boxed{ii)\implies i)}$ Supongamos que m no es irreducible $\implies m = ab$ con a y b divisores propios $\implies m$ no divide a $a$ y m no divide a $b$ $\implies [a] \ne 0 $ y $[b] \ne 0$ en $A/(m)$ pero $[a][b] = [ab] = [m] = [0] = 0$ y como [a] y [b] son distintos de cero $\implies A/(m)$ no es DI, en contradicción con la hipótesis.
\end{proof}

En $\ent_n$ si $p \geq 2$ es un irreducible y $\ent_p $ es un cuerpo. En general, si $K$ es un cuerpo y $f(x) = \sum a_i x^i \in K[x]$ es de grado $n \implies K[x]_{f(x)}$
 es un cuerpo $\iff f(x)$ es irreducible, con $K[x]_{f(x)} = \{b_0 +b_1 x + ... + b_{n-1}x^{n-1} :  b_i \in K \}$


En particular, si $p$ es un irreducible de $\ent$ y $f(x)$ es un irreducible de $\ent_p[x]$ de grado $n \implies \ent_p[x]_{f(x)}$ es un cuerpo con $p^n$ elementos. Lo notamos $F_pn = \ent_p[x]_{f(x)}$

Salvo isomorfismos es el único cuerpo con $p^n$ elementos , al variar p y n obtenemos todos los cuerpos finitos que existen.
\subsection{Ecuaciones en $\ent_n$}

Vamos a intentar ahora encontrar una solución para una ecuación $ax = b$ en $\ent_n$ con $a \ne 0$.

\begin{enumerate}
	\item Tiene solución $\iff d =(a,n)/b$
	
	\item Si tiene solución, tiene exactamente $d$ soluciones distintas.
\end{enumerate}

\begin{proof}
Utilizaremos el siguiente isomorfismo para simplificar esta prueba: $\ent_n \cong \ent/(n)$.\\
\begin{enumerate}
	\item $\exists x \in Z_n :  ax = b \iff \exists[x] \in \ent/(n) : [a][x] = [b] \iff \exists x \in \ent /(n) : ax \equiv_n b \iff d/b$. Quedando probado 1.
	\item  Para demostrar 2., suponemos que d/b.\\
	 Sean $a' = \frac{a}{b}, b' = \frac{b}{d}, n' = \frac{n}{d}$. Recuperamos la propiedad anterior, $a'x \equiv b'mod(n')$. De este expresión obtenemos la solución óptima: $x_0 : a'x_0 \equiv b'mod(n'), 0 \le x_0 \textless n'$. Siendo la solución general, $x \equiv x_0 mod(n')$.\\
Ahora, si $x = x_0 + kn' \quad k \in \ent$, los $x$ que satisfacen nuestro problema original son los restos de estos elementos:\quad $\{x_0,\ x_0 + n',\ x_0 +2n',...,\ x_0 + (d-1)n'\}$, si $k \in \ent, 0 \le k \textless d \implies x_0 + kn' \textless \frac{n}{d} + (d-1)\frac{n}{d} = \frac{dn}{d} = n$. Por tanto, estas son las únicas soluciones.

Podemos expresar las soluciones como $\{ [x] \in \ent /(n) : x = x_0 + kn', k \in \ent\}$. Si $k \in \ent \ y \ k = qd + r \ con \ 0 \le r \textless d, \ x_0 + kn' = x_0 + (qd + r)n' = x_0 + rn' + qdn'= x_0 + rn' +qn \implies [x_0 + kn'] = [x_0 + rn']$. Las soluciones de la ecuación original serán $\{[x_0],\ [x_0 + n'],\ [x_0 +2n'],...,\ [x_0 + (d-1)n']\} \subseteq \ent /(n) \cong \ent _n, \forall r, 0 \le r \le d - 1.$
$\{x_0,\ x_0 + n',\ x_0 +2n',...,\ x_0 + (d-1)n'\} \subseteq \ent _n$
\end{enumerate}
	
	
	
\end{proof}

\section{Función de Euler.}
\[
  \varphi : \mathbb{N}-\{0\} \longrightarrow \mathbb{N}  
\]

definida de la siguiente forma $\forall n\ge 1$:
\[
  \varphi(n) = |\{m\in \mathbb{N} : 1\le m \le n \text{ y } (m,n)=1\}|
\]

Que es igual al número de naturales menores que $n$ y primos con él.
%\vspace{5mm}

\begin{nprop}
  Si $(m,n) = 1 \implies \varphi(mn) = \varphi(m)\varphi(n)$
\end{nprop}

Necesitamos algunos resultados para probar esto:

\begin{ndef}[Anillo producto.]
Si $A$ y $B$ son anillos:
\[
  A\times B = \{(a,b) : a\in A, b\in B\}
\]

Donde se definen:
\begin{align*}
  (a_1,b_1)+(a_2,b_2) &= (a_1+a_2, b_1+b_2)\\
  (a_1,b_1)(a_2,b_2) &= (a_1a_2, b_1b_2)
\end{align*}

$U(A\times B) = U(A)\times U(B)$

\end{ndef}

\begin{nota} Un anillo producto nunca es un cuerpo.
  
\end{nota}

\begin{nth}[Versión clásica del teorema chino del resto.]
Si $(m,n)=1 \implies \mathbb{Z}_{mn} \cong \mathbb{Z}_m \times \mathbb{Z}_n \iff \mathbb{Z}_{|(mn)} = \mathbb{Z}_{|(m)} \times \mathbb{Z}_{|(n)}$ 
\end{nth}

\begin{proof}
\begin{align*}
f:& \mathbb{Z} \longrightarrow \mathbb{Z}_{|(m)}\times \mathbb{Z}_{|(n)}\text{ es un homomorfismo de anillos.}\\
& a \longmapsto ([a]_m, [a]_n)
\end{align*}

Probaremos que es sobreyectivo:\\\\
$([b]_m,[c]_n)$ ¿$\forall b,c \in \mathbb{Z}\quad \exists a \in \mathbb{Z} : [a]_m = [b]_m \text{ y } [a]_n = [c]_n$?

¿$\forall b,c \in \mathbb{Z}\quad \exists a \in \mathbb{Z} : \begin{cases}
  a \equiv b mod(m)\\
  a \equiv c mod(n)
\end{cases}$?$ \iff b \equiv c mod(m,n) \implies b \equiv_1 c$

Sí es sobreyectiva.

$ker(f) = (mn)$, pues $m$ y $n$ son primos entre sí. Ahora, como $f$ es sobreyectiva, por el teorema de isomorfía tenemos el resultado.
\end{proof}

\begin{ncor}
  Si $(m,n) = 1 \implies \varphi(mn) = \varphi(m)\varphi(n)$
\end{ncor}

\begin{proof}
  $\varphi(mn) = |U(\mathbb{Z}_{mn})| = |U(\mathbb{Z}_{|m}\times\mathbb{Z}_{|n})| = |U(\mathbb{Z}_{|m}) \times U(\mathbb{Z}_{|n})| = \varphi(m)\varphi(n)$
\end{proof}
\begin{nota}
  Como sabemos (aunque no lo hayamos probado), si $n\in \mathbb{N}, n= p_1^{e_1}\dots p_n^{e_n} \text{ con } p_i \ne p_j,\; p_i\ \text{primo de $\mathbb{Z}$ (irreducible)}$. Así, $\varphi(n) = \varphi(p_1^{e_1})\dots\varphi(p_n^{e_n})$.
  
  $\varphi(p^e) = p^e(1-\frac{1}{p}) = p^e-p^{e-1}$
\end{nota}



\begin{nth}
	Si $(a,m)=1 \implies a^{\varphi(m)} = 1$ en $ \mathbb Z_{m}$ \hspace{0.5cm} $a \in \mathbb Z_{m} $

Por tanto, $ a^{\varphi(m)-1} = a^{-1} $ en $ \mathbb Z_{m} $

\end{nth}


\begin{nth}[Teorema de Euler]
	$ \forall a \in \mathbb Z $ si $(a,m) = 1 \implies a^{\varphi(m)} \equiv 1 $ mod(m)
\end{nth}


\begin{nth}[Teorema pequeño de Fermat]
	Si p es irreducible, $ \forall a \in \mathbb Z, $ $ a^{p} \equiv a \cdot mod(p) $ 
\end{nth}



\begin{ncor}
	Si p es irreducible, $ \forall a \in \mathbb Z_{p} $	 \hspace{0.25cm} $ a^{p} = a $ en $ \mathbb Z_{p} $
\end{ncor}


\begin{proof}
	Partiendo de la hipótesis del teorema demostraremos el corolario:

$ m = p, $ $ a \neq 0 $ en  $ \mathbb Z_{p}$ (si $a=0$ es trivial) $ \implies (a,p) = 1 \implies a^{\varphi(p)} = 1 $ en $ \mathbb Z_{p} $

$ \varphi(p) = p(1-\frac{1}{p}) = p-1 \implies a^{p-1} = 1 $ en $ \mathbb Z_{p} \implies a^p=a $ en $ \mathbb Z_{p} $
	
\end{proof}



\section{Dominio de Factorización Única (DFU)}

Un Dominio de Integridad, A, es llamado un DFU si $\forall a \in A, a\ne 0$ y $a\notin U(A)$, entonces $\exists$ irreducibles $q_1 ,...,q_r \in A : a = q_1 ...q_r$ tales que la factorización es esencialmente única en el sentido de que si $q_1' ,...,q_s' \in A $ con $q_j$ irreducible $\implies r = s$ y $\exists \sigma:\{1,...,r\} \cong \{1,...,r\}$ una permutación tal que si $q_i'$ es asociado con $q_{\sigma(i)}$

\begin{ndef}[Conjunto representativo de los irreducibles de A]
	Si A es un DFU, vamos a denotar $\mathcal{P}= $ un conjunto representativo de los irreducibles de A.
\end{ndef}

\begin{itemize}
\item $\forall p \in \mathcal{P}$, p es un irreducible
\item $\forall p, q \in \mathcal{P}$, $p$ y $q$ no son asociados entre sí
\item $\forall p $ irreducible de A, $\exists q \in \mathcal{P} : p \sim q$   
\end{itemize}

Supongamos ahora que estamos en un DFU y hemos seleccionado un conjunto $\mathcal{P}$.

Si tenemos un $a\in A,\ a \notin U(A), a \ne 0$, por definición existirán $q_1 ,...,q_r \in A $ irreducibles tales que $a = q_1...q_r $. Entonces, $\forall i = 1,...,r\ \exists p_1,...,p_r \in \mathcal{P} : q_i = u_ip_i $ con $u_i \in U(A)$. Así, $a$ se puede expresar como: $a = (u_1...u_r)p_1...p_r$ pero todos los $u_i$ son unidades del anillo, luego $\exists p_1,...,p_r \in \mathcal{P} $ y $u \in U(A) : a = u(p_1...p_r)$. Esta descomposición es esencialmente única pero de forma más fuerte que antes. Además, es única salvo orden de escritura de los $p_i$.

\begin{ejemplo}
	En $\ent$ el $-6$ se puede escribir como $(-1)*2*3$ ó como $(-1)*3*2$
\end{ejemplo}

Estos $p_i$ pueden repetirse, así que si agrupamos en términos obtenemos:

$\forall a \in A,\ a \ne 0\ \exists p_1,...,p_s \in \mathcal{P}$ con $p_i \ne p_j$, $e_1,...,e_s \in \ent$ y $ u \in U(A) : a = u(p_1^{e_1}...p_s^{e^s})$


\begin{ndef}[]
	Si $p\in \mathcal{P}$ y $a \in A,\ a \ne 0$ denotamos $e(p,a)$ como:
	\begin{nlist}
	\item exponente con que $p$ aparece en la factorización de $a$, si aparece. $e(p_i,a) = e_i$ \quad $i = 1,...,s$
	\item 0 en otro caso. $e(p,a) = 0 \quad \forall p \notin \{p_1,...,p_s\}$

\end{nlist}
	
	Vamos a asumir a partir de ahora que $a^0  = 1 $ en cualquier anillo. Así, podemos ver que:
	
	$$\forall a \in A, \quad a = u(\prod _{p \in \mathcal{P}}p^{e(p,a)})$$
\end{ndef}

\textbf{Propiedades:}
\begin{nlist}
	\item $e(p,ab) = e(p,a) + e(p,b)$.\\
	
	\begin{proof}
	
	Con el $a$ anterior y $b= v(\prod _{p \in \mathcal{P}}p^{e(p,b)})$. Entonces $ab = uv(\prod _{p \in \mathcal{P}}p^{e(p,a)+ e(p,b)})$
\end{proof}

\item $a,c \ne 0$ y $a/c \iff \forall p \in \mathcal{P}, \quad e(p,a) \leq e(p,c)$\\

\begin{proof}
	$\boxed{\Rightarrow} \quad \exists b: c =ab \implies \forall p \in \mathcal{P}, \quad e(p,c) = e(p,ab) = e(p,a)+e(p,b) \geq e(p,a) $
	
	$\boxed{\Leftarrow} $ ¿Existe un $b$ tal que $ab= c$? 
	
	Si $c$ es: $c = v(\prod _{p \in \mathcal{P}}p^{e(p,c)})$
	
	Si tomamos $b = (u^{-1}v)(\prod _{p \in \mathcal{P}}p^{e(p,c) - e(p,a)}) $ y multiplicamos por $a$, obtenemos $c$.
\end{proof}
\end{nlist}

\begin{nprop}
En un DFU existen mcd y mcm de cualesquiera elementos. Así:

\[
\forall a,b \ne 0 \quad (a,b) = (\prod _{p \in \mathcal{P}}p^{min\{e(p,a),e(p,b)\}})
\]
\[
[a,b] = (\prod _{p \in \mathcal{P}}p^{max\{e(p,a),e(p,b)\}})
\]
	
\end{nprop}
\begin{proof}
	Probaremos el caso del mcd, el caso de mcd se hace de la misma forma.\\
	Vamos a llamar $d=(a,b)$. $d$ divide a $a$ porque $e(p,d) = min \{e(p,a), e(p,b)\} \leq e(p,a)$ y, por tanto, a $b$ también. Ahora, si tenemos un divisor común cualquiera , digamos $c \implies c/a$  y $c/b \implies e(p,c) \leq e(p,a),e(p,b) \implies e(p,c) \leq e(p,d) \implies c/d$ luego $d$ es un máximo común divisor.
\end{proof}

\begin{ndef}[Elemento Primo]
	Si $A$ es un D.I. un elemento $p\in A , p \notin U(A), p \ne 0$ es llamado “primo" si se verifica la siguiente propiedad:
	
	Si $p$ no divide a un elemento $a$ ni a un elemento $b \implies p$ no divide a su producto.\\
	Equivalentemente: si $p/ab \implies p/a$ o $p/b$  
	
\end{ndef}

\begin{nprop}
\begin{nlist}
	\item Todo primo es irreducible en cualquier anillo A
	\item Si A es un DFU, entonces todo irreducible es primo.
\end{nlist}
\end{nprop}
\begin{proof}
	\begin{nlist}
	\item Sea $p$ un elemento primo. Supongamos que $p=ab$, producto de dos elementos, bastaría ver que uno de ellos es un asociado solo. Ahora, como $p/p \implies p/ab \implies  p/a $ o $p/b \implies a \sim p$ o $b \sim p$
	
	\item $p \in \mathcal{P}$, veamos que $p$ es primo.
	
Supongamos que $p/ab$. Veamos que $p$ divide a $a$ o a $b$. 
Si $p/ab\implies e(p,ab) \geq 1$ pero sabemos que $e(p,ab) = e(p,a)+e(p,b) \implies e(p,a) \geq 1$ o $e(p,b) \geq 1$. Si ocurre lo primero, $p/a$ y si ocurre lo segundo $p/b$ luego si $p$ divide a un producto, entonces $p$ divide a uno de los dos elementos del producto.
\end{nlist}
\end{proof}



\begin{nth}
	Sea A un D.I. Entonces, son equivalentes:
	\begin{nlist}
	\item A es un DFU
	\item \begin{enumerate}
	\item Todo elemento no nulo ni unidad de A factoriza como producto de irreducibles
	\item Todo irreducible de A es primo
\end{enumerate}
	\item \begin{enumerate}
	\item Idem
	\item $\forall a,b \in A \ , \exists \ mcd(a,b)$
\end{enumerate}
\end{nlist}
\end{nth}
\begin{proof}
	Que $(i)\implies (ii)$ es trivial. Veamos que $(ii) \implies (i)$
	
	Lo único que falta para probar que es un DFU es probar que las factorizaciones son únicas. Sea $a= p_1...p_r = q_1...q_s$ con $p_i,q_j$ irreducibles. Vamos a ver que $r=s$. Para ello, vamos a hacer una inducción en $r$.
	
	\begin{itemize}
	\item Caso $r=1 \implies p_1 = q_1...q_s$. Ahora, ¿puede ser $s>1$? Los $q$ no son unidades, pues son irreducibles, por tanto, si $s$ fuese mayor que 1 serían los divisores propios de $p_1$, pero eso no puede ocurrir porque $p_1$ es irreducible. Como no se puede dar que $s>1$ $\implies$ $s=1=r \implies p_1 = q_1$
	
	\item Si $r> 1$ y usando la hipótesis de inducción, entonces $s > 1$.
	
	Nos fijamos en $p_1$, que es claro que divide a $a\implies p/(q_1...q_s)\implies^{p_1 \ primo} \exists j : p_1 /q_j$ y reordenando podemos suponer que $p_1/q_1$.
	
	Esto implica que $p_1 \sim q_1 \implies \exists u \in U(A): q_1 = up_1$. Ahora nos podemos llevar la expresión a la igualdad de a($a= p_1...p_r = q_1...q_s$ ) $\implies p_1...p_r = up_1q_2...q_s $ y podemos reducir dividiendo por $p_1$ y nos queda $p_2...p_r = uq_2...q_s$. 
	
	Ahora, usando la hipótesis de inducción, nos queda en cada lado $r-1$ elementos y $s-1$ elementos y por tanto $r-1 =s-1\implies r=s$
\end{itemize}

Ahora, que $(i) \implies (iii)$ es trivial. Como (i) y (ii) son equivalentes, basta probar que $(iii) \implies (ii)$

Queremos probar que todo irreducible es primo. Sea $p$ un irreducible. Supongamos que $p$ no divide ni a $a$ ni a $b$. Probaremos que entonces, no divide al producto $ab$

Es fácil ver que $(p,a) = 1$ y que $(p,b) = 1$. Ahora, por la propiedad del mcd que asegura que:
\[
(a,b) = 1 \ y \ (a,c) = 1 \iff (a,bc) = 1
\]

Entonces, $(p,ab) = 1 \implies$ $p$ es primo relativo con el producto, por tanto, $p$ no divide al producto y así $p$ es primo.
\end{proof}

	\textbf{Lema previo:} En un DIP, toda cadena ascendente de ideales es estacionaria. En otras palabras, si A es un DIP, $I_1 \subseteq I_2 \subseteq ... \subseteq I_{n}$ es una sucesión de ideales creciente respecto a la inclusión (cada uno está incluido en el siguiente). $\implies \exists m : I_m = I_{m+1} = ... = I_{m+k} \quad k \geq 1$.
	
	\begin{proof}[Demostración del lema.]
	Podemos ver que: 
	\[
	I = U_{n\geq 1} I_n = \{a \in A : \exists n \ con \ a \in I_n\} \quad \forall a,b \in I \implies \exists n: a,b \in I_n\]\[ \implies a+b \in I_n \implies a+b \in I
	\]
	Realizando la prueba análoga para el producto, I es un ideal y por estar en un DIP, es principal $\implies \exists a \in A : I = (a) = \{ax : x \in A\}$. Que es no vacío, pues $a \in I$.
	
	Si $a\in I$, en particular estará en alguno de los $I_i$ de la unión $\implies \exists m : a \in I_m \implies (a) \subseteq I_m$, pero  $I = (a) \subseteq I_m \subseteq I_{m+k} \subseteq I \implies I_i = I_j$ para todo $i$ y $j$. 
	
\end{proof}

\begin{nth}
	Todo DIP es un DFU (Lo cual implica que todo DE es un DFU)
\end{nth}
\begin{proof}
	Tenemos que probar que en un DIP todo elemento se puede descomponer como producto de irreducibles. Para ello, vamos a negar la tesis. Supongamos que estamos en un DIP y que en ese anillo existen elementos distintos de cero que no son unidades y no se pueden descomponer como producto de irreducibles.
	
	Supongamos que $a$ es un elemento de esa “clase”. Entonces $\exists  \ a'$ divisor propio de $a$ que también es de esa “clase de elementos”.
	
	Podemos asegurar que $a$ no es un irreducible, pues no admite factorización en irreducibles y si fuera irreducible, él mismo sería una factorización como irreducibles. $\implies \exists b,c : a = bc$ con b y c divisores propios. Entonces, uno de los dos (b ó c) no puede admitir una factorización como producto de irreducibles, pues si no, $a$ admitiría esa factorización. Entonces, llamamos $a'$ a $b$ o a $c$ según sea el que no admita esa factorización.
	
	Ahora, vamos a construir una sucesión $\{a_n\} \in A$ con $a_1 = a,\ a_{n+1}=a'_n$. Cada elemento siguiente, es un divisor propio del anterior y es de la “clase” que establecimos al principio (no es cero, ni una unidad, ni se puede factorizar en producto de irreducibles). Esto implica que $a_{n+1} = a_n'$ y $a_{n+1}$ no es sociado con $a_n$.
	
	Si consideramos los ideales principales generados por los elementos de esta sucesión, podemos ver que (como $a_{n+1}$ es divisor de $a_n$):
	
	\[
	\implies (a_1) \subset (a_2) \subset ... \subset (a_n) \subset (a_{n+1}) \subset ...	\]
	
	Y en esta cadena no hay igualdades, pues si $(a_n) = (a_{n+1}) \implies a_{n+1} \in (a_n) \implies a_n/ a_{n+1}$ y esto no puede ocurrir.
	
	Pero esto contradice el lema que hemos visto anteriormente, por tanto hemos probado así que todos los elementos deben tener una factorización y por tanto estamos en un DFU.
\end{proof}


\begin{nprop}
	Si $\alpha \in \ent[\sqrt n]$ es un divisor propio de $\beta$ en $\ent[\sqrt n]$ entonces $N(\alpha)$ es un divisor propio de $N(\beta)$ en $\ent$
\end{nprop}
\begin{proof}
	Como $\alpha$ es un divisor de $\beta$ entonces $\exists \gamma \in \ent[\sqrt n] : \beta = \alpha \gamma \implies N(\beta) = N(\alpha)N(\gamma) \implies N(\alpha)/N(\beta)$.
	
	Ahora, la norma de $\alpha$ no puede ser ni 1, ni -1 pues si no sería una unidad y, por tanto, no sería divisor propio; $\alpha$ no puede ser un asociado pues si no, $\gamma$ sería un divisor propio también, luego $N(\alpha)$ tiene que ser un divisor propio de $N(\beta)$

\end{proof}
\begin{ncor}
	Si $N(\alpha) = \pm p$ con $p $ un primo de $\ent, p\geq 2\implies \alpha$ es irreducible en $\ent[\sqrt n]$ 
\end{ncor}
\begin{ncor}
	Si $\alpha$ es primo en $\ent[\sqrt n]\implies N(\alpha) = \pm p $ ó $\pm p^2$ con $p \geq 2 $ un primo de $\ent$. Además, si $N(\alpha) = \pm p^2 \implies \alpha \ y \ p$ son asociados en $\ent[\sqrt n]$
\end{ncor}
\begin{proof}
	Supongamos que $\alpha \in \ent[\sqrt n]$, primo. Consideramos su norma: $N(\alpha)$ que no es ni 1 ni -1 pues si no sería una unidad. Así: $N(\alpha) = p_1...p_r$ con $p_i\in \ent$ primos, lo que implica que $\alpha \bar{\alpha} = p_1...p_r \implies \alpha/p_1...p_r $ pero $\alpha$ es primo, luego $\exists i \in \{1,...,r\}: \alpha / p_i \implies \exists p \geq 2 $ primo de $\ent$ tal que $\alpha / p \ en \ \ent[\sqrt n]$.
	
	 Esto implica $p = \alpha \beta \ con \ \beta \in \ent[\sqrt n] \implies p^2 = N(\alpha) N(\beta) \implies N(\alpha)/p^2 \implies N(\alpha) = \pm p$ ó $\pm p^2$, como queríamos.\\
	 Si $N(\alpha) = p^2 \implies N(\beta) = 1 \implies \beta$ es una unidad $\implies \alpha$ y $p$ son asociados
\end{proof}
\begin{nota}
	Si estuviéramos en un DFU, ser irreducible y ser primo son equivalentes, luego estos enunciados valdrían igual para elementos primos.
\end{nota}

\begin{ejemplo}
	Factorizar $2i$ y $11+7i$ en producto de irreducibles(primos por estar en un DFU).
	
	\begin{enumerate}
	\item Primero, calcularemos su norma. $N(11+7i) = 11^2 + 7^2 = 170$
	\item Factorizamos la norma en $\ent$. $170 = 2*85 = 2*5 *17$.
	\item Ahora, los factores irreducibles serán los enteros de Gauss cuya norma sea un primo o el cuadrado de un primo. Por tanto, un divisor de este número será un entero de Gauss $Z[i]$ cuya norma sea un divisor de la norma de $11+7i$, por tanto su norma será 2, 5 ó 17 o producto entre esos números.
	\item  $N(a+bi) = a^2 +b^2 = 2 \iff a=\pm 1$ y $b= \pm 1.$ Los enteros de Gauss de norma 2 son: $1+i,1-i,-1+i,-1-i$, es decir $1+i$ y sus 3 asociados.
	\item Ahora, tenemos que plantearnos si $1+i / 11+7i$, vemos que la división es: $11+7i/1+i = 9-2i\in \ent[i]$. Además, como $1+i$ tiene norma 2, que es un primo de $\ent$ luego ya tenemos un irreducible por el corolario 7. 
	\item Tenemos que repetir el proceso para $9-2i$.
	\item Su norma es $N(9-2i) = 5*17$ pues es el de antes quitándole el irreducible cuya norma vale 2.
	\item Buscamos los enteros de Gauss cuya norma valga 5. $N(a+bi) = a^2 +b^2 = 5 \iff a = \pm 1$ y $b =\pm 2$ ó $a \pm 2$ y $b = \pm 1$.
	
	Estos son: $1+2i$ y sus asociados para el primer caso y $2+i$ y sus asociados para el segundo caso.
	
	\item Ahora, tenemos que ver si estos dividen a $9-2i$. 
	\begin{itemize}
	\item $9-2i/2+1  = \frac{16}{5} + \frac{13}{5}i \notin \ent[i]$
	\item $9-2i /1+2i = 1-4i \in \ent[i]$
\end{itemize}

Por lo que tenemos que $11+7i = (1+i)(1+2i)(1-4i)$ y ahora tenemos justo 3 irreducibles con las normas que buscábamos, luego tenemos hecha la factorización en irreducibles.
	
\end{enumerate}

Ahora, haciendo lo mismo para $2i$ vemos que $2i = (1+i)^2$.

	
\end{ejemplo}
\begin{ejemplo}[2]
	Vamos a factorizar 180 en $\ent[i\sqrt2]$. Para ello, vemos que $180 = 2^2 * 3^2 * 5$. Recordamos que en este anillo, $N(a+b\sqrt2) = a^2 +2b^2$ y $U(\ent[i\sqrt2]) = \pm 1$
	
	Ahora, como $N(2) = 4$, un divisor propio del 2 tendrá por norma un divisor propio del 4 en $\ent$. En $\ent$, sólo el 2 es divisor propio del 4. Por ello, tenemos que plantearnos la ecuación $a^2+2b^2 = 2$. Entonces, los únicos elementos que hay que tienen son $\sqrt{-2}$ y $-\sqrt{-2}$. 
	Ahora vemos si alguno de estos divide a 2:
	\[
	\frac{2}{\sqrt{-2}} =  \frac{2*(-\sqrt{-2})}{\sqrt{-2}*(-\sqrt{-2})}= \frac{2(-\sqrt{-2})}{2} = -\sqrt{-2} \ \in \ent[i\sqrt2]
	\]
	Ahora, como $N(\sqrt{-2})=2$ que es un primo en $\ent \implies \sqrt{-2}$ es un primo de $\ent[\sqrt{-2}] = \ent[i\sqrt2]$ y por ello $2=-(\sqrt{-2})^2$.
	
	Seguimos, haciendo lo mismo con el 3. $N(3) = 9$. ¿Existen a y b : $a^2 + 2b^2 = 3$? Vemos que tomando $a=\pm 1$ y $b=\pm1$ se puede llegar a la igualdad. Es decir, tenemos los elementos: $\{1+\sqrt{-2}, \ 1-\sqrt{-2}, \ -1-\sqrt{-2}, \ - 1 + \sqrt{-2}\}$.
	
	Probamos dividiendo $3/(1+\sqrt{-2}) = 1 - \sqrt{-2}\implies 3 = (1+\sqrt{-2})(1-\sqrt{-2})$ y ambos son irreducibles.
	
	Hacemos lo mismo con el 5. Tenemos que discutir la ecuación: $a^2+2b^2 = 5$. Sin embargo, en este caso no hay ningún elemento que tenga solución luego 5 es primo en $\ent[i\sqrt2]$.
	
	Por tanto, la factorización de 180 en $\ent[i\sqrt2]$ es: $180 = (\sqrt{-2})^4*(1+\sqrt{-2})^2*(1-\sqrt{-2})^2*5$
\end{ejemplo}

\begin{ejemplo}[Ejemplo de anillo que no es un DFU] $\quad$

Como ejemplo, vamos a probar que \textbf{\underline{$\ent[\sqrt{-5}]$ no es un DFU.}} En este anillo, $N(a+bi\sqrt5) = a^2 + 5b^2$ y $U(\ent[\sqrt{-5}])= \{\pm 1\}$

Vamos a considerar el elemento $1+i\sqrt5$. Su norma es: $N(1+i\sqrt5) = (1+i\sqrt5)(1-i\sqrt5) = 6 = 2 * 3$. ¿Es este elemento irreducible?

Vamos a plantearnos qué elementos del anillo $\ent$ tienen norma 2 o norma 3.
\begin{itemize}
	\item En la ecuación $a^2+5b^2= 2$ no hay soluciones en $\ent[i\sqrt5]$
	\item En la ecuación $a^2+5b^2 = 3$ tampoco hay soluciones en este anillo.
\end{itemize}
	Como no tiene divisores propios, entoces este elemento es irreducible. Su conjugado, por el mismo motivo, también es un irreducible.
	
	Ahora, por la norma de $1+i\sqrt5$ hemos obtenido una factorización del 6 en producto de irreducibles. Pero el 6 también es $2*3$ en este anillo. Esta podría ser otra factorización en irreducibles de 6 en $\ent [i\sqrt5]$. El 2 no tiene ningún divisor propio en este anillo, pues es el único divisor de 4 en $\ent$ luego el 2 es irreducible en este anillo. Lo mismo ocurre con el 3.
	
	Por tanto, tenemos dos descomposiciones del 6 en producto de irreducibles, que no son iguales ni asociados luego este anillo no puede ser un DFU.
	
	Ahora, el 2 es un irreducible, \underline{¿es 2 primo?} Vemos que $2/6$, y $6 = (1+i\sqrt5)(1-i\sqrt5)$. Si fuese primo, necesitaríamos $2/1+i\sqrt5$ ó $2/1-i\sqrt5$ y eso no ocurre pues sus normas no se dividen en $\ent$ por tanto el 2 no es primo.
\end{ejemplo}

\subsection{$\ent[x]$ es un DFU y no es un DIP}

Vamos a estudiar ahora que este anillo es un DFU sin ser un DIP ni un DE.

Bastaría de hecho tomar los elementos 2 y x para ver que $(2,x) = 1$ y no existen los coeficientes de Bezout para estos elementos, es decir:
\[
\nexists f,g \in \ent[x] : 1 = 2f(x)+ xg(x)
\]

Vamos ahora a enunciar y a demostrar el Teorema de Gauss sobre los DFU. Sin embargo, antes debemos aclarar algunos conceptos.
\begin{ndef}
Si $f \in A[x],\ gr(f) \geq 1$, se define su contenido como el m.c.d. de sus coeficientes. Lo denotamos por $c(f)$. (Si $f= \sum a_i x^i \implies c(f) = mcd(a_0,...,a_n)$	)\\
Se dice que $f$ es primitivo si $c(f) = 1$. El contenido es único salvo asociados.
\end{ndef}
\textbf{Lema.} $ \ c(af) = ac(f)$. Esta propiedad es consecuencia directa de que $(ab,ac) = a(b,c)$
\textbf{Lema.} Todo $f: gr(f) \geq 1$ se puede factorizar de la forma $f = af'$ con $f'$ primitivo. Además, esta factorización es esencialmente única.
\begin{proof}
	Sea $f$ un polinomio y $a = c(f)$. Entonces, $a/a_i \ \forall i$. Tomamos $a_i' = \frac{a_i}{a} \in A$. Sea también $f' = \sum a_i'x^i$. Ahora, si tomáramos $af'= \sum aa_i' x^i = f \implies a = c(f) = c(af') = ac(f') \implies c(f') = 1$ simplificando por a, luego hemos encontrado una factorización $f=af'$.
	
	Además, podemos ver que la factorización es única, pues si $f=bg$ con $c(g)=1$ y $f=af'$, entonces $a = c(f) = b c(g) = b \implies a=b$ y $g=f'$
\end{proof}

\textbf{Lema.} Todo $\phi,\ gr(\phi) \geq 1$ se puede factorizar de forma esencialmente única como $\phi = \frac{a}{b} f$ con $f$ primitivo.
\begin{proof}
	Sea $\phi = \sum \frac{a_i}{b_i} x^i$. Tomamos $b= \prod b_i \implies b_i/b \ \forall i \implies b_i /ba_i \ \forall i \implies b \phi = \sum \frac{ba_i}{b_i}x^i$ Pero el numerador es un múltiplo del denominador, luego $g = \sum \frac{ba_i}{b_i}x^i \in A[x]$.
	
	De esta forma, $b\phi = af \implies \phi = \frac{a}{b} f$.
	
	Además, veamos que es única. Sea ahora $\phi = \frac{c}{d} f'$ con $f'$ primitivo$\implies \frac{a}{b} f = \frac{c}{d} f' \implies daf=bcf' \implies da = bc$ y $f=f'$
\end{proof}

Enunciaremos también un lema muy importante, el lema de Gauss.\\

\textbf{Lema de Gauss.} Sea A un DFU. Si tenemos $f,g \in A\implies c(fg) = c(f)c(g)$. En particular, el producto de polinomios primitivos es primitivo.\\
\begin{proof}
	Probaremos en primer lugar que el producto de polinomios primitivos es primitivo. Sean $f= \sum a_i x^i $ y $g= \sum b_j x^j$ ambos en $A[x]$ y primitivos. Consideramos también su producto $fg = \sum c_k x^k$. Sabemos que  $c_k = \sum_{i+j=k}a_ib_j$. Negaremos la tesis e intentaremos llegar a una contradicción.
	
	Supongamos que $fg$ no es primitivo. Entonces, $c(fg) \ne 1$. Como A es un DFU, el contenido se podrá expresar como un producto de primos (irreducibles). Existirá al menos un primo que lo divida. $\exists p $ primo de $A$ con $p/c(fg) \implies p/c_k \ \forall k$. Sin embargo, ese p no puede dividir a $a_i \ \forall i$ pues f es primitivo. Por la misma razón, no puede dividir a todos los $b_j$. 
	
	Por tanto, sea $r$ el primer índice tal que $p$ no divide a $a_r$. Tomemos también $s$ el primer índice tal que $s$ no divide a $b_s$. Ahora, vamos a fijarnos en el coeficiente $c_{r+s}$ de $fg$. Este coeficiente se expresa como:
	 \[c_{r+s} = \sum _{i+j=r+s}a_ib_j = \sum _{i+j =r+s \ \ i < r}a_ib_j + a_rb_s + \sum_{i+j=r \ \ i > r}a_ib_j\]
	 De esta expresión, podemos ver que $p/c_{r+s}$, que $\forall i < r$ entonces $p/a_ib_j \implies p/\sum_{i+j =r+s}a_ib_j$ por dividir a los $a_i$. Ocurre lo mismo con el mismo en el término de $i > r $ pues p divide a $b_j$. Si despejáramos $a_rb_s$ veríamos que $p/a_rb_s$ pues divide a todos los sumandos, pero p es un primo, por tanto si divide a un producto tiene que dividir a alguno de los factores, pero como habíamos dicho que no divide a ninguno de los dos, hemos llegado a una contradicción y, por tanto, $fg$ es primitivo.
	 
	 Ahora demostraremos c(fg) = c(f)c(g). Podemos poner $f=af'$ con $f'$ primitivo e igual con $g=bg'$. Entonces $fg= abf'g'\implies c(fg)=c(abf'g') = abc(f'g')$ pero $f'$ y $g'$ son primitivos luego su contenido es 1. Así, el contenido de f es $a$ y el de g es $b$ luego $c(fg) = c(f)c(g)$ probando así el lema.
\end{proof}
\newpage
\textbf{Lema.} 
\begin{nlist}
	\item Si $a \in A$, $a$ es irreducible en A[x] $\Leftrightarrow$ $a$ es irreducible en A.
	\item Si $gr(f) \geq 1$, f es irreducible en A[x] $\Leftrightarrow$ f es primitivo y es irreducible en K[x]. 
 \end{nlist}

\begin{proof}
$(i)$ Una factorización de un polinomio de grado 0 será irreducible si es irreducible en A (porque U(A[x]) = U(A)).\\
$(ii)$ $\displaystyle \boxed{\Rightarrow}\ $ Si f es irreducible, el contenido tiene que ser o una unidad o un asociado. No puede ser asociado porque...
Como c(f)/f, si f es irreducible $\Rightarrow$ c(f) = 1 $\Rightarrow$ f es primitivo. Vamos a probar que es irreducible en K[x] por contradicción. Supongamos que f no es irreducible en K[x] $\Rightarrow$ f = $\phi \psi$ con $gr(\phi) \geq 1$ y $gr(\psi) \geq 1$
$$
 \left .
      \begin{matrix} 
        \phi =  \frac{a}{b}g : g\ es\ primitivo\\
        \psi =  \frac{c}{d}h : h\ es\ primitivo
      \end{matrix} 
   \right \} 
   \Longrightarrow \quad f = \frac{a}{b} \frac{c}{d} gh \Rightarrow bdf = acgh
  $$
  Aplicando contenidos sobre esta igualdad nos queda, $c(bdf) = c(acgh) \Rightarrow bdc(f) = acc(gh) \Rightarrow bd = ac \Rightarrow f = gh $ CONTRADICCIÓN f es irreducible y ninguno es unidad ya que el grado es mayor o igual que 1.
  
$\displaystyle \boxed{\Leftarrow}\ $Suponemos que $f=gh$ en A[x], $g$ y $h$ no unidades. Entonces, $gr(g)$ y $gr(h)$ es mayor o igual que 1. (Suponemos $gr(g)=0$, $g = a \in A; f=ah \Rightarrow 1 = ac(h) \Rightarrow g=a \in U(A)=U(A[x])$, llegamos a una contradicción con la suposición inicial).  \\
$f=gh$ se da en K[x] y muestra que f no es irreducible en K[x], contradicción con la hipótesis.
En K[x] no hay irreducibles de grado 0, las constantes tienen inverso.
\end{proof}

\textit{Observación:} Sea $\phi \in$ K[x], $gr(\phi) \geq$ 1, $\phi=\frac{a}{b}f$ con $f$ primitivo. $\phi$ es irreducible en K[x] $\Leftrightarrow$ f es irreducible en A[x].

\begin{nth}[Teorema de Gauss]
	Si A es un DFU $\implies A[x]$ es también un DFU.
\end{nth}

\begin{proof}
Sea $f \in$ A[x], $f \neq 0$ y $f \notin$ U(A[x])=U(A)
\begin{itemize}
	\item Caso $gr(f) = 0$. $f = a \in$ A, $a \neq 0$ y $a \notin$ U(A). Como A es un DFU, existen $p_1,...,p_r$ irreducibles de A (también de A[x]) tales que $a = p_1...p_r$
	\item Caso $gr(f) \geq 1$ y $f$ primitivo. Existen $\phi_1,...,\phi_r \in$ K[x] irreducibles tales que $f=\phi_1...\phi_r$. $\phi_i = \frac{a_i}{b_i}f_i : f_i \in$ A[x], primitivo $\implies f = \frac{a}{b}f_1...f_r$ donde $a=\prod a_i,\ b=\prod b_i \implies bf = af_1...f_r$ $\implies^{(1)} b=a \implies f = f_1...f_r$. Por el lema anterior, como $f_i$ son irreducibles en K[x] y, además, son primitivos, $f_i$ son irreducibles en A[x].
	\item Caso general, $gr(f) \geq 1$ y $f$ no primitivo. Tomamos $f=af'$
con $f'$ primitivo. $c(f) = a \neq 0,\ a \notin$ U(A)\\
$^{(1)}$ Como $f_i$ son primitivos
\end{itemize} 
\end{proof}


En lo sucesivo, vamos a notar $A\subseteq K = Q(A)$. De esta forma, también sucede $A[x] \subseteq K[x]$. También vamos a notar:
\begin{itemize}
	\item $a,b,c.... \in A$
	\item $f,g,h... \in A[x]$
	\item $\phi,\psi, ... \in K[x]$
\end{itemize}

\begin{ncor}
	Si $f\in A[x]$ es irreducible en $A[x]$ con $gr(f) \geq 1\implies$ $f$ es primo en $A[x]$.
\end{ncor}
\begin{proof}
	Supongamos que $f/gh$ en $A[x]\subseteq K[x]$. Como $f$ es irreducible en $A[x]$, lo  es en $K[x]$ y $K[x]$ es un DFU por ser K un cuerpo, entonces $f$ es primo en $K[x]$. Entonces, podemos asegurar que o bien $f/g$ o bien $f/h$ en $K[x]$.
	
	Para lo que sigue, supongamos que $f/g$ en $K[x]$ (Si quisiéramos para hacerlo para $h$, sólo habría que cambiar las letras). Entonces, $f\phi = g $ y $\phi = \frac{a}{b}f'$ con $f'$ un polinomio de $K[x]$ primitivo. Esto implica: $$\frac{a}{b}ff'=g \implies aff' = bg(1)$$
	
	Ahora, calcularemos los contenidos aplicando el lema de Gauss. Como f y $f'$ son primitivos,
	
	$$ a c(f)c(f') = bc(g) \implies a = bc(g)$$
	
	Ahora, si nos llevamos esta igualdad a (1) obtenemos:
	\[
	bc(g)ff'=bg \implies f(c(g)f')=g \implies f/g \ en \ A[x]
	\]
	Y por tanto, obtenemos que $f$ es primo.
\end{proof}

\begin{nota}
	De esta forma, podemos ver por inducción que cualquier anillo de la forma $A[x_1,...,x_r] $ es un DFU si A es un DFU.
\end{nota}
\begin{nota}[2]
	En $\ent$ hay infinitos primos. 

\end{nota}

Vamos a intentar ahora a buscar una factorización de un $f \in A[x]$. ¿Cuándo es f irreducible?

 Sabemos que $gr(f) = 0 \iff f = p \in A$ que sería irreducible si $p$ lo es en A.
 
 Supongamos ahora que $gr(f) \geq 1$, $f$ es irreducible en $A[x] \iff$ f es primitivo e irreducible en $K[x]$

\begin{nota}
	Si K es un cuerpo, todo polinomio de grado 1 es irreducible.
	\begin{proof}
	Supongamos que $\phi = \phi_1 \phi_2 \implies 1 = gr(\phi) =  gr(\phi_1)+gr(\phi_2) \implies \phi_1$ es una unidad o lo es $\phi_2$, luego necesariamente $\phi$ es irreducible.  
\end{proof}
\end{nota}


\textbf{Ejercicio.} Factorizar $(120x+100)^2$ en $\ent[x]$.

\[
120x+100 = 20(6x+5) = 2^2*5*(6x+5)
\]
Pero, $6x+5$ es un polinomio de grado 1, primitivo en $\ent[x]$. Además, 2 y 5 son irreducibles en $\ent$, por tanto la factorización del polinomio al cuadrado es:
\[
2^4*5^2(6x+5)^2
\]

\begin{nota}
	Si $K$ es un cuerpo, $\phi \in K[x]$ tiene un factor de grado 1 $\iff \phi$ tiene una raíz en $K$
	\begin{proof}
	$$\alpha x + \beta = \alpha(x-(\frac{-\beta}{\alpha}) = \alpha(x - \gamma) \implies \alpha x+b/\phi \iff x-\gamma / \phi \iff^{Rufinni} \phi(\gamma) = 0$$
	Donde $\gamma = \frac{-\beta}{\alpha}$
\end{proof}
\end{nota}

\begin{nprop}
	Sea $f=a_0+a_1x+...+a_nx^n \in A[x]$. Entonces $f$ tiene un factor irreducible de grado 1 en $K[x]$ si y solamente si tiene una raíz en $K$.
	
	Equivalentemente, $(a,b)=1$ y $f(\frac{a}{b})=0 \iff bx-a/f$ en $A[x]$. En tal caso, esta posible raíz verifica que $a/a_0$ y $b/a_n$ en $A$.
\end{nprop}
\begin{proof}
	Si tenemos que $f(\frac{a}{b})=0 \iff x-\frac{a}{b}/f$ en $K[x]$.
	
	$\boxed{\Rightarrow}$ Esto implica que $f=(x-\frac{a}{b})\phi$ con $\phi=\frac{c}{d}g$ con $g \in$ A primitivo, $$\implies f= \frac{c}{d}(x-\frac{a}{b})g = \frac{c}{bd}(bx-a)g \implies bdf=c(bx-a)g$$
	Si aplicamos contenidos:
	\[
	bdc(f)=c
	\]
	Volviendo a la igualdad anterior:
	\[
	f=(bx-a)(c(f)g)
	\]
	Por tanto, $bx-a/f$ en $A[x]$.
	
	$\boxed{\Leftarrow}$ Si $f=(bx-a)g \implies f(\frac{a}{b}) = (b\frac{a}{b}-a)g(\frac{a}{b})=0$
\end{proof}
\textbf{Criterio de la raíz:}
En $K[x]$ todo polinomio de grado 1 es irreducible y es asociado a uno de la forma $x-\alpha$ con $x-\alpha/\phi(x)\iff \phi(\alpha) = 0$.

Así, también en $A[x]$ los polinomios irreducibles de grado 1 son los de la forma $bx-a$ con $(a,b)=1$ por lo que $(a,b)=1$ y $bx-a/f(x)\iff f(\frac{a}{b})= 0$.

Si $f(x)= a_0+a_1x+...+a_nx^n$ y si $f(\frac{a}{b}) \implies a / a_0$ y $b/a_n$ en $A$.

Esto lo podríamos ver como una condición de irreducibilidad, pues si un polinomio fuera irreducible, no podría tener ninguna raíz, pues carecen de factores de grado 1.

\begin{ejemplo}
	Factorizar en producto de irreducibles el polinomio $f=6x^4+3x^3-18x^2+33x+21$ en $\ent[x]$.

Lo primero que deberíamos hacer es buscar el contenido de este polinomio:
\[
f=3(2x^4+x^3-6x^2+11x+7)
\]
El contenido de $f$ es 3 y $f'=2x^4+x^3-6x^2+11x+7$. 3 es un primo en $\ent $ y por tanto lo es en $\ent[x]$ y por tanto nos centramos en el primitivo asociado. Aplicamos el criterio de la raíz. Las posibles raíces que el polinomio tuviera en $\rac$ serían $\{\pm 1, \pm \frac{1}{2}, \pm 7, \pm \frac{7}{2}\}$. Calculamos las imágenes de estos puntos para ver si alguno es cero:
\[
f'(1) = 2+1-6+11 + 7 \ne 0; \quad f'(-1) = 2-1-6-11+7 \ne 0;\]
\[f'(\frac{-1}{2}) = 2*(1/2)-(1/2^3)-6*(1/4)-11*(1/2) + 7 = 0
\]
Y por tanto hemos encontrado una raíz, es decir:
\[
2x+1/f'(x) \ en \ \ent[x]
\]
Ahora, dividiremos $f'(x)$ entre $2x+1$ para reducirlo. Al dividirlo nos queda cociente $x^3-3x+7$ y resto 0. Por tanto, f nos queda factorizado como:

\[
f(x)=3*(2x+1)*(x^3-3x+7)
\]
Tenemos ya uno de grado 3, por tanto es irreducible si y solo si carece de raíces en $\rac$. Para ello, debemos buscar fracciones que sean divisores del 7, que son $\{\pm 1, \pm 7\}$ y haríamos sus imágenes por el polinomio $x^3-3x+7$ y vemos que ninguna da de resultado cero, por tanto estas no son raíces suyas y por tanto es irreducible en $\ent[x]$ y la factorización del polinomio en producto de primos sería justo la que acabamos de obtener.

Si nos planteamos la discusión en $\rac[x]$: 3 y $x^3-3x+7$ son mónicos, $2x+1$ no, pero podemos sacar factor común para hacerlo mónico y nos quedaría como resultado final:
\[
f(x)= 6(x+\frac{1}{2})(x^3-3x+7) \quad \in \rac[x]
\]
\end{ejemplo}
\begin{ejemplo}[2]
	Factorizaremos el polinomio: $20x^3 +10x^2-80x+30 \in \ent[x]$. Para resolverlo, había que observar que su contenido es 10 y que por tanto:
	\[
	f=2*5*(2x^3+x^2-8x+3)
	\]
	Y por tanto habría que buscar sus raíces entre $\{\pm 1, \pm \frac{1}{2}, \pm 3 \pm \frac{3}{2}\}$ y veríamos que en $\frac{3}{2}$ el polinomio evaluado da cero, por tanto factoriza como:
	\[
	2*5*(2x-3)(x^2+2x+2) 
	\]
	Y podemos ver que $x^2+2x+2$ no tiene raíces y por tanto ese es el polinomio descompuesto en primos.
\end{ejemplo}
\begin{ejemplo}[3]
	Consideremos el polinomio: $x^4+x^3+x^2+x+1$ en $\ent_2[x]$. ¿Este polinomio es irreducible?. \\Estamos en $\ent_2$, luego sustituyendo el 0 y el 1 vemos que no tiene de raíces. Como no tiene raíces no tiene factores de grado 1. 
	
	Ahora, si no fuese irreducible habría que descomponerlo como 2 factores de grado 2. Si dividimos por el polinomio $x^2+x+1$, que es irreducible en $\ent_2[x]$ el resto no es 0, luego no es divisible y por tanto el polinomio inicial ya es irreducible en $\ent_2[x]$.
\end{ejemplo}

\begin{nprop}[Criterio de Eisenstein]
	Sea $f(x)=a_0+a_1x+...+a_nx^n \in A[x]$ primitivo y siendo $A$ un DFU.
Si existe un primo $p\in A$ cumpliendo alguna de estas dos:
	\begin{enumerate}
	\item $p/a_i \ \ \forall i = 0,...,n-1$ y $p^2$ no divide a $a_0$ 
	\item $p/a_i \ \ \forall i=1,...,n$ y $p^2$ no divide a $a_n$
\end{enumerate}
Entonces $f$ es irreducible.

\end{nprop}
\begin{proof}
	Vamos a demostrarlo para el primer caso, la demostración para el segundo es equivalente. Vamos a negar la tesis y veremos qué ocurre.
	
	Supongamos que $f=(b_0+...+b_mx^m)(c_0+...+c_rx^r)$ con $r+m=n$ y $r,m \geq 1$. Entonces, $a_0=b_0c_0$ para que fuera el producto. Como $p$ es primo, tiene que dividir a $b_0$ o a $c_0$. Además, no puede dividir a ambos pues si no $p^2$ dividiría a $a_0$. Supongamos que $p/b_0$ y por tanto que no divide a $c_0$. 
	
	Supongamos también que $p$ no divide a $a_n$ pues si no, formaría parte del contenido. Además, como hemos obtenido una factorización, $a_n = b_mc_r$ y como $p$ no divide a $a_n$ entonces no divide ni a $b_m$ ni a $c_r$.
	
	Sea $i$ el primer índice tal que $p$ no divide a $b_i$. Es claro que $0<i \leq m < n$, que existirá pues al menos es el último. Como lo habíamos factorizado, entonces $a_i = b_ic_0+(b_{i-1}c_1+...+b_0c_i)$. Entonces, $p/b_j \ \ \forall j < i \implies p/b_ic_0$ pero $p$ es primo luego divide o a $b_i$ o a $c_0$, pero ya habíamos dicho que no dividía a ninguno de los dos anteriormente en la demostración, luego hemos llegado a una contradición. Por tanto, f es irreducible.
\end{proof}

\begin{ejemplo}
	Podemos ver que $3x^7-70x^3+140 \in \ent[x]$ por el criterio de Eisenstein para el primo $p=5$, que divide a todos los coeficientes menos al coeficiente líder pero su cuadrado $p^2 = 25$ no divide a ninguno de los coeficientes. 
\end{ejemplo}
\begin{ejemplo}[2]
	El polinomio : $y^3 + x^2y^2 +xy + x \in \ent[x,y]$, pues el primo $p=x$ divide a todos los coeficientes menos al líder y su cuadrado no divide a todos los coeficientes.
\end{ejemplo}

\begin{nprop}[Cambio de Anillo]
	Sean A y B dominios de integridad (así, el grado del producto es el producto de los grados). Sea $\phi : A[x] \to B[x]$ un homomorfismo en el que $gr(\phi(f))  \leq gr(f)$.
	
	Si $f\in A[x] : gr(\phi(f))= gr(f)$ y $\phi(f)$ no tiene divisores de grado $r \implies f$ tampoco tiene divisores de grado $r$ 
\end{nprop}
\begin{proof}
	Supongamos que $f=gh$ donde $gr(f) = n$, $gr(g) = r$ y $gr(h) = s$ con $r+s = n$.
	Aplicamos entonces el homomorfismo:
	\[
	\phi(f) = \phi(gh) = \phi(g)\phi(h)
	\]
	Y ahora, $gr(\phi(f)) = n$, $gr(\phi(g)) \leq r$ y $gr(\phi(h)) \leq s$ y entonces implica que $gr(\phi(g)) = r$ (Si $r < gr(\phi(g)) \implies r+s \neq n$, contradicción)
\end{proof}
\begin{ejemplo}
	$x^4+1$ es irreducible en $\ent[x]$.
	
	Si tomamos $\phi:\ent[x] \to \ent[x]$ que lleva $x \mapsto x+1$ el homomorfismo de evaluación en $x+1$. ($\phi(\sum a_i x^i) = \sum a_i (x+1)^i$).
	
	Así, $\phi(x^4+1) = (x+1)^4+1 = x^4 +4x^3 +6x^2 +4x + 2 \in \ent[x]$, que es irreducible por el criterio de Eisenstein para el primo $p=2$. Ahora, por el criterio del cambio de anillo, entonces $x^4+1$ es también irreducible en $\ent[x]$
\end{ejemplo}

\begin{nprop}[Criterio de reducción módulo un primo]
	En las condiciones anteriores, consiste en aplicar el homomorfismo:
	\[
	R_p:\ent[x] \to \ent_p[x] : \quad R_p(\sum a_i x^i) =  \sum R_p(a_i) x^i \in \ent_p[x]
	\]
	Para utilizar este criterio, es muy útil conocer los irreducibles mónicos de los anillos $Z_p[x]$
\end{nprop}

\textbf{Irreducibles mónicos de $\ent_p[x]$:}
\begin{itemize}
	\item de grado 2 en $\ent_2[x]$: $x^2+x+1$
	\item de grado 3 en $\ent_2[x]: x^3+x+1,\ x^3+x^2+1$
	\item de grado 2 en $\ent_3[x]: x^2+1,\ x^2+x+2,\ x^2+2x+2$
\end{itemize}
\begin{ejemplo}
	Sea $f = x^4 +3x^2 -2x +5 \in \ent[x]$. Si le aplicamos el criterio de reducción módulo un primo, en este caso el 2, el polinomio queda: $R_2(f) = x^4 + x^2 +1 \in \ent_2[x]$ y por el criterio de reducción, como este polinomio no tiene divisores de grado ni 1 ni 3 entonces podemos asegurar que $f$ tampoco los tiene.
	
	Si lo volvemos a hacer para el primo $p=3$, queda $R_3(f) = x^4 + x + 2 \in \ent_3[x]$ , que nos da exactamente la misma información que la reducción anterior. Como conocemos los mónicos de $\ent_3[x]$, dividimos este polinomio entre los 3 irreducibles mónicos de $\ent_3[x]$ y vemos que los restos son no nulos. Así, este polinomio es irreducible en $\ent_3[x]$ y por el criterio de reducción módulo un primo, el polinomio primero es irreducible en $\ent[x]$.
\end{ejemplo}



\section{Matrices sobre Anillos conmutativos}
Sea $K=$ un cuerpo, y consideremos $\mathcal{M}_{mxn}$ y  $\mathcal{M}_{n}$. Imaginemos ahora que $A\leq K$ es un subanillo. De esta forma, podemos concebir las matrices que están este en este conjunto : $\mathcal{M}_{mxn}(A)$ ó  $\mathcal{M}_{n(A)}$ .

Así, si sumáramos dos matrices que estuvieran en  $\mathcal{M}_{mxn}(A)$, nos quedaría una matriz cuyas entradas están en el subanillo A. Lo mismo ocurre en la multiplicación. 


Podemos considerar también:
\[
GL_n(K) = U (\mathcal{M}_n(K)) = \{ M \in \mathcal{M}_n(K) :  \ \exists M^{-1} \in \mathcal{M}_n(K) \ \ con \ \ MM^{-1} = M^{-1}M\}
\]

Que sabemos que es igual al conjunto $\{M \in \mathcal{M}_n(K) : | M | \ne 0\}$. Ahora, podemos considerar como hemos hecho anteriormente, lo mismo pero en el subanillo A.

Dada $M \in \mathcal{M}_n(A)$, si $M \in GL_n(A)$, cumplirá que $MM^{-1} = I$, tomando determinantes vemos que: $|I|=|M||M^{-1}|$. El determinante de M y el determinante de $M^{-1}$ están en A, y es un producto de dos que da como resultado 1, por lo que podemos asegurar que $|M| \in U(A)$.

Veámoslo al revés. Si $M\in \mathcal{M}_n(A)$ con $|M\in U(A)| \implies 1/|M| \in A$. Si consideramos la adjunta de M, esta también pertenece a $\mathcal{M}_n$ y por tanto $M \in GL_n(A)$ pues tiene una inversa cuyas entradas están todas en el subanillo A.
Con todo esto, hemos deducido la siguiente proposición:

\begin{nprop}
	Sea $M \in \mathcal{M}_n(A)$, entonces $M \in GL_n(A) \iff |M| \in U(A)$
\end{nprop}

\begin{ejemplo}
	El grupo lineal de $\ent$ es: $GL_n(Z) = \{M \in \mathcal{M}_n(A) :  |M| = \pm 1\}$
\end{ejemplo}

\begin{ndef}[Matrices equivalentes]
	Dos matrices M y N $\in \mathcal{M}_{mxn}(A)$ son equivalentes si $\exists P,Q \in GL_n(A)$ tal que
	\[
	M = PNQ
	\]
Estando en un cuerpo, dos matrices serán equivalentes $\iff$ tienen el mismo rango.
\end{ndef}

\begin{nth}[Teorema de la forma Normal de Smith]
	Si A es un Dominio Euclídeo, toda matriz  $M \in \mathcal{M}_{mxn}(A)$ es equivalente a una de la forma:
	\[ \left( \begin{array}{ccccc}
 d_{11}&  &  & & \\
 &   d_{22} & & \\
 &  & ... &  &  \\
 & & &d_{rr} &\\
 & & & & 0
 \end{array} \right)\]


con $d_{11},...,d_{rr} \in A$ ubicados en la diagonal, donde $ 0 \leq r \leq min{m,n} $ y cada $d_{ii}\ne 0$ y $d_{ii}/ d_{i+1 \ i+1}$.  Además, estos $d_{11}, ...,d_{rr}$ son únicos para la clase de equivalencia de M y se llaman los \textbf{factores invariantes} de la matriz.

Dos matrices $M,N \in \mathcal{M}_{mxn}(A)$ son equivalentes $\iff$ tienen los mismos factores invariantes.
\end{nth}

Sea esta matriz

	\[P= \left( \begin{array}{ccccc}
 1&  &  & & \\
 &   ... & a & \\
 &  & ... &  &  \\
 & & & & 1
 \end{array} \right)\]
Que tiene en la posición $M_{ij}$ un elemento $a\in A$. Entonces, esta matriz es invertible y además su inversa es igual pero cambiando a por $-a$.

Si:
	\[ N =  \left( \begin{array}{c}
 F_1 \\
 ...\\
 F_n
 \end{array} \right)   \]
 Y multiplicáramos $PN$, entonces quedaría $F_k' = F_k $ si $k \ne i$ con ese i el número de fila que tenía el elemento a y . Lo mismo ocurriría con las columnas si...\\
\textbf{FALTA CONTENIDO DE MATRICES, añadirla.}
 


\begin{nota}
	Si multiplicáramos en una matriz todos los elementos de una fila o los de una columna por una unidad del anillo, entonces la matriz inicial y la resultante son equivalentes.
\end{nota}
\begin{nota}
	Si permutamos dos filas, o dos columnas, las matrices inicial y final son equivalentes.
\end{nota}

\begin{ndef}
	Una matriz $M \in \mathcal{M}_{mxn}$ que sea:
		\[ \begin{pmatrix}
 d_{11}&  &  & & \\
 &  & \ddots &  &  \\
 & & &d_{rr} &\\
 & & & & 0 \\
 & & & & & \ddots& \\
 & & & &  & & 0 \\
\end{pmatrix} \]
 con $d_1,...,d_r \in A \ \ d_i \ne 0$ y $d_i / d_{i+1}$, en las posiciones (i,i) con $1 \leq i \leq r$ y 0 en todas las demás posiciones, se dice matriz Normalizada (Smith)
\end{ndef}

Vamos a explicar ahora un método a partir del cual, teniendo una matriz cualquiera podemos obtener una matriz normalizada que es equivalente a la primera. No olvidemos que estamos trabajando en Dominios euclídeos.

Sea $A \in \mathcal{M}_{mxn}$. $A \ne 0$. Sea $\varphi:A -\{0\} \to \nat$ la función euclídea.

Vamos a llamar $\varphi(M) = min\{\varphi(a_{ij}) : a_{ij} \ne 0\}$
Primero, tomaremos un $a_{ij}$ donde la función euclídea tome el valor más pequeño que tome en toda la matriz. Ahora, por operaciones elementales podemos llevarlo a la posición $d_{11}$.

Supongamos ahora que $a_{11}$ no divide a $a_{1k}$. como estamos en un dominio euclídeo, podemos expresar el cociente entre ellos como:
\[
a_{1k} = a_{11}q+ b_{1k}
\]
donde $b_{1k} \ne 0$ y $\varphi(b_{1k}) < \varphi(a_{11})$.

Ahora, realizamos la operación:
\[
C_k \mapsto C_k - q C_1
\]

Reiteramos el proceso las veces que haga falta, hasta encontrarnos una matriz que sea de la forma $b_{ij}$. Ahora, si $b_{11}$ no divide a $b_{k1}$, entonces este último se expresará como $b_{k1} = b_{11}q' + c_{k1}$ con $c_{k1}\ne 0$ y $\varphi(c_{k1}) < \varphi(b_{11})$

Entonces, realizamos la operación:
\[
F_k \mapsto F_k - q'F_1
\]
Volvemos a reiterar el procedimiento cuantas veces sea falta.

Tras una serie de pasos, llegaremos a una matriz equivalente a la original en la que $b_{11}/b_{k1}$ y $b_{11}/b_{1k} \ \ \forall k$. LLegados a esta situación,por transformaciones elementales podemos obtener una matriz que sea del tipo:
\[
\begin{pmatrix}
 b_{11} & 0 & \cdots & 0 \\
 0 & c_{22} & \cdots & c_{2n} \\
 \vdots &  & \ddots& \\
 0 & c_{m2} & \cdots & c_{mn} \\
\end{pmatrix} 
\]

Ahora, tenemos una matriz en la que $b_{11}/ c_{ij}$ para cada i y j. Ahora, lo que haremos es tomar la submatriz que empieza en $c_{22}$ y volveremos a realizar el mismo proceso.

\begin{ejemplo}
	Sea $A \in matrices$ tal que:
	\[
	\begin{pmatrix}
 14 & 10 \\
 22 & 14\\
 10 & 10\\ 
\end{pmatrix} 
	\]
	Ahora, tomamos el numero que tenga la $\varphi$ menor, en este caso el 10 y hacemos transformaciones elementales para llevarlo a la primera posición.
	\[\begin{pmatrix}
 14 & 10 \\
 22 & 14\\
 10 & 10\\ 
\end{pmatrix} \mapsto^{C_2,C_1}
	\begin{pmatrix}
 10 & 14 \\
 14 & 22\\
 10 & 10\\ 
\end{pmatrix} \mapsto^{C_2 - C_1} \begin{pmatrix}
 14 & 4 \\
 22 & 8\\
 10 & 0\\ 
\end{pmatrix}\mapsto^{C_2,C_1}
\begin{pmatrix}
 4 & 10 \\
 8 & 14\\
 0 & 10\\ 
\end{pmatrix}\mapsto^{C_2 - 2C_1}  \begin{pmatrix}
 4 & 2 \\
 8 & -2\\
 0 & 10\\ 
\end{pmatrix}
	\]
	\[
	\mapsto\begin{pmatrix}
 2 & 4 \\
 -2 & 8\\
 10 & 0\\ 
\end{pmatrix} \mapsto^{C_2-2C_1}
\begin{pmatrix}
 2 & 0 \\
 -2 & 12\\
 10 & -20\\ 
\end{pmatrix}\mapsto^{F_2+F_1}
\begin{pmatrix}
 2 & 0 \\
 0 & 12\\
 10 & -20\\ 
\end{pmatrix} \mapsto^{F_3-F_1}  
\begin{pmatrix}
 2 & 0 \\
 0 & 12\\
 0 & -20\\ 
\end{pmatrix} 
	\]
	Y ahora, tendríamos que hacer lo mismo para la submatriz resultante:
	\[
	\begin{pmatrix}
 12 \\
 -20 \\ 
\end{pmatrix}\mapsto^{F_3+2F_2} 
\begin{pmatrix}
  4 \\
 12 \\ 
\end{pmatrix}\mapsto^{F_3 - 3F_2}
\begin{pmatrix}
  4 \\
 0 \\ 
\end{pmatrix}  
	\]
	Y nos queda la matriz :
	\[
	\begin{pmatrix}
 2 & 0 \\
 0 & 4\\
 0 & 0\\ 
\end{pmatrix} 
	\]
	Que es equivalente a la inicial.
\end{ejemplo}

\begin{ndef}[Rango de la matriz]
	El entero $r$ tla que $M$ tiene al menos un menor de orden $r$ no nulo y todos los de orden $r+1$ son nulos se llama Rango de la matriz.
\end{ndef}

Vamos a definir ahora $\Delta_s(M) = mcd\{$menores de orden s de M no nulos $\}$
Así, $\Delta_1(M)$ sería el máximo común divisor de todos los $m_{ij}$. Ahora, con esta definición, podemos decir que:
\[
rg(M) = max\{s : \Delta_s(M) \ne 0\}
\]
Lo que implica que si $rg(M) = r \implies \Delta_1(M),...,\Delta_r(M)$ son $\ne 0$ y $\Delta_{r+1}(M),...$ son nulos.

Sea $P \in \mathcal{M}_m(A)$. Construimos la matriz $PM\in \mathcal{M}_{mxn}(A)$(tenemos la matriz $M \in \mathcal{M}_{mxn}(A)$). Si miramos el elemento de PM en la posición $(k,j)$,este es: $\sum_{i=1}^n P_{ki}a_{ij}$. Así, la fila $k-$ésima de $PM$ es:
\[
(\sum_{i=1}^n P_{ki}a_{i1}, \cdots , \sum_{i=1}^n P_{ki}a_{in}) = \sum_{i=1}^n P_{ki}(a_{i1},\cdots , a_{in})
\]
Por tanto las filas de la matriz $PM$ son combinación lineal de la matriz M. Por tanto, también podemos afirmar que los menores de orden $s$ de la matriz PM son combinación lineal de los menores de orden $s$ de la matriz M.

Si $P\in GL_m(A)\implies \Delta_s(PM) : \Delta_s(P^{-1}PM) = \Delta_s(M)$.

Si $Q \in GL_m(A) \implies \Delta_s (M) = \Delta(MQ)$

\begin{nota}
	$\Delta(M) = \Delta(M^t)$. Del mismo modo, $\Delta_S(MQ) = \Delta_s((MQ)^t) =  \Delta_s (Q^tM^t) = \Delta_s (M^t) =  \Delta_s(M)$
\end{nota}

Si $M \sim M' \sim \begin{pmatrix}
 d_{11}'&  &  & & \\
 &  & \ddots &  &  \\
 & & &d_{rr}' &\\
 & & & & 0 \\
 & & & & & \ddots& \\
 & & & &  & & 0 \\
\end{pmatrix}  \implies r = r'$

Tenemos que $\Delta_1(M) = d_1 = \Delta_1 (M') = d_1' \implies d_1 = d_1'$ ; y si lo hacemos en general vemos que:
\[
\Delta_s(M) = d_1...d_s = A_s(M') = d_1'...d_s' \implies ds =  \frac{\Delta_s(M)}{\Delta_{s-1}(M)} =  \frac{\Delta_s(M')}{\Delta_{s-1}(M')} = d_s'
\]

\section{A-módulo}
\begin{ndef}[A-módulo]
	Es un conjunto no vacío M dotado de dos operacions:
	\begin{itemize}
	\item $MxM \to M$ tal que $(x,y) \mapsto x+y$ que es interna
	\item $Ax M \to M$ tal que $(a,x) \mapsto ax$
\end{itemize}
que verifican las propiedades asociativa, conmutativa, elemento neutro y elemento opuesto. También verifican la distributiva respecto de la suma y respecto del producto, asociatividad del producto y neutro del producto.

\end{ndef}
Durante todo el apartado nos referiremos a $x, y,...$ como elementos de $M$ y a $a, b, ...$ como elementos de A.\\
\textbf{Propiedades:}
\begin{nlist}
	\item Unicidad del 0 y del opuesto.
	\item $0x = 0$ y $a0 = 0$
	\item $(-a)x = -(ax) = a(-x)$
	\item Si para una lista de elementos $(x_1,...,x_n)$ de M con al menos 2 elementos definimos su suma simultánea $\sum_i^n x_i$ por inducción:$ \sum_i^n x_i = (\sum_{i=0}^{n-1}x_i) +x_n$
	\item $\forall i < r < n,$ entonces $\sum_i^n x_i = (\sum_{i=0}^{r}x_i) + (\sum_{i=r+1}^{n}x_i) $
	\item $ (\sum_i^n a_i)(\sum_j^m x_i) = \sum_i^n \sum_j^m a_ix_j$
\end{nlist}

Además, si $I \leq A$ es un ideal, si hacemos $M= A/I$, entonces se verfican:
\begin{itemize}
	\item $[x]+[y] = [x+y]$
	\item $a[x] = [ax] = [a][x]$
\end{itemize}

\begin{ndef}[Submódulos]
	Si M es un A-Módulo, un submódulo es $\emptyset \ne N \subseteq M$ que es cerrado para sumas ($x,y\in N \implies x+y \in N$) y para múltiplos($a\in A,\ x\in N \implies ax\in N$).
	
	Con estas operaciones de suma y producto restringidas, $N$ también es un A-módulo. Por tanto, todo submódulo es un módulo.
\end{ndef}
\begin{nprop}
	Si $M_1,...,M_r \subseteq M$ son submódulos, existe siempre un módulo que los contiene. Se le llama le submódulo suma y se representa como:
	\[
	\sum_i^r M_i =\{\sum_i^r x_i : x_i \in M_i\}
	\]
	Además, se puede usar la conmutatividad de la suma y la asociatividad generalizada.

\end{nprop}
\begin{ndef}[Suma directa]
	Se dice que la suma de $M_1,...,M_r$ es directa y se representa:
	\[
	\sum_i^r = \oplus_i^r M_i \ si \ (\sum_i^r x_i =  \sum_i^r y_i) \ con \ x_i,y_i \in M_i \iff xi = y_i	\]
\end{ndef}
\begin{ndef}[Sistema de generadores]
	Si $x\in M$, siempre existe un submódulo de $M$ que es el menor submódulo que lo contiene. Más aún, que contiene al elemento $x$ y está contenido en cualquier otro submódulo que contenga a $x$. Se suele denotar como $Ax = \{ax : a \in A\}$ pues es donde están todos los múltiplos del elemento. Se le llama el submódulo cíclico de x. También se suele notar "(x)".
\end{ndef}
\begin{ndef}
	Si $x_1,...,x_r \in M$, también existe un módulo $M$ que es el más pequeño módulo que los contiene. Este es:
	\[
	\sum Ax_i = \{\sum a_ix_i : a_i \in A\}
	\]
	Estos son combinaciones lineales de $x_1,...,x_r$, y se le llama submódulo generado por $x_1,...,x_n$, y se suele representar como $(x_1,...,x_n)$
\end{ndef}
\begin{nota}
	Si $N=(x_1,...,x_r)$ y $N'= (y_1,...,y_s)$, entonces
	\[
	N \subseteq N' \iff x_i \ es \ c.l. \ de \ y_1,...,y_s \ \forall i
	\]
	Y $N=N'\iff$ todo $x_i$ es c.l. de $y_1,...,y_s$ y todo $y_j$ es c.l. de $x_1,...,x_r$
\end{nota}
\begin{ndef}
	Si $x_1,...,x_r\in M : M = (x_1,...,x_r)$ se dice que $x_1,...,x_r$ es un sistema de generadores de M  y se dice que M es finitamente generado.
\end{ndef}
\begin{nprop}
	Si A es un dominio euclídeo, todo submódulo de un módulo finitamente generado es finitamente generado.
\end{nprop}
\begin{ndef}
	Un sistema de generadores $x_1,...,x_r$ de un módulo $M$ se dice que es una base si se verifica:
	\[
	\sum_i^r a_i x_i = \sum_i^n b_i x_i \iff a_i = b_i \ \forall i
	\]
	O, equivalentemente:
	\[
	\sum_i^r a_ix_i = 0 \iff a_i  = 0  \ \forall i
	\]
\end{ndef}
\begin{ndef}[Linealmente independientes]
	Un conjunto de elementos de un A-Módulo $x_1,...,x_r$ se dice que son l.i. si la única c.l. de ellos que da como resultado 0 es aquella en la que todos los coeficientes son cero.
\end{ndef}
\begin{nota}
	En los A-módulos, es falso decir que todo A-módulo tiene una base o que todo sistema de generadores contiene a una base. Tampoco podemos hacer siempre que un conjunto linealmente independiente se pueda extender a una base.
\end{nota}
\begin{ndef}[Módulo Libre]
	Un módulo que contenga una base se dice que es un módulo libre.
\end{ndef}

\begin{ndef}
	Si M, N son dos módulos, y $\varphi:M \to N$ una aplicación, la llamamos lineal u homomorfismo entre módulos si:
	\[
	\begin{rcases}
	\varphi(x+y) = \varphi(x) + \varphi(y)\\
	\varphi(ax) = a\varphi(x)
\end{rcases} \iff \varphi(\sum_i^n a_ix_i) = \sum_i^n a_i \varphi(x_i)
	\]
	También puede ser un monomorfismo(si y solo si su núcleo es 0), epimorfismo o un isomorfismo.
\end{ndef}
\begin{ejemplo}
	Si $m\in A$, $A/(m)$ es un A-módulo ( $[a]+[b] = [a+b]$ y $[a]b = [ab]$. Si A tiene unicidad de cocientes y restos($\ent,K[x]$), entonces $A_m$ es un A-módulo con $r+r' = R_m(r+r')$ y $ar = R_m(ar)$.
	
	Recordemos que existe un isomorfismo entre $A/(m) \cong A_m$ que lleva $[a] \mapsto R_m(a)$
\end{ejemplo}

\begin{ejemplo}[2]
	Si $M_1,...,M_n$ son A-módulos, su "producto" $M_1 \times ... \times M_n =  \prod M_i$ es un A-módulo con $(x_1,...,x_n) + (y_1,...,y_n) = (x_1+y_1,...,x_n+y_n)$  y $a(x_1,...,x_n)= (ax_1,...,ax_n)$
\end{ejemplo}
\begin{nprop}
	Supongamos que $M_1,...,M_n$ son submódulos de M y que $M = M_1 \oplus ... \oplus M_n = \oplus_i M_i$
	Entonces, existe:
	\begin{align}
	\varphi:\prod_i M_i to M=\oplus_i M_i \\
	(x_1,...,x_n) \mapsto x_1+...+x_n = \sum_i x_i
\end{align}
que es un isomorfismo.
\end{nprop}	

\section{Clasificación de los módulos finitamente generados sobre un Dominio Euclídeo.}

Para comprenderlo mejor, vamos a ver primero un caso particular sencillo.
\subsection{Módulos cíclicos}
\begin{ndef}[Anulador minimal de M]
	Si M es cualquier módulo, entonces
	\[
	\{a: ax = 0 \ \forall x \in M\}\subseteq A \ \  \text{es un ideal}
	\]
	Además, como estamos en un DE, todos los ideales son principales y por tanto este también lo es, así que existe un elemento que genera ese ideal al que se le llama anolador minimal de M, denotado por
\[
\mu(M)
\]
y que es único salvo asociados
\end{ndef}

También, dado un cierto $x\in M$, podemos ver el conjunto $\{a : ax = 0\} \subseteq A$. Este también es un ideal y también es principal. A este se le llama el anulador minimal de x y se le denota:
\[
\mu(x)
\]

\begin{nprop}
	Si expresamos $M= (x_1,...,x_r) = Ax_1 + ...+ Ax_r \implies \mu(M) = mcm(\mu(x_1),...,\mu(x_r))$
\end{nprop}
\begin{proof}
	$\forall i \ \mu(M)x_i = 0 \implies \forall i \ \ \mu(x_i) / \mu(M)$.
	
	Por tanto, si $a\in A: \mu(x_i)/a \ \forall i \implies ax_i = 0 \ \forall i \implies a\sum_i a_ix_i = \sum_i a_i(ax_i) = 0$
	$\implies ax = 0 \ \forall x \in M \implies \mu(M) / a$
\end{proof}

En particular, si estamos en un módulo cíclico y tenemos por tanto un solo elemento, entonces $\mu(M) = \mu(x)$ salvo asociados.

\begin{nota}
	En estos casos, si $\mu(x) = 0$, la aplicación $\varphi: A \to M$ que lleva a cada elemento de A en M de forma $a \mapsto ax$, esa aplicación es lineal, sobreyectiva y es inyectiva, pues $ker(\varphi) = \emptyset$ pues el anulador minimal es el cero, por tanto es un isomorfismo, $M \cong A$.
\end{nota}

\begin{nota}
	Si $\mu(M) = \mu(x) \ne 0$, la aplicación $\varphi :A/\mu(x) \to M$ tal que $[a]\mapsto ax$ está bien definida ( si $a \equiv b mod(\mu(x)) \implies a = b+q\mu(x) \implies ax = bx +q\mu(x)x = bx$), es lineal, sobreyectiva e inyectiva (si $\varphi([a]) = 0 \implies ax = 0 \implies  \mu(x)/a \implies [a] = 0$) , por lo que
	\[
	M  \cong A/(\mu(M))
	\]
\end{nota}

\begin{ndef}[Presentacion de M por generadores y relaciones]
	Sea M un A-módulo finitamente generado. Se llama entonces "Presentación de M por generadores y relaciones" a toda expresión de la forma
	\[
	\langle x_1,...,x_n : \sum_i^n a_{i1}x_i = 0,...,\sum_i^n a_{im}x_i = 0 \rangle
	\]
	Donde $(x_1,...,x_n)$ es un sistema de generadores de $M$ y lo que hay a la derecha son relaciones de dependencia lineal entre ellos que son verificadas en el módulo tales que cualquier relación de dependencia entre $(x_1,...,x_n)$ es combinación lineal de las que aparecen en la presentación
	
\end{ndef}

\begin{nprop}
		Si 
	\[
	\langle x_1,...,x_n : \sum_i^n a_{i1}x_i = 0,...,\sum_i^n a_{im}x_i = 0 \rangle
	\]
	es una presentación del módulo M, la matriz:
	\[
	\begin{pmatrix}
 a_{11} & \cdots & a_{1n}\\
 \vdots & & \vdots\\
 a_{m1} & \cdots & a_{mn} 
\end{pmatrix} 
	\]
	cuyas filas son exactamente los coeficientes que aparecen en la presentación. A esta matriz $\mathcal{M}_R$ la llamamos matriz de relaciones de los generadores de M
	
	Decir entonces que las combinaciones lineales de la presentación son cero, es lo mismo que decir:
	\[
	\begin{pmatrix}
 a_{11} & \cdots & a_{1n}\\
 \vdots & & \vdots\\
 a_{m1} & \cdots & a_{mn} 
\end{pmatrix} \begin{pmatrix}
 x_1  \\
 \vdots\\
 x_n 
\end{pmatrix} =  \begin{pmatrix}
 0  \\
 \vdots\\
 0
\end{pmatrix} 
	\]
\end{nprop}

	
	Si tenemos un módulo $A^n$, podemos ver que este es libre y finitamente generado, además de que tiene una base (la canónica de $\mathbb{R}^n$. Vamos a considerar una aplicación $\varphi A^n \to M$ tal que $\varphi (a_1,...a_n) = a_1x_1+...+a_nx_n$. Esta aplicación es un epimorfirmo.

\begin{nprop}
	\[
	\sum_i^n a_i x_i  = 0 \iff (a_1,...a_n) \in Ker(\varphi)
	\]
	En particular, que se verifiquen las relaciones de dependencia que aparecen en la presentación, significa que $F_1,...,F_m \in Ker(\varphi)$.
	Y decir que cualquier relación de dependencia entre $x_1,...,x_n$ es combinación lineal de las que aparecen en la presentación significa que si $F((a_1,...a_n) \in Ker(\varphi) \implies F$ es combinación lineal de $F_1,...,F_m$, por tanto, estamos diciendo que
	\[
	Ker(\varphi) = (F_1,...,F_m)
	\]
\end{nprop}

\begin{nota}
	Es importante recordar que estos sistemas de generadores pueden ser sometidos a operaciones elementales sin dejar de ser sistemas de generadores. Las operaciones pueden ser:
	\begin{itemize}
	\item Permutar dos de ellos
	\item Multiplicar uno por una unidad del anillo
	\item Sumar a uno de ellos un múltiplo de otro.
\end{itemize}
Aún así, la matriz de relaciones se verá alterada y tendremos 'otra presentación distinta'
\end{nota}

Realizando estas operaciones, siempre podemos llegar a obtener la forma normal de Schmidt de una matriz.

\begin{nprop}
	$\forall M $, existe una presentación cuya matriz de relaciones está normalizada, que tiene la forma de una matriz de Schmidt. Es decir, de la forma:
	\[
	M : \langle \ \  y_1,...,y_n : \ d_1y_1 = 0 \ ,..., \ d_ry_r = 0 \ \  \rangle
	\]
	
\end{nprop}
Es más, si ahora consideráramos un nuevo epimorfismo $\varphi: A^n \to M$ tal que $\varphi (a_1,...a_n) 0 \sum a_i y_i$ y 
\[
Ker(\varphi): \ \langle d_1e_1 \ \ , .... , \ \ d_r e_r \ \ \rangle
\]
con $e_i$ el vector de la base canónica (tiene un 1 en la posición i-ésima y lo demás son ceros).

\begin{nprop}
	Se puede comprobar que:
	\[
	ay_i = 0 \implies \varphi (ae_i) = 0 \implies ae_i \in Ker(\varphi) \implies ae_i = b_1d_1e_1+...+b_rd_re_r
	\]
	\[
	\implies b_1d_1e_1+...+b_rd_re_r - ae_i = 0
	\]
	\begin{itemize}
	\item  si $i>r \implies a = 0 \implies \mu(y_i) = 0$.
	\item si $i \leq r \implies a = b_id_i$ pues son todos linealmente independientes $\implies d_i/a \implies \mu(y_i) = d_i$
\end{itemize}
\end{nprop}

\begin{nprop}
	$M \oplus_i AY_i$, todo elemento de M se expresa de forma única como combinación lineal de elementos de $A$
\end{nprop}
\begin{proof}
	Supongamos que $\sum a_i y_i = 0 \implies \varphi(\sum a_i e_i ) = 0 \implies \sum a_ie_i \in Ker (\varphi) $\\
	$\implies a_1e_1+...+ a_re_r+...+a_ne_n = b_1d_1e_1+...+b_rd_re_r \implies a_i = 0 \ \ \forall i > r$\\
	$\implies a_iy_i = 0$
	
	Y si $i \leq r\implies a_i = b_id_i \implies a_iy_i = 0$
\end{proof}
También podemos decir si $d_1 = ... = d_s = 1 \implies y_1 = y_2 = ... = y_s = 0$ y por tanto tendríamos que:
\[
M = Ay_{s+1}  \oplus ... \oplus Ay_r \oplus ... \oplus Ay_n
\]
Y tenemos que $\mu(y_i) = d_i \in A$, no es cero ni unidad $\forall i = s+1,...,r$ y cada $d_i/d_{i+1}$ y $\mu(y_{r+1}) = ... \mu(y_n) = 0$

\begin{nprop}
	En estas condiciones,
	\[
	M \cong Ay_{s+1} \times ... \times Ay_r \times ... \times Ay_n
	\]
	Siendo $Ay_{s+1},..., Ay_r$ módulos cíclicos y desde $r+1$ hasta $n$, $Ay_j$ es isomorfo al propio anillo, por lo que podemos decir que:
	\[
	M \cong Ad_{s+1}\times ... \times Ad_r \times A^{n-r}
	\]
\end{nprop}

\begin{nota}
	El hecho que hemos expuesto se conoce como el Teorema de estructura de módulos cíclicos
\end{nota}
A $n-r$ se le llama el rango del módulo y $d_{s+1},...,d_s$ es la lista de los factores invariantes del módulo M

La unicidad de estos invariantes es consecuencia de la unicidad de los invariantes de la forma normal de Schmidt.

\begin{ncor}
	Para conocer un módulo, basta conocer la lista de sus factores y su rango.
\end{ncor}
\begin{ncor}
	Si el rango de un módulo ($n-r$) es mayor quecero, entonces $\mu(M) = 0$.\\
	
	Si $n-r$ = 0, entonces $\mu(M) $ es el mayor de los factores invariantes, el $d_r$
\end{ncor}

A continuación, ilustraremos el teorema que hemos dado con ejemplos:

\subsubsection{$\ent$-Módulos. Grupos abelianos.}

Vamos a tratar de llevarnos los conceptos que hemos obtenido con el teorema de estructura al caso de los $\ent-$módulos.\\
Si M es un $\ent$-módulo, $n\in \ent$ con $n>0$ y $x\in M$, entonces $nx = (\sum_i^n 1)x = \sum_i^n 1*x = \sum_i^n x$. Lo mismo podemos ver si lo hacemos con $n$ negativo.

Entonces, un $\ent$-módulo es un grupo abeliano. Hablaremos de aquí en adelante de grupos abelianos, aunque sabremos que es lo mismo que un $\ent-$módulo.

Si $M$ es un grupo abeliano y $x\in M$, entonces $\ent x = (x) = \{nx : n \in \ent  \}$.
Tenemos también que $\mu(x) = d \iff dx = 0$ y $mx = 0 \implies d/m$.

Si $M=\ent x$ y $\mu(x) = 0 \implies M \cong \ent$

\begin{nth}[Teorema de estructura en grupos abelianos]\hfill \\
	Para todo grupo abeliano finitamente generado $M$ existen enteros positivos $d_1,...,d_s$ con $d_i \geq 2$ tales que $d_i/d_{i+1}$ y existe un $r> 0$ tal que 
	\[
	M \cong \ent_{d_1} \times ... \times \ent_{d_s} \times \ent^n
	\]
	y a estos $d_i$ se les llama factores invariantes del grupo abeliano y $r$ es su rango.
	A la descomposición que hemos obtenido se le llama descomposición cíclica del grupo abeliano.
\end{nth}

\begin{nprop}
	Dos grupos abelianos son isomorfos $\iff$ tienen los mismos factores invariantes y el mismo rango. Por tanto, son isomorfos $\iff$ tienen la misma descomposición cíclica.
\end{nprop}

\begin{nprop}
	$M$ es finito $\iff$ su rango es 0. En ese caso, el tamaño del grupo es el producto de sus invariantes  $|M| = d_1...d_s$ y en tal caso $\mu(M) = d_s$
\end{nprop}
\begin{nprop}
	Si el rango de un módulo $M$ no es cero, entonces $\mu(M) =0$
\end{nprop}

\begin{ejemplo}
	Un grupo abeliano $M$ está dado por:
	\[
	M: \langle \ x,y,z :  12x+26y+13z = 0  \ , \ 6x+12y+6z = 0 \ , \ 6x+26y+13z = 0 \ \rangle
	\]
	¿Cuál es la descomposición cíclica de este grupo?
\end{ejemplo}
\begin{proof}[Solución]
	Lo primero que habría que hacer es construir la matriz de relaciones.
	
	\[
	\begin{pmatrix}
 12 & 26 & 13 \\
 6 & 12 & 6 \\
 6 & 26 & 13 
\end{pmatrix} 
	\]
	
	Y ahora buscamos su forma normal de Schmidt, para ver cuál es el rango de esta matriz.
	
	\[
	\begin{pmatrix}
 12 & 26 & 13 \\
 6 & 12 & 6 \\
 6 & 26 & 13 
\end{pmatrix} \sim
\begin{pmatrix}
 6 & 12 & 6 \\
 12 & 26 & 13 \\
 6 & 26 & 13 
\end{pmatrix} \sim^{C_2 - 2C_1}
\begin{pmatrix}
 6 & 0 & 6 \\
 12 & 2 & 13 \\
 6 & 14& 13 
\end{pmatrix} 
	\] 
	Seguimos haciendo este método a llegar a una de la forma de Schmidt y nos queda la matriz:
	\[
	\begin{pmatrix}
 1 & 0 & 0 \\
 0 & 6 & 0 \\
 0 & 0 & 0 
\end{pmatrix} 
	\]
	(comprobar si está bien).
	
	Así, los factores invariantes de la matriz son 1 y 6, y los factores invariantes de M serán , de los obtenidos de la matriz, los que no sean unidades, por lo que en este caso es el 6. Cuidado, porque si aparecieran dos 6, los factores invariantes serían $\{6,6\}$. 
	
	El rango de M, es el número de columnas nulas de esta matriz, que es 1. 
	
	Por tanto, la descomposición cíclica es:
	\[
	M \cong \ent_6 \times \ent
	\]
\end{proof}

\begin{ejemplo}
	Sea 
	\[
	M: \langle \ x,y,z :  2x+y+3z = 0  \ , \ 3x+6y-2z = 0 \ , \ 2x+4y+4z = 0 \ \rangle
	\]
	¿Cuál es la descomposición cíclica de este grupo?
\end{ejemplo}
\begin{proof}[Solución]:

De nuevo, hallamos su matriz y le hacemos su forma normal de Schmidt
	
\[
	\begin{pmatrix}
 2& 1  & 3 \\
 3 & 6 & -2 \\
 2 & 4 & 4 
\end{pmatrix}  \sim 	\begin{pmatrix}
 1 & 0 & 0 \\
 0 & 1 & 0 \\
 0 & 0 & 48 
\end{pmatrix} 
\]
Y hacemos el mismo procedimiento de antes para hallar sus invariantes (solo el 48), su rango (0) y su descomposición cíclica
\[
M \cong \ent_{48} 
\]

	
\end{proof}

\begin{nprop}
	Si $(m,n) = 1 \implies \ent_{mn} \cong \ent_m \times \ent_n$.
Si $d = p_1^{e_1}...p_r^{e_r}$ con $p_i$ primo distintos entre sí y $e_i \geq 1$, entonces:
\[
\ent_d \cong \ent_{p_1^{e_1}} \times ... \times \ent_{p_r^{e_r}}
\]
\end{nprop}

\begin{nprop}
	Si
	\[
	M \cong \ent_{d_1} \times ... \times \ent_{d_s} \times \ent^r
	\]
	con $d_i \geq 2 $ y $d_i / d_{i+1}$, entonces, la descomposición de invariantes en primos estará dada por los mismos primos pero si $d_i = p_1^{e_{i1}}...p_r{e_{ir}}$ y tenedremos $0 \leq e_{ij} \leq e_{i+1j}$
	
	Así, podemos tener:
	\[
	M \cong \prod_i^s \prod_j ^k \ent_{p_i^{e_{ij}}} \times \ent^r
	\]
	A la que llamamos descomposición cícica Primaria. En ella, la lista ${p_i^{e_{ij}}}$ es la lista de divisores elementales de $M$.
\end{nprop}
\begin{ejemplo}
	Un grupo abeliano tiene como divisores elementales: $\{3,9,5,25,125 y 7\} $ (que son potencias de primos) y rango 0. Así, ese grupo es, en su descomposición primaria:
	\[
	\ent_3 \times \ent_9 \times \ent_5 \times \ent_{25} \times \ent_{125} \ent_7
	\]
	
Sus invariantes serían $d_1 = 9*125*7 = 7875$, $d_2 = 3*5^2 = 75$, $d_3 = 5$ y ahora los ordenamos, quedando que: $d_3 = 9*125*7 = 7875$, $d_2 = 3*5^2 = 75$, $d_3 = 5$, por lo que su descomposición cíclica sería:
\[
\ent_5 \times \ent_{75} \times \ent_{7875}
\]
\end{ejemplo}

\end{document}

