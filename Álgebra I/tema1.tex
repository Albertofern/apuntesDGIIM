%%%%%%%%%%%%%%%%%%%%%%%%%%%%%%%%%%%%%%%%%%%%%%%%%%%%%%%%%%%%%%%%
%
% Apuntes de la asignatura Análisis Matemático I.
% Doble Grado de Informática y Matemáticas.
% Universidad de Granada.
% Curso 2016/17.
% 
% 
% Colaboradores:
% Javier Sáez (@fjsaezm)
% Daniel Pozo (@danipozodg)
% Pedro Bonilla (@pedrobn23)
% Guillermo Galindo
% Antonio Coín (@antcc)
%
% Agradecimientos:
% Andrés Herrera (@andreshp) y Mario Román (@M42) por
% las plantillas base.
%
% Sitio original:
% https://github.com/libreim/apuntesDGIIM/
%
% Licencia:
% CC BY 4.0 (https://creativecommons.org/licenses/by/4.0/)
%
%%%%%%%%%%%%%%%%%%%%%%%%%%%%%%%%%%%%%%%%%%%%%%%%%%%%%%%%%%%%%%%


%------------------------------------------------------------------------------
%   ACKNOWLEDGMENTS
%------------------------------------------------------------------------------

%%%%%%%%%%%%%%%%%%%%%%%%%%%%%%%%%%%%%%%%%%%%%%%%%%%%%%%%%%%%%%%%%%%%%%%%
% Plantilla básica de Latex en Español.
%
% Autor: Andrés Herrera Poyatos (https://github.com/andreshp) 
%
% Es una plantilla básica para redactar documentos. Utiliza el paquete  fancyhdr para darle un
% estilo moderno pero serio.
%
% La plantilla se encuentra adaptada al español.
%
%%%%%%%%%%%%%%%%%%%%%%%%%%%%%%%%%%%%%%%%%%%%%%%%%%%%%%%%%%%%%%%%%%%%%%%%%

%%%
% Plantilla de Trabajo
% Modificación de una plantilla de Latex de Frits Wenneker para adaptarla 
% al castellano y a las necesidades de escribir informática y matemáticas.
%
% Editada por: Mario Román
%
% License:
% CC BY-NC-SA 3.0 (http://creativecommons.org/licenses/by-nc-sa/3.0/)
%%%

%%%%%%%%%%%%%%%%%%%%%%%%%%%%%%%%%%%%%%%%
% Short Sectioned Assignment
% LaTeX Template
% Version 1.0 (5/5/12)
%
% This template has been downloaded from:
% http://www.LaTeXTemplates.com
%
% Original author:
% Frits Wenneker (http://www.howtotex.com)
%
% License:
% CC BY-NC-SA 3.0 (http://creativecommons.org/licenses/by-nc-sa/3.0/)
%
%%%%%%%%%%%%%%%%%%%%%%%%%%%%%%%%%%%%%%%%%


% Tipo de documento y opciones.
\documentclass[11pt, a4paper, titlepage]{article}


%---------------------------------------------------------------------------
%   PAQUETES
%---------------------------------------------------------------------------

% Idioma y codificación para Español.
\usepackage[utf8]{inputenc}
\usepackage[spanish, es-tabla, es-lcroman, es-noquoting]{babel}
\selectlanguage{spanish} 
%\usepackage[T1]{fontenc}

% Fuente utilizada.
\usepackage{courier}    % Fuente Courier.
\usepackage{microtype}  % Mejora la letra final de cara al lector.

% Diseño de página.
\usepackage{fancyhdr}   % Utilizado para hacer títulos propios.
\usepackage{lastpage}   % Referencia a la última página.
\usepackage{extramarks} % Marcas extras. Utilizado en pie de página y cabecera.
\usepackage[parfill]{parskip}    % Crea una nueva línea entre párrafos.
\usepackage{geometry}            % Geometría de las páginas.

% Símbolos y matemáticas.
\usepackage{amssymb, amsmath, amsthm, amsfonts, amscd}
\usepackage{upgreek}

% Otros.
\usepackage{enumitem}   % Listas mejoradas.


%---------------------------------------------------------------------------
%   OPCIONES PERSONALIZADAS
%---------------------------------------------------------------------------

% Redefinir letra griega épsilon.
\let\epsilon\upvarepsilon

% Formato de texto.
\linespread{1.1}            % Espaciado entre líneas.
\setlength\parindent{0pt}   % No indentar el texto por defecto.
\setlist{leftmargin=.5in}   % Indentación para las listas.

% Estilo de página.
\pagestyle{fancy}
\fancyhf{}
\geometry{left=3cm,right=3cm,top=3cm,bottom=3cm,headheight=1cm,headsep=0.5cm}   % Márgenes y cabecera.

% Redefinir entorno de demostración (reducir espacio superior)
\makeatletter
\renewenvironment{proof}[1][\proofname] {\vspace{-15pt}\par\pushQED{\qed}\normalfont\topsep6\p@\@plus6\p@\relax\trivlist\item[\hskip\labelsep\it#1\@addpunct{.}]\ignorespaces}{\popQED\endtrivlist\@endpefalse}
\makeatother


%---------------------------------------------------------------------------
%   COMANDOS PERSONALIZADOS
%---------------------------------------------------------------------------

% Números enteros: \ent
\providecommand{\ent}{\mathbb{Z}}

% Números racionales: \rac
\providecommand{\rac}{\mathbb{Q}}


% Valor absoluto: \abs{}
\providecommand{\abs}[1]{\lvert#1\rvert}    

% Fracción grande: \ddfrac{}{}
\newcommand\ddfrac[2]{\frac{\displaystyle #1}{\displaystyle #2}}

% Texto en negrita en modo matemática: \bm{}
\newcommand{\bm}[1]{\boldsymbol{#1}}

% Línea horizontal.
\newcommand{\horrule}[1]{\rule{\linewidth}{#1}}


%---------------------------------------------------------------------------
%   CABECERA Y PIE DE PÁGINA
%---------------------------------------------------------------------------

% Cabecera del documento.
\renewcommand\headrule{
	\begin{minipage}{1\textwidth}
		\hrule width \hsize 
	\end{minipage}
}

% Texto de la cabecera.
\lhead{\subject}  % Izquierda.
\chead{}            % Centro.
\rhead{\docauthor}    % Derecha.

% Pie de página del documento.
\renewcommand\footrule{                                 
	\begin{minipage}{1\textwidth}
		\hrule width \hsize   
	\end{minipage}\par
}

% Texto del pie de página.
\lfoot{}                                                 % Izquierda
\cfoot{}                                                 % Centro.
\rfoot{Página\ \thepage\ de\ \protect\pageref{LastPage}} % Derecha.


%---------------------------------------------------------------------------
%   ENTORNOS PARA MATEMÁTICAS
%---------------------------------------------------------------------------

% Nuevo estilo para definiciones.
\newtheoremstyle{definition-style} % Nombre del estilo.
{10pt}               % Espacio por encima.
{10pt}               % Espacio por debajo.
{}                   % Fuente del cuerpo.
{}                   % Identación.
{\bf}                % Fuente para la cabecera.
{.}                  % Puntuación tras la cabecera.
{.5em}               % Espacio tras la cabecera.
{\thmname{#1}\thmnumber{ #2}\thmnote{ (#3)}}     % Especificación de la cabecera (actual: nombre en negrita).

% Nuevo estilo para notas.
\newtheoremstyle{remark-style} 
{10pt}                
{10pt}                
{}                   
{}                   
{\itshape}          
{.}                  
{.5em}               
{}                  

% Nuevo estilo para teoremas y proposiciones.
\newtheoremstyle{theorem-style}
{10pt}                
{10pt}                
{\itshape}           
{}                  
{\bf}             
{.}                
{.5em}               
{\thmname{#1}\thmnumber{ #2}\thmnote{ (#3)}}                   

% Nuevo estilo para ejemplos.
\newtheoremstyle{example-style}
{10pt}                
{10pt}                
{}                  
{}                   
{\scshape}              
{:}                 
{.5em}               
{}                   

% Teoremas, proposiciones y corolarios.
\theoremstyle{theorem-style}
\newtheorem*{nth}{Teorema}
\newtheorem*{nprop}{Proposición}
\newtheorem{ncor}{Corolario}

% Definiciones.
\theoremstyle{definition-style}
\newtheorem*{ndef}{Definición}

% Notas.
\theoremstyle{remark-style}
\newtheorem*{nota}{Nota}

% Ejemplos.
\theoremstyle{example-style}
\newtheorem*{ejemplo}{Ejemplo}

% Listas ordenadas con números romanos (i), (ii), etc.
\newenvironment{nlist}
{\begin{enumerate}
\renewcommand\labelenumi{(\emph{\roman{enumi})}}}
{\end{enumerate}}


%---------------------------------------------------------------------------
%   PÁGINA DE TÍTULO
%---------------------------------------------------------------------------

% Título del documento.
\newcommand{\subject}{Álgebra I}

% Autor del documento.
\newcommand{\docauthor}{Doble Grado de Informática y Matemáticas}

% Título
\title{
  \normalfont \normalsize 
  \textsc{Universidad de Granada} \\ [25pt]    % Texto por encima.
  \horrule{0.5pt} \\[0.4cm] % Línea horizontal fina.
  \huge \subject\\ % Título.
  \horrule{2pt} \\[0.5cm] % Línea horizontal gruesa.
}

% Autor.
\author{\Large{\docauthor}}

% Fecha.
\date{\vspace{-1.5em} \normalsize Curso 2016/17}


%---------------------------------------------------------------------------
%   COMIENZO DEL DOCUMENTO
%---------------------------------------------------------------------------
\begin{document}
\maketitle

\section{Anillo conmutativo}

\begin{ndef}[Anillo conmutativo]
Un conjunto A es un anillo conmutativo si en él hay definidas dos operaciones; una aplicacion de adición y una aplicación de multiplicación, tales que cumplen las siguientes propiedades:
\begin{nlist}
\item Asociativa: $a+(b+c) = (a+b)+c$\hspace{1cm} $a(bc) = (ab)c$
\item Conmutativa: $a+b = b+a$ \hspace{1cm} $ab = ba$
\item Existencia elemento neutro: $a+0 = a$ \hspace{1cm} $a*1 = a$
\item Existencia del elemento opuesto: $a+(-a) = 0$
\item Distributiva del producto en la suma: $a(b+c) = ab+ ac$

\end{nlist}

\end{ndef}

\begin{ndef}[Grupo conmutativo]
	Denominamos un grupo conmutativo o abeliano a aquellos conjuntos que cumplen las propiedades asociativa, conmutativa y existencia de elemento neutro para la suma, y existencia de elemento opuesto.
\end{ndef}

\begin{ndef}[monoide]
	Denominamos monoide a un conjunto con una operación binaria interna que cumple la propiedad asociativa y tiene un elemento neutro a izquierda y derecha. En el caso del producto, se denomina monoide multiplicativo.
\end{ndef}

\begin{nota}
	Llamaremos anillo aquellos conjuntos que cumplan todas las propiedades excepto la propiedad conmutativa para la multiplicación.
\end{nota}


\section{Caracterización de $\mathbb{Z}_{n}$.}

Llamaremos $R_n:\mathbb{N} \rightarrow \mathbb{Z}_n$ a la aplicación definida como:

\[
R_n(a) = a - nq = a- nE(\frac{a}{n})
\] 
Para esta aplicación, definimos las siguientes propiedades:

\begin{itemize}
\item Si $0 \leq a < n-1  \rightarrow R_n(a) = a$
\item $\forall a,b \in \mathbb{N}$

\begin{itemize}
	\item  $R_n(a+b) = R_n(R_n(a) + R_n(b))$
	\item $R_n(ab) = R_n(R_n(a)*R_n(b))$
\end{itemize}

\end{itemize} 
\pagebreak

Una vez que tenemos definida una suma y producto con la aplicación $R_n$, definimos las suma y el producto de $\mathbb{Z}_n$.
\begin{ndef}[Suma y producto en $\ent_n$] Se define la suma y el producto en $\ent_n$ de la forma:
	\begin{itemize}
	\item $a\oplus b = R_n(a+b)$
	\item $a\otimes b = R_n(ab)$
\end{itemize}

\end{ndef}



Es fácil verificar que $\mathbb{Z}_n$ es un anillo conmutativo con estas operaciones.\\

\begin{ndef}[Unidad]
	Si A es un anillo conmutativo (a.c) $a \in A$ es una "\textbf{unidad}" o "\textbf{invertible}" si $\exists a^{-1}$ t.q. $ aa^{-1} = 1$.\\ $U(A) = \{ a \in A$ t.q. a es una unidad\} = conjunto de las unidades de A.\\
\end{ndef}



\begin{ndef}[Cuerpo] Se dice que A es un \textbf{cuerpo} si siendo un anillo conmutativo, $U(A) = A - \{0\}$, es decir, $\exists a^{-1}$ $\forall a \in A$ con $a \neq 0$.
\end{ndef}


\begin{nprop}[Asociatividad generalizada]
	Sea A un anillo conmutativo, y $a_1 ... a_n$ una lista de elementos de A.
La propiedad de la \textbf{asociatividad generalizada} nos dice que:
$\forall m$ tal que $1 \leq m < n$ entonces: \[\sum_{i=1}^{n}a_i = (\sum_{i=1}^{m}a_i) + (\sum_{i=m+1}^{n}a_i)\]\[\prod_{i=1}^{n}a_i = (\prod_{i=1}^{m}a_i)(\prod_{i=m+1}^{n}a_i)\]\\
\end{nprop}


\begin{ndef}[Distributividad generalizada]
Definimos también la distributividad generalizada en un anillo como:
	\[(\sum_{i=0}^{n} a_i)(\sum_{j=1}^{m}b_j) = \sum_{i=1}^{n}\sum_{j=1}^{m}a_i b_j \hspace{1cm} \forall a,b\in A\]
\end{ndef}

\begin{ndef}[Subanillo]

Si A es un anillo conmutativo y B es un subconjunto de A. Se dice que B es un \textbf{subanillo} de A ($B \leq A$) si se verifican:
\begin{itemize}
\item $1,-1 \in B$
\item B es cerrado para la suma y el producto.
\end{itemize}
	
\end{ndef}


\subsection{Ejemplos- Anillos de números cuadráticos}

\begin{itemize}
\item $\mathbb{Z}[\sqrt{n}]$.
Definimos este conjunto de la siguiente forma:
\begin{center}$\mathbb{Z}[\sqrt{n}] = \{a+b\sqrt{n} \in \mathbb{C}$ : $a,b \in \mathbb{Z} \}$ $\leq \mathbb{C}$\end{center}

Podemos definir también $\mathbb{Q}[\sqrt{n}]$ de la misma forma:

\begin{center}$\mathbb{Q}[\sqrt{n}] = \{a+b\sqrt{n} \in \mathbb{C}$ : $a,b \in \mathbb{Q} \}$ $\leq \mathbb{C}$\end{center}

Se puede comprobar que $\mathbb{Z}[\sqrt{n}]$ $\leq$ $\mathbb{Q}[\sqrt{n}]$ y que $\mathbb{Q}[\sqrt{n}]$ es un cuerpo.

\begin{ndef}[Conjugado]
	Si $\alpha = a+b\sqrt{n}\in \mathbb{Q}[\sqrt{n}]$ se define su conjugado como $\bar{\alpha} = a - b\sqrt{n}$. Este verifica que:
\begin{enumerate}

\item $ \overline{(\alpha+ \beta)} $ = $\bar{\alpha} + \bar{\beta}$
\item $\overline{\alpha \beta}$ = $\bar{\alpha}\bar{\beta}$
\item $\alpha = \bar{\alpha} \Leftrightarrow b = 0$
	
\end{enumerate} 
\end{ndef}

\begin{ndef}[Norma]
	Se define entonces la Norma $N(\alpha) = \alpha \bar{\alpha} = a^2 - nb^2 \in \rac $. Así:
\begin{enumerate}
	\item $N(\alpha \beta) = N(\alpha) * N(\beta)$
	\item $N(\alpha) = 0 \iff \alpha = 0$
\end{enumerate}
\end{ndef}


\begin{nprop}
	$\alpha \in a + b \sqrt{n} \in \ent[\sqrt{n]}$ es invertible $\iff$ $N(\alpha) \in \{-1,1\}$
\end{nprop}



\item Anillos de series.
\begin{ndef}
	Si A es un anillo conmutativo y X es un símbolo que no denota ningún elemento de A. El anillo de series con coeficientes en A, denotado con A[[x]] esta definido como:\begin{center}
$A[[x]] = \{a = \sum_{i=1}^{n}a_i x^i = a_0 + a_1 x^1 + ... + a_n x^n\}$ $a_i \in A$\end{center}
Y definimos la suma y el producto de la siguiente forma:\\
\[(a+b) = \sum_{i=0}^{n}(a_i+b_i)x^i\]\[ (ab) = \sum_{k=0}^n\sum_{i=0}^{k}a_ib_{k-i}\]
\end{ndef}
\end{itemize}

Se puede probar que con estas operaciones de suma y producto, $A[[x]]$ es un anillo y $A[x]$ es un subanillo de $A[[x]]$  

\section{Homomorfismos}

\begin{ndef}
Si $A,B$ son anillos conmutativos, una aplicacion $\varphi: A \to B$ es un homomorfismo si:

\begin{enumerate}
	\item $\varphi(1) = 1$
	\item $\varphi(a+b) = \varphi(a) + \varphi(b)$
	\item $\varphi(ab) = \varphi(a)  \varphi(b)$
	
\end{enumerate}
	
\end{ndef}

Además, decimos que:

\begin{enumerate}
	\item Es monomorfismo si es inyectivo.
	\item Es epimorfismo si es sobreyectivo.
	\item Es isomorfismo si es biyectivo.
\end{enumerate}

\textbf{Propiedades de los homomorfismos}

\begin{itemize}

\item $\varphi(0) = 0$

\item $\varphi(-a) = -\varphi(a) $

\item $\varphi(\sum_{i = 1}^n a_i) = \sum_{i = 1}^n\varphi(a_i)$. 

$\varphi(\prod_{i = 1}^n a_i) = \prod_{i = 1}^n\varphi(a_i)$

\item $\varphi(na) = n\varphi(a)$

	
\end{itemize}


Ya sabemos que $Im(\varphi) = \{ \varphi(x): x \in A\} \leq B$ es un subanillo.

\begin{nprop}
	Si $\varphi$ es monomorfismo, entonces la aplicación restringida:

\[
A \to Im(\varphi)\]
\[
a \mapsto \varphi(a)
\]

es un epimorfismo y por ello es un isomorfismo, podemos decir que $A \cong Im(\varphi)$.\\
\end{nprop}


\begin{nota}
	Se puede probar que $R_n: \ent \to \ent_n$ es un homomorfismo, llamado \emph{Homomorfismo de reducción módulo n}\\
\end{nota}


\begin{nprop}[1]
 Dado $A$ cualquier anillo conmutativo, conocido $A[x]$.
 
 Si $\varphi:A \to B$ es homomorfismo de anillos conmutativos, entonces:
 \[
 \exists \varphi: A[x] \to B[x] : \varphi\left(\sum_i a_i x^i\right) = \sum_i\varphi(a_i) x^i
 \]

	
\end{nprop}

\begin{nprop}[Sustición en un polinomio(2)]
	
Si $A$ es cualquier conjunto y $a \in A$ entonces: existe un homomorfismo $E_a: A[x] \to A$ tal que $E_a(\sum_i a_i x^i) = \sum_i a_i a^i$.
	
\end{nprop}

\begin{nprop}[3]
Si $A \leq B$ es un subanillo y $b\in B$, la aplicación $E_b:A[x] \to B$ definida como $E_b(\sum_i a_i x_i) = \sum_i a_i b^i$ es un homomorfismo
	
\end{nprop}

\begin{nprop}[Engloba a las anteriores]
Si $\varphi:A \to B$ es un homomorfismo y $b\in B$, la aplicación $\Phi:A[x] \to B$ definida como $\Phi ( \sum_i a_i x_i) = \sum_i \varphi(a_i)b^i \in B$ es un homomorfismo 

\end{nprop}

\begin{proof}
Veamos primero cómo (4) engloba a las demás:
\begin{nlist}

\item $4 \Rightarrow 3$. Se ve tomando como $\varphi$ la inclusión en $B$
\item $4 \Rightarrow 2$. Tomamos esta vez como $\varphi$ la identidad
\item $4 \Rightarrow 1$. Suponemos 4 válido. Probaremos que $\exists\varphi: A \to B[x]$ que lleva $a \to \varphi(a)$.
Ahora, podemos ver que esa aplicación es como usar primero $\varphi$ para ir de A a B y luego usar la inclusión de $B$ en $B[x]$:
\[
A\to B \to B[x]
\]
\[
a \to a \to \varphi(a)
\]

De esta forma, tomamos $x\in B[x]$. Entonces:
\begin{center}
	$A[x]\to B[x]$\\
	$\sum_i a_i x_i \to \sum_i \varphi(a_i)x_i$
\end{center}

Que es justamente el enunciado de la primera proposición.

\end{nlist}

Pasamos ahora a la demostración de la Proposición 4.


Sean $f = \sum a_ix_i$ y $g = \sum b_i x_i \in A[x]$. Entonces: $f+g = \sum c_i x_i$ con $c_i = a_i + b_i$

Si ahora aplicamos $\Phi(f+g) = \sum \varphi(c_i)b^i = \sum \varphi(a_i + b_i)b^i$. 

Como $\varphi$ es homomorfismo , eso es igual a:
 $\sum (\varphi(a_i) + \varphi(b_i))b^i$.
 
 Usando que $B$ es un anillo y por ello hay distributividad, eso es igual a: $\sum (\varphi(a_i)b^i + \varphi(b_i)b^i$.
 
 Por la asociatividad generalizada eso es igual a: $\sum \varphi(a_i)b^i + \sum \varphi(b_i)b^i = \Phi(f) + \Phi(g)$ Por lo que queda probado para la suma.
 
 Ahora probaremos el producto:
 
 $fg = \sum c_i x^i$ con $c_i  = \sum_{i+j = n} a_ib_j$\\
 
 Así: \[\Phi(f+g) = \sum_n\varphi(c_n)b^n = \sum \varphi(\sum_{i+j = n} a_ib_j)b^n = \sum_n ( \sum_{i+j = n} \varphi(a_i)\varphi(b_j))b^n\]
 
 Desarrollamos por otro lado
 \[
 \Phi(f) + \Phi(g) = (\sum_i \varphi(a_i)b^i)(\sum_j \varphi(b_j)b^i) =^{(1)} \sum_{i,j} \varphi(a_i)b^i\varphi(b_j)b^j =^{(2)} \sum_{i,j} \varphi(a_ib_j)b^{i+j}=
 \]
 
 \[
  = \sum_n(\sum_{i,j: i+j = n} \varphi(a_ib_j)b^n)
 \]
 
 Donde en $(1)$ hemos usado la distributividad general y en $(2)$ hemos usado que estamos en un anillo conmutativo y que $\varphi$ es un homomorfismo. 
 
 Así, hemos llegado a dos expresiones que son iguales, probando así el resultado.
 
\end{proof}


Sabemos que cada polinomio $f(x)$ constituye una función de evaluación $f(x) \in A[x]$
\[
f(x):B \to B
\]
\[
\hspace{1cm}b \to f(b)
\]

Sin embargo, un polinomio es mucho más que la función de evaluación que él mismo define. Estudiaremos el caso $A[x_1,...,x_r]$

\begin{ndef}[Polinomios de r variables con coeficientes en A]
Sea A un anillo conmutativo. Consideramos $A[x_1,...x_r]$ inductivamente en $r$:\\
Si $r>1$ entonces $A[x_1,...,x_r] = A[x_1,...,x_{r-1}][x_r]$\\
	
\begin{proof}\hfill 

\begin{itemize}

	\item $r=1$:
\[
f(x_1) \in A[x_1] \hspace{1cm} \sum_{i \geq 0}a_i x_i \hspace{0.5cm} a_i \in A \hspace{0.5cm} \exists K: a_{i1} = 0 \hspace{0.3cm} \forall i > K
\]
	\item $r> 1$

\[
f(x_1,...,x_r) = \sum_{i1,...ir} a_{i1},...,a_{ir} x_1^{i1},...,x_r^{ir}: \hspace{0.3cm} \exists K: a_i,...,a_r  = 0 \iff i_s > K
\]
Ahora, si vemos que:
\[
f_{ir}(x_1,...,x_{r-1}) = \sum_{i1,...ir > 0} a_{i1},...,a_{ir} x_1^{i1},...,x_r^{ir-1} \in A[x_1,...,x_{r-1}]
\]
Entonces:
\[
\sum_{ir \geq 0} f_{ir}(x_1,...,x_{r-1})x_r^{ir}= \sum_{ir \geq 0}(\sum_{i1,...,ir > 0}a_1,...,a_r x_1^{i1},...,x_{r-1}^{ir-1})x_r^{ir} = 
\]
\[
=\sum_{i1,...ir} a_{i1},...,a_{ir} x_1^{i1},...,x_r^{ir}
\]

Ahora, definimos $g(x_1,...,x_r) = \sum_{i1,...ir} b_{i1},...,b_{ir} x_1^{i1},...,x_r^{ir}$. Ahora, sumamos:

\[
f(x_1,...,x_r)+g(x_1,...,x_r) = \sum_{i1,...ir} a_{i1},...,a_{ir} x_1^{i1},...,x_r^{ir} + \sum_{i1,...ir} b_{i1},...,b_{ir} x_1^{i1},...,x_r^{ir} = 
\]
\[
 = \sum_{i1,...,ir} (a_i+bi)x^{i1+ij}
\]

Ahora, podemos desarrollar de la misma forma el producto y ver que:

\[
(ax_1^{i1},...,x_r^{ir})(bx_1^{j1},...,x_r^{jr}) = abx_i^{i+j}x_2^{i_2+j_2}...x_r^{i_r+j_r}
\]

\end{itemize}
Por lo que queda probado nuestro resultado.
\end{proof}

\end{ndef}

\begin{ndef} \textbf{(A[x][y])}
\hfill \\
Definimos $f=\sum f_i y^i | f_i \in A[x] : f_i = \sum_j a_{ij}x^j$\\
Luego, $f = \sum_i (\sum_j a_{ij}x^j)y^i = \sum_{i,j} a_{ij}x^iy^j$\\

Ahora, tomamos $g =\sum_{i,j} b_{ij}x^iy^j$ y sumamos:

\[
f+g = \sum_{i,j}(a_{ij}+b_{ij})x^iy^j
\]

Y , si $A[x][y]$ es un anillo, vemos que la multiplicación se realiza:
\[
(a_{ij}x^iy^j)(b_{mn}x^my^n) = a_{ij}b_{mn} x^{i+m}y^{j+n}
\]


Además, como es un anillo conmutativo $\Rightarrow$ $A[x][y] = A[y][x] = A[x,y]$

	
\end{ndef}


\begin{ndef}
	$A[x_1,...,x_n] = A[x_1,...,x_{n-1}][x_n]$

Se puede probar que $A[x_1,...,x_n] = A[x_{\sigma(1)},...,x_{\sigma(n)}]$ siendo $\sigma$ una permutación de $\{1,2,...,n \}$
\end{ndef}

\begin{nprop}
	
Si $\varphi :A \to B$ es un homomorfismo, $\forall(b_1,...,b_n) \in B^n$ la aplicación:

\[
\Phi:A[x_1,...x_n] \to B \iff \Phi:(\sum_{i1,...,in}a_1...a_nx^{i1}... x^{in}) = \sum_{i1,...,in}a_{i1}...a_{in}b^{i1}... b^{in} \in B
\]

es un homomorfismo de anillos conmutativos. Es conocido como evaluación de un polinomio en n variables.

\end{nprop}



\begin{nprop}
	Si $\varphi:A \to B$ es un homomorfismo, $\forall b \in B \exists ! $ homomorfismo definido como:
	
\[
\Phi:A[x] \to B : \begin{cases}
	\Phi(a) = \varphi(a) \forall a \in A \\
	\Phi(x) = b 
\end{cases}
\]
\[
\Phi(\sum a_i x_i) = \sum \Phi(a_i x_i) =  \sum \Phi(a_i) \Phi(x)^i = \sum \varphi(a_i) b^i
\]

Además, ya se probó que esto es un homomorfismo de anillos conmutativos.

\end{nprop}

\begin{ncor}
	$A \leq B$ subanillo, $\forall b \in B$ $\exists !$ homomorfismo
\[
E_b: A[x] \to B: \begin{cases}
E_a(a) = a \hspace{0.3cm} \forall a \in A\\
E_b(x) = b
	
\end{cases}
\] 
\end{ncor}

\begin{nota}
	Si $f(x)\in A[x]$ denota un polinomio de $A[x]$, notaremos: $E_b(f(x)) = f(b)$.
	De la misma forma, si $f(x) = \sum a_i x^i \Rightarrow E_b(f(x)) = \sum a_i b^i$
\end{nota}


\begin{nprop}[Evaluación en r-variables]
  Si $\varphi : A \longrightarrow B$ es un homomorfismo de anillos conmutativos, y $b_1,\cdots, b_r \in B$ una lista ordenada. Entonces
  \[
    \exists! \phi : A[x_1,\cdots,x_r] \longrightarrow B :\begin{cases}
      \phi(a) = \varphi(a) & \forall a\in A\\
      \phi(x_1) = b_1\\
      \vdots\\
      \phi(x_r) = b_r
    \end{cases}
  \]
\end{nprop}

\begin{proof}
  Si $r=1$, ya está probado. Para $r>1$:
  \[
    \exists \psi : A[x_1,\cdots,x_r] \longrightarrow B : \begin{cases}
      \psi(a) = \psi(a)\\
      \psi(x_i) = b_i \forall i=1,\cdots,r-1
    \end{cases}
  \]
  \[  \exists \phi : A[x_1,\cdots,x_r] \longrightarrow B \begin{cases}
      \phi(a) = \psi(a) = \varphi(a)\\
      \phi(x_i) = \psi(x_i) = b_i \forall i=1,\cdots,r-1\\
      \phi(x_r) = b_r
    \end{cases}
  \]
  
  ¿Es único?
  \[
    \phi(\sum_{i1,\cdots,ir} a_{1i}\cdots a_{ir}x_1^{i1}\cdots x_r^{ir} )= \sum _{i1,\cdots,ir} \varphi(a_{i1}\cdots a_{ir})b_1^{i1}\cdots b_r^{ir}
  \]
\end{proof}

\begin{nprop}[Evaluación en subanillos r-variables]
	Si $A\leq B, \forall b_1,\cdots,b_r \in B$ lista ordenada:
	
\[
\exists ! E_{b_1,\cdots,b_r}: A[x_1,\cdots,x_r] \to B :
\begin{cases}
	a \to a\\
	x_i \to b_i
\end{cases}
\]

Se suele notar $f(x_1,\cdots,x_r) \to f(b_1,\cdots,b_r)$
\end{nprop}

\section{Dominio de Integridad}

\begin{ndef}[Dominio de integridad]
A es un dominio de integridad si verifica la propiedad:
\[
a \neq 0, b \neq 0 \Rightarrow ab \neq 0 \iff si \hspace{0.1cm} ab=0 \begin{cases}
	a = 0\\
	b = 0
\end{cases}
\]
	
\end{ndef}

\begin{nprop}[Propiedad de simplificación]
	
A es un dominio de integridad $\iff$	$ax=ay$ con $a\neq 0 \Rightarrow x = y$
\end{nprop}
\begin{proof}
	\boxed{\Rightarrow} $a(x-y) = 0$, por ser A dominio de integridad, $x-y = 0 \Rightarrow x=y$ \\
	\boxed{\Leftarrow} $ab = 0$ con $a\neq 0 \Rightarrow b= 0$ pues $a0 = 0; ab = a0; b = 0;$
\end{proof}

\begin{ndef}[Divisor de 0]
	$a\in A$ es divisor de 0 si $\exists b\neq 0 : ab = 0$
\end{ndef}

\begin{nprop}
	Si A es un dominio de integridad $\Rightarrow$ el 0 es el único divisor de 0.
	
	Equivalentemente: A es dominio de integridad $\iff$ no tiene divisores de cero no nulos.
\begin{nlist}
	\item $A\leq B$ y $B$ es D.I. $\Rightarrow$ A es D.I.
	\item Todo cuerpo es D.I.
	\item Si $u \in U(A) \Rightarrow$ u no es divisor de 0 (Supongamos $u*b = 0 \Rightarrow u*u^{-1}*b = u^{-1}*0 \Rightarrow b=0$)
\end{nlist}
\end{nprop}

\begin{nprop}
	Si $|A|< \infty$, A es dominio de integridad $\iff$ A es un cuerpo
\end{nprop}
\begin{proof}
	\boxed{\Leftarrow} Trivial\\
	\boxed{\Rightarrow} $0 \neq a \in A$.
Tomo $\{1,a,a^2,\cdots,a^n\} = \{a^n: n\in N\} \subseteq A$
Como tiene cardinalidad finita: $\exists k \in N : a^n = a^{n+k}$.\\
Pero, por ello: $a^n = a^n a^k; a^n * 1 = a^n * a^k$, luego $a^n$ no es 0 porque A es Dominio de integridad y por ser D.I entonces:
\[
1=a^k \begin{cases}
	k = 1 \Rightarrow a = 1\\
	k > 1 \Rightarrow a^{k-1}*a = 1
	
\end{cases}\]
Con lo que $\exists$ inverso de a = $a^{k-1}$ y como a es un elemento cualquiera, todo elemento tiene inverso, luego es un cuerpo.
\end{proof}


\begin{nprop}
	Todo D.I. es un subanillo de un cuerpo.
\end{nprop}


Primero, presentaremos otros conceptos:

\begin{ndef}[Cuerpo de fracciones de un D.I.]
Sea A un dominio de integridad con $|A| \geq 2$. Consideramos $A$x$A-\{0\} = \{(a,b), a,b \in A\hspace{0.3cm} | b\neq 0\}$

\begin{ndef} Decimos que (a,b) es equivalente a (c,d):
	$(a,b)\sim(c,d) \iff ad = bc$
	
Esta relación es reflexiva, simétrica y transitiva.
\end{ndef}
	
Ahora, considero $a,b\in A \hspace{0.2cm}$. Llamo $\frac{a}{b} = \{(c,d)\hspace{0.2cm} c,d \in A : (c,d) \sim (a,b)\} \subseteq A$x$A-\{0\}$

Y llamo a $\frac{a}{b}$ la fracción a entre b.

\begin{ncor}
	\[\frac{a}{b}= \frac{u}{v} \iff av = bu \iff (a,b) \sim (u,v)\]
\end{ncor}
\begin{proof}
	
	\boxed{\Rightarrow} $(a,b) \in \frac{a}{b} = \frac{u}{v} \Rightarrow (a,b) \sim (u,v) \Rightarrow (av = ub)$
	
	\boxed{\Leftarrow} $(a,b) \sim (u,v)$ Por la transitividad: $\frac{a}{b} \subseteq \frac{u}{v}$ y $\frac{u}{v} \subseteq \frac{a}{b} \Rightarrow \frac{a}{b}= \frac{u}{v}$
\end{proof}
\end{ndef}

Ahora, llamamos $Q(A) = \{\frac{a}{b} | a,b \in A: \hspace{0.25cm} b\neq 0\}$ que es un conjunto de conjuntos, pues ya habíamos definido la fracción $\frac{a}{b}$ como un conjunto.

Sobre él, definimos unas operaciones que nos permitirán ver que es un cuerpo:
\begin{nlist}
	\item Suma: \[\frac{a}{b}+ \frac{c}{d} = \frac{ad + cb}{bd}\]

Ahora, como la fracción $\frac{a}{b}$ es un conjunto, hay que probar que el resultado es único, es decir:
\[
\frac{a}{b}= \frac{a'}{b'} \hspace{0.25cm}y\hspace{0.25cm} \frac{c}{d}= \frac{c'}{d'}  \Rightarrow ab' = a'b \hspace{0.25cm} y \hspace{0.25cm} cd' = c'd
\]
Hay que probar que se cumple:
\[
\frac{ad+cb}{bd} = \frac{a'd'+c'b'}{b'd'}
\]
Equivalementente, tenemos que probar que se cumple:
\[
b'd'(ad+cb) = bd(a'd'+c'b')
\]

Desarrollamos en la izquierda:
\[
b'd'(ad+cb) = b'd'ad + b'd'cb =^{(1)} a'bd'd + b'bc'd
\]
Donde en (1) hemos usado la equivalencia que habíamos dado de $ab' = a'b$ y $cd' = c'd$. Ahora, desarollamos el producto de la derecha y veremos que es igual al resultado obtenido

\[
bd(a'd'+c'b') = bda'd' + bdc'b' = a'bdd' + bb'c'd
\]

Probando la unicidad. 
	
    \item Producto: \[\frac{a}{b} \frac{c}{d} = \frac{ac}{bd}\]

La unicidad del producto se hace desarrollando de la misma manera.
\end{nlist}



Para finalizar, se puede probar que es un cuerpo probando las propiedades de anillo conmutativo y que existe inverso para todo $\frac{a}{b}$.

\begin{nprop}[Fracciones de denominador 1]
  Existe un homomorfismo
  \begin{align}
    i :& A \longrightarrow \mathbb{Q}(A)\notag\\
     \quad  &a \longmapsto \frac{a}{1} = i(a)\notag
  \end{align}

  Que cumple que $i(a+b) = i(a)+i(b)$ y que $i(ab)=i(a)i(b)$, y además es un monomorfismo.
  Así, $A \stackrel{i}{\cong} Img(i) = \{\frac{a}{1} : a\in A\}$ es un isomorfismo y $A\le\mathbb{Q}(A) \text{ con } a=\frac{a}{1}$.
  Con esta identificaión $\frac{a}{b} = \frac{a}{1}\frac{1}{b} = ab^{-1}$
\end{nprop}

\begin{nprop}
  Sea $K$ un cuerpo y $A\le K$, $a,b\in A b\not=0$.
  \begin{align}
    \implies& a\in K \text{ y } b^{-1}\in K \implies ab^{-1}\in K\notag\\
    \implies&\mathbb{Q}(A)\le K\notag
  \end{align}
\end{nprop}

\begin{nota}
 Sea $K$ un cuerpo. Entonces  $\mathbb{Q}(K)$ es el cuerpo más pequeño que contiene a $K$.
\end{nota}

\begin{nota}
  $A \subseteq \mathbb{Q}(A), A=\text{D.I.} \implies \rac(\rac(A)) = \rac(A)$
\end{nota}


\begin{nprop}
  Sea K un cuerpo, $A\le K$. Si $\forall \alpha\in K\quad \exists a\in A\;\; a \not= 0 : a\alpha\in A \implies \rac(A)=K$
\end{nprop}

\begin{proof}
  $\alpha\in K,\; \exists a\in A \ne 0 : a\alpha=b\in A \implies \alpha = ba^{-1} = \frac{b}{a}\in \rac(A)$
\end{proof}

\begin{ejemplo}
  $\ent[i] = \{a+bi : a,b\in \ent\} \le \rac[i] = \{a+bi : a,b\in \rac\} \implies \rac(\ent[i]) = \rac[i]$
  \[
    \alpha\in\rac[i]\implies\alpha=\frac{m}{n}+\frac{m'}{n'}i\; \implies \ent[i] \ni nn'\alpha = n'm+nm'i \in \ent[i]\notag
  \]
\end{ejemplo}

\begin{nprop}
	Si $A$ es un D.I. $\implies$ $A[x]$ es un D.I. 
\end{nprop}

\begin{ndef}[Grado de un polinomio]

Si $f=\sum a_i x^i \neq 0$ $\implies gr(f)=n\in \mathbb{N}$ si $a_n\neq 0$ y $a_m=0 \hspace{0.3cm} \forall m>n$ 

El coeficiente $a_n$ se denomina coeficiente líder.

\begin{itemize}
	\item Si A es D.I, $f,g \in A[x]\implies gr(fg) = gr(f)+gr(g)$
\end{itemize}
	
\end{ndef}

\begin{ndef}[Divisibilidad en D.I.]
	
Sea A un D.I. Sean $a,b\in A$. Decimos entonces que a divide a b (a es un divisor de b, b es un múltiplo de a):
\begin{align}
	\iff& \exists c \in A : b = ac 
\\
	\iff& La \hspace{0.2cm} ecuacion \hspace{0.2cm} ax=b \hspace{0.2cm} tiene\hspace{0.2cm}solucion
\\
\iff& \frac{b}{a} \in A
\end{align}

\begin{proof}
	$\boxed{\implies}$ Si a divide a b $\implies$ $\exists c: b=ac \implies \frac{b}{a} = \frac{ac}{a} = \frac{c}{1} = c \in A$\\
	$\boxed{\Longleftarrow}$ si $\frac{b}{a}\in A \implies \frac{b}{c} = \frac{c}{1} \implies b = ac$
\end{proof}
\end{ndef}
\textbf{Notación:} Si a divide a b, escribiremos $a/b$

\begin{nlist}
	\item Los divisores de 1 son las unidades del anillo, los elementos del grupo $U(A)$
	\item Las unidades son divisores de todos los elementos del anillo.
	\item Dado $a\in A$, los elementos $ua$ con $u\in U(A)$ se llaman \emph{asociados de a}.
	\item Si $u\in U(A)$, $\forall a \in A$, $ua/a$
	
\end{nlist}

\begin{ndef}
	Los divisores triviales de un número son las unidades y sus asociados.
\end{ndef}

\begin{nprop}
	Sean $a,b\neq 0$. Son equivalentes:
	\begin{nlist}
	\item a es asociado de b
	\item b es asociado de a
	\item $a/b \wedge b/a$ los asociados son los elementos que se dividen mutuamente
	
\end{nlist}
\end{nprop}

\begin{ndef}[Irreducible]
	
	Sea $a\in A, a\neq 0, a \notin U(A)$ es irreducible si sus únicos divisores son los triviales

\begin{align}
	\iff& si\hspace{0.2cm} b/a \implies b\in U(A) \vee b \sim a
\\
	\iff& si \hspace{0.2cm} a=bc \implies b\in U(A) \vee c \in U(A)
\\
	\iff& si \hspace{0.2cm} a=bc \implies a\sim b \vee c \sim a
\\
	\iff& si \hspace{0.2cm}a=bc \wedge b \notin U(A) \implies c \in U(A)
\end{align}
	
Propiedades elementales:
\begin{nlist}
	\item Reflexión: $a/a$
	\item Transitividad: $a/b \wedge b/c \implies a/c$
	\item Si $a/b \vee a/c \implies a/bx +cy\hspace{0.2cm} \forall x,y \in A$ 
	\item Si $a/b \implies \hspace{0.2cm} \forall c \hspace{0.2cm} a/bc$
	\item Si $c\neq 0$ entonces $a/b \iff ac/bc$
\end{nlist}
\end{ndef}

\begin{ndef}[Dominios euclídeos]

	Un dominio euclídeo es un dominio de integridad, A, tal que haya definida una función $\varphi: A- \{0\} \to \mathbb{N} $ verificando:

\begin{nlist}
	\item $\varphi(ab) \ge \varphi(a)$
	\item $\forall a,b \in A,\ b \neq 0 \quad \exists q,r \in A : a = bq + r $ con $r=0 \vee \varphi(r) < \varphi(b)$
	\item $\forall a,b \in A,\ b \neq 0 \quad \exists q \in A : a - bq = 0 \vee  \varphi(a-bq) < \varphi(b)$
\end{nlist}	
\end{ndef}

\begin{nota}
	Si A es dominio euclídeo, entonces: 
$b/a \iff$ un resto de dividir a entre b es cero $\iff$ cualquier resto de dividir a entre b es 0 
\end{nota}
\begin{proof}
	$\boxed{\implies}$ Por definicion de b/a, $\implies \exists c \in A$ tal que $a=bc$ y por ser A un dominio euclídeo, $\implies \exists q,r \in 		A : 	a = bq + r $ con $r=0 \vee \varphi(r) < \varphi(b)$. La solución 	es evidentemente correcta para $r = 0$, veamos que sucede para $r \neq 	0$.
\\
Supongamos $r \neq 0$, entonces $\varphi(r) < \varphi(b)$. 
$$r = a - bq = bc - bq = b(c-q) \quad \quad c-q \neq 0$$
$$\varphi(r) = \varphi(b(c-q)) \ge \varphi(b) \implies \textup{CONTRADICCIÓN}$$  
\end{proof}

\begin{nth}[Teorema de Euclídes]

	$\forall a,b \in \ent, b \neq 0, \exists q,r \in \ent$ tales que $a = bq + r $ con $0 \le r < b$
\end{nth}

\begin{ncor}
$\ent$ es un dominio de euclídes con $\varphi = \abs{.}: \ent \to \nat$
\[
\varphi(a) \begin{cases}
	a  \quad \textup{si } a \ge 0\\
	-a \quad \textup{si } a < 0
	
\end{cases}\]
\end{ncor}

\end{document}
