%%%%%%%%%%%%%%%%%%%%%%%%%%%%%%%%%%%%%%%%%%%%%%%%%%%%%%%%%%%%%%%%%%%%%%%%%%%%%%%%%
%%                                                                             %%
%% Este fichero contiene ejercicios tipo con sus respectivas soluciones.       %%
%% Autor: Ignacio Aguilera Martos                                              %%
%% https://github.com/nacheteam                                                %%
%%                                                                             %%
%%%%%%%%%%%%%%%%%%%%%%%%%%%%%%%%%%%%%%%%%%%%%%%%%%%%%%%%%%%%%%%%%%%%%%%%%%%%%%%%%

%%%%%%%%%%%%%%%%%%%%%%%%%%%%% Desintegración radioactiva %%%%%%%%%%%%%%%%%%%%%%%%%%%%%

\section{Desintegración radioactiva}
\begin{ejer}
	Un reactor transforma plutonio 239 en uranio 238 que es relativamente estable para
	uso industrial. Después de 15 años se determina que 0.0043 por ciento de la cantidad
	inicial A 0 de plutonio se ha desintegrado. Determina la semivida (tiempo necesario
	para que la cantidad inicial de los átomos se reduzca a la mitad) de este isótopo si
	la rapidez de desintegración es proporcional a la cantidad restante.
\end{ejer}
\begin{sol}
	Se toma como ecuación diferencial para la desintegración radioactiva la $m'(t)=-\lambda \cdot m(t)$ siendo $m(t)$ la masa en cada instante t.  
	Sabemos que después de 15 años hay $0.9957 \cdot A_0$ de masa siendo $A_0$ la masa que había inicialmente.  
	Integrando la ecuación diferencial dada obtenemos que $m(t) = c\cdot e^{-\lambda \cdot t}$. Con esto procedemos a obtener la constante $\lambda$.  
	$0.9957\cdot A_0 = A_0\cdot e^{-\lambda \cdot 15}$ con lo que $0.9957 = e^{-\lambda \cdot 15}$ de donde sacamos $ln(0.9957) = -15\cdot \lambda$ y por lo tanto obtenemos como constante $\lambda = \frac{ln(0.9957)}{-15}$.  
	Para calcular el tiempo de semivida tenemos ahora que ver en qué instante t obtenemos la mitad de la cantidad inicial con la constante que hemos despejado.  
	$\frac{1}{2}\cdot A_0 = A_0\cdot e^{\frac{ln(0.9957)}{15}\cdot t} \rightarrow \frac{ln(\frac{1}{2})}{\frac{ln(0.9957)}{15}}=t \rightarrow t = \frac{15\cdot ln(\frac{1}{2})}{ln(0.9957)} = 2412.753$ años.  
	La solución es que el tiempo de semivida es de 2412.753 años.
\end{sol}

%%%%%%%%%%%%%%%%%%%%%%%%%%%%% Poblaciones con ley de Malthus %%%%%%%%%%%%%%%%%%%%%%%%%%%%%

\section{Poblaciones}
\begin{ejer}
	La población de Malthusilandia (país cuyo crecimiento sigue la ley de Malthus) era de 20 millones en 1980 y se había duplicado en 1990. ¿Qué población tendrá en el año 2000?
\end{ejer}
\begin{sol}
	Para este tipo de problemas usamos la ecuación diferencial $m'(t) = \lambda \cdot m(t)$ de donde obtenemos integrando $m(t) = c\cdot e^{\lambda \cdot t}$.  
	Sabemos que en el instante t=0 la población es de 20 millones, por lo tanto $20M = c\cdot e^{\lambda \cdot 0} = c$. Por lo tanto c=20M.  
	Sabemos que la población 10 años después es de 40 millones, por lo tanto $m(10) = 40M \Rightarrow 40M = 20M \cdot e^{\lambda \cdot 10} \Rightarrow 2 = e^{\lambda \cdot 10} \Rightarrow \lambda = \frac{ln(2)}{10}$.  
	Si queremos saber que población habrá en el 2000, es decir, en t = 20 sólo tenemos que sustituir en la fórmula.  
	$m(20) = 20M\cdot e^{\frac{ln(2)}{10}\cdot 20} = 80M$.  
	La población en el año 2000 será de 80 millones.
\end{sol}

%%%%%%%%%%%%%%%%%%%%%%%%%%%%% Ecuaciones de Bernoulli %%%%%%%%%%%%%%%%%%%%%%%%%%%%%

\section{Ecuaciones de Bernoulli}
\begin{ejer}
	Resuelve la siguiente ecuación de Bernoulli: $(t^2\cdot x^2-1)\cdot x' + 2\cdot t\cdot x^3=0$ haciendo $x=z^\alpha$
\end{ejer}
\begin{sol}
	Las ecuaciones de bernoulli son de la forma $x' = a(t)\cdot x^q + b(t)\cdot x$ y debemos aplicar el siguiente cambio de variable para resolverlas:  
	$\begin{cases}
	s=t \\
	y=x^{1-q}
	\end{cases}$  
	Ponemos la ecuación dada en forma normal:  
	$x' = \frac{-2\cdot t\cdot x^3}{t^2\cdot x^2-1} = \frac{-2\cdot t\cdot z^{3\cdot \alpha}}{t^2\cdot z^{2\cdot \alpha}-1} = -2+z^{3\cdot \alpha}\cdot (\frac{1}{t^2}\cdot z^{-2\cdot \alpha}-1) = -2\cdot \frac{1}{t}\cdot z^{\alpha} - 2\cdot t\cdot z^{3\cdot \alpha} = \frac{-2}{t}\cdot x -2\cdot t\cdot x^3$  
	Después del cambio sugerido por el enunciado hemos obtenido una ecuación de Bernoulli.  
	Después de este cambio para resolver la ecuación tenemos que hacer el cambio de variable propuesto inicialmente:  
	$\begin{cases}
	s = t \\
	y = x^{-2}
	\end{cases}$  
	De donde obtenemos $y' = -2\cdot \frac{1}{x^3}\cdot x' = \frac{4}{x^2}\cdot (\frac{1}{t}-t\cdot x^2) = \frac{4}{t}\cdot y - 4\cdot t$  
	Con lo que hemos obtenido una ecuación lineal.
\end{sol}

%%%%%%%%%%%%%%%%%%%%%%%%%%%%% Curva ortogonal %%%%%%%%%%%%%%%%%%%%%%%%%%%%%

\section{Curva ortogonal}
\begin{ejer}
	Obtén la familia de curvas ortogonales a la familia de curvas:  
	$y^2 = 2\cdot x^2\cdot (1-c\cdot x)$ con $c\in \mathbb{R}$
\end{ejer}
\begin{sol}
	En primer lugar obtenemos la expresión de c:  
	$y^2 = 2\cdot x^2 \cdot (1-c\cdot x)\Rightarrow \frac{y^2}{2\cdot x^2} = 1-c\cdot x \Rightarrow c\cdot x = 1-\frac{y^2}{2\cdot x^2} \Rightarrow c = \frac{1}{x} - \frac{y^2}{2\cdot x^3}$  
	Obtenemos ahora la expresión de y' y sustituimos la expresión de c obtenida:  
	$2\cdot y\cdot y' = 4\cdot x - c\cdot 6\cdot x^2$  
	$y' = \frac{4\cdot x}{2\cdot y}-\frac{c\cdot 6\cdot x^2}{2\cdot y} = 2\cdot \frac{x}{y}-c\cdot 3\cdot \frac{x^2}{y} = 2\cdot \frac{x}{y}-(\frac{1}{x}-\frac{y^2}{2\cdot x^3})\cdot 3\cdot \frac{x^2}{y} = 2\cdot \frac{x}{y}-3\cdot \frac{x}{y} + \frac{3}{2}\cdot \frac{y}{x} = \frac{3}{2}\cdot \frac{y}{x} - \frac{x}{y}$  
	Para obtener la familia ortogonal tenemos que cambiarle el signo y hacer el inverso:  
	$y' = -\frac{2}{3}\cdot \frac{x}{y} + \frac{y}{x}$  
	Esta ecuación es una de tipo homogénea por lo que hacemos el cambio $u = \frac{y}{x} \Rightarrow y = u\cdot x\Rightarrow y' = u+x\cdot u'$  e igualamos las expresiones de y' obtenidas:  
	$\frac{-2}{3\cdot u} + u = u + x\cdot u'\Rightarrow u' = \frac{-2}{3\cdot u\cdot x} = \frac{-2}{3\cdot u}\cdot \frac{1}{x}$  
	Nos ha salido una ecuación de variables separadas que resolvemos:  
	$\frac{du}{dx} = \frac{-2}{3\cdot u}\cdot \frac{1}{x}\Rightarrow \frac{-3}{2}\cdot \int u\cdot du = \int \frac{1}{x}\cdot dx$  
	$-\frac{3}{2}\cdot \frac{u^2}{2} = ln|x| + c \Rightarrow u = \sqrt{-\frac{4}{3}\cdot(ln|x| + c)}$  
	Deshaciendo el cambio de variable:  
	$y = \sqrt{-\frac{4}{3}\cdot (ln|x| + c)}\cdot x$
\end{sol}

%%%%%%%%%%%%%%%%%%%%%%%%%%%%% Ecuaciones lineales %%%%%%%%%%%%%%%%%%%%%%%%%%%%%

\section{Ecuaciones lineales}

\begin{ejer}
	Resuelve la siguiente ecuación lineal: $x'-t\cdot x = 3\cdot t$
\end{ejer}
\begin{sol}
	Ponemos la ecuación en forma normal $x' = t\cdot x + 3\cdot t$ y realizamos el cambio de variable correspondiente para resolver las ecuaciones lineales.  
	$\begin{cases}
	s = t \\
	y = l(t)\cdot x
	\end{cases}$  
	Derivamos el cambio de variable haciendo la derivada de y con respecto a s.  
	$\frac{dy}{ds} = l'(s)\cdot x(s) + l(s)\cdot x(s)' = l'(s)\cdot x(s) + l(s)\cdot (s\cdot x + 3\cdot s) = x\cdot [l'(s) + l(s)\cdot s] + l(s)\cdot 3\cdot s$  
	Imponemos que $l'(s) + s\cdot l(s) = 0 \Rightarrow l'(s) = -s\cdot l(s)$ y resolvemos esta ecuación diferencial para obtener l(s) como una ecuación de variables separadas.  
	$l(s) = e^{-\frac{1}{2}\cdot t^2}$  
	Con esto ya lo podemos sustituir en la ecuación de y(s) dada inicialmente y resolver el problema.  
	$y(s) = \int_{s_0}^{s}l(u)\cdot 3\cdot u\cdot du = \int_{s_0}^{s}e^{-\frac{1}{2}\cdot u^2}\cdot 3\cdot u\cdot du = -3\cdot \int_{s_0}^{s}e^{-\frac{1}{2}\cdot u^2}\cdot (-u)\cdot du = -3\cdot e^{\frac{1}{2}\cdot u^2} \Big|_{s_0}^{s} + c$  
	Deshacemos el cambio de variable que hicimos al principio:  
	$\begin{cases}
	t=s \\
	x = \frac{y}{l(t)}
	\end{cases}$
	
	Obtenemos la x:  
	$x = \frac{y(t)}{l(t)} = e^{\frac{1}{2}\cdot t^2}\cdot (-3\cdot e^{-\frac{1}{2}\cdot t^2}\Big|_{t_0}^{t} + c)$  
	Con lo que habríamos obtenido la solución del problema.
\end{sol}

%%%%%%%%%%%%%%%%%%%%%%%%%%%%% Ecuaciones exactas %%%%%%%%%%%%%%%%%%%%%%%%%%%%%

\section{Ecuaciones exactas}

\begin{ejer}
	Resuelve la ecuación diferencial $sen(t\cdot x) + t\cdot x\cdot cos(t\cdot x) + t^2\cdot cos(t\cdot x)\cdot x'$  
\end{ejer}
\begin{sol}
	Este tipo de ecuaciones tienen la forma $P(t,x) + Q(t,x)\cdot x' = 0$.  
	Debemos comprobar que se da la condición de exactitud, es decir:  
	$\frac{\partial P(t,x)}{\partial x} = \frac{\partial Q(t,x)}{\partial t}$  
	Si esta condición se cumple y estamos en un dominio estrellado como es nuestro caso, entonces sabemos que existe la función solución U(t,x).  
	$\frac{\partial P(t,x)}{\partial x} = cos(t\cdot x)\cdot t + t\cdot cos(t\cdot x) - t^2\cdot x\cdot sen(t\cdot x)$  
	$\frac{\partial Q(t,x)}{\partial t} = 2\cdot t\cdot cos(t\cdot x) - t^2\cdot x\cdot sen(t\cdot x)$  
	Como podemos comprobar en este caso se cumple la condición de exactitud y estamos en un dominio estrellado por lo que sabemos que $\exists U(t,x)$ tal que $\frac{\partial U(t,x)}{\partial t} = P(t,x)$ y $\frac{\partial U(t,x)}{\partial x} = Q(t,x)$  
	Para obtener la función U(t,x) vamos a integrar la función P(t,x) con respecto a t.  
	$\int P(t,x)\cdot dt = \int sen(t\cdot x) + t\cdot x\cdot cos(t\cdot x)\cdot dt = \int sen(t\cdot x)\cdot dt + x\cdot \int t\cdot cos(t\cdot x)\cdot dt$  
	La primera integral la resolvemos de manera inmediata:  
	$\int sen(t\cdot x)\cdot dt = \frac{-cos(t\cdot x)}{x} + c$  
	La segunda la tenemos que resolver por partes:  
	$$
	\begin{cases}
	u = t \\
	dv = cos(t\cdot x)
	\end{cases}
	\begin{cases}
	du = 1 \\
	v = \frac{sen(t\cdot x)}{x}
	\end{cases}
	$$  
	Resolviendo con la fórmula de integración por partes:  
	$\int t\cdot cos(t\cdot x)\cdot dt = \frac{t\cdot sen(t\cdot x)}{x}-\int \frac{sen(t\cdot x)}{x}\cdot dt$  
	$\int \frac{sen(t\cdot x)}{x}\cdot dt = \frac{-cos(t\cdot x)}{x^2}$  
	$x\cdot \int t\cdot cos(t\cdot x)\cdot dt = t\cdot sen(t\cdot x) + \frac{cos(t\cdot x)}{x}$  
	Con lo que:  
	$\int P(t,x)\cdot dt = -\frac{cos(t\cdot x)}{x} + t\cdot sen(t\cdot x) + \frac{cos(t\cdot x)}{x} + c = t\cdot sen(t\cdot x) + c$  
	De donde obtenemos que $U(t,x) = t\cdot sen(t\cdot x) + c + \phi (x)$  
	Para obtener este factor en función de x que nos queda tenemos que derivar con respecto a x e igualarlo con Q(t,x) para sacarlo.  
	$\frac{\partial U(t,x)}{\partial x} = t^2\cdot cos(t\cdot x) + \phi '(x)$  
	De donde sacamos que $\phi '(x) = 0$ y por lo tanto es una constante que podemos agrupar con la constante de integración.  
	$U(t,x) = t\cdot sen(t\cdot x) + c$
\end{sol}

%%%%%%%%%%%%%%%%%%%%%%%%%%%%% Factores integrantes %%%%%%%%%%%%%%%%%%%%%%%%%%%%%

\section{Factores integrantes}
\begin{ejer}
	Resuelve la ecuación diferencial $(3\cdot x\cdot y^2 - 4\cdot y) + (3\cdot x - 4\cdot x^2\cdot y)\cdot y' = 0$  buscando un factor integrante del tipo $\mu (x,y) = \mu (x^n\cdot y^m)$
\end{ejer}
\begin{sol}
	En este caso tenemos que:  
	$P(x,y) = 3\cdot x\cdot y^2 - 4\cdot y$  
	$Q(x,y) = 3\cdot x - 4\cdot x^2\cdot y$  
	Con estas dos ecuaciones tenemos que no se cumple la condición de exactitud. Si multiplicamos por el factor integrante ambas funciones obtenemos la nueva ecuación diferencial sobre la que obtendremos condiciones para el factor integrante.  
	$\tilde{P}(x,y) = \mu (x^n\cdot y^m)\cdot (3\cdot x\cdot y^2 - 4\cdot y)$  
	$\tilde{Q}(x,y) = \mu (x^n\cdot y^m)\cdot (3\cdot x - 4\cdot x^2\cdot y)$  
	Obtenemos la condición de exactitud para $\tilde{P}(x,y)$ y $\tilde{Q}(x,y)$ para obtener las condiciones necesarias para el factor integrante.  
	$\frac{\partial \tilde{P}(x,y)}{\partial y} = m\cdot y^{m-1}\cdot x^n\cdot \mu '(x^n\cdot y^m)\cdot (3\cdot x\cdot y^2-4\cdot y) + \mu (x^n\cdot y^m)\cdot (6\cdot x\cdot y - 4)$  
	$\frac{\partial \tilde{Q}(x,y)}{\partial x} = n\cdot x^{n-1}\cdot y^m\cdot \mu '(x^n\cdot y^m)\cdot (3\cdot x - 4\cdot x^2\cdot y) + \mu (x^n\cdot y^m)\cdot (3-8\cdot x\cdot y)$  
	Igualamos ambas para obtener condiciones sobre $\mu (x^n\cdot y^m)$  
	$\mu '(x^n\cdot y^m)\cdot (m\cdot y^{m-1}\cdot x^n\cdot (3\cdot x\cdot y^2 - 4\cdot y)-n\cdot x^{n-1}\cdot y^m\cdot (3\cdot x - 4\cdot x^2\cdot y)) = \mu (x^n\cdot y^m)\cdot (7-14\cdot x\cdot y)$   
	$\frac{\mu '(x^n\cdot y^m)}{\mu (x^n\cdot y^m)} = \frac{7-14\cdot x\cdot y}{y^m\cdot x^n\cdot((3\cdot m + 4\cdot n)\cdot x\cdot y - 4\cdot m - 3\cdot n)}$  
	Igualamos lo del paréntesis con el numerador para que sean iguales y los podamos eliminar:  
	$$
	\begin{cases}
	3\cdot m + 4\cdot n = -14 \\
	-4\cdot m - 3\cdot n = 7
	\end{cases}
	\begin{cases}
	12\cdot m + 16\cdot n = -56 \\
	-12\cdot m - 9\cdot n = 21
	\end{cases}
	\begin{cases}
	7\cdot n = -35 \\
	-4\cdot m  +15 = 7
	\end{cases}
	$$
	De donde obtenemos que $n=-5$ y $m=2$. Por lo tanto nos queda:  
	$\frac{\mu '(x^n\cdot y^m)}{\mu (x^n\cdot y^m)} = \frac{1}{x^{-5}\cdot y^2}$  
	Resolvemos como una ecuación de variables separadas llamando $u = x^{-5}\cdot y^2$  
	$\frac{d\mu}{du} = \frac{1}{u}\cdot \mu \Rightarrow \frac{d\mu}{\mu} = \frac{1}{u}\cdot du$  
	Integrando obtenemos que $\mu  = u$ y por tanto el factor integrante obtenido es:  
	$\mu (x,y) = x^{-5}\cdot y^2$
\end{sol}