%%%%%%%%%%%%%%%%%%%%%%%%%%%%%%%%%%%%%%%%%%%%%%%%%%%%%%%%%%%%%%%%%%%%%%%%%%%%%%%%%
%%                                                                             %%
%% Este fichero contiene ejercicios tipo con sus respectivas soluciones.       %%
%% Autor: Ignacio Aguilera Martos                                              %%
%% https://github.com/nacheteam                                                %%
%%                                                                             %%
%%%%%%%%%%%%%%%%%%%%%%%%%%%%%%%%%%%%%%%%%%%%%%%%%%%%%%%%%%%%%%%%%%%%%%%%%%%%%%%%%

%%%%%%%%%%%%%%%%%%%%%%%%%%%%% Desintegración radioactiva %%%%%%%%%%%%%%%%%%%%%%%%%%%%%

\section{Desintegración radioactiva}
\begin{ejer}
	Un reactor transforma plutonio 239 en uranio 238 que es relativamente estable para
	uso industrial. Después de 15 años se determina que 0.0043 por ciento de la cantidad
	inicial A 0 de plutonio se ha desintegrado. Determina la semivida (tiempo necesario
	para que la cantidad inicial de los átomos se reduzca a la mitad) de este isótopo si
	la rapidez de desintegración es proporcional a la cantidad restante.
\end{ejer}
\begin{sol}
	Se toma como ecuación diferencial para la desintegración radioactiva la $m'(t)=-\lambda \cdot m(t)$ siendo $m(t)$ la masa en cada instante t.  
	Sabemos que después de 15 años hay $0.9957 \cdot A_0$ de masa siendo $A_0$ la masa que había inicialmente.  
	Integrando la ecuación diferencial dada obtenemos que $m(t) = c\cdot e^{-\lambda \cdot t}$. Con esto procedemos a obtener la constante $\lambda$.  
	$0.9957\cdot A_0 = A_0\cdot e^{-\lambda \cdot 15}$ con lo que $0.9957 = e^{-\lambda \cdot 15}$ de donde sacamos $ln(0.9957) = -15\cdot \lambda$ y por lo tanto obtenemos como constante $\lambda = \frac{ln(0.9957)}{-15}$.  
	Para calcular el tiempo de semivida tenemos ahora que ver en qué instante t obtenemos la mitad de la cantidad inicial con la constante que hemos despejado.  
	$\frac{1}{2}\cdot A_0 = A_0\cdot e^{\frac{ln(0.9957)}{15}\cdot t} \rightarrow \frac{ln(\frac{1}{2})}{\frac{ln(0.9957)}{15}}=t \rightarrow t = \frac{15\cdot ln(\frac{1}{2})}{ln(0.9957)} = 2412.753$ años.  
	La solución es que el tiempo de semivida es de 2412.753 años.
\end{sol}

%%%%%%%%%%%%%%%%%%%%%%%%%%%%% Poblaciones con ley de Malthus %%%%%%%%%%%%%%%%%%%%%%%%%%%%%

\section{Poblaciones}
\begin{ejer}
	La población de Malthusilandia (país cuyo crecimiento sigue la ley de Malthus) era de 20 millones en 1980 y se había duplicado en 1990. ¿Qué población tendrá en el año 2000?
\end{ejer}
\begin{sol}
	Para este tipo de problemas usamos la ecuación diferencial $m'(t) = \lambda \cdot m(t)$ de donde obtenemos integrando $m(t) = c\cdot e^{\lambda \cdot t}$.  
	Sabemos que en el instante t=0 la población es de 20 millones, por lo tanto $20M = c\cdot e^{\lambda \cdot 0} = c$. Por lo tanto c=20M.  
	Sabemos que la población 10 años después es de 40 millones, por lo tanto $m(10) = 40M \Rightarrow 40M = 20M \cdot e^{\lambda \cdot 10} \Rightarrow 2 = e^{\lambda \cdot 10} \Rightarrow \lambda = \frac{ln(2)}{10}$.  
	Si queremos saber que población habrá en el 2000, es decir, en t = 20 sólo tenemos que sustituir en la fórmula.  
	$m(20) = 20M\cdot e^{\frac{ln(2)}{10}\cdot 20} = 80M$.  
	La población en el año 2000 será de 80 millones.
\end{sol}

%%%%%%%%%%%%%%%%%%%%%%%%%%%%% Ecuaciones de Bernoulli %%%%%%%%%%%%%%%%%%%%%%%%%%%%%

\section{Ecuaciones de Bernoulli}
\begin{ejer}
	Resuelve la siguiente ecuación de Bernoulli: $(t^2\cdot x^2-1)\cdot x' + 2\cdot t\cdot x^3=0$ haciendo $x=z^\alpha$
\end{ejer}
\begin{sol}
	Las ecuaciones de bernoulli son de la forma $x' = a(t)\cdot x^q + b(t)\cdot x$ y debemos aplicar el siguiente cambio de variable para resolverlas:  
	$\begin{cases}
	s=t \\
	y=x^{1-q}
	\end{cases}$  
	Ponemos la ecuación dada en forma normal:  
	$x' = \frac{-2\cdot t\cdot x^3}{t^2\cdot x^2-1} = \frac{-2\cdot t\cdot z^{3\cdot \alpha}}{t^2\cdot z^{2\cdot \alpha}-1} = -2+z^{3\cdot \alpha}\cdot (\frac{1}{t^2}\cdot z^{-2\cdot \alpha}-1) = -2\cdot \frac{1}{t}\cdot z^{\alpha} - 2\cdot t\cdot z^{3\cdot \alpha} = \frac{-2}{t}\cdot x -2\cdot t\cdot x^3$  
	Después del cambio sugerido por el enunciado hemos obtenido una ecuación de Bernoulli.  
	Después de este cambio para resolver la ecuación tenemos que hacer el cambio de variable propuesto inicialmente:  
	$\begin{cases}
	s = t \\
	y = x^{-2}
	\end{cases}$  
	De donde obtenemos $y' = -2\cdot \frac{1}{x^3}\cdot x' = \frac{4}{x^2}\cdot (\frac{1}{t}-t\cdot x^2) = \frac{4}{t}\cdot y - 4\cdot t$  
	Con lo que hemos obtenido una ecuación lineal.
\end{sol}

%%%%%%%%%%%%%%%%%%%%%%%%%%%%% Curva ortogonal %%%%%%%%%%%%%%%%%%%%%%%%%%%%%

\section{Curva ortogonal}
\begin{ejer}
	Obtén la familia de curvas ortogonales a la familia de curvas:  
	$y^2 = 2\cdot x^2\cdot (1-c\cdot x)$ con $c\in \mathbb{R}$
\end{ejer}
\begin{sol}
	En primer lugar obtenemos la expresión de c:  
	$y^2 = 2\cdot x^2 \cdot (1-c\cdot x)\Rightarrow \frac{y^2}{2\cdot x^2} = 1-c\cdot x \Rightarrow c\cdot x = 1-\frac{y^2}{2\cdot x^2} \Rightarrow c = \frac{1}{x} - \frac{y^2}{2\cdot x^3}$  
	Obtenemos ahora la expresión de y' y sustituimos la expresión de c obtenida:  
	$2\cdot y\cdot y' = 4\cdot x - c\cdot 6\cdot x^2$  
	$y' = \frac{4\cdot x}{2\cdot y}-\frac{c\cdot 6\cdot x^2}{2\cdot y} = 2\cdot \frac{x}{y}-c\cdot 3\cdot \frac{x^2}{y} = 2\cdot \frac{x}{y}-(\frac{1}{x}-\frac{y^2}{2\cdot x^3})\cdot 3\cdot \frac{x^2}{y} = 2\cdot \frac{x}{y}-3\cdot \frac{x}{y} + \frac{3}{2}\cdot \frac{y}{x} = \frac{3}{2}\cdot \frac{y}{x} - \frac{x}{y}$  
	Para obtener la familia ortogonal tenemos que cambiarle el signo y hacer el inverso:  
	$y' = -\frac{2}{3}\cdot \frac{x}{y} + \frac{y}{x}$  
	Esta ecuación es una de tipo homogénea por lo que hacemos el cambio $u = \frac{y}{x} \Rightarrow y = u\cdot x\Rightarrow y' = u+x\cdot u'$  e igualamos las expresiones de y' obtenidas:  
	$\frac{-2}{3\cdot u} + u = u + x\cdot u'\Rightarrow u' = \frac{-2}{3\cdot u\cdot x} = \frac{-2}{3\cdot u}\cdot \frac{1}{x}$  
	Nos ha salido una ecuación de variables separadas que resolvemos:  
	$\frac{du}{dx} = \frac{-2}{3\cdot u}\cdot \frac{1}{x}\Rightarrow \frac{-3}{2}\cdot \int u\cdot du = \int \frac{1}{x}\cdot dx$  
	$-\frac{3}{2}\cdot \frac{u^2}{2} = ln|x| + c \Rightarrow u = \sqrt{-\frac{4}{3}\cdot(ln|x| + c)}$  
	Deshaciendo el cambio de variable:  
	$y = \sqrt{-\frac{4}{3}\cdot (ln|x| + c)}\cdot x$
\end{sol}

%%%%%%%%%%%%%%%%%%%%%%%%%%%%% Ecuaciones lineales %%%%%%%%%%%%%%%%%%%%%%%%%%%%%

\section{Ecuaciones lineales}

\begin{ejer}
	Resuelve la siguiente ecuación lineal: $x'-t\cdot x = 3\cdot t$
\end{ejer}
\begin{sol}
	Ponemos la ecuación en forma normal $x' = t\cdot x + 3\cdot t$ y realizamos el cambio de variable correspondiente para resolver las ecuaciones lineales.  
	$\begin{cases}
	s = t \\
	y = l(t)\cdot x
	\end{cases}$  
	Derivamos el cambio de variable haciendo la derivada de y con respecto a s.  
	$\frac{dy}{ds} = l'(s)\cdot x(s) + l(s)\cdot x(s)' = l'(s)\cdot x(s) + l(s)\cdot (s\cdot x + 3\cdot s) = x\cdot [l'(s) + l(s)\cdot s] + l(s)\cdot 3\cdot s$  
	Imponemos que $l'(s) + s\cdot l(s) = 0 \Rightarrow l'(s) = -s\cdot l(s)$ y resolvemos esta ecuación diferencial para obtener l(s) como una ecuación de variables separadas.  
	$l(s) = e^{-\frac{1}{2}\cdot t^2}$  
	Con esto ya lo podemos sustituir en la ecuación de y(s) dada inicialmente y resolver el problema.  
	$y(s) = \int_{s_0}^{s}l(u)\cdot 3\cdot u\cdot du = \int_{s_0}^{s}e^{-\frac{1}{2}\cdot u^2}\cdot 3\cdot u\cdot du = -3\cdot \int_{s_0}^{s}e^{-\frac{1}{2}\cdot u^2}\cdot (-u)\cdot du = -3\cdot e^{\frac{1}{2}\cdot u^2} \Big|_{s_0}^{s} + c$  
	Deshacemos el cambio de variable que hicimos al principio:  
	$\begin{cases}
	t=s \\
	x = \frac{y}{l(t)}
	\end{cases}$
	
	Obtenemos la x:  
	$x = \frac{y(t)}{l(t)} = e^{\frac{1}{2}\cdot t^2}\cdot (-3\cdot e^{-\frac{1}{2}\cdot t^2}\Big|_{t_0}^{t} + c)$  
	Con lo que habríamos obtenido la solución del problema.
\end{sol}

%%%%%%%%%%%%%%%%%%%%%%%%%%%%% Ecuaciones exactas %%%%%%%%%%%%%%%%%%%%%%%%%%%%%

\section{Ecuaciones exactas}

\begin{ejer}
	Resuelve la ecuación diferencial $sen(t\cdot x) + t\cdot x\cdot cos(t\cdot x) + t^2\cdot cos(t\cdot x)\cdot x'$  
\end{ejer}
\begin{sol}
	Este tipo de ecuaciones tienen la forma $P(t,x) + Q(t,x)\cdot x' = 0$.  
	Debemos comprobar que se da la condición de exactitud, es decir:  
	$\frac{\partial P(t,x)}{\partial x} = \frac{\partial Q(t,x)}{\partial t}$  
	Si esta condición se cumple y estamos en un dominio estrellado como es nuestro caso, entonces sabemos que existe la función solución U(t,x).  
	$\frac{\partial P(t,x)}{\partial x} = cos(t\cdot x)\cdot t + t\cdot cos(t\cdot x) - t^2\cdot x\cdot sen(t\cdot x)$  
	$\frac{\partial Q(t,x)}{\partial t} = 2\cdot t\cdot cos(t\cdot x) - t^2\cdot x\cdot sen(t\cdot x)$  
	Como podemos comprobar en este caso se cumple la condición de exactitud y estamos en un dominio estrellado por lo que sabemos que $\exists U(t,x)$ tal que $\frac{\partial U(t,x)}{\partial t} = P(t,x)$ y $\frac{\partial U(t,x)}{\partial x} = Q(t,x)$  
	Para obtener la función U(t,x) vamos a integrar la función P(t,x) con respecto a t.  
	$\int P(t,x)\cdot dt = \int sen(t\cdot x) + t\cdot x\cdot cos(t\cdot x)\cdot dt = \int sen(t\cdot x)\cdot dt + x\cdot \int t\cdot cos(t\cdot x)\cdot dt$  
	La primera integral la resolvemos de manera inmediata:  
	$\int sen(t\cdot x)\cdot dt = \frac{-cos(t\cdot x)}{x} + c$  
	La segunda la tenemos que resolver por partes:  
	$$
	\begin{cases}
	u = t \\
	dv = cos(t\cdot x)
	\end{cases}
	\begin{cases}
	du = 1 \\
	v = \frac{sen(t\cdot x)}{x}
	\end{cases}
	$$  
	Resolviendo con la fórmula de integración por partes:  
	$\int t\cdot cos(t\cdot x)\cdot dt = \frac{t\cdot sen(t\cdot x)}{x}-\int \frac{sen(t\cdot x)}{x}\cdot dt$  
	$\int \frac{sen(t\cdot x)}{x}\cdot dt = \frac{-cos(t\cdot x)}{x^2}$  
	$x\cdot \int t\cdot cos(t\cdot x)\cdot dt = t\cdot sen(t\cdot x) + \frac{cos(t\cdot x)}{x}$  
	Con lo que:  
	$\int P(t,x)\cdot dt = -\frac{cos(t\cdot x)}{x} + t\cdot sen(t\cdot x) + \frac{cos(t\cdot x)}{x} + c = t\cdot sen(t\cdot x) + c$  
	De donde obtenemos que $U(t,x) = t\cdot sen(t\cdot x) + c + \phi (x)$  
	Para obtener este factor en función de x que nos queda tenemos que derivar con respecto a x e igualarlo con Q(t,x) para sacarlo.  
	$\frac{\partial U(t,x)}{\partial x} = t^2\cdot cos(t\cdot x) + \phi '(x)$  
	De donde sacamos que $\phi '(x) = 0$ y por lo tanto es una constante que podemos agrupar con la constante de integración.  
	$U(t,x) = t\cdot sen(t\cdot x) + c$
\end{sol}

%%%%%%%%%%%%%%%%%%%%%%%%%%%%% Factores integrantes %%%%%%%%%%%%%%%%%%%%%%%%%%%%%

\section{Factores integrantes}
\begin{ejer}
	Resuelve la ecuación diferencial $(3\cdot x\cdot y^2 - 4\cdot y) + (3\cdot x - 4\cdot x^2\cdot y)\cdot y' = 0$  buscando un factor integrante del tipo $\mu (x,y) = \mu (x^n\cdot y^m)$
\end{ejer}
\begin{sol}
	En este caso tenemos que:  
	$P(x,y) = 3\cdot x\cdot y^2 - 4\cdot y$  
	$Q(x,y) = 3\cdot x - 4\cdot x^2\cdot y$  
	Con estas dos ecuaciones tenemos que no se cumple la condición de exactitud. Si multiplicamos por el factor integrante ambas funciones obtenemos la nueva ecuación diferencial sobre la que obtendremos condiciones para el factor integrante.  
	$\tilde{P}(x,y) = \mu (x^n\cdot y^m)\cdot (3\cdot x\cdot y^2 - 4\cdot y)$  
	$\tilde{Q}(x,y) = \mu (x^n\cdot y^m)\cdot (3\cdot x - 4\cdot x^2\cdot y)$  
	Obtenemos la condición de exactitud para $\tilde{P}(x,y)$ y $\tilde{Q}(x,y)$ para obtener las condiciones necesarias para el factor integrante.  
	$\frac{\partial \tilde{P}(x,y)}{\partial y} = m\cdot y^{m-1}\cdot x^n\cdot \mu '(x^n\cdot y^m)\cdot (3\cdot x\cdot y^2-4\cdot y) + \mu (x^n\cdot y^m)\cdot (6\cdot x\cdot y - 4)$  
	$\frac{\partial \tilde{Q}(x,y)}{\partial x} = n\cdot x^{n-1}\cdot y^m\cdot \mu '(x^n\cdot y^m)\cdot (3\cdot x - 4\cdot x^2\cdot y) + \mu (x^n\cdot y^m)\cdot (3-8\cdot x\cdot y)$  
	Igualamos ambas para obtener condiciones sobre $\mu (x^n\cdot y^m)$  
	$\mu '(x^n\cdot y^m)\cdot (m\cdot y^{m-1}\cdot x^n\cdot (3\cdot x\cdot y^2 - 4\cdot y)-n\cdot x^{n-1}\cdot y^m\cdot (3\cdot x - 4\cdot x^2\cdot y)) = \mu (x^n\cdot y^m)\cdot (7-14\cdot x\cdot y)$   
	$\frac{\mu '(x^n\cdot y^m)}{\mu (x^n\cdot y^m)} = \frac{7-14\cdot x\cdot y}{y^m\cdot x^n\cdot((3\cdot m + 4\cdot n)\cdot x\cdot y - 4\cdot m - 3\cdot n)}$  
	Igualamos lo del paréntesis con el numerador para que sean iguales y los podamos eliminar:  
	$$
	\begin{cases}
	3\cdot m + 4\cdot n = -14 \\
	-4\cdot m - 3\cdot n = 7
	\end{cases}
	\begin{cases}
	12\cdot m + 16\cdot n = -56 \\
	-12\cdot m - 9\cdot n = 21
	\end{cases}
	\begin{cases}
	7\cdot n = -35 \\
	-4\cdot m  +15 = 7
	\end{cases}
	$$
	De donde obtenemos que $n=-5$ y $m=2$. Por lo tanto nos queda:  
	$\frac{\mu '(x^n\cdot y^m)}{\mu (x^n\cdot y^m)} = \frac{1}{x^{-5}\cdot y^2}$  
	Resolvemos como una ecuación de variables separadas llamando $u = x^{-5}\cdot y^2$  
	$\frac{d\mu}{du} = \frac{1}{u}\cdot \mu \Rightarrow \frac{d\mu}{\mu} = \frac{1}{u}\cdot du$  
	Integrando obtenemos que $\mu  = u$ y por tanto el factor integrante obtenido es:  
	$\mu (x,y) = x^{-5}\cdot y^2$
\end{sol}

%%%%%%%%%%%%%%%%%%%%%%%%%%%%% Ecuaciones homogéneas %%%%%%%%%%%%%%%%%%%%%%%%%%%%%

\section{Ecuaciones homogéneas}
\begin{ejer}
	Resuelve la siguiente ecuación diferencial: $x+(x-t)\cdot x' = 0$
\end{ejer}
\begin{sol}
	Ponemos la ecuación diferencial en forma normal.  
	$x' = \frac{-x}{x-t} = \frac{-\frac{x}{t}}{\frac{x}{t}-1}$  
	De esta forma ya la tenemos como una ecuación diferencial homogénea, es decir, en función de $\frac{x}{t}$.  
	Hacemos el cambio de variable $u = \frac{x}{t}$ de donde obtenemos que $x = u\cdot t \Rightarrow x' = u + t\cdot u'$.  
	Igualando las dos expresiones que tenemos de x':  
	$\frac{-u}{u-1} = u + t\cdot u' \Rightarrow u' = (\frac{-u}{u-1}-u)\cdot \frac{1}{t}$.  
	De donde hemos obtenido una ecuación resoluble por variables separadas.
	$u' = (\frac{-u-u\cdot (u-1)}{u-1})\cdot \frac{1}{t}$  
	$\int \frac{u-1}{-u-u\cdot (u-1)}\cdot du = \int \frac{1}{t}\cdot dt$
	Obtenemos la descomposición del denominador para poder integrar como una racional.  
	$u^2 -2\cdot u = u\cdot (u-2)$  
	$\frac{u-1}{u\cdot (u-2)} = \frac{A}{u} + \frac{B}{u-2}$  
	$u-1 = A\cdot (u-2)  + B\cdot u \Rightarrow A=\frac{1}{2} \ y \ B=\frac{1}{2}$  
	Terminamos por lo tanto la integral:  
	$\frac{1}{2}\cdot \int \frac{1}{u}\cdot du + \frac{1}{2}\cdot \int \frac{1}{u-2}\cdot du = \int \frac{1}{t}\cdot dt$  
	$\frac{1}{2}\cdot ln|u| + \frac{1}{2}\cdot ln|u-2| = ln|t| + c$  
	$ln|u\cdot (u-2)|^{\frac{1}{2}} = ln|t| + c$  
	$\sqrt{u\cdot (u-2)} = t + e^c$  
	$u\cdot (u-2) = t^2 + 2\cdot t\cdot e^c + e^{2\cdot c}$  
	Deshaciendo el cambio del principio:  
	$\frac{x}{t}\cdot (\frac{x}{t}-2) = t^2 + 2\cdot t\cdot e^c + e^{2\cdot c}$  
	$\frac{x^2}{t^2}-2\cdot \frac{x}{t} = t^2 + 2\cdot t\cdot e^c + e^{2\cdot c} \Rightarrow x^2 - 2\cdot t\cdot x = t^4 + 2\cdot t^3\cdot e^c + t^2\cdot e^{2\cdot c}$
\end{sol}

%%%%%%%%%%%%%%%%%%%%%%%%%%%%% Ecuaciones reducibles a homogéneas %%%%%%%%%%%%%%%%%%%%%%%%%%%%%

\section{Ecuaciones reducibles a homogéneas}

\begin{ejer}
	Resolver las siguientes ecuaciones:  
	- $y' = \frac{x-y+1}{x+y-3}$   
	- $y' = \frac{2\cdot x-y+2}{4\cdot x - 2\cdot y + 3}$  
\end{ejer}
\begin{sol}
	- En el primero de los casos comenzamos obteniendo el punto de corte de las rectas:  
	$\begin{cases}
	x-y+1 = 0 \\
	x+y-3 = 0
	\end{cases}$
	
	De donde obtenemos sumando las dos ecuaciones:  
	
	$$
	\begin{cases}
	2\cdot x-2 = 0 \\
	x+y-3 = 0
	\end{cases}
	$$  
	
	De donde obtenemos que los puntos de corte son $x=1$ e $y=2$.  
	Hacemos el cambio de variable siguiente:  
	
	$$
	\begin{cases}
	x=u+1\\
	y = v+2
	\end{cases}
	$$ 
	
	Sustituyendo este cambio de variable en la ecuación original:  
	
	$\frac{dv}{du} =  \frac{u+1-v-2+1}{u+1+v+2-3} = \frac{u-v}{u+v}$  
	Con lo que hemos convertido la ecuación inicial en una homogénea.  
	
	- Intentamos hallar el punto de corte de las dos rectas dadas pero vemos que son paralelas. Si nos fijamos vemos que $2\cdot x -y$ es factor común de ambas rectas y con ello vamos a hacer el cambio de variable.  
	$v = 2\cdot x - y \Rightarrow dv = 2\cdot dx - dy \Rightarrow dy = 2\cdot dx - dv$  
	$(2\cdot dx - dv)\cdot (2\cdot v + 3) = v+2\cdot dx$  
	$4\cdot v\cdot dx - 2\cdot v\cdot dv + 6\cdot dx - \cdot dv = v\cdot dx + 2\cdot dx$  
	$3\cdot v\cdot dx - 2\cdot v \cdot dv + 4\cdot dx -3\cdot dv = 0 \Rightarrow (3\cdot v + 4)\cdot dx = (2\cdot v + 3)\cdot dv$  
	$dx = \frac{2\cdot v + 3}{3\cdot v + 4}\cdot dv$  
\end{sol}

%%%%%%%%%%%%%%%%%%%%%%%%%%%%% Ecuaciones de Riccati %%%%%%%%%%%%%%%%%%%%%%%%%%%%%

\section{Ecuaciones de Riccati}

\begin{ejer}
	Resolver la ecuación diferencial $y' = y^2 - x\cdot y + 1$ con $y_p(x) =x$
\end{ejer}
\begin{sol}
	Hacemos el cambio de variable $w = \frac{1}{y-x}$  
	$w' = \frac{-y' + 1}{(y-x)^2} = \frac{x\cdot y - y^2}{(y-x)^2} = \frac{y\cdot (x-y)}{(y-x)^2} = \frac{y}{x-y} = -y\cdot w$  
	Integrando la ecuación que hemos obtenido para w obtenemos que $w = e^{-y\cdot x}$  
	Deshaciendo el cambio de variable de antes
	$w\cdot (y-x) = 1 \Rightarrow w\cdot y - w\cdot x = 1 \Rightarrow y = \frac{1+x\cdot w}{w}$  
	Con lo que obtenemos la solución $y = \frac{1+e^{-y\cdot x}\cdot x}{e^{-y\cdot x}} = e^{y\cdot x} + x$  
	La solución general de la ecuación de riccati se obtiene sumando la particular dada por el enunciado más la que hemos obtenido en la resolución.  
	$y(t) = e^{y\cdot x} + x + x = e^{y\cdot x} + 2x$
\end{sol}

%%%%%%%%%%%%%%%%%%%%%%%%%%%%% Variables separadas %%%%%%%%%%%%%%%%%%%%%%%%%%%%%

\section{Variables separadas}

\begin{ejer}
	Resuelve la siguiente ecuación diferencial:  
	$x' = e^t - \frac{2\cdot}{t^2-1}$ 
\end{ejer}
\begin{sol}
	$x' = f(t)\cdot g(x) \Rightarrow \frac{dx}{dt} = f(t)\cdot g(x) \Rightarrow \frac{dx}{g(x)} = f(t)\cdot dt$  
	Al hacer esto sólo tenemos que integrar en ambos lados para obtener la ecuación diferencial que queremos.  
	$\int dx = \int e^t - \frac{2\cdot t}{t^2-1}\cdot dt = \int e^t - \int \frac{2\cdot t}{t^2-1} = e^t - ln|t^2-1| + c$  
	Por lo tanto hemos obtenido nuestra solución:  
	$x(t) = e^t - ln|t^2-1| + c$
\end{sol}

%%%%%%%%%%%%%%%%%%%%%%%%%%%%% Rebajamiento de orden %%%%%%%%%%%%%%%%%%%%%%%%%%%%%

\section{Rebajamiento de orden}

\begin{ejer}
	Resuelve la siguiente ecuación diferencial previo rebajamiento de orden: $t^2\cdot x'' + t\cdot (t-4)\cdot x' + 2\cdot (3-t)\cdot x = 2\cdot t\cdot e^t$ con $x_1(t) = t^2$
\end{ejer}
\begin{sol}
	Hacemos el cambio de variable $x= u\cdot t^2$  
	$x' = u'\cdot t^2 + u\cdot 2\cdot t$  
	$x'' = 2\cdot u + 4\cdot u'\cdot t + u''\cdot t^2$  
	Sustituimos en la ecuación diferencial del principio obteniendo:
	$u'+u'' = 2\cdot e^t$  
	Hacemos el cmabio de variable $u' = v$ que nos lleva a la ecuación $v+v' = 2\cdot e^t$  
	$v' = 2\cdot e^t  - v$ que es una ecuación lineal.  
	Como solución nos queda $x = t^2\cdot e^t - t^2\cdot e^{-t}\cdot c + t^2\cdot k$  
	De esta forma obtenemos las soluciones de la ecuación que nos dan como SFS $-t^2\cdot e^{-t}$, $t^2$ y como solución particular de la completa $t^2\cdot e^t$.
\end{sol}

%%%%%%%%%%%%%%%%%%%%%%%%%%%%% Ecuaciones de segundo orden con coeficientes indeterminados %%%%%%%%%%%%%%%%%%%%%%%%%%%%%

\section{Ecuaciones de segundo orden con coeficientes indeterminados}

\begin{ejer}
	A continuación se describe el modelo de resolución de estos ejercicios.
\end{ejer}
\begin{sol}
	Cuando tenemos una ecuación diferencial de orden superior completa tenemos que es de la forma $a(x)\cdot y'' + b(x)\cdot y' + c(x) = r(x)$  
	Para este método necesitamos que $r(x)$ sea un polinomio, seno, coseno, exponencial o una mezcla de estas. Podemos distinguir dos casos:  
	- Caso 1: No hay relación entre las soluciones de la homogénea y r(x)  
	En este caso vamos a proponer como son el tipo de soluciones particulares de la homogénea que debemos encontrar. Para ello vamos a usar la ecuación diferencial de orden superior homogénea $y'' - 5\cdot y' + 6\cdot y = 0$ que tiene como soluciones $y_h = c_1\cdot e^{3\cdot x} + c_2\cdot e^{2\cdot x}$  
	
	1. $y''-5\cdot y'+6\cdot y = x^3 + x$  
	En este caso se propone como solución particular de la homogénea un polinomio general del mismo grado que $r(x)$, es decir, $y_p = A+B\cdot x + C\cdot x^2 + D\cdot x^3$  
	2. $y''-5\cdot y' + 6\cdot y = 20\cdot sen(8\cdot x)$ en este caso debemos proponer una solución del tipo $y_p = A\cdot sen(8\cdot x) + B\cdot cos(8\cdot x)$. Esto se tiene en consideración siempre que aparezca una función seno o coseno.
	3. $y''-5\cdot y' + 6\cdot y = 12\cdot e^{5\cdot x}$. Para el caso de las exponenciales debemos proponer como solución $y_p = A\cdot e^{5\cdot x}$   
	
	Para resolver los coeficientes que nos quedan debemos aplicar la ecuación diferencial a la solución particular obtenida e igualar los coeficientes.
	1. $y_p = A+B\cdot x + C\cdot x^2 + D\cdot x^3$  
	$y_p' = B+2\cdot C\cdot x + 3\cdot D\cdot x^2$  
	$y_p'' = 2\cdot C + 6\cdot D\cdot x$  
	Sustituyendo en la ecuación diferencial original:  
	$2\cdot C + 6\cdot D\cdot x - 5\cdot D - 10\cdot C\cdot x - 15\cdot D\cdot x^2 + 6\cdot A + 6\cdot B\cdot x + 6\cdot X\cdot x^2 + 6\cdot D\cdot x^3 =\\= x^3\cdot 6\cdot D + x^2\cdot (6\cdot C - 15\cdot D) + x\cdot (6\cdot D - 10\cdot C + 6\cdot B) + 2\cdot C - 5\cdot B + 6\cdot A$  
	Igualamos los coeficientes:  
	
	$$
	\begin{cases}
	1 = 6D \\
	0 = 6\cdot C-15\cdot D \\
	1 = 6\cdot D - 10\cdot C + 6\cdot B \\
	0 = 2\cdot C - 5\cdot B + 6\cdot A
	\end{cases}
	\Rightarrow
	\begin{cases}
	D = \frac{1}{6} \\
	C = \frac{5}{12} \\
	B = \frac{19}{36} \\
	A = \frac{69}{216} \\
	\end{cases}
	$$
	Con lo que obtenemos la solución particular $y_p = \frac{69}{216} + \frac{19}{36}\cdot x + \frac{5}{12}\cdot x^2 + \frac{1}{6}\cdot x^3$  
	
	2. $y_p = A\cdot sen(8\cdot x) + B\cdot cos(8\cdot x)$  
	$y_p' = 8\cdot A\cdot cos(8\cdot x) - 8\cdot B\cdot sen(8\cdot x)$  
	$y_p'' = -16\cdot A\cdot sen(8\cdot x) - 16\cdot B\cdot cos(8\cdot x)$  
	Sustituyendo de nuevo en la ecuación diferencial:  
	$sen(8\cdot x)\cdot (-16\cdot A + 40\cdot B + 6\cdot A) + cos(8\cdot x)\cdot (-16\cdot B + 40\cdot A + 6\cdot B)$
	$$
	\begin{cases}
	40\cdot B - 10\cdot A = 20 \\
	40\cdot A - 10\cdot B = 0
	\end{cases}
	\Rightarrow
	\begin{cases}
	40\cdot B - 10\cdot A = 20 \\
	160\cdot A - 40\cdot B = 0
	\end{cases}
	\Rightarrow
	\begin{cases}
	A = \frac{2}{15} \\
	B = -\frac{8}{15}
	\end{cases}
	$$
	Con lo que obtenemos la solución particular $y_p = \frac{2}{15}\cdot sen(8\cdot x) - \frac{8}{15}\cdot cos(8\cdot x)$
	
	Si tenemos una suma de funciones de este tipo la solución particular se da sumando las soluciones particulares correspondientes a cada una de las funciones.  
	En el caso del producto se multiplican las soluciones particulares:  
	- $y'' - 5\cdot y' + 6\cdot y = x^2\cdot e^{6\cdot x}$  
	$y_p = (A+B\cdot x + C\cdot x^2)\cdot e^{6\cdot x}$  
	- $y'' - 5\cdot y' + 6\cdot y = x^3\cdot sen(3\cdot x)$  
	$y_p = (A+B\cdot x + C\cdot x^2 + D\cdot x^3)\cdot sen(3\cdot x) + (E + F\cdot x + G\cdot x^2 + H\cdot x^3)\cdot cos(3\cdot x)$  
	- $y'' - 5\cdot y' + 6\cdot y = sen(5\cdot x)\cdot e^{-2\cdot x}$  
	$y_p = (A\cdot sen(5\cdot x) + B\cdot cos(5\cdot x))\cdot e^{-2\cdot x}$  
	
	- Caso 2: r(x) tiene funciones en común con las soluciones de la homogénea.  
	En este caso debemos exponer la solución que daríamos en el caso 1 y multiplicar por x hasta que no encontremos funciones en común.  
	$y'' + 6\cdot y' + 13\cdot y =e^{-3\cdot x}\cdot cos(2\cdot x)$ tiene como soluciones de la homogénea $y_h = e^{-3\cdot x}\cdot (c_1\cdot cos(2\cdot x) + c_2\cdot sen(2\cdot x))$  
	Proponemos la solución como si estuviéramos en el caso 1:  
	$y_p = (A\cdot sen(2\cdot x) + B\cdot cos(2\cdot x))\cdot e^{-3\cdot x}$  
	Esta solución claramente comparte funciones con las soluciones de la homogénea, multiplicamos por x:  
	$y_p = x\cdot (A\cdot sen(2\cdot x) + B\cdot cos(2\cdot x))\cdot e^{-3\cdot x}$  
	Esta ya no tiene funciones en común y nos vale como solución particular.
\end{sol}

%%%%%%%%%%%%%%%%%%%%%%%%%%%%% Ecuaciones de segundo orden homogéneas %%%%%%%%%%%%%%%%%%%%%%%%%%%%%

\section{Ecuaciones de segundo orden homogéneas}

\begin{ejer}
	A continuación se describe el modelo de resolución.
\end{ejer}
\begin{sol}
	En primer lugar tenemos que resolver la ecuación homogénea que se nos plantee, por ejemplo:  
	$y'' - 5\cdot y' + 6\cdot y = 0$  
	Para ello encontramos las raíces del polinomio asociado:  
	$\lambda ^2 - 5\cdot \lambda + 6 = 0$ que en este caso son $\lambda = 3$ y $\lambda  = 2$  
	Por ello el sistema fundamental de soluciones es $y_h = c_1\cdot e^{3\cdot x} + c_2\cdot e^{2\cdot x}$  
	Este es el caso si tenemos raíces reales diferentes, en caso de tener raíces repetidas, por ejemplo $\lambda = 1$ raíz doble el sistema fundamental de soluciones sería $c_1\cdot e^x + c_2\cdot x\cdot e^x$.  
	Si tenemos raíces complejas $\lambda = \alpha \pm \beta \cdot i$ entonces el sistema fundamental de soluciones sería $c_1\cdot e^{\alpha \cdot x}\cdot sen(\beta \cdot x) + c_2\cdot e^{\alpha \cdot x}\cdot cos(\beta \cdot x)$
\end{sol}

%%%%%%%%%%%%%%%%%%%%%%%%%%%%% Ecuaciones de segundo orden con variación de las constantes %%%%%%%%%%%%%%%%%%%%%%%%%%%%%

\section{Ecuaciones de segundo orden con variación de las constantes}

\begin{ejer}
	A continuación se describe el modelo de resolución.
\end{ejer}
\begin{sol}
	En el método de variación de las constantes vamos a proponer como soluciones para la particular de la homogénea soluciones del tipo $x(t) = c_1(t)\cdot \phi_1 (t) + c_2(t)\cdot \phi_2 (t)$  
	Esto se hace para ecuaciones con la forma $x'' + a_1 (t)\cdot x' + a_0 (t)\cdot x = b(t)$  
	Se imponen las ligaduras siguientes para obtener condiciones sobre $c_1$ y $c_2$:  
	$$
	\begin{cases}
	c_1'\cdot \phi_1 + c_2' \cdot \phi_2 = 0 \\
	c_1'\cdot \phi_1' + c_2'\cdot \phi_2' = b
	\end{cases}
	$$
	En el caso de las ecuaciones de tercer grado tenemos las ligaduras:  
	$$
	\begin{cases}
	c_1'\cdot \phi_1 + c_2'\cdot \phi_2 + c_3'\cdot \phi_3 = 0 \\
	c_1'\cdot \phi_1' + c_2'\cdot \phi_2' + c_3'\cdot \phi_3' = b
	\end{cases}
	$$
	Con esto obtenemos expresiones de la derivada de $c_i$ y obtenemos cada constante integrando.
\end{sol}

%%%%%%%%%%%%%%%%%%%%%%%%%%%%% Sistemas de ecuaciones diferenciales %%%%%%%%%%%%%%%%%%%%%%%%%%%%%

\section{Sistemas de ecuaciones diferenciales}

\begin{ejer}
	A continuación se describe el modelo de resolución.
\end{ejer}
\begin{sol}
	Tenemos una ecuación de la forma $x' = A\cdot x + b$ con A una matriz.  
	Lo primero que debemos hacer es obtener los valores propios de la matriz para obtener una forma de Jordan a la que sea semejante. Esta forma de Jordan nos va a facilitar obtener $e^{A\cdot t}$.  
	Cuando obtenemos los valores propios podemos proponer cuáles son las posibles formas de Jordan y estudiando el rango de $A-\lambda \cdot I$ para cada valor propio distinguimos la forma de Jordan. Después de esto proponemos que como $A\cdot P = P\cdot J$ con una $P = (v1|v2|v3)$ genérica y obtenemos condiciones de ellos del tipo $A\cdot v_i = a_1\cdot v1 + a_2\cdot v_2 + a_3\cdot v_3$  
	A partir de estas condiciones obtenemos si están o no en el núcleo y calculando el núcleo de cada matriz que nos quede obtenemos los vectores $v_1, v_2, v_3$.  
	A partir de la matriz de Jordan obtenemos que $e^{A\cdot t} = P\cdot e^{t\cdot J}\cdot P^{-1}$ teniendo en consideración que:  
	- Si
	$$A=
	\begin{pmatrix}
	\lambda & 0 & ... & 0 \\
	0 & \lambda & ... & 0 \\
	. & . & ... & . \\
	. & . & ... & . \\
	. & . & ... & . \\
	0 & 0 & ... & \lambda
	\end{pmatrix}
	$$  
	entonces la matriz $e^{t\cdot A}$ es:  
	$$
	\begin{pmatrix}
	e^{\lambda \cdot t} & 0 & ... & 0 \\
	0 & e^{\lambda \cdot t} & ... & 0 \\
	. & . & ... & . \\
	. & . & ... & . \\
	. & . & ... & . \\
	0 & 0 & ... & e^{\lambda \cdot t}
	\end{pmatrix}
	$$
	- Si
	$$
	\begin{pmatrix}
	\lambda & 1 & 0 & ... & 0 \\
	0 & \lambda & 1 & ... & 0 \\
	. & . & . & ... & . \\
	. & . & . & ... & . \\
	. & . & . & ... & . \\
	0 & 0 & 0 & ... & 1 \\
	0 & 0 & 0 & ... & \lambda \\
	\end{pmatrix}
	$$  
	entonces la matriz $e^{t\cdot A}$ es:  
	$$
	e^{\lambda \cdot t}
	\begin{pmatrix}
	1 & t & t^2 & ... & \frac{t^{n-1}}{(n-1)!} \\
	0 & 1 & t & ... & \frac{t^{n-2}}{(n-2)!} \\
	. & . & . & ... & . \\
	. & . & . & ... & . \\
	. & . & . & ... & . \\
	0 & 0 & 0 & ... & t \\
	0 & 0 & 0 & ... & 1 \\
	\end{pmatrix}
	$$
	- Si
	$$
	\begin{pmatrix}
	a & b \\
	-b & a
	\end{pmatrix}
	$$
	entonces tenemos que $e^{A\cdot t}$ es:
	$$
	e^{a\cdot t}
	\begin{pmatrix}
	cos(b\cdot t) & sen(b\cdot t) \\
	-sen(b\cdot t) & cos(b\cdot t)
	\end{pmatrix}
	$$
	
	Esta matriz obtenida es la matriz fundamental principal en 0 y por tanto podemos expresar todas las soluciones como esta matriz por un vector de constantes.  
	Para obtener la solución del sistema de ecuaciones completo tenemos que:  
	$x_p(t) = \Phi(t)\cdot \int_{t_0}^{t}\Phi^{-1}(s)\cdot b(s)\cdot ds$  
	Las soluciones del sistema completo son las soluciones de la homogénea mas la solución particular obtenida de la completa.
\end{sol}