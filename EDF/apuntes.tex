
\section{Primeras definiciones}

En esta asignatura estudiaremos ecuaciones diferenciales, es decir, ecuaciones que relacionan una función $x \in \mathcal{C}^k(I)$
con sus derivadas. Para formalizar este concepto, vamos a dar algunas definiciones.

\begin{ndef}[Dominio]
  Sea $D \subseteq \R^N$. Diremos que es un dominio si es abierto y conexo.
\end{ndef}

Ya sabemos que un dominio, por ser abierto y conexo, es conexo por arcos.

\begin{ndef}[Ecuación diferencial]
  Sean $D \subseteq \R^{k+2}$ un dominio, $\upphi : D \to \R$ continua, $I \subseteq \R$ un intervalo y $x \in \mathcal{C}^k(I)$
  tal que $(t, x(t), x^{(1)}(t), \dots, x^{(k)}(t)) \in D$.

  Entonces la ecuación
  \[\upphi(t, x(t), x^{(1)}(t), \dots, x^{(k)}(t)) = 0\]
  es una ecuación diferencial de orden $k$.
\end{ndef}

\section{Ecuaciones de primer orden}
\[ \upphi(t, x(t), x'(t)) = 0,   \upphi(t,x,y) = 0  \]

\begin{ejemplo}
\[x(t)^2 + x'(t)^2 = 4 \]
\[ \upphi(t,x,y) = x^2+y^2 = 4\]
\[ x_1(t) = 2 \]
\[ x_2(t) = -2 \]
\[ x_3(t) = \sqrt{2}\sen t \]
\[ x_4(t) = \sqrt{2}\cos t \]
\[ x_5(t) = \sqrt{2}\cos(t+\alpha) \text{ con } \alpha\in\R \]
\end{ejemplo}

\begin{ejemplo}

\[ x'(t) = 7x(t) \]
\[ \upphi(t,x,y) = y - 7x \]
\[ x_1(t) = e^{7t} \]
\[ x_2(t) = ke^{7t} \text{ con } k \in \R \] 


\end{ejemplo}

Dada $\upphi: D\subset\R^3\to\R$ continua.
$D$ dominio abierto y conexo.
$\upphi(t,x,y)\in\R$

\begin{ndef}

Una solución de la ecuación $\upphi(t, x(t), x'(t)) = 0$ es una función $x:I\to\R$ que cumple:
\begin{itemize}
\item $I$ es un intervalo abierto de $\R$
\item $\exists x'(t), \forall t \in I$
\item $(t, x(t), x'(t)) \in D, \forall t \in I$
\item $\upphi(t,x(t),x'(t)) = 0, \forall t \in I$
\end{itemize}

\end{ndef}


\begin{ejemplo}
$x(t)x'(t) = 1$\\
$ \upphi(t,x,y) = xy - 1$\\
$ \varphi(t) = \sqrt{2t+1}  $\\

$\varphi$ esta definida en $[-1/2, \inf)$ y $\varphi$ es derivable en $(-1/2, \infty)$\\
¿Es la función $\varphi$ en $I = (-1/2, \infty)$ solución de la ecuación?

\begin{itemize}
\item $I$ es abierto de $\R$
\item $\exists \varphi'(t) = \frac{1}{\sqrt{2t+1}}, \forall t \in I$
\item $(t, \varphi(t), \varphi'(t)) \in D = \R^3, \forall t \in I$
\item $\upphi(t, \varphi(t), \varphi'(t)) = 0$
\end{itemize}

Cuando en una ecuación diferencial aparezca la $x'(t)$ despejada, se dice que esta en forma \textbf{normal}. 

$x'(t) = F(t,x(t))$ 1º orden\\

$x'(t) = \frac{1}{x(t)} \implies \upphi(t,x,y) = y - 1/x$\\
$D_1 = \R \times (0,+\infty )\times \R$\\
$D_1 = \R \times (-\infty ,0)\times \R$\\
Como el 0 no era solución son la misma ecuación.


\end{ejemplo}

\section{Ecuación de crecimiento constante}

$x(t)$ cantidad o proporción de población en el instante t.\\
$x'(t) = kx(t)$ k constante.\\
$x(t) = Ae^{kt}$ es solución, $\forall A \in \R$\\
$\upphi(t,x,y)= y - kx,$    $ \upphi:\R^3 \to \R$ continua.\\

\begin{nth}
Sea $\varphi(t)$ una solución definida en $I\subset \R$, entonces, $\varphi (t) = Ae^{kt} \forall t \in I$ para algún $A\in \R$

\end{nth}

\begin{proof}
Sea $\varphi(t)$ solución de $x'(t) = kx(t)$ definida en $I$. Para cada $t\in I$ considero $e^{-kt}\varphi (t)$ es derivable.\\
$(e^{-kt}\varphi (t))' = -ke^{-kt}\varphi (t) + e^{-kt}\varphi' (t) = -ke^{-kt}\varphi (t) + e^{-kt}k\varphi (t) =$\\$ e^{-kt}\varphi (t) (-k+k) = 0, \forall t \in I$\\
Como I es conexo $\implies e^{-kt}\varphi = A$, $\forall t \in I$ $\implies \varphi(t)=Ae^{kt}$, $\forall t \in I$
\end{proof}


Un problema de valores iniciales consiste en buscar la o las soluciones de la ecuación diferencial que pasan por un punto concreto, es decir, $x(t_0)=x_0$.

\section{Variables separadas}

$x' = f(t)g(x)$ con $f:(a,b)\to \R$ y $g:(c,d) \to \R$ continuas.

Suponemos ademas que $g(x) \neq 0$, $x\in (c,d)$.\\
Sea $x(t)$ solución de la ecuación $\implies x:I \to \R$ tal que $I\subset (a,b),$ con $ x(t)\in (c,d), \forall t \in I$

$\exists x'(t), \forall t \in I$ tal que $\frac{x'(t)}{g(x(t))} = f(t), \forall t \in I$\\
Fijado $t_0 \in I$\\
\[\int_{t_0}^{t}\frac{x'(t)}{g(x(t))} ds = \int_{t_0}^{t} f(s) ds, \forall t \in I \implies G(x(t)) - G(x(t_0)) = \int_{t_0}^{t}f(s) ds\] \\
$G'(u) = \frac{1}{g(u)}$ tiene signo constante. Es estrictamente monótona. Por el teorema de la función inversa:
\[x(t) = G^{-1}(G(x_0) + \int_{t_0}^{t}f(s) ds), \forall t \in I\]

\begin{nth}
Sea $f\in C(a,b), g\in C(c,d)$, con $g(u) \neq 0, \forall u \in (c,d)$. Entonces dado $t_0 \in (a,b), x_0 \in (c,d)$, existe una única solución de $x' = f(t)g(x)$ que cumple $x(t_0) = x_0$\\
Si $x_1$ es otra solución $\implies x_1 = x|_{I_1}$ con $I_1\subset I$.

\end{nth}

\begin{proof}



\end{proof}
