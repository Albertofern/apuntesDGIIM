
\section{Primeras definiciones}

En esta asignatura estudiaremos ecuaciones diferenciales, es decir, ecuaciones que relacionan una función $x \in \mathcal{C}^k(I)$
con sus derivadas. Para formalizar este concepto, vamos a dar algunas definiciones.

\begin{ndef}[Dominio]
  Sea $D \subseteq \R^N$. Diremos que es un dominio si es abierto y conexo.
\end{ndef}

Ya sabemos que un dominio, por ser abierto y conexo, es conexo por arcos.

\begin{ndef}[Ecuación diferencial]
  Sean $D \subseteq \R^{k+2}$ un dominio, $\upphi : D \to \R$ continua.

  Entonces la expresión
  \[\upphi(t, x(t), x^{(1)}(t), \dots, x^{(k)}(t)) = 0\]
  es una ecuación diferencial de orden $k$.
\end{ndef}

\begin{ndef}[Solución de una ecuación diferencial]
  Sean 
  \begin{itemize}
  \item $\upphi : D \subseteq \R^{k+2} \to \R$ una función de una ecuación diferencial
  \item $I \subseteq \R$ un intervalo
  \item $x \in \mathcal{C}^k(I)$ tal que $(t, x(t), x^{(1)}(t), \dots, x^{(k)}(t)) \in D$
  \end{itemize}

  Entonces, $x$ es solución de la ecuación diferencial dada por $\upphi$ si

  \[
    \upphi(t, x(t), x^{(1)}(t), \dots, x^{(k)}(t)) = 0 \ \forall t \in I
  \]
\end{ndef}

\section{Ecuaciones de primer orden}
\[ \upphi(t, x(t), x'(t)) = 0,   \upphi(t,x,y) = 0  \]

\begin{ejemplo}
\[x(t)^2 + x'(t)^2 = 4 \]
\[ \upphi(t,x,y) = x^2+y^2 = 4\]
\[ x_1(t) = 2 \]
\[ x_2(t) = -2 \]
\[ x_3(t) = \sqrt{2}\sen t \]
\[ x_4(t) = \sqrt{2}\cos t \]
\[ x_5(t) = \sqrt{2}\cos(t+\alpha) \text{ con } \alpha\in\R \]
\end{ejemplo}

\begin{ejemplo}

\[ x'(t) = 7x(t) \]
\[ \upphi(t,x,y) = y - 7x \]
\[ x_1(t) = e^{7t} \]
\[ x_2(t) = ke^{7t} \text{ con } k \in \R \] 


\end{ejemplo}

Dada $\upphi: D\subset\R^3\to\R$ continua.
$D$ dominio abierto y conexo.
$\upphi(t,x,y)\in\R$

\begin{ndef}

Una solución de la ecuación $\upphi(t, x(t), x'(t)) = 0$ es una función $x:I\to\R$ que cumple:
\begin{itemize}
\item $I$ es un intervalo abierto de $\R$
\item $\exists x'(t), \forall t \in I$
\item $(t, x(t), x'(t)) \in D, \forall t \in I$
\item $\upphi(t,x(t),x'(t)) = 0, \forall t \in I$
\end{itemize}

\end{ndef}


\begin{ejemplo}
$x(t)x'(t) = 1$\\
$ \upphi(t,x,y) = xy - 1$\\
$ \varphi(t) = \sqrt{2t+1}  $\\

$\varphi$ esta definida en $[-1/2, \infty)$ y $\varphi$ es derivable en $(-1/2, \infty)$\\
¿Es la función $\varphi$ en $I = (-1/2, \infty)$ solución de la ecuación?

\begin{itemize}
\item $I$ es abierto de $\R$
\item $\exists \varphi'(t) = \frac{1}{\sqrt{2t+1}}, \forall t \in I$
\item $(t, \varphi(t), \varphi'(t)) \in D = \R^3, \forall t \in I$
\item $\upphi(t, \varphi(t), \varphi'(t)) = 0$
\end{itemize}

Cuando en una ecuación diferencial aparezca la $x'(t)$ despejada, se dice que esta en forma \textbf{normal}. 

$x'(t) = F(t,x(t))$ 1º orden\\

$x'(t) = \frac{1}{x(t)} \implies \upphi(t,x,y) = y - 1/x$\\
$D_1 = \R \times (0,+\infty )\times \R$\\
$D_1 = \R \times (-\infty ,0)\times \R$\\
Como el 0 no era solución son la misma ecuación.


\end{ejemplo}

\section{Ecuación de crecimiento constante}

$x(t)$ cantidad o proporción de población en el instante t.\\
$x'(t) = kx(t)$ k constante.\\
$x(t) = Ae^{kt}$ es solución, $\forall A \in \R$\\
$\upphi(t,x,y)= y - kx,$    $ \upphi:\R^3 \to \R$ continua.\\

\begin{nth}
Sea $\varphi(t)$ una solución definida en $I\subset \R$, entonces, $\varphi (t) = Ae^{kt} \forall t \in I$ para algún $A\in \R$

\end{nth}

\begin{proof}
Sea $\varphi(t)$ solución de $x'(t) = kx(t)$ definida en $I$. Para cada $t\in I$ considero $e^{-kt}\varphi (t)$ es derivable.\\
$(e^{-kt}\varphi (t))' = -ke^{-kt}\varphi (t) + e^{-kt}\varphi' (t) = -ke^{-kt}\varphi (t) + e^{-kt}k\varphi (t) =$\\$ e^{-kt}\varphi (t) (-k+k) = 0, \forall t \in I$\\
Como I es conexo $\implies e^{-kt}\varphi = A$, $\forall t \in I$ $\implies \varphi(t)=Ae^{kt}$, $\forall t \in I$
\end{proof}


Un problema de valores iniciales consiste en buscar la o las soluciones de la ecuación diferencial que pasan por un punto concreto, es decir, $x(t_0)=x_0$.

\section{Variables separadas}

Consideramos la ecuación

\[
x' = f(t)g(x)
\]

con $\ f:(a,b)\to \R \ $ y $\ g:(c,d) \to \R$ continuas.

Sabemos que $x(t) = k$, $k \in \R$ es una familia de soluciones, para las que $x' = 0$. Ahora, si suponemos que $x$
no es constante, vamos a dar condiciones bajo las cuales la solución a un problema de valores iniciales dado por esa
ecuación es única.

Suponemos ademas que $g(x) \neq 0$, $x\in (c,d)$.\\
Sea $x(t)$ solución de la ecuación $\implies x:I \to \R$ tal que $I\subset (a,b),$ con $ x(t)\in (c,d), \forall t \in I$

$\exists x'(t), \forall t \in I$ tal que $\frac{x'(t)}{g(x(t))} = f(t), \forall t \in I$\\
Fijado $t_0 \in I$\\
\[\int_{t_0}^{t}\frac{x'(t)}{g(x(t))} ds = \int_{t_0}^{t} f(s) ds, \forall t \in I \implies G(x(t)) - G(x(t_0)) = \int_{t_0}^{t}f(s) ds\] \\
$G'(u) = \frac{1}{g(u)}$ tiene signo constante. Es estrictamente monótona. Por el teorema de la función inversa:
\[x(t) = G^{-1}(G(x_0) + \int_{t_0}^{t}f(s) ds), \forall t \in I\]

\begin{nth}
Sea $f\in C(a,b), g\in C(c,d)$, con $g(u) \neq 0, \forall u \in (c,d)$. Entonces dado $t_0 \in (a,b), x_0 \in (c,d)$, existe una única solución de $x' = f(t)g(x)$ que cumple $x(t_0) = x_0$\\
Si $x_1$ es otra solución $\implies x_1 = x|_{I_1}$ con $I_1\subset I$.

\end{nth}

% TODO: revisar prueba.


\begin{proof}

$g(u) \neq 0 \ \forall u \in (c,d) \ \implies \ \frac{1}{g} \in C(c,d) \ \implies \ G \text{ una primitiva de } \frac{1}{g}, \ G\in C^{1}(c,d)$\\

$G(u) = \frac{1}{g(u)} \neq 0, \ \forall u \in (c,d) \ \implies \ \exists G^{-1}:V\subset \R \to \R, \ G^{-1}$ derivable.\\

Sea $x_0 \in (c,d)$. $G(x_0) \in  V$, $\ F(t) = \int_{t_0}^t f(s) ds$, $\ F(t_0) = 0$.\\

$t \to G(x_0)+\int_{t_0}^t f(s) ds  \in V$ cuando $t,t_0 \in I$.\\

$\exists t_0 \in I$ t.q. $\forall t \in I, \ G^{-1}(G(x_0) + F(t))$ esta bien definida.\\

$x'(t) = \frac{1}{G'(G^{-1}(G(x_0) + F(t)))}$


\end{proof}

  
\section{Cambio de variable en ecuaciones diferenciales}
El objetivo del cambio de variable será transformar una ecuación en una más fácilmente resoluble.
Vamos a estudiar el caso de las ecuaciones diferenciales de primer orden.

% TODO: definir difeomorfismo. ¿En esta sección?

Consideramos una ecuación de este tipo en forma normal:

\[
  x' = F(t, x),\ F : D \subseteq \R^2 \to \R \ \ \ \ (E)
  \]

  y un difeomorfismo

  \[
  \begin{array}{lll}
    \varphi : & D \to \tilde{D} \\ 
  \end{array}
  \]

  de forma que, si $\tilde{F} = F \circ \varphi$, la ecuación
  \[ \dfrac{dy}{ds} = \tilde{F}(s, y) \ \ \ \ \tilde{E}\]
  es equivalente a la que estamos considerando.


\begin{nprop}

Dado $\varphi = (\varphi _1, \varphi _2) : D \to \tilde{D}$ un difeomorfismo, tal que 
\[
\dfrac{\partial \varphi_1}{\partial t}(t, x(t)) + \dfrac{\partial \varphi_1}{\partial x}(t, x(t)) F(t,x(t)) \neq 0 \forall (t,x) \in D
\]

Entonces el cambio de variable:
\[
s := \varphi _1(t,x)\\
y := \varphi _2 (t,x)
\]
transforma (E) en otra equivalente $(\tilde{E}) \ \ \ \ \frac{dy}{ds} = \tilde{F}(s,y)$ en el sentido de que para cualquier solución $x = x(t)$ de (E) definida en un intervalo I, existe una función $y=y(s)$ solución de ($\tilde{E}$) definida en un intervalo $J$ y $\varphi(t, x(t)) = (s(t), \ \ y(s(t))) \ \forall t \in I$ y reciprocamente.
\end{nprop}

\begin{proof}

  De una parte, sabemos que $\dfrac{dy}{ds} = \dfrac{dy}{dt}\dfrac{dt}{ds}$. De otra, si $x(t)$
  es solución de la ecuación

  % WARNING: justificar el uso del teorema de la función inversa
  
  \[
  \begin{cases}
    \dfrac{ds(t)}{dt} = \dfrac{\partial \varphi_1}{\partial t}(t, x(t)) + \dfrac{\partial \varphi_1}{\partial x}(t, x(t))x'(t) = \dfrac{\partial \varphi_1}{\partial t}(t, x(t)) + \dfrac{\partial \varphi_1}{\partial x}(t, x(t))F(t, x) \\ \\

 \dfrac{dy(t)}{dt} = \dfrac{\partial \varphi_2}{\partial t}(t, x(t)) + \dfrac{\partial \varphi_2}{\partial x}(t, x(t))x'(t) = \dfrac{\partial \varphi_2}{\partial t}(t, x(t)) + \dfrac{\partial \varphi_2}{\partial x}(t, x(t))F(t, x)
   
  \end{cases}
  \]

  Con lo cual

  \[
  \dfrac{dy}{ds} = \dfrac{dy}{dt}\dfrac{dt}{ds} = \dfrac{\dfrac{\partial \varphi_1}{\partial t}(t, x(t)) + \dfrac{\partial \varphi_1}{\partial x}(t, x(t))F(t, x)}{\dfrac{\partial \varphi_2}{\partial t}(t, x(t)) + \dfrac{\partial \varphi_2}{\partial x}(t, x(t))F(t, x)}
  = \dfrac{\dfrac{\partial \varphi_1}{\partial t}(\varphi^{-1}(s, y(s))) + \dfrac{\partial \varphi_1}{\partial x}(\varphi^{-1}(s, y(s)))F(\varphi^{-1}(s, y))}{\dfrac{\partial \varphi_2}{\partial t}(\varphi^{-1}(s, y(s))) + \dfrac{\partial \varphi_2}{\partial x}(\varphi^{-1}(s, y(s)))F(\varphi^{-1}(s, y))} = \tilde{F}(s,y)
  \]

  % TODO: insertar ejemplo de uso

\end{proof}

\subsection{Ecuación homogénea.}

Sea $f:(a,b) \to \R$ continua. Tenemos una ecuación diferencial de la forma $x' = f(\frac{x}{t}) \ \implies \ F(t,x) = f(\frac{x}{t})$, con $t \neq 0, \frac{x}{t} \in (a,b)$.\\
$$Dom(F) = \{(t,x)\in \R ^2 : t \neq 0, \ \frac{x}{t} \in (a,b)\}$$\\
Podemos dividir el dominio de F en dos dominios:\\
$D_1 = \{(t,x) \in \R ^2 : t > 0, \ at < x < bt\}$\\
$D_2 = \{(t,x) \in \R ^2 : t < 0, \ at > x > bt\}$\\

Realizamos un cambio de variable:
\[
s = t\\
y = \frac{x}{t}
\]

Los nuevos dominios son:
\[
\tilde{D}_1 = \{s > 0, a < y < b\}\\
\tilde{D}_2 = \{s < 0, a > y > b\}
\]

La ecuación diferencial queda como una función de variables separadas:

$$y' = \frac{1}{s}(f(y) - y)$$


