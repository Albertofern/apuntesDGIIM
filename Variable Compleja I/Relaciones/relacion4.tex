\subsection{Ejercicio 1}
\subsubsection{b)}


\textbf{Solución}

$\sum_{n\geq 0} z^{2n} = \sum_{n\geq 0} \alpha_nz^n$
con $\alpha_nz{2n+1} =0, \alpha_{2n} = 1$
hay que ver el límite superior de la sucesión
$\limsup |\alpha_n| = 1 \implies R=\frac{1}{1} = 1$

\subsubsection{c)}

\textbf{Solución}

$\sum_{n\geq 0} 2^nz^{n!}$,
si $|z|\geq 1$, entonces la serie diverge, 
si $|z|<1$
$\sqrt[n]{|2^nz^{n!}|} = \sqrt[n]{2^n|z|^{n!}} = 2*|z|^{(n-1)!} \rightarrow 0$
por tanto
$\sum_{n\geq 0} |2^nz^{n!}| < \infty \implies \sum_{n\geq 0} 2^nz^{n!} < \infty$, entonces $R=1$.

\subsubsection{e)}

\textbf{Solución}

$\sum_{n\geq 0} (n+a^n)z^n$ con $a\in\mathbb{R}^+$.
Lo separamos en dos series, calculamos los radios de convergencia por separado y cogemos el menor.

$\sum_{n\geq 0} nz^n$ y $\sum_{n\geq 0} a^nz^n$

% ---------------------- hay que arreglarlo
$a^n \leq n+a^n \leq (n+1)a^n$
en los extremos el radio de convergencia es $1/a$, por tanto el radio de convergencia de $n+a^n$ es $1/a$

$\sum_{n\geq 0} a^n |z|^n \leq \sum_{n\geq 0} (n+a^n)|z|^n \leq \sum_{n\geq 0} (n+1)a^n|z|^n$

Por tanto el radio de convergencia es $1/a$...................................
% ----------------------

//Podemos usar el criterio de la raíz para sucesiones 
$\sum_{n\geq 0} (n+a^n)z^n$ con $a\in\mathbb{R}^+$
Calculamos $\lim_{n\rightarrow\infty} \sqrt[n]{n+a^n}$
$ = \lim_{n\rightarrow\infty} \frac{(n+1)+a^{n+1}}{n+a^n} = a (\text{si a>1}), 1 (\text{ si a<=1})$
por tanto
$R = 1/a (a>1), 1 (a<=1)$

\subsubsection{c)}
$\sum_{n\geq 0} a^{n^2}z^n,a\in\mathbb{C}$
raíz n-esima 


\subsection{Ejercicio 2}
\subsubsection{a)}
No se reduce el radio de convergencia, se puede intentar el radio de convergencia de las derivadas


\subsection{Ejercicio 3}
Convergen en todo el plano las que a partir de cierto término son 0, o sea, las que son una suma finita

Sea $\sum_{n\geq 0} \alpha_n (z-a)^n$ que converge uniformemente,
$S_n=\sum_{k=0}^{n-1} \alpha_k (z-a)^k$ c.u. $\Longleftrightarrow S_n$ es uniformemente de Cauchy $\Longleftrightarrow \forall\epsilon>0,\exists m\in\mathbb{N}: p\geq q\geq n \implies |S_p(z)-S_q(z)| < \epsilon \forall z\in\mathbb{C}$
Vemos que los polinomios divergen en infinito.
Sea $p\in\mathbb{P}[\mathbb{C}]$, $\lim_{|z|\rightarrow\infty} p(z) = \infty \Longleftrightarrow gr(p)>0 $

Vemos que $|S_p(z)-S_q(z)|$ es un polinomio que está acotado, tiene que ser constante para no diverger en infinito, en particular es la constante $0$, ya que $S_p(a)=S_q(a)$, por tanto $\sum_{q=m}^{p-10} \alpha_n (z-\alpha)^n = 0 \implies  \alpha_n=0 \forall n\geq m$

\subsection{Ejercicio 4}
Idea:

llamar w a (w-1)/(z-1), y mirar cuando la función tiene imagen que cae en el disco de centro 0 y radio 1, afinar para saber cuando un subconjunto de C po esta función cae dentro de un subconjunto compacto donde el sumatorio converja
