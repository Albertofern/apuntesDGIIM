\subsection{Ejercicio 1}
\textbf{Enunciado}

\textbf{Solución}

$\gamma :[0,r] \rightarrow \mathbb{C}$,
$\gamma (s) = \alpha + s, s\in[0,r]$

$\int_{[\alpha,\alpha + r]} f(z)dz = \int_{0}^{r} f(\alpha +s)\gamma ' (s)ds$, donde $\gamma ' (s) = 1 \forall s\in [0,r]$

$\int_{0}^{r} f(\alpha +s)\gamma ' (s)ds = \int_{0}^{r} f(\alpha + s)ds$

$\xi : [0,r] \rightarrow \mathbb{C}$, $\xi(s) = \alpha +is \forall s\in[0,r]$

$\int_{[\alpha,\alpha + ir]} f(z)dz = \int_0^r f(\alpha + is) \gamma ' (s)ds = i\int_0^r f(\alpha +is)ds$



\subsection{Ejercicio 2}
\textbf{Enunciado}

\textbf{Solución}
 
Tenemos que probar que
$\lim_{r\rightarrow\infty} \int_{\gamma_r} \frac{zdz}{z^3+1} = 0$

$| \int_{\gamma_r} \frac{z}{z^3+1}dz | \leq \int_{\gamma_r} \frac{|z|}{|z^3+1|} dz \leq l(\gamma_r)M(r)$
donde $M(r)>0$ satisface que $\frac{|z|}{|z^3+1|} \leq M(r) \forall z\in\gamma_r{\ast}$

Dado $z\in\mathbb{C}(0,r)^{\ast}$, $|z| = r $

$\frac{|z|}{|z^3+1|} \leq \frac{r}{|z^3|-1} = \frac{r}{r^3-1}$

Por tanto
$l(\gamma_r)M(r) = 2\pi r \frac{r}{r^3-1}$ y $\lim_{r\rightarrow\infty} \frac{r}{r^3-1} = 0$

\

Para la otra integral tenemos que probar que

$\lim_{r\rightarrow\infty} \int_{\gamma_r} \frac{z^2 e^z}{z+1} dz = 0$ 

$| \int_{\gamma_r} \frac{z^2 e^z}{z+1}dz | < \int_{\gamma_r} \frac{|z|^2 |e^z|}{z+1}dz$

Si $z\in\gamma_r^{\ast}$
$$ |z|^2 = |-r+is|^2 \leq (r+1)^2 \hspace{1cm} s\in[0,1] $$
$$ |z+1| \geq r-1 $$

Como $z\in\gamma_r^{\ast}$, $|e^z| = e^{Rez} = e^{-r}$




\subsection{Ejercicio 3}
\textbf{Enunciado}

\textbf{Solución}

$\int_{\mathbb{C}(0,r)} \frac{\log(i+z)}{z} dz$

donde $0<r<1$, $\mathbb{C}(0,r)^{\ast} \subset D(0,1)$ definimos $f(z) = \log(1+z)$, $f\in\mathcal{H}(D(0,1))$

Usando la fórmula de Cauchy para la circunferencia:

$\int_{\mathbb{C}(0,r)} \frac{\log(i+z)}{z} dz = f(0)2\pi i = 0$

\

Sin usar la fórmula de Cauchy para la circunferencia vemos:

$\log (1+z) = \sum_{n=1}^{\infty} (-1)^{n+1} \frac{z^n}{n}$,
la serie $\sum_{n\geq 1} (-1)^{n+1} \frac{z^n}{n}$ converge uniformemente sobre compactos de $D(0,1)$.

La sucesión $\{ \frac{1}{z} \}$ está acotada en el compacto $\mathbb{C}(0,1)^{\ast}$ 

Usando \textbf{observación} 
$\frac{1}{z} \sum_{n\geq 1} (-1)^{n+1} \frac{z^n}{n}$ converge uniformemente en $\mathbb{C}(0,r)^{\ast}$ a $\frac{\log (1+z)}{z}$

\textbf{Observación}

$\{ f_n \} $ converge uniformemente en $B$ a $f$  
y $g:B\rightarrow\mathbb{C}$ está acotada en $B$, entonces
$\{ gf_n \}$ converge uniformemente en $B$ a $gf$

($\exists M>0 : |g(z)| \leq M \hspace{0.5cm} \forall z\in B  $)


---------------------

$0 = \int_{\mathbb{C}(0,r)} \frac{\log(1+z)}{z} dz = \int_{-\pi}^{\pi}  \frac{\log(1+re^{it})}{re^{it}} dt$
$= i\int_{-pi}^{\pi} \log(1+re^{it}) dt$
$i\int_{-\pi}^{\pi} \ln|1+re^{it}|dt -\int_{-\pi}^{\pi} arg(1+re^{it})dt$
$\implies \int_{-\pi}^{\pi} \ln |1+re^{it}|dt = 0$

Sabemos que $\log(1+re^{it}) = \ln|1+re^{it}| + i*arg(1+re^{it})$
y que
$\gamma(t) = re^{it}, \gamma:[-\pi,\pi] \rightarrow \mathbb{C}$, $\gamma '(t) = ire^{it}$


también sabemos que
$|1+re^{it}| = ((1+r\cos(t))^2 + r^2\sin^2(t))^{1/2} = (1+2r\cos(t) + r^2\cos^2(t) + r^2\sin^2(t))^{1/2} = (1+2r\cos(t)+r^2)^{1/2}$


por tanto
$0 = \int_{-\pi}^{\pi} \ln|1+re^{it}| dt = 1/2 \int_{-\pi}^{\pi} \ln(1++2r\cos(t) +r^2)dt$

Como sabemos que el coseno es par
$ 1/2 \int_{-\pi}^{\pi} \ln(1++2r\cos(t) +r^2)dt = \int_{0}^{\pi} \ln (1+2r\cos(t)+r^2) dt$


\subsection{Ejercicio 4}
\textbf{Enunciado}

\textbf{Solución}

Sabemos que $f\in D(0,1)$, teniendo que $|f(z)-1|<1, \forall z\in D(0,1)$

Sabemos que $f$ no se anula en $D(0,1)$ y deducimos que
$f(D(0,1)) \subset D(1,1)$.
Por tanto $\log(f) \in \mathcal{H}(D(0,1))$

$\log(f(z))' = \frac{f'(z)}{f(z)} \implies \frac{f'}{f}$ admite primitiva holomorfa en $D(0,1)$
y $\mathbb{C}(0,r) \subset D(0,1)$ es un camino cerrado, por tanto
$\int_{\mathbb{C}(0,r)} \frac{f'(z)}{f(z)}dz = 0$



\subsection{Ejercicio 5}
\textbf{Enunciado}

\textbf{Solución}

$\frac{1}{1+z^2} = \frac{1}{(z-i)(z+i)} = \frac{A}{z-i} + \frac{B}{z+i} = \frac{1}{2i} \left[ \frac{1}{z-i} - \frac{1}{z+i} \right]$

Por tanto

$\int_{\mathbb{C}(i,1)} \frac{1}{1+z^2} dz = \frac{1}{2i} \left[ \int_{\mathbb{C}(i,1)}\frac{1}{z-i}dz - \int_{\mathbb{C}(i,1)} \frac{1}{z+i}dz \right] \not = 0$

La primera es $2\pi i$

La segunda integral es $0$ por el teorema de Cauchy para dominios estrellados, teniendo que $\mathbb{C}(i,1) \subset D(i,3/2)$.



\subsection{Ejercicio 6}
\textbf{Idea}

$\Omega = \mathbb{C}\backslash \{ iy : |y|>1 \}$
$\arctan \in \mathcal{H}(\Omega)$ y $\arctan$ es una primitiva de $\frac{1}{1+z^2}$

$\sigma ^{\ast} \subset \Omega$ y $\sigma$ es un camino cerrado.

Con esa información sabremos que 
$\int_{\sigma} \frac{1}{1+z^2} dz = 0$

