\subsection{Ejercicio 1.} Probar que el conjunto de matrices
$$ M = \left\{ 
	\left( \begin{array}{cc}
	 a & b \\
	-b & a \end{array} \right)
	 : \ a, b\in \mathbb{R}\right\} $$
con las operaciones de suma y producto de matrices, es un cuerpo isomorfo a C .

\

\textbf{Solución}

\





\subsection{Ejercicio 3.}
Sea $U=\{ z\in\mathbb{C} : |z|<1 \}$.
Fijado $a \in U$, se considera la función $f: U\rightarrow\mathbb{C}$ dada por
----------------------------


\textbf{Solución}

$|f(z)| < 1 \ \forall z : |z| <1$
$f^{-1}(z)= \frac{z+a}{1+\overline{a}z}$
Si $|z|=1$, entonces $|f(z)| = |\frac{z-a}{a-\overline{a}z}|$
multiplicando en esta expresión por $\overline{z}$
$ \frac{z-a}{1-\overline{a}z} \overline{z}  =  \frac{1-a\overline{z}}{1-\overline{a}z}$
$f$ tenemos que es holomorfa en el disco, lleva la frontera en la frontera.



\subsection{Ejercicio 4.}
Dados $z_1, z_2, . . . , z_n \in C^{\ast}$ , encontrar una condición necesaria y suficiente para que se
verifique la siguiente igualdad:
$$ |\sum_{k=1}^n z_k| = \sum_{k=1}^n |z_k|$$

\textbf{Solución}

Por inducción, todos los números complejos deben tener el mismo argumento, son vectores linealmente dependientes sin que se invierta el signo de ninguno de ellos.

$\exists\lambda_1,...,\lambda_n>0 : \lambda_1 z_1 = \lambda_2 z_2 = ... = \lambda_n z_n$

para probar que es necesaria no hace falta hacer inducción
$ |\sum_{k=1}^n z_n| = |\sum_{k=1}^n \frac{\lambda_1}{\lambda_k} z_1| 
=
|z_1|\sum_{k=1}^n \frac{\lambda_1}{\lambda_k} 
=
|z_1|\sum_{k=1}^n \frac{|z_k|}{|z_1|} 
=
\sum_{k=1}^n |z_k|$

$n=2 \implies |z_1+z_2| = |z_1| + |z_2| \longleftrightarrow z_2 = \lambda z_1$ con $\lambda \in \mathbb{R}$

Sea cierto para $n\in\mathbb{N}$
$|\sum_{k=1}^{n+1} z_k| 
=
|z_{n+1} + \sum_{k=1}^n z_k|
=
|z_{n+1}| + |\sum_{k=1}^n z_k|$

Hemos usado en este último paso que
$|z_{n+1}|+|\sum_{k=1}^n| \geq |\sum_{k=1}^{n+1} z_k|=\sum_{k=1}^{n+1} |z_k| = |z_{n+1}| + \sum_{k=1}^n |z_k|$
donde la otra desigualdad entre los extremos se sabe de la desigualdad triangular.
%Ahora usamos la hipótesis de inducción
$|z_{n+1}| + |\sum_{k=1}^n z_k| 
=
|z_{n+1}| + \sum_{k=1}^n |z_k|$
Por la hipótesis de inducción para $k=2$, tenemos que 
$\exists \lambda >0 : z_{n+1} = \lambda \sum_{k=1}^n z_k$

Nos queda por hacer lo análogo con el resto de números complejos $z_k,k=1...n$
Por la hipótesis de inducción para $k=n, \exists \mu_1,...\mu_n : \mu_1 z_1 = \mu_2 z_2=...= \mu_n z_n$,
como $z_{n+1} = \lambda \sum_{k=1}^n z_k 
=
\lambda \sum_{k=1}^n \frac{\mu_1}{\mu_k} z_1$

$$ |z_1+z_2| = |z_1|+|z_2| 
\Longleftrightarrow
|z_1+z_2|^2 = (|z_1|+|z_2|)^2 
=
|z_1|^2 + 2|z_1||z_2|+|z_2|^2$$
 entonces tenemos
$$z_1 \overline{z_2}=\lambda>0 \text{ y que } z_1 = \frac{\lambda}{|z_2|^2 }z_2 
$$


\subsection{Ejercicio 5.}
Describir geométricamente los subconjuntos del plano dados por
$$ A=\{ z\in\mathbb{C} : |z+i|=2|z-i| \} \text{ y } B=\{ z\in\mathbb{C} : |z-i| + |z+i| = 4 \} $$

Veamos el conjunto A, tenemos en cuenta $z=(a,b)$,
$A$ cumple $ \sqrt{a^2+(b+1)^2} = 2\sqrt{a^2+(b-1)^2}$
, entonces $a^2+b^2+1+2 b = a^2+(b+1)^2 = 4(a^2+(b-1)^2) = 4a^2+4b^2+4-8b$, entonces
$3a^2+3b^2-10b+3=0$, $a^2+b^2-\frac{10}{3}b+1=0$, sumamos y restamos $\frac{25}{9}$, $a^2+(b-\frac{5}{3})^2-\frac{25}{9}+1$, 
$a^2+(b-\frac{5}{3})^2=\frac{16}{9}$
Vemos que A es la circunferencia con $c=(0, \frac{5}{3})$, $r=\frac{4}{3}$

Veamos el B, elevando dos veces al cuadrado tenemos como resultado una elipse
$$ \frac{a^2}{\sqrt{3}^2} + \frac{b^2}{2^2} = 1 $$


\subsection{Ejercicio 6.}
Probar que $arg z = 2\arctan( \frac{Im z}{Re z + |z|} )$ para todo  $z\in\mathbb{C}^{\ast}\backslash\mathbb{R}^{-}$.

\textbf{Solución}

Hemos usado la fórmula del ángulo doble
$$ \cos(2\alpha) = \cos^2(\alpha)-\sin^2(\alpha) $$
$$ \sin(2\alpha) = 2\sin(\alpha)\cos(\alpha) $$

$\phi = \arctan(\frac{Im z}{Re z + |z|})$
$\phi = 2\alpha = \arctan(\frac{Im z}{Re z + |z|}) $

\textbf{Pista para otra forma de hacerlo}
$$ \phi = 2\arctan ( \frac{Im z}{Re z + |z|} ) 
|z|(\cos(\phi)+i\sin(\phi)) =? z $$
Para ello usamos, si $t=\theta/2$
$$ \cos(\theta) = \frac{1-\tan^2(\theta/2)}{1+\tan^2(\theta/2)}
\sin(\theta) = \frac{2\tan(\theta/2)}{1+\tan^2(\theta/2)} $$

\subsection{Ejercicio 8.}

\textbf{Solución}

Haremos la prueba por inducción, para el caso inicial $n=1$ es trivial, suponemos cierto para un $n\in\mathbb{N}$ genérico.
Probamos que en ese caso es cierto para $n+1$.

$$ 
\cos((n+1)\theta)+i\sin((n+1)\theta) 
=
\cos(n\theta+\theta)+i\sin(n\theta+\theta)
$$
$$ =
\cos(n\theta) \cos(\theta)- \sin(n\theta)\sin(\theta) +i\cos(n\theta)\sin(\theta) + i\sin(n\theta)\cos(\theta) $$
$$ =
\cos(n\theta)(cos(\theta)+i\sin(\theta)) + i\sin(n\theta)(cos(\theta)+i\sin(\theta))
$$
$$=
(\cos(theta)+i\sin(\theta)) (\cos(n\theta)+i\sin(n\theta))
=
(cos(\theta)+i\sin(theta))^{n+1}
$$
La última igualdad la tenemos por la hipótesis de inducción.


\subsection{Ejercicio 10.}
\textbf{Pista}
Probar las dos simultáneamente y usar la fórmula de moivre
$ \sum_{k=0}^n ( cos(x)+sin(x) )^k $

\textbf{Solución}

$$\sin(x/2) \sum_{k=0}^{n} \cos(kx) + i\sin(x/2) \sum_{k=0}^{n} \sin(kx)$$
Sacamos factor común 
$$ \sin(x/2)( \sum_{k=0}^{n} \cos(kx)+i\sin(kx) ) $$
Usamos la fórmula de Moivre
$$ \sin(x/2) \sum_{k=0}^{n} (\cos(x)+i\sin(x)^k $$
Usamos que $x\not\in 2\pi\mathbb{Z}$
$$ \sin(x/2) \frac{1-(\cos(x)+i\sin(x)^{n+1}}{1-(\cos(x)+i\sin(x)} $$
$$= \sin(x/2) \frac{ 1- (\cos((n+1)x)+i\sin((n+1)x)) }{1 - \cos(x) - i\sin(x)} $$
Usamos que $2\sin^2(x/2) = 1-\cos(x)$ y que $\sin(x) = \sin(x/2)\cos(x/2)$,
$$ 
\sin(x/2) \frac{ 1- (\cos((n+1)x)+i\sin((n+1)x)) }{ 2\sin^2(x/2)-2i\sin(x/2)\cos(x/2) }
=
\frac{ 1- (\cos((n+1)x)+i\sin((n+1)x)) }{ 2\sin(x/2)-2i\cos(x/2) }
 $$
$$
\frac{ 2\sin^2 (\frac{n+1}{x}x)-i2\sin(\frac{n+1}{2}x)\cos(\frac{n+1}{2}x) }{ 2\sin(x/2) -2i\cos(x/2) }
$$
Multiplicamos numerador y denominador por el conjugado del denominador
$$
\sin(\frac{n+1}{2}x)( \sin(\frac{n+1}{2}x)-i\cos(\frac{n+1}{2}x) )( \sin(x/2)+i\cos(x/2) )
$$
$$ 
=  \sin(\frac{n+1}{2}x) ( \cos(\frac{n+1}{2}x)\cos(x/2) + \sin(\frac{n+1}{2}x)\sin(x/2)  + i( \sin(\frac{n+1}{2}x)\cos(x/2) - \cos(\frac{n+1}{2}x)\sin(x/2) ) )
$$
$$ = \sin(\frac{n+1}{2}x) (\cos(\frac{nx}{2}) + i\sin(\frac{nx}{2}))
$$

